
%==============================
??????????????????????????


یہاں شکل \حوالہ{شکل_مویج_مستطیل_گھمکیا} کو دوبارہ دیکھیں۔اس کا بایاں طرف \عددیء{x=0} اور دایاں طرف \عددیء{x=x_1} ہیں۔ان دونوں موصل اطراف پر متوازی برقی میدان صفر ہو گا۔یوں \عددیء{X(0)Y(y)Z(z)=0} اور \عددیء{X(x_1)Y(y)Z(z)=0} ہوں گے جن سے \عددیء{X(0)=0} اور \عددیء{X(x_1)=0} لکھے جا سکتے ہیں۔یہ شرائط مساوات \حوالہ{مساوات_مویج_گھمکی_عمومی_الف} میں پر کرنے سے
\begin{align}
c_1&=0\\
k_x &=\frac{l \pi}{x_1} \quad \quad (l=1,2,3\cdots)
\end{align}
حاصل ہوتے ہیں۔اسی طرح شکل \حوالہ{شکل_مویج_مستطیل_گھمکیا} کے بقایا اطراف پر برقی میدان صفر پر کرنے سے
\begin{align}
c_3&=0\\
k_y &=\frac{n \pi}{y_1} \quad \quad (n=1,2,3\cdots)
\end{align}
اور
\begin{align}
c_5&=0\\
k_z &=\frac{m \pi}{z_1} \quad \quad (m=1,2,3\cdots)
\end{align}
حاصل ہوتے ہیں۔ان معلومات سے
\begin{gather}
\begin{aligned}
M(x,y,z)&=E_{x0} \sin k_x x \sin k_y y \sin k_z z \\
&=E_0 \sin \frac{l \pi x}{x_1}\sin \frac{n\pi y}{y_1}\sin \frac{m \pi z}{z_1}
\end{aligned}
\end{gather}
حاصل ہوتا ہے جہاں \عددیء{c_2 c_4 c_6=E_{x0}} لکھا گیا ہے۔مساوات \حوالہ{مساوات_مویج_مستطیلی_گھمکیا_ث} سے برقی میدان کی مکمل مساوات
\begin{align}
E(x,y,z,t)=E_{x0} \sin k_x x \sin k_y y \sin k_z z e^{j \omega t}
\end{align}
حاصل ہوتی ہے۔مندرجہ بالا مساوات نیم دوری سمتیہ کی شکل میں ہے۔اس کا حقیقی جزو اصل میدان ہو گا یعنی
\begin{align}
E(x,y,z,t)=E_{x0} \sin k_x x \sin k_y y \sin k_z z \cos \omega t
\end{align}
چونکہ \عددیء{\cos \omega t} اور \عددیء{\cos (-\omega t)} برابر ہوتے ہیں لہٰذا \عددیء{T=c_t e^{+j \omega r}} اور \عددیء{T=c_t e^{-j \omega t}} میں سے کسی ایک کو منتخب کیا جا سکتا ہے۔مندرجہ بالا مساوات میں \عددیء{c_t} کو بھی \عددیء{E_{x0}} میں زم کیا گیا ہے۔

مساوات  \حوالہ{مساوات_مویج_مستطیلی_گھمکی_علیحدگی_پ} سے
\begin{align}
k^2=\left(\frac{m \pi}{x_1}\right)^2+\left(\frac{m \pi}{y_1}\right)^2 +\left(\frac{m \pi}{z_1}\right)^2
\end{align}
حاصل ہوتا ہے۔
