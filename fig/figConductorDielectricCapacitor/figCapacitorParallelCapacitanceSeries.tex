\documentclass[b5paper]{standalone} 
\usepackage{fontspec}
\usepackage{tikz}
\usetikzlibrary{intersections,decorations.markings}
\usepackage{commath}
\usepackage{polyglossia}                %is loaded the last

\input{../../tex/myEMTvectors.tex}

\setmainlanguage[numerals=maghrib]{arabic}     %for english numbers use numerals=maghrib, for arabic numerals=arabicdigits
\setotherlanguages{english}


\newfontfamily\arabicfont[Scale=1.0,Script=Arabic]{Jameel Noori Nastaleeq}
%\newfontfamily\urdufont[Scale=1.0,Script=Arabic]{XB Tabriz}
\newfontfamily\urdufont[Scale=1.25,WordSpace=6.0,,Script=Arabic]{Jameel Noori Nastaleeq}
%




 \pgfmathsetmacro{\delX}{0.7}            %start points, lower left corner
 \pgfmathsetmacro{\delL}{4}
 \pgfmathsetmacro{\delZ}{0.7}


\begin{document}
\begin{urdufont}
\begin{tikzpicture}[x={(-0.5cm,-0.5cm)},y={(1cm,0cm)},z={(0cm,1cm)}]
%text
\draw [stealth-stealth] (1,1.5,0)--++(0,0,-1)node[pos=0.5,fill=white]{$d_1$};
\draw [stealth-stealth] (1,1.5,0)--++(0,0,1)node[pos=0.5,fill=white]{$d_2$};
\draw (1,0.5,0.5)node{$\epsilon_2$};
\draw (1,0.5,-0.5)node{$\epsilon_1$};
\draw (0.5,1,1)node{$S$};
%upper front and side
\shade[opacity=0.5] (1,0,1)--++(0,0,-1)--++(0,2,0)--++(0,0,1)--cycle;
\shade[opacity=0.5] (1,2,0)--++(-1,0,0)--++(0,0,1)--++(1,0,0)--cycle;
%lower front and side
\shade[opacity=0.5] (1,0,0)--++(0,0,-1)--++(0,2,0)--++(0,0,1)--cycle;
\shade[opacity=0.5] (1,2,-1)--++(-1,0,0)--++(0,0,1)--++(1,0,0)--cycle;
%top and bottom conductors
\draw[thick,fill=gray,opacity=0.5] (0,0,1)coordinate(a)--++(1,0,0)coordinate(b)--++(0,2,0)coordinate(c)--++(-1,0,0)coordinate(d)--cycle;
\draw[thick] (b)++(0,0,-2)--++(0,2,0)--++(-1,0,0);
%text
\draw[stealth-] (1,0,1)++(0,-0.2,0) to [out=180,in=30] (1,-1.5,0.2)node[left]{\RL{موصل سطحیں}};
\draw[stealth-] (1,0,-1)++(0,-0.2,0) to [out=180,in=-30] (1,-1.5,0);
\draw (0,2.5,-0.5)node[right]{$C_1=\frac{d_1}{\epsilon_1 S}$};
\draw (0,2.5,0.5)node[right]{$C_2=\frac{d_2}{\epsilon_2 S}$};
\draw(0,5,0)node[right]{$\frac{1}{C}=\frac{1}{C_1}+\frac{1}{C_2}$};
\end{tikzpicture}%
\end{urdufont}
\end{document}
