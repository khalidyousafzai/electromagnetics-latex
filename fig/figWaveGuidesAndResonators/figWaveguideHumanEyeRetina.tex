\documentclass{standalone}
\usepackage{fontspec}
\usepackage{tikz}
\usepackage{pgfplots}
\pgfplotsset{compat=1.9}
\usepackage[europeanresistors]{circuitikz}
\usetikzlibrary{intersections,decorations.markings,calc,decorations.pathreplacing}
\usepackage{polyglossia}                %is loaded the last
\usepackage{siunitx}
\usepackage{amsmath}
\input{../../tex/myEMTvectors.tex}

\setmainlanguage[numerals=maghrib]{arabic}     %for english numbers use numerals=maghrib, for arabic numerals=arabicdigits
\setotherlanguages{english}


\newfontfamily\arabicfont[Scale=1.0,Script=Arabic]{Jameel Noori Nastaleeq}
%\newfontfamily\urdufont[Scale=1.0,Script=Arabic]{XB Tabriz}
\newfontfamily\urdufont[Scale=1.25,WordSpace=6.0,,Script=Arabic]{Jameel Noori Nastaleeq}

%draws left and right arrows where needed e.g.  
% \draw[->-=0.5] (0,0)--(3,0); draws arrow at the middle
\tikzset{->-/.style={decoration={markings, mark=at position #1 with {\arrow{latex}}},postaction={decorate}}}
\tikzset{-<-/.style={decoration={markings, mark=at position #1 with {\arrow{latex reversed}}},postaction={decorate}}}


\pgfmathsetmacro{\cX}{1} 
\pgfmathsetmacro{\cY}{0.3} 
\pgfmathsetmacro{\axon}{0.5} 
\pgfmathsetmacro{\dendrite}{0.2} 

\def\coneEye at (#1,#2){\draw(#1,#2) to [out=0,in=-160] ++(\cX,1/3*\cY) --++(0,1/3*\cY) to [out=160,in=0] ++(-\cX,1/3*\cY) .. controls (#1-5/3*\cY,#2+1.5*\cY) and (#1-5/3*\cY,#2-0.5*\cY) .. (#1,#2);
\path(#1,#2+\cY) .. controls (#1-5/3*\cY,#2+1.5*\cY) and (#1-5/3*\cY,#2-0.5*\cY) .. (#1,#2)coordinate[pos=0.5](coneOut);
\draw(coneOut)--++(-\axon,0)++(-\dendrite,0) circle (\dendrite)++(-\dendrite,0)--++(-\axon,0)++(-\dendrite,0) circle (\dendrite); }
%
\def\rodEye at (#1,#2){\draw(#1,#2) to [out=0,in=-160] ++(\cX,1/3*\cY) --++(0,1/3*\cY) to [out=160,in=0] ++(-\cX,1/3*\cY) -- (#1,#2);
\draw(#1,#2+\cY/2)--++(-\axon/2,0)++(-\dendrite,0) circle (\dendrite);
}
\pgfmathsetmacro{\yStep}{0.6} 

\begin{document}
\begin{urdufont}
\begin{tikzpicture}
%grid
%\draw[gray,thick] (-3,0) grid (1,5);
%\draw[gray,xstep=0.1,ystep=0.1] (-3,0) grid (1,5);
%rods and cones
\foreach \m in {0,1,2,3,4}{\rodEye at (0,\m*\yStep);} 
\foreach \m in {0,1,2}{\coneEye at (0,5*\yStep+\m*\yStep);} 
%rods in parallel, bottom to top
\draw (-\axon/2-2*\dendrite,\yStep+\cY/2)--++(-\axon/2,0)++(-\dendrite,0)circle(\dendrite);
\path[name path=rodsConnection](-\axon/2-2*\dendrite,\yStep+\cY/2)++(-\axon/2,0)++(-\dendrite,0)circle(\dendrite);
\path[name path=lowerConn](-\axon/2-2*\dendrite,0+\cY/2) to [out=180,in=-90] ++(-\axon/2-\dendrite,\yStep);
\draw[name intersections={of=rodsConnection and lowerConn}](-\axon/2-2*\dendrite,0+\cY/2) to [out=180,in=-90] (intersection-1);
%
\path[name path=upperConn](-\axon/2-2*\dendrite,2*\yStep+\cY/2) to [out=180,in=90] ++(-\axon/2-\dendrite,-\yStep);
\draw[name intersections={of=rodsConnection and upperConn}](-\axon/2-2*\dendrite,2*\yStep+\cY/2) to [out=180,in=90] (intersection-1);
%
\draw[name path=rodsConnectionUpper] (-\axon/2-2*\dendrite,4*\yStep-\cY/2)++(-\axon/2-\dendrite,0) circle(\dendrite);
\path[name path=lowerConnKK](-\axon/2-2*\dendrite,3*\yStep+\cY/2) to [out=180,in=-90] ++(-\axon/2-\dendrite,\yStep/2);
\path[name path=upperConnKK](-\axon/2-2*\dendrite,4*\yStep+\cY/2) to [out=180,in=90] ++(-\axon/2-\dendrite,-\yStep/2);
\draw[name intersections ={of=lowerConnKK and rodsConnectionUpper}](-\axon/2-2*\dendrite,3*\yStep+\cY/2) to [out=180,in=-45] (intersection-1);
\draw[name intersections ={of=upperConnKK and rodsConnectionUpper}](-\axon/2-2*\dendrite,4*\yStep+\cY/2) to [out=180,in=45] (intersection-1);
%
\draw(-\axon-4*\dendrite,\yStep+\cY/2)--++(-\axon,0)++(-\dendrite,0) circle(\dendrite);
\draw(-\axon-4*\dendrite,3.5*\yStep+\cY/2)--++(-\axon,0)++(-\dendrite,0) circle(\dendrite);
%cones to optic nerve. top to bottom
\path[name path=opticEnds] (0,8*\yStep+0.2)--++(-3.5,0);
\path(0,7*\yStep+\cY) .. controls (0-0.5,7*\yStep+1.5*\cY) and (0-0.5,7*\yStep-0.5*\cY) .. (0,7*\yStep)coordinate[pos=0.5](coneOut);
\draw(coneOut)++(-2*\axon-4*\dendrite,0)--++(-0.1,0) to [out=180,in=-90]++(-0.2,0.2)coordinate(toOptic);
\path[name path=optic](toOptic)--++(0,0.5);
\draw[name intersections={of=optic and opticEnds}] (toOptic)--(intersection-1);
%
\path(0,6*\yStep+\cY) .. controls (0-0.5,6*\yStep+1.5*\cY) and (0-0.5,6*\yStep-0.5*\cY) .. (0,6*\yStep)coordinate[pos=0.5](coneOut);
\draw(coneOut)++(-2*\axon-4*\dendrite,0)--++(-0.2,0) to [out=180,in=-90]++(-0.2,0.2)coordinate(toOptic);
\path[name path=optic](toOptic)--++(0,1*\yStep+0.5);
\draw[name intersections={of=optic and opticEnds}] (toOptic)--(intersection-1);
%
\path(0,5*\yStep+\cY) .. controls (0-0.5,5*\yStep+1.5*\cY) and (0-0.5,5*\yStep-0.5*\cY) .. (0,5*\yStep)coordinate[pos=0.5](coneOut);
\draw(coneOut)++(-2*\axon-4*\dendrite,0)--++(-0.3,0) to [out=180,in=-90]++(-0.2,0.2)coordinate(toOptic);
\path[name path=optic](toOptic)--++(0,2*\yStep+0.5);
\draw[name intersections={of=optic and opticEnds}] (toOptic)--(intersection-1);
%rods to optic nerve. top to bottom
\draw(-2*\axon-6*\dendrite,3.5*\yStep+\cY/2)--++(-0.4,0) to [out=180,in=-90] ++(-0.2,0.2) coordinate(toOptic);
\path[name path=optic](toOptic)--++(0,3.5*\yStep+0.5);
\draw[name intersections={of=optic and opticEnds}] (toOptic)--(intersection-1);
%
\draw(-2*\axon-6*\dendrite,1*\yStep+\cY/2)--++(-0.5,0) to [out=180,in=-90] ++(-0.2,0.2) coordinate(toOptic);
\path[name path=optic](toOptic)--++(0,7.5*\yStep+0.5);
\draw[name intersections={of=optic and opticEnds}] (toOptic)--(intersection-1);
%external boundary
\draw(0,-0.2)--++(0,7*\yStep+\cY+0.4);
\draw[fill=gray] (1.1,-0.2)--++(0.2,0)--++(0,7*\yStep+\cY+0.4)--++(-0.2,0)--cycle;
\draw(1.1,-0.2)--++(-4.3,0)--++(0,7*\yStep+\cY+0.4)--++(4.3,0);
%dimensions
\draw[stealth-] (-3.2,0.4)--++(-0.5,0);
\draw[stealth-] (1.3,0.4)--++(0.5,0)node[right]{$\SI{0.5}{\milli\meter}$};
\draw[stealth-] (1.2,-0.2) to [out=-90,in=180]++(0.5,-0.3) node[right]{\RL{غیر شفاف تہہ}};
\draw[stealth-] (0.4,7*\yStep+\cY) to [out=90,in=180] ++(1,0.5) node[right]{\RL{مخروط}};
\draw[stealth-] (0.6,4*\yStep+\cY) to [out=90,in=180] ++(1,0.2) node[right]{\RL{سلاخ}};
\draw[stealth-] (-1.2,4.55) to [out=120,in=180] ++(0.5,1.3) node[right]{\RL{دو قطبی خلیہ}};
\draw[stealth-] (-\axon/2-\dendrite,\cY/2-\dendrite) to [out=-90,in=0]++(-0.5,-0.5)node[left]{مرکزہ};
\draw[stealth-] (-0.2,4.55) to [out=90,in=180]++(0.5,0.75)node[right]{مرکزہ};
\draw [decorate,decoration={brace,amplitude=5pt},xshift=0,yshift=0pt] (-3.1,5) -- (-2.3,5) node [black,midway,yshift=0.5cm] {\RL{عصب بصری}};
\draw[stealth-](0,-0.2) to [out=-90,in=180]++(0.3,-0.7) node[right]{\RL{خلوی جھلیاں}};
\draw[stealth-](-\axon-3*\dendrite,\cY/2+\yStep)++(-135:\dendrite) to [out=-135,in=0]++(-0.5,-0.5) node[left]{\RL{دو قطبی خلیہ}};
\draw[stealth-](-1.3*\axon-4*\dendrite,\cY/2+\yStep) to [out=90,in=-60]++(-0.2,0.3) node[above]{\RL{عصبی ریشہ}};
\draw[stealth-](-2*\axon-5*\dendrite,\cY/2+3.5*\yStep)++(90:\dendrite) to [out=90,in=180]++(0.2,0.2) node[shift={(0:0.3)}]{\RL{شجریہ}};
\end{tikzpicture}%
\end{urdufont}
\end{document} 
