\حصہء{دیباچہ}
میں نے تقریباً چودہ برس پہلے اس کتاب کو لکھنے کی ناکام کوشش کی۔ کئی برس گذرنے کے بعد آج میں ایسا کرنے میں کامیاب ہوا ہوں۔

مجھے طلباء و طالبات کی طرف سے بھر پور حوصلہ ملا ہے جو آئے دن مجھ تک کتاب کی غلطیاں پہنچاتے ہیں۔اس سے بھی زیادہ پُر امید میں اس وقت ہوا جب مجھے معلوم ہوا کہ کئی طلباء اور طالبات میری کتاب سے پڑھ رہے ہیں۔میں امید کرتا ہوں کہ برقی و مقناطیسیات کو بھی درست کرنے میں آپ مدد کریں گے۔

میں یہاں بالخصوص رانا لیاقت، ہرا یوصفزئی، انیلا تبسم اور ماجد بلال خان کا شکریہ ادا کرتا ہوں جنہوں نے کتاب کے دیگر حصے پڑھ کر درست کئے۔عابد نے گزشتہ کتاب کی طرح برقی و مقناطیسیات کو بھی کتاب کی شکل دی۔

یہ کتاب \عددی{XeLatex} میں لکھی گئی جبکہ سوالات کے جوابات مرتب کرنے میں \عددی{wxMaxima} کا سہارا لیا گیا۔

اس سے پہلے میں تین کتابیں لکھ چکا ہوں لیکن پھر بھی برقی و مقناطیسیات لکھنے کی خاطر میں بیٹھ نہیں پا رہا تھا۔کتاب لکھنے کے لئے جتنی توجہ درکار ہوتی ہے اتنی میں کر نہیں پا رہا تھا۔انہیں کوششوں میں تقریباً ایک مہینا گذرا ہو گا کہ پشاور میں آرمی پبلک اسکول کا واقع ہوا۔ 
