\باب{دیباچہ}
میں نے تقریباً چودہ برس قبل اس کتاب کو لکھنے کی پہلی مرتبہ ناکام کوشش کی۔ کئی برس گذرنے کے بعد آج میں ایسا کرنے میں کامیاب ہوا ہوں۔یہ کتاب اس امید کے ساتھ لکھی گئی ہے کہ یہ ایک دن برقی انجنیئرنگ کی نصابی کتاب کے طور پر پڑھائی جائے گی۔امید کی جاتی ہے کہ اب بھی طلبہ و طالبات اس سے استفادہ حاصل کر سکیں گے۔برقی و مقناطیسیات کا شعبہ انتہائی دلچسپ ہے۔میں نے پوری کوشش کی ہے کہ یہ کتاب بھی پر کشش ہو۔


اس کتاب میں تقریباً \عددی{121} حل شدہ مثال اور \عددی{245} اشکال پائے جاتے ہیں۔اس کے علاوہ \عددی{370} سوالات دئے گئے ہیں۔تمام کے تمام سوالات کے جوابات بھی دئے گئے ہیں۔

برقی و مقناطیسیات کو خطی الجبرا اور سمتیات کی مدد سے سمجھنا زیادہ آسان ثابت ہوتا ہے۔اس کتاب کے پہلے باب میں درکار خطی الجبرا اور سمتیات پر غور کیا گیا ہے۔کارتیسی محدد کے علاوہ نلکی محدد اور کروی محدد متعارف کرائے گئے ہیں۔عمومی محدد پر بھی تبصرہ کیا گیا ہے۔اس باب کو مکمل طور پر سمجھنا نہایت ضروری ہے۔دوسرے باب میں کولمب کے قانون پر غور کے بعد تیسرے باب میں پھیلاو متعارف کرایا گیا ہے۔چوتھے باب میں برقی دباو اور ڈھلوان پر غور کیا گیا ہے۔پانچویں باب میں برق گیر جبکہ آٹھویں باب میں امالہ پر غور کیا گیا ہے۔ساتویں باب میں گردش، مسئلہ سٹوکس اور ایمپیئر کے دوری قانون پر بحث کی گئی ہے۔ 

میکس ویل مساوات کے بعد حرکت کرتے میدان پر تبصرہ کیا گیا ہے۔ حرکت کرتا میدان انتہائی دلچسپ موضوع ہے جسے پڑھ کر ایسے حقائق جیسے  آئینے میں عکس کیوں بنتا ہے یا پھر شیشے میں آر پار کیوں نظر آتا ہے کی سمجھ پیدا ہوتی ہے۔اس کے علاوہ خالی خلاء میں برقی و مقناطیسی امواج کی رفتار، خالصاً میکس ویل مساوات سے حاصل کی جاتی ہے۔خالی خلاء کی قدرتی رکاوٹ بھی انہیں مساوات سے حاصل کی جاتی ہے۔

میکس ویل مساوات کے بعد پوئنٹنگ سمتیا متعارف کرایا جاتا ہے جو منتق کے بالکل برعکس بتلاتا ہے کہ برقی طاقت، منبع سے برقی بوجھ تک، ہرگز موصل تار کے ذریعہ نہیں پہنچتی بلکہ ایسا تار کے گرد خلاء میں برقی و مقناطیسی میدان کے ذریعہ ہوتا ہے۔موصل تار صرف اور صرف ان امواج کو منبع سے مزاحمت تک کی راہ دکھلاتی ہے۔ترسیلی تار کے ذریعہ ساکن موج پر بھرپور تبصرہ کیا گیا ہے جسے آپ ضرور پسند کریں گے۔

برقی ادوار پڑھنے سے ایسا معلوم ہوتا ہے جیسے برقی طاقت کے منتقلی کے لئے دو عدد موصل تاروں کا ہونا ضروری ہے۔مویج کا باب اس حقیقت کے بالکل برعکس ہے جہاں صرف ایک عدد تار  ہی برقی و مقناطیسی امواج کو منبع سے بوجھ تک راہ دکھاتی ہے۔

کتاب کے آخر میں اینٹینا پر تبصرہ کیا گیا ہے جہاں منبع سے بوجھ تک طاقت بغیر کسی راہ دکھاتے موصل تار کے پہنچتی ہے۔

مجھے طلباء و طالبات کی طرف سے بھر پور حوصلہ ملا ہے جو آئے دن مجھ تک کتاب کی غلطیاں پہنچاتے ہیں۔اس سے بھی زیادہ پُر امید میں اس وقت ہوا جب مجھے معلوم ہوا کہ کئی طلباء اور طالبات میری کتاب سے پڑھ رہے ہیں۔میں امید کرتا ہوں کہ برقی و مقناطیسیات کو بھی درست کرنے میں آپ مدد کریں گے۔

میں یہاں بالخصوص رانا لیاقت، حرا خان، انیلا تبسم، ماجد بلال خان اور سید منیر کا شکریہ ادا کرتا ہوں جنہوں نے کتاب کے دیگر حصے پڑھ کر درست کئے۔میں عابد حسن مجتبٰے کا بھی شکریہ ادا کرنا چاہتا ہوں جو \تحریر{XeLatex} کے معاملات سنبھالے ہوئے ہیں۔

یہ کتاب \تحریر{Ubuntu} استعمال کرتے ہوئے \تحریر{XeLatex} میں تشکیل دی گئی جبکہ سوالات کے جوابات کے حصول میں \تحریر{wxMaxima} کا سہارا لیا گیا۔یہ کتاب خطِ جمیل نوری نستعلیق میں لکھی گئی ہے۔

یہ کتاب لکھتے ہوئے مندرجہ ذیل کتابوں سے مدد لی گئی
{
\begin{otherlanguage}{english}
\begin{itemize}
\item
Engineering Electromagnetics by William H. Hayt, Jr
\item
Electromagnetics by John D. Kraus
\item
Antennas And Radiowave Propagation by R.E. Collin
\end{itemize}
\end{otherlanguage}
}
جبکہ اردو اصطلاحات چننے میں درج ذیل لغت سے استفادہ حاصل کیا گیا۔
{
\begin{otherlanguage}{english}
\begin{itemize}
\item
http:/\!\!/www.urduenglishdictionary.org
\item
http:/\!\!/www.nlpd.gov.pk/lughat/
\end{itemize}
\end{otherlanguage}
}
آپ سے گزارش ہے کہ اس کتاب کو زیادہ سے زیادہ طلبہ و طالبات تک پہنچائیں اور کتاب میں غلطیوں کی نشاندہی میرے  برقیاتی پتہ

{
\begin{otherlanguage}{english}
khalidyousafzai@comsats.edu.pk
\end{otherlanguage}
}

 پر کریں۔میری تمام کتابوں کی مکمل \تحریر{XeLatex} معلومات

{
\begin{otherlanguage}{english}
https:/\!\!/www.github.com/khalidyousafzai
\end{otherlanguage}
}

سے حاصل کی جا سکتی ہیں جنہیں آپ مکمل اختیار کے ساتھ استعمال کر سکتے ہیں۔
\vspace{5mm}

{\raggedleft{
خالد خان یوسفزئی

8 جنوری 2017؁}}


