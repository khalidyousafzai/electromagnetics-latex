\باب{سوالات}
%=====================
\ابتدا{سوال}
میدان \عددی{\kvec{E}=1.5\az \, \si{\volt\per\meter}} میں الیکٹران  حرکت کرتا ہے۔لمحہ \عددی{t=0} پر الیکٹران نقطہ \عددی{(0,0,0)} پر پایا جاتا ہے جبکہ اس کی سمتی رفتار \عددی{\kvec{v}=3\times 10^{5}\ax \, \si{\meter\per\second}} ہے۔الیکٹران کا چارج \عددی{\SI{-1.6e-19}{\coulomb}} اور اس کی کمیت \عددی{\SI{3.1e-31}{\kilo\gram}} ہے۔نیوٹن کے قوانین حرکت سے تفرقی مساوات لکھ کر اسے حل کرتے ہوئے لمحہ \عددی{t=\SI{150}{\nano\second}} پر الیکٹران کی سمتی رفتار، مقام اور حرکی توانائی دریافت کریں۔

جوابات:\عددی{\kvec{v}=\num{300000}\ax-\num{116129}\az \, \si{\meter\per\second}}، \عددی{(0.045,0,-3.48)}، \عددی{\SI{1.63e-20}{\joule}}
\انتہا{سوال}
%=====================
\ابتدا{سوال}
مقناطیسی میدان \عددی{\kvec{B}=0.3\ax-0.2\ay-0.4\az \, \si{\tesla}} میں لمحہ \عددی{t=0} پر الیکٹران کی سمتی رفتار \عددی{\kvec{v}=10^6 \az \si{\meter\per\second}} ہے۔الیکٹران پر قوت دریافت کریں۔ایسا برقی میدان حاصل کریں جس کی موجودگی میں مقناطیسی اور برقی میدان مل کر اس الیکٹران پر صفر قوت  پیدا کرتے ہیں۔

جواب:\عددی{\kvec{F}=-32\ax-48\ay \, \si{\femto\newton}}، \عددی{\kvec{E}=-200\ax-300\ay \, \si{\volt\per\meter}}
\انتہا{سوال}
%====================
\ابتدا{سوال}
میدان \عددی{\kvec{B}=2\ax-1\ay+3\az \, \si{\tesla}} اور \عددی{\kvec{E}=3\ax+2\ay-1\az \, \si{\volt\per\meter}} میں چارج \عددی{\SI{1.2}{\micro\coulomb}} حرکت کر رہا ہے۔لمحہ \عددی{t=0} پر اس کی رفتار \عددی{\kvec{v}=10\ax-30\ay+20\az \, \si{\kilo\meter\per\second}} ہے۔یہ چارج \عددی{\SI{5}{\micro\gram}} کے کمیت پر پایا جاتا ہے۔لمحہ \عددی{t=0} پر چارج کی اسراع حاصل کریں۔

جواب:\عددی{\kvec{a}=-16.8\ax+2.4\ay+12\az \, \si{\mega\meter\per\second\squared}}
\انتہا{سوال}
%======================
\ابتدا{سوال}
محدد \عددی{z} پر پڑی لامحدود لمبائی کے تار میں \عددی{5\az \, \si{\ampere}} برقی رو گزر رہی ہے۔اس کے قریب سطح \عددی{x=0} پر موصل تار \عددی{N_1(0,1,0)}، \عددی{N_2(0,4,0)}، \عددی{N_3(0,4,2)} اور \عددی{N_4(0,1,2)} نقطوں کو جوڑ کر مستطیل بناتی ہے جس میں \عددی{N_1} سے \عددی{N_2} جانب \عددی{\SI{2}{\ampere}} برقی رو چکر  لگا رہی ہے۔چکور کے چاروں اطراف پر قوت دریافت کرتے ہوئے پورے چکور پر قوت حاصل کریں۔

جوابات:تار \عددی{N_1(0,1,0)} تا \عددی{N_2(0,4,0)} پر قوت \عددی{2.77\az \, \si{\micro\newton}} ہے۔گھڑی کے الٹ سمت چلتے ہوئے بقایا قوت  \عددی{-1\ay\, \si{\micro\newton}}، \عددی{-2.77\az \, \si{\micro\newton}} اور \عددی{4\ay\, \si{\micro\newton}} ہیں۔یوں مستطیل پر کل قوت \عددی{3\ay\, \si{\micro\newton}} ہے۔
\انتہا{سوال}
%======================
\ابتدا{سوال}
محدد \عددی{z} پر پڑی لامحدود لمبائی کے تار میں \عددی{10\az \, \si{\ampere}} برقی رو گزر رہی ہے۔اس کے قریب نقطہ \عددی{N_1(2,1,3)} سے \عددی{N_2(5,4,7)} تک سیدھی موصل تار میں \عددی{N_1} سے \عددی{N_2} جانب \عددی{\SI{4}{\ampere}} برقی رو گزر رہی ہے۔چھوٹی تار پر قوت حاصل کریں۔

جواب:\عددی{\kvec{F}=-6.74\ax-4.49\ay+8.42\az \, \si{\micro\newton}}
\انتہا{سوال}
%========================
\ابتدا{سوال}
سطح \عددی{x=0} پر مقناطیسی میدان کا \عددی{z} جزو \عددی{B_z=\tfrac{200}{z^2+1} \, \si{\micro \tesla}} پایا جاتا ہے۔اس مقناطیسی جزو سے خطہ \عددی{1<y<3}، \عددی{-\infty <z< \infty} میں کثافت \عددی{\kvec{K}=0.2\ay \si{\ampere\per\meter}} پر قوت حاصل کریں۔

جواب:\عددی{251\ax \, \si{\micro\newton}}
\انتہا{سوال}
%=======================
\ابتدا{سوال}
\عددی{z} محدد پر پڑی لامحدود لمبائی کے تار میں \عددی{\SI{2.2}{\ampere}} برقی رو پائی جاتی ہے۔سطح \عددی{y=0} پر خطہ \عددی{\SI{1}{\milli\meter} < x < \SI{5}{\milli\meter}} پر \عددی{\az} سمت میں کل \عددی{\SI{8}{\ampere}} برقی رو گزر رہی ہے۔اس خطے کی فی میٹر لمبائی پر مقناطیسی قوت حاصل کریں۔محدد \عددی{z} پر پڑی تار پر بھی فی میٹر قوت حاصل کریں۔

جواب:\عددی{-1.4\ax \, \si{\milli\newton}}، \عددی{1.4\ax \, \si{\milli\newton}}
\انتہا{سوال}
%=======================
\ابتدا{سوال}
موصل تار نقطہ \عددی{N_1(2,0,0)}، \عددی{N_2(5,0,0)}، \عددی{N_3(5,0,4)} اور \عددی{N_4(2,0,4)} کو جوڑ کر مستطیل بناتی ہے۔مثبت \عددی{y} محدد کی جانب سے دیکھتے ہوئے، اس مستطیل میں \عددی{\SI{6}{\ampere}} برقی رو سمت گھڑی گردش کر رہی رہی ہے۔الف)  یکساں میدان \عددی{\kvec{B}=5\ax \, \si{\tesla}} کی صورت میں \عددی{z} محدد  کو محور لیتے ہوئے مروڑ حاصل کریں۔ ب) سطح \عددی{y=0} پر لکیر \عددی{x=3} کو محور لیتے ہوئے اسی یکساں میدان میں دوبارہ مروڑ حاصل کریں۔ پ) یکساں میدان کی جگہ اگر \عددی{z} محدد پر لامحدود لمبائی کے تار میں \عددی{\az} جانب \عددی{\SI{25}{\ampere}} برقی رو میدان پیدا کرے تب محدد کے مرکز \عددی{(0,0,0)} کو محور لیتے ہوئے مروڑ حاصل کریں۔ 

جوابات:\عددی{360\az \, \si{\newton\meter}}، \عددی{360\az \, \si{\newton\meter}}، \عددی{72\ay \, \si{\micro\newton\meter}}
\انتہا{سوال}
%======================
