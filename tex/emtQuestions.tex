\باب{سوالات}

\حصہء{مستوی امواج}
%====================
\ابتدا{سوال}
خالی خلاء میں \عددی{\az} سمت میں حرکت کرتی، \عددی{\SI{600}{\mega \hertz}} تعدد  کے مستوی برقی موج \عددی{\kvec{E}} کی  چوٹی لمحہ \عددی{t=\SI{1}{\nano\second}} پر \عددی{z=\SI{0.3}{\meter}}  پر پائی جاتی ہے۔یہ چوٹی \عددی{\SI{310}{\volt\per\meter}} کے برابر ہے۔الف) برقی میدان \عددی{\ax} سمت میں ہونے کی صورت میں سائن نما  \عددی{\kvec{E}} اور \عددی{\kvec{H}} امواج کے مساوات لکھیں۔ ب) میدان سمتیہ \عددی{5\ax-2\ay} کی سمت میں ہونے کی صورت
 میں سائن نما  \عددی{\kvec{E}_s} اور \عددی{\kvec{H}_s}  امواج کی مساوات لکھیں۔

جواب:\عددی{\kvec{E}=310 \ax \cos (12\pi \times 10^8 t -4\pi z)}، \عددی{\kvec{H}=\tfrac{31}{12\pi} \ay \cos (12\pi \times 10^8 t -4\pi z)}،\\ \عددی{\kvec{E}_s=310\left[\tfrac{5}{\sqrt{29}}\ax-\tfrac{2}{\sqrt{29}}\ay\right]e^{-j4\pi z}}،  \عددی{\kvec{H}_s=\tfrac{31}{12\pi}\left[\tfrac{2}{\sqrt{29}}\ax+\tfrac{5}{\sqrt{29}}\ay\right]e^{-j4\pi z}}
\انتہا{سوال}
%===================
\ابتدا{سوال}
خالی خلاء میں  نقطہ \عددی{N(3,-2,5)} پر \عددی{\az} جانب حرکت کرتی، \عددی{\SI{200}{\mega\hertz}} تعدد کے برقی میدان کی سائن نما مستوی موج کی چوٹی لمحہ \عددی{t=0} پر \عددی{\kvec{E}_0=150\ax+210\ay \, \si{\volt\per\meter}} پائی جاتی ہے۔ الف) \عددی{\lambda}، \عددی{\beta}، \عددی{\kvec{a}_E}، \عددی{\kvec{a}_H}، \عددی{H_0} اور مقناطیسی موج \عددی{\kvec{H}_s} حاصل کریں۔  ب) لمحہ \عددی{t=0} پر نقطہ \عددی{N} پہ برقی میدان کی شدت  حاصل کریں۔ پ) لمحہ \عددی{t=\SI{1.5}{\nano\second}} پر نقطہ \عددی{N} پہ برقی میدان کی شدت  حاصل کریں۔ ت) نقطہ \عددی{P(5,3,7)} پہ لمحہ \عددی{t=\SI{2}{\nano\second}} پر برقی میدان کی شدت حاصل کریں۔ 

جوابات:\عددی{\lambda=\tfrac{3}{2}\, \si{\meter}}، \عددی{\beta=4.2 \, \si{\radian \per \meter}}، \عددی{\kvec{a}_E=0.51\ax+0.86\ay}، \عددی{\kvec{a}_H=-0.86\ax+0.51\ay}، \\
\عددی{H_0=\SI{0.7733}{\ampere\per\meter}}،  \عددی{\kvec{H}_s=0.7733(-0.86\ax+0.51\ay)e^{-j4.2 z}}، \عددی{\SI{292}{\volt\per\meter}}، \عددی{\SI{-90}{\volt\per\meter}}، \عددی{\SI{266}{\volt\per\meter}}
\انتہا{سوال}
%==================

\ابتدا{سوال}
خالی خلاء میں مستوی موج \عددی{\kvec{E}_s=\kvec{E}_0 e^{-j 6z}} دی گئی ہے۔ الف) موج کی تعدد \عددی{\omega} حاصل کریں۔ ب) برقی میدان کا حیطہ بالترتیب \عددی{\kvec{E}_0=50\ax}، \عددی{\kvec{E}_0=(5+j10)\ax}، \عددی{\kvec{E}_0=50\ax+80\ay} اور \عددی{\kvec{E}_0=(30\phase{45^{\circ}})\ax} ہونے کی صورت میں لمحہ \عددی{t=0} پر نقطہ \عددی{N(0,0,0)} پہ \عددی{\abs{\kvec{E}}} حاصل کریں۔


جوابات: \عددی{\SI{1.8}{\giga \radian\per\second}}، \عددی{\SI{50}{\volt\per\meter}}، \عددی{\SI{11.18}{\volt\per\meter}}،  \عددی{\SI{94.3}{\volt\per\meter}}، \عددی{\SI{11.18}{\volt\per\meter}}،  \عددی{\SI{21.2}{\volt\per\meter}}،
\انتہا{سوال}
%====================
\ابتدا{سوال}
خالی خلاء میں \عددی{\SI{350}{\mega\hertz}} تعدد کی مستوی موج \عددی{\kvec{E}_s=(5+j2)(3\ax-j4\ay)e^{j \beta z} \, \si{\volt\per\meter}} پائی جاتی ہے۔ \عددی{\lambda} اور \عددی{\beta} کی قیمتیں دریافت کریں۔لمحہ \عددی{t=\SI{1.4}{\nano\second}} پر نقطہ \عددی{z=\SI{40}{\centi\meter}} پہ \عددی{\kvec{E}} حاصل کریں۔موج کا حیطہ حاصل کریں۔

جواب:\عددی{\lambda=\tfrac{6}{7} \, \si{\meter}}، \عددی{\beta=\tfrac{7\pi}{3}\,\si{\radian \per \meter}}، \عددی{\kvec{E}(z=\si{40}{\centi\meter},t=\SI{1.4}{\nano\second})=13.96\ax-10.84\ay \, \si{\volt\per\meter}}، \عددی{\abs{\kvec{E}}_{\textrm{بلندتر}}=\SI{26.9}{\volt\per\meter}}
\انتہا{سوال}
%====================
\ابتدا{سوال}
ایسا خطہ جس کے مستقل \عددی{\mu_R=1}، \عددی{\epsilon_R=4.4} اور \عددی{\sigma=0} ہیں میں بڑھتے \عددی{x} محدد کی جانب حرکت کرتی، \عددی{\SI{250}{\mega\hertz}} تعدد کی مستوی برقی موج پائی جاتی ہے۔برقی میدان \عددی{\ay} سمت میں ہے۔مندرجہ ذیل حاصل کریں۔ \عددی{v_p}، \عددی{\beta}، \عددی{\lambda}، \عددی{\eta} \عددی{\kvec{E}_s}، \عددی{\kvec{H}_s} اور \عددی{\pmb{\mathscr{P}}_{\text{اوسط}}}؛   

جوابات:\عددی{v_p=\SI{1.429e8}{\meter\per\second}}، \عددی{\beta=\SI{10.99}{\radian\per\meter}}، \عددی{\lambda=\SI{57.2}{\centi\meter}}، \عددی{\eta=\SI{179.6}{\ohm}}،  \عددی{\kvec{E}_s=E_0 e^{-j 10.99 x} \ay \, \si{\volt\per\meter}}، 
\عددی{\kvec{H}_s=\tfrac{E_0}{179.6} e^{-j 10.99 x}\az \, \si{\ampere\per\meter}}،
 \عددی{\pmb{\mathscr{P}}_{\text{اوسط}}=\tfrac{E_0^2}{359.2} \ax \, \si{\watt\per\meter\squared}}
\انتہا{سوال}
%===================
\ابتدا{سوال}
مستوی برقی موج \عددی{\kvec{E}=E_0 e^{-\alpha z}\sin(\omega t - \beta z)\ay \, \si{\volt\per\meter}} اور \عددی{\eta = \abs{\eta_0} e^{j \phi}} دئے گئے ہیں۔ الف) دوری سمتیات \عددی{\kvec{E}_s} اور \عددی{\kvec{H}_s} حاصل کریں۔ ب) \عددی{\pmb{\mathscr{P}}_{\text{اوسط}}} حاصل کریں۔

جوابات:\عددی{\kvec{E}_s=E_0 e^{-\alpha z} e^{-j (\beta z +\pi)} \ay \, \si{\volt\per\meter}}،
 \عددی{\kvec{H}_s=-\tfrac{E_0}{\abs{\eta_0}} e^{-\alpha z }e^{-j(\beta z +\pi+\phi)}\ax \, \si{\ampere\per\meter}}،
 \عددی{\pmb{\mathscr{P}}_{\text{اوسط}}=\tfrac{E_0^2}{2\abs{\eta_0}}e^{-2\alpha z} \cos \phi  \az \, \si{\watt\per\meter\squared}}
\انتہا{سوال}
%================
\ابتدا{سوال}
خالی خلاء میں  \عددی{\kvec{E}=(30\ay+22\az)\cos(\omega t -60 x) \, \si{\volt\per\meter}} پایا جاتا ہے۔ الف) \عددی{\lambda} اور \عددی{\omega} حاصل کریں۔ ب) دوری سمتیات \عددی{\kvec{E}_s} اور \عددی{\kvec{H}_s} لکھیں۔ پ) \عددی{\pmb{\mathscr{P}}_{\text{اوسط}}} حاصل کریں۔

جوابات:\عددی{\lambda=\tfrac{\pi}{30} \, \si{\meter}}، \عددی{\omega=\SI{1.8e10}{\radian\per\second}}، \عددی{\kvec{E}_s=(30\ay+22\az)e^{-j 60 x} \, \si{\volt \per\meter}}، \\
 \عددی{\kvec{H}_s=\tfrac{1}{120 \pi}(-22\ay+30\az) e^{-j 60 x} \, \si{\ampere\per\meter}}، 
\عددی{\pmb{\mathscr{P}}_{\text{اوسط}}=\tfrac{173}{30\pi} \ax \, \si{\watt\per\meter\squared}}
\انتہا{سوال}
%================
\ابتدا{سوال}
مستوی مقناطیسی موج کا دوری سمتیہ \عددی{\kvec{H}_s=(5\ax+j4\az)e^{j20y} \, \si{\volt\per\meter}} اور تعدد \عددی{\SI{200}{\mega\hertz}} ہے۔ برقی موج کا زیادہ سے زیادہ حیطہ \عددی{\SI{1200}{\volt\per\meter}} ہے۔حاصل کریں \عددی{\beta}، \عددی{\lambda}، \عددی{\eta}، \عددی{v_p}، \عددی{\epsilon_R}، \عددی{\mu_R} اور \عددی{\kvec{H}(x,y,z,t)}؛

جوابات:\عددی{\beta=\SI{20}{\radian\per\meter}}، \عددی{\lambda=\tfrac{\pi}{10} \, \si{\meter}}، \عددی{\eta=\SI{187.4}{\ohm}}،
 \عددی{v_p=\SI{6.28e7}{\meter\per\second}}، \عددی{\epsilon_R=9.6}، \عددی{\mu_R=2.4}، \عددی{\kvec{H}=5\cos(2\pi\times 200\times 10^6 t +20y)\ax-4\sin(2\pi\times 200 \times 10^6 t +20y)\az \, \si{\ampere\per\meter}}
\انتہا{سوال}
%===================
\ابتدا{سوال}
میدان \عددی{\kvec{E}(y,t)=700\cos(2.5\times 10^7  t -\beta y) \ax \, \si{\volt\per\meter}} اور \عددی{\kvec{H}(y,t)=1.5\cos(2.5 \times 10^7  t -\beta y) \ay \, \si{\ampere\per\meter}} مستوی موج کو ظاہر کرتے ہیں۔یہ موج \عددی{\SI{1.7e8}{\meter\per\second}} رفتار سے حرکت کر رہی ہے۔حاصل کریں \عددی{\beta}، \عددی{\lambda}، \عددی{\eta}، \عددی{\epsilon_R} اور \عددی{\mu_R}؛

جوابات:\عددی{\beta=\SI{0.147}{\radian\per\meter}}، \عددی{\lambda=\SI{42.7}{\meter}}، \عددی{\eta=\SI{467}{\ohm}}،
 \عددی{\epsilon_R=1.4}، \عددی{\mu_R=2.2}
\انتہا{سوال}
%===================
