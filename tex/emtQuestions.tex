\باب{سوالات}

\حصہء{مویج}
%====================
\ابتدا{سوال}
ہوا سے بھرے مستطیل مویج کے اطراف کی لمبائی \عددی{\SI{25}{\milli\meter}} اور \عددی{\SI{50}{\milli\meter}} ہے۔اس میں کمتر انقطاعی تعدد کے \عددی{1.7} گنا تعدد کی موج پائی جاتی ہے۔ الف) کم تر انقطاعی طول موج دریافت کریں۔ ب) مویج میں دوری رفتار حاصل کریں۔

جوابات:\عددی{\SI{100}{\milli\meter}}، \عددی{\SI{3.843e8}{\meter\per\second}} 
\انتہا{سوال}
%===================
\ابتدا{سوال}
ہوا سے بھرے \عددی{\SI{50}{\milli\meter}} لمبائی کے اطراف کے چکور مویج میں \عددی{\SI{40}{\milli\meter}} سے زیادہ طول موج کے تمام ممکنہ \عددی{\TE{}} اور \عددی{\TM{}} امواج دریافت کریں۔

جوابات:\عددی{\TE{10}}، \عددی{\TE{01}}، \عددی{\TE{11}}، \عددی{\TE{20}}، \عددی{\TE{02}}، \عددی{\TE{21}}،\عددی{\TE{12}}، \عددی{\TM{11}}، \عددی{\TM{21}}،\عددی{\TM{12}}
\انتہا{سوال}
%====================
\ابتدا{سوال}
ہوا سے بھرے \عددی{\SI{20}{\milli\meter}} اور \عددی{\SI{100}{\milli\meter}} لمبائی کے اطراف کے  مستطیل مویج میں \عددی{\SI{150}{\milli\meter}} سے زیادہ طول موج کے تمام ممکنہ \عددی{\TE{}} اور \عددی{\TM{}} امواج دریافت کریں۔

جوابات:\عددی{\TE{10}}
\انتہا{سوال}
%====================
\ابتدا{سوال}
ہوا سے بھرے نلکی مویج کا رداس \عددی{\SI{75}{\milli\meter}} ہے۔اس میں کم ضیاعی \عددی{\TE{01}} موج کی انقطاعی طول موج اور غالب \عددی{\TE{11}} موج کی انقطاعی طول موج دریافت کریں۔

جوابات:\عددی{\SI{123}{\milli\meter}}، \عددی{\SI{256}{\milli\meter}}
\انتہا{سوال}
%=====================
\ابتدا{سوال}
ہوا سے بھرے نلکی مویج کا رداس \عددی{\SI{100}{\milli\meter}} ہے۔ اس میں مندرجہ ذیل بلند درجی امواج کے انقطاعی طول موج دریافت کریں۔ \عددی{\TM{01}}، \عددی{\TM{02}}، \عددی{\TM{11}}، \عددی{\TM{12}}، \عددی{\TM{21}}،  \عددی{\TM{31}}، \عددی{\TM{41}}

جوابات:\عددی{\SI{261}{\milli\meter}}،\عددی{\SI{114}{\milli\meter}}،\عددی{\SI{164}{\milli\meter}}،\عددی{\SI{89}{\milli\meter}}،\عددی{\SI{122}{\milli\meter}}،\عددی{\SI{98}{\milli\meter}}،\عددی{\SI{83}{\milli\meter}}
\انتہا{سوال}

\ابتدا{سوال}
ہوا سے بھرے نلکی مویج کا رداس \عددی{\SI{100}{\milli\meter}} ہے۔ اس میں مندرجہ ذیل بلند درجی امواج کے انقطاعی طول موج دریافت کریں۔ \عددی{\TE{01}}، \عددی{\TE{02}}، \عددی{\TE{11}}، \عددی{\TE{12}}، \عددی{\TE{21}}، \عددی{\TE{22}}، \عددی{\TE{31}}، \عددی{\TE{41}}

جوابات:\عددی{\SI{164}{\milli\meter}}،\عددی{\SI{89}{\milli\meter}}،\عددی{\SI{341}{\milli\meter}}،\عددی{\SI{118}{\milli\meter}}،\عددی{\SI{206}{\milli\meter}}،\عددی{\SI{94}{\milli\meter}}،\عددی{\SI{149}{\milli\meter}}،\عددی{\SI{118}{\milli\meter}}
\انتہا{سوال}
%=======================
\ابتدا{سوال}
ثابت کریں کہ کامل موصل کے نلکی مویج میں \عددی{\TE{11}} بلند درجی انداز اوسطاً \عددی{\frac{\omega \mu \beta \rho_0^4 \abs{H_0}^2}{82}} واٹ کی طاقت ترسیل کرتی ہے۔
\انتہا{سوال}
%=======================
\ابتدا{سوال}
ثابت کریں کہ متوازی دو عدد لامحدود موصل چادروں کے مویج میں انقطاعی تعدد سے بلند تعدد پر \عددی{\TE{10}} موج کی تضعیفی مستقل
\begin{align*}
 \alpha=\frac{2}{d}\frac{Z_{c,h}}{Z_{d,h}}\frac{\left(\frac{\lambda_0}{2d}\right)^2}{\sqrt{1-\left(\frac{\lambda_0}{2d}\right)^2}} \quad \quad (\si{\neper/ \meter})
\end{align*}
 ہے جہاں
\begin{description}
\شے{$Z_{c,h}$} مویجی موصل چادر کے قدرتی رکاوٹ کا حقیقی جزو،
\شے{$Z_{d,h}$} مویج میں بھرے ذو برق کی قدرتی رکاوٹ کا حقیقی جزو،
\شے{$d$} دو لامحدود چادروں میں فاصلہ اور
\شے{$\lambda_0$} خالی خلاء میں طول موج ہیں
\end{description}
\انتہا{سوال}
%==========================
\ابتدا{سوال}
ثابت کریں کہ متوازی دو عدد لامحدود موصل چادروں کے مویج میں انقطاعی تعدد سے بلند تعدد پر \عددی{\TE{m0}} موج کی تضعیفی مستقل
\begin{align*}
 \alpha=\frac{2}{d}\frac{Z_{c,h}}{Z_{d,h}}\frac{\left(\frac{\lambda_0}{\lambda_{0c}}\right)^2}{\sqrt{1-\left(\frac{\lambda_0}{\lambda_{0c}}\right)^2}} \quad \quad (\si{\neper/ \meter})
\end{align*}
 ہے جہاں
\begin{description}
\شے{$Z_{c,h}$} مویجی موصل چادر کے قدرتی رکاوٹ کا حقیقی جزو،
\شے{$Z_{d,h}$} مویج میں بھرے ذو برق کی قدرتی رکاوٹ کا حقیقی جزو،
\شے{$d$} دو لامحدود چادروں میں فاصلہ اور
\شے{$\lambda_0$} خالی خلاء میں طول موج ہیں
\end{description}
\انتہا{سوال}
%===========================
\ابتدا{سوال}
لامحدود جسامت کے تانبے کے دو چادروں کے درمیان \عددی{\SI{18}{\milli\meter}} کا فاصلہ ہے۔اس میں \عددی{\SI{10}{\mega\hertz}} تعدد کی \عددی{\TEM} موج کا تضعیفی مستقل اور \عددی{\TE{10}} موج  کا تضعیفی مستقل دریافت کریں۔ 

جواب:\عددی{\alpha=\SI{3.85}{\milli\neper\per\meter}}، \عددی{\alpha=\SI{9.67}{\milli\neper\per\meter}}
\انتہا{سوال}
%==========================
