\باب{سوالات}

\حصہء{اینٹینا}
%========================
\ابتدا{سوال}
غیر سمتی اینٹینا \عددی{E=\frac{25 I}{r}} میدان پیدا کرتی ہے جہاں اینٹینا کا داخلی موثر برقی رو \عددی{I} اور اینٹینا سے فاصلہ \عددی{r} ہے۔اس اینٹینا کی اخراجی مزاحمت حاصل کریں۔

جواب:\عددی{\SI{20.8}{\ohm}}
\انتہا{سوال}
%=======================
\ابتدا{سوال}
اینٹینا کی شعاع \عددی{0<\theta<30^{\circ}}، \عددی{0<\phi<2\pi} خطے میں یکساں میدان پیدا کرتی ہے جبکہ \عددی{30^{\circ}<\theta<180^{\circ}} خطے میں میدان صفر کے برابر ہے۔ الف) اینٹینا کا اخراجی ٹھوس زاویہ \عددی{\Omega_A} حاصل کریں۔ ب) شعاع کی سمتیت \عددی{D} دریافت کریں۔

جوابات: \عددی{\SI{0.842}{\steradian}}، \عددی{14.9}
\انتہا{سوال}
%=====================
\ابتدا{سوال}

اینٹینا کی شعاع \عددی{0<\theta<60^{\circ}}، \عددی{0<\phi<2\pi} خطے میں یکساں میدان پیدا کرتی ہے جبکہ \عددی{60^{\circ}<\theta<180^{\circ}} خطے میں میدان صفر کے برابر ہے۔ الف) اینٹینا کا اخراجی ٹھوس زاویہ \عددی{\Omega_A} حاصل کریں۔ ب) شعاع کی سمتیت \عددی{D} دریافت کریں۔ پ) اینٹینا کا اخراجی رقبہ \عددی{A_e} حاصل کریں۔ ت) اینٹینا کا داخلی موثر برقی رو \عددی{\SI{12}{\ampere}} ہونے کی صورت میں اینٹینا سے \عددی{\SI{164}{\meter}} کے فاصلے پر موثر برقی 
میدان \عددی{\SI{7}{\volt\per\meter}} ہے۔ اینٹینا کا اخراجی مزاحمت \عددی{R_{\text{اخراجی}}} دریافت کریں۔



جوابات: \عددی{\SI{3.142}{\steradian}}، \عددی{4}، \عددی{0.318\lambda^2}، \عددی{\SI{76.3}{\ohm}}
\انتہا{سوال}
%===================
\ابتدا{سوال}
اینٹینا کی شعاع \عددی{45^{\circ}<\theta<60^{\circ}}، \عددی{0^{\circ}<\phi<120^{\circ}} خطے میں یکساں ہے۔بقایا خطے میں میدان صفر کے برابر ہے۔اینٹینا سے \عددی{\SI{1000}{\meter}} کے فاصلے پر اس خطے میں \عددی{\SI{2}{\volt\per\meter}} برقی میدان حاصل کرنے کی خاطر \عددی{\SI{4}{\ampere}} موثر داخلی برقی رو درکار ہے۔ اینٹینا کی اخراجی مزاحمت \عددی{R_{\text{اخراجی}}} دریافت کریں۔

جواب:\عددی{\SI{288}{\ohm}}
\انتہا{سوال}
%=================
\ابتدا{سوال}
اینٹینا کی مرکزی شعاع  \عددی{0^{\circ}<\theta<45^{\circ}} خطے میں یکساں پائی جاتی ہے جبکہ اس کی ثانوی شعاع \عددی{120^{\circ}<\theta<180^{\circ}} خطے میں یکساں پائی جاتی ہے۔میدان \عددی{\phi} کے ساتھ تبدیل نہیں ہوتا۔مرکزی شعاع میں میدان ثانوی شعاع کے میدان کے چار گنا ہے۔ الف) اینٹینا کی سمتیت \عددی{D} دریافت کریں۔ب) مرکزی شعاع میں اینٹینا سے \عددی{\SI{350}{\meter}} فاصلے پر \عددی{E_{\text{موثر}}=\SI{6}{\volt\per\meter}} برقی میدان کے حصول کے لئے اینٹینا کو \عددی{\SI{6}{\ampere}}  موثر داخلی برقی رو مہیا کیا جاتی ہے۔اینٹینا کی اخراجی مزاحمت \عددی{R_{\text{اخراجی}}} دریافت کریں۔

جوابات:\عددی{D=6.17}، \عددی{\SI{662}{\ohm}}
\انتہا{سوال}
%======================
\ابتدا{سوال}
دو عدد غیر سمتی، ہم قدم منبع کے درمیان فاصل \عددی{2\lambda} ہے۔ الف) نقش کے صفر حاصل کریں۔ ب) نقش کی چوٹیاں حاصل کریں۔

جوابات:الف) \عددی{\mp 41.4^{\circ}}، \عددی{\mp 75.5^{\circ}}، \عددی{\mp 104.5^{\circ}}، \عددی{\mp 138.6^{\circ}}؛ ب) \عددی{0^{\circ}}، \عددی{\mp 60^{\circ}}، \عددی{\mp 90^{\circ}}، \عددی{\mp 120^{\circ}}، \عددی{180^{\circ}}
\انتہا{سوال}
%=====================
%======================
\ابتدا{سوال}
دو عدد غیر سمتی، منبع کے درمیان فاصل \عددی{\tfrac{3\lambda}{2}} ہے جبکہ ان میں زاویائی فرق \عددی{180^{\circ}} ہے۔ الف) نقش کے صفر حاصل کریں۔ ب) نقش کی چوٹیاں حاصل کریں۔

جوابات:الف) \عددی{\mp 90^{\circ}}، \عددی{\mp 48.2^{\circ}}، \عددی{\mp 131.8^{\circ}}؛ ب) \عددی{0^{\circ}}، \عددی{\mp 70.5^{\circ}}، \عددی{\mp 109.5^{\circ}}
\انتہا{سوال}
%=====================
