\باب{سوالات}
\حصہ{لاپلاس}
%======================
\ابتدا{سوال}
برقی دباو \عددی{V=0.002x^2yz^3 \, \si{\volt}} ہے۔نقطہ \عددی{N(2,-3,-4)} پر \عددی{V}، \عددی{\kvec{E}} اور \عددی{\abs{\rho_h}} حاصل کریں۔نقطہ \عددی{N} پر ہم قوہ سطح اور سمت بہاو خط کے مساوات حاصل کریں۔کیا برقی دباو لاپلاس کی مساوات پر پورا اترتا ہے؟

جوابات: \عددی{\SI{1.536}{\volt}}، \عددی{\kvec{E}=-1.536\ax+0.512\ay+1.152\az \, \si{\volt\per\meter}}،
 \عددی{\abs{\rho_h}=\SI{1.344}{\coulomb\per\meter\squared}}، \عددی{x^2yz^3-768=0}؛ سمت بہاو خط ان مساوات سے ظاہر ہو گی: \عددی{2y^2-x^2=14} اور \عددی{2z^2-3x^2=6}؛ چونکہ حاصل کردہ حجمی کثافت چارج صفر کے برابر نہیں ہے لہٰذا لاپلاس کی مساوات پر برقی دباو پورا نہیں اترتا۔
\انتہا{سوال}
%========================
\ابتدا{سوال}
دباو کا میدان \عددی{V=xy^2z-kxz^3} لاپلاس مساوات پر پورا اترتا ہے۔اس میں مستقل \عددی{k} کی قیمت حاصل کرتے ہوئے نقطہ \عددی{N(5,2,4)} پر \عددی{\kvec{E}} کی سمت میں اکائی سمتیہ دریافت کریں۔

جوابات:\عددی{k=\tfrac{1}{3}}، \عددی{0.053\ax-0.799\ay+0.599\az}
\انتہا{سوال}
%=========================
\ابتدا{سوال}
خلاء میں نقطہ \عددی{N(2,-3,1)} پر میدان \عددی{V=x+y^2(z^3-x^2)} اور \عددی{V=3x^2+y^2-4z^2} میں \عددی{\rho_h} حاصل کریں۔

جوابات:\عددی{\SI{-0.265}{\nano\coulomb\per\meter^3}}، \عددی{\SI{0}{\coulomb\per\meter^3}}
\انتہا{سوال}
%========================
\ابتدا{سوال}
محدد کے مرکز \عددی{(0,0,0)} پر \عددی{V=3x^3+y^4+2z}  اور \عددی{V=e^{2x}\sin 2y} کے لاپلاس کی قیمت حاصل کریں۔ کیا یہ تفاعل لاپلاس مساوات پر پورا اترتے ہیں؟
جوابات: \عددی{0}، \عددی{0}، نہیں، جی ہاں
\انتہا{سوال}
%========================
\ابتدا{سوال}
میدان \عددی{V=5\rho^2 \sin 2\phi} کا لاپلاس حاصل کریں۔

جواب: \عددی{\nabla^2 V=0}
\انتہا{سوال}
%========================
\ابتدا{سوال}
ثابت کریں کہ \عددی{V=\rho V_0 \cos \phi} لاپلاس مساوات پر پورا اترتا ہے۔اسی برقی دباو کو کارتیسی محدد میں لکھتے ہوئے \عددی{V=0} اور \عددی{V=V_0} سطحیں دریافت کریں۔

جوابات:\عددی{V=V_0 x}، \عددی{x=0}، \عددی{x=1}
\انتہا{سوال}
%======================
\ابتدا{سوال}
متوازی چادر کپیسٹر میں \عددی{{V=10x+15y-30z+55}} ہے۔ چادر کا رقبہ \عددی{\SI{100}{\centi\meter\squared}} جبکہ ان کے درمیان فاصلہ \عددی{\SI{0.5}{\milli\meter}} ہے۔کپیسٹر پر برقی دباو کی قیمت حاصل کریں۔اس کی کپیسٹنس بھی حاصل کریں۔

جوابات:\عددی{\SI{17.5}{\milli\volt}}، \عددی{\SI{177}{\pico\farad}}
\انتہا{سوال}
%====================
