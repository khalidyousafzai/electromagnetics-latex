\باب{سوالات}

\حصہء{مستوی امواج}
%====================
\ابتدا{سوال}
خالی خلاء میں \عددی{\az} سمت میں حرکت کرتی، \عددی{\SI{600}{\mega \hertz}} تعدد  کے برقی موج \عددی{\kvec{E}} کی  چوٹی لمحہ \عددی{t=\SI{1}{\nano\second}} پر \عددی{z=\SI{0.3}{\meter}}  پر پائی جاتی ہے۔یہ چوٹی \عددی{\SI{310}{\volt\per\meter}} کے برابر ہے۔الف) میدان سمتیہ \عددی{\ax} کی سمت میں ہونے کی صورت میں سائن نما  \عددی{\kvec{E}} اور \عددی{\kvec{H}} امواج کے مساوات لکھیں۔ ب) میدان سمتیہ \عددی{5\ax-2\ay} کی سمت میں ہونے کی صورت میں سائن نما  \عددی{\kvec{E}}  موج کی مساوات لکھیں۔

جواب:\عددی{\kvec{E}=310 \ax \cos (12\pi \times 10^8 t -4\pi z)}، \عددی{\kvec{E}=310\left[\tfrac{5}{\sqrt{29}}\ax-\tfrac{2}{\sqrt{29}}\ay\right]\cos (12\pi \times 10^8 t -4\pi z)}
\انتہا{سوال}
%===================
\ابتدا{سوال}
خالی خلاء میں نقطہ \عددی{N(3,-2,5)} پر \عددی{\SI{200}{\mega\hertz}} تعدد کے برقی میدان کی سائن نما موج کی چوٹی لمحہ \عددی{t=0} پر \عددی{\kvec{E}_s=150\ax+210\ay \, \si{\volt\per\meter}} پائی جاتی ہے۔ الف) لمحہ \عددی{t=0} پر نقطہ \عددی{N} پہ برقی میدان کی شدت  حاصل کریں۔ ب) لمحہ \عددی{t=\SI{1.5}{\nano\second}} پر نقطہ \عددی{N} پہ برقی میدان کی شدت  حاصل کریں۔ پ) نقطہ \عددی{P(5,3,7)} پہ لمحہ \عددی{t=\SI{2}{\nano\second}} میدان کی شدت حاصل کریں۔ 

جوابات:\عددی{\SI{292}{\volt\per\meter}}، \عددی{\SI{-90}{\volt\per\meter}}، \عددی{\SI{266}{\volt\per\meter}}
\انتہا{سوال}
%==================

\ابتدا{سوال}
خالی خلاء میں موج \عددی{\kvec{E}_s=\kvec{E}_0 e^{-j 6z}} دی گئی ہے۔ الف) موج کی تعدد \عددی{\omega} حاصل کریں۔ ب) \عددی{\kvec{E}_0=50\ax}، \عددی{\kvec{E}_0=(5+j10)\ax}، \عددی{\kvec{E}_0=50\ax+80\ay} اور \عددی{\kvec{E}_0=(30\phase{45^{\circ}})\ax} ہونے کی صورت میں لمحہ \عددی{t=0} پر نقطہ \عددی{N(0,0,0)} پہ موج کا حیطہ حاصل کریں۔


جوابات: \عددی{\SI{1.8}{\giga \radian\per\second}}، \عددی{\SI{50}{\volt\per\meter}}، \عددی{\SI{11.18}{\volt\per\meter}}،  \عددی{\SI{94.3}{\volt\per\meter}}، \عددی{\SI{11.18}{\volt\per\meter}}،  \عددی{\SI{21.2}{\volt\per\meter}}،
\انتہا{سوال}
%====================
\ابتدا{سوال}
خالی خلاء میں \عددی{\SI{350}{\mega\hertz}} تعدد کی موج \عددی{\kvec{E}_s=(5+j2)(3\ax-j4\ay)e^{j \beta z} \, \si{\volt\per\meter}} پائی جاتی ہے۔ \عددی{\lambda} اور \عددی{\beta} کی قیمتیں دریافت کریں۔لمحہ \عددی{t=\SI{1.4}{\nano\second}} پر نقطہ \عددی{z=\SI{40}{\centi\meter}} پہ \عددی{\kvec{E}} حاصل کریں۔موج کا زیادہ سے زیادہ حیطہ حاصل کریں۔

جواب:\عددی{\kvec{E}(z=\si{40}{\centi\meter},t=\SI{1.4}{\nano\second})=13.96\ax-10.84\ay \, \si{\volt\per\meter}}، \عددی{\abs{\kvec{E}}_{\textrm{بلندتر}}=\SI{26.9}{\volt\per\meter}}
\انتہا{سوال}
%====================

