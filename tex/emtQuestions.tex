\باب{سوالات}

\حصہء{مویج}
%====================
\ابتدا{سوال}
ہوا سے بھرے مستطیل مویج کے اطراف کی لمبائی \عددی{\SI{25}{\milli\meter}} اور \عددی{\SI{50}{\milli\meter}} ہے۔اس میں کمتر انقطاعی تعدد کے \عددی{1.7} گنا تعدد کی موج پائی جاتی ہے۔ الف) کم تر انقطاعی طول موج دریافت کریں۔ ب) مویج میں دوری رفتار حاصل کریں۔

جوابات:\عددی{\SI{100}{\milli\meter}}، \عددی{\SI{3.843e8}{\meter\per\second}} 
\انتہا{سوال}
%===================
\ابتدا{سوال}
ہوا سے بھرے \عددی{\SI{50}{\milli\meter}} لمبائی کے اطراف کے چکور مویج میں \عددی{\SI{40}{\milli\meter}} سے زیادہ طول موج کے تمام ممکنہ \عددی{\TE{}} اور \عددی{\TM{}} امواج دریافت کریں۔

جوابات:\عددی{\TE{10}}، \عددی{\TE{01}}، \عددی{\TE{11}}، \عددی{\TE{20}}، \عددی{\TE{02}}، \عددی{\TE{21}}،\عددی{\TE{12}}، \عددی{\TM{11}}، \عددی{\TM{21}}،\عددی{\TM{12}}
\انتہا{سوال}
%====================
\ابتدا{سوال}
ہوا سے بھرے \عددی{\SI{20}{\milli\meter}} اور \عددی{\SI{100}{\milli\meter}} لمبائی کے اطراف کے  مستطیل مویج میں \عددی{\SI{150}{\milli\meter}} سے زیادہ طول موج کے تمام ممکنہ \عددی{\TE{}} اور \عددی{\TM{}} امواج دریافت کریں۔

جوابات:\عددی{\TE{10}}
\انتہا{سوال}
%====================
\ابتدا{سوال}
ہوا سے بھرے نلکی مویج کا رداس \عددی{\SI{75}{\milli\meter}} ہے۔اس میں کم ضیاعی \عددی{\TE{01}} موج کی انقطاعی طول موج اور غالب \عددی{\TE{11}} موج کی انقطاعی طول موج دریافت کریں۔

جوابات:\عددی{\SI{123}{\milli\meter}}، \عددی{\SI{256}{\milli\meter}}
\انتہا{سوال}
%=====================
\ابتدا{سوال}
ہوا سے بھرے نلکی مویج کا رداس \عددی{\SI{100}{\milli\meter}} ہے۔ اس میں مندرجہ ذیل بلند درجی امواج کے انقطاعی طول موج دریافت کریں۔ \عددی{\TM{01}}، \عددی{\TM{02}}، \عددی{\TM{11}}، \عددی{\TM{12}}، \عددی{\TM{21}}،  \عددی{\TM{31}}، \عددی{\TM{41}}

جوابات:\عددی{\SI{261}{\milli\meter}}،\عددی{\SI{114}{\milli\meter}}،\عددی{\SI{164}{\milli\meter}}،\عددی{\SI{89}{\milli\meter}}،\عددی{\SI{122}{\milli\meter}}،\عددی{\SI{98}{\milli\meter}}،\عددی{\SI{83}{\milli\meter}}
\انتہا{سوال}

\ابتدا{سوال}
ہوا سے بھرے نلکی مویج کا رداس \عددی{\SI{100}{\milli\meter}} ہے۔ اس میں مندرجہ ذیل بلند درجی امواج کے انقطاعی طول موج دریافت کریں۔ \عددی{\TE{01}}، \عددی{\TE{02}}، \عددی{\TE{11}}، \عددی{\TE{12}}، \عددی{\TE{21}}، \عددی{\TE{22}}، \عددی{\TE{31}}، \عددی{\TE{41}}

جوابات:\عددی{\SI{164}{\milli\meter}}،\عددی{\SI{89}{\milli\meter}}،\عددی{\SI{341}{\milli\meter}}،\عددی{\SI{118}{\milli\meter}}،\عددی{\SI{206}{\milli\meter}}،\عددی{\SI{94}{\milli\meter}}،\عددی{\SI{149}{\milli\meter}}،\عددی{\SI{118}{\milli\meter}}
\انتہا{سوال}
%=======================
\ابتدا{سوال}
ثابت کریں کہ کامل موصل کے نلکی مویج میں \عددی{\TE{11}} بلند درجی انداز اوسطاً \عددی{\frac{\omega \mu \beta \rho_0^4 \abs{H_0}^2}{82}} واٹ کی طاقت ترسیل کرتی ہے۔
\انتہا{سوال}
%=======================
\ابتدا{سوال}
ثابت کریں کہ متوازی دو عدد لامحدود موصل چادروں کے مویج میں انقطاعی تعدد سے بلند تعدد پر \عددی{\TE{10}} موج کی تضعیفی مستقل
\begin{align*}
 \alpha=\frac{2}{d}\frac{Z_{c,h}}{Z_{d,h}}\frac{\left(\frac{\lambda_0}{2d}\right)^2}{\sqrt{1-\left(\frac{\lambda_0}{2d}\right)^2}} \quad \quad (\si{\neper/ \meter})
\end{align*}
 ہے جہاں
\begin{description}
\شے{$Z_{c,h}$} مویجی موصل چادر کی قدرتی رکاوٹ کا حقیقی جزو،
\شے{$Z_{d,h}$} مویج میں بھرے ذو برق کی قدرتی رکاوٹ کا حقیقی جزو،
\شے{$d$} دو لامحدود چادروں میں فاصلہ اور
\شے{$\lambda_0$} خالی خلاء میں طول موج ہیں
\end{description}
\انتہا{سوال}
%==========================
\ابتدا{سوال}
ثابت کریں کہ متوازی دو عدد لامحدود موصل چادروں کے مویج میں انقطاعی تعدد سے بلند تعدد پر \عددی{\TE{m0}} موج کی تضعیفی مستقل
\begin{align*}
 \alpha=\frac{2}{d}\frac{Z_{c,h}}{Z_{d,h}}\frac{\left(\frac{\lambda_0}{\lambda_{0c}}\right)^2}{\sqrt{1-\left(\frac{\lambda_0}{\lambda_{0c}}\right)^2}} \quad \quad (\si{\neper/ \meter})
\end{align*}
 ہے جہاں
\begin{description}
\شے{$Z_{c,h}$} مویجی موصل چادر کے قدرتی رکاوٹ کا حقیقی جزو،
\شے{$Z_{d,h}$} مویج میں بھرے ذو برق کی قدرتی رکاوٹ کا حقیقی جزو،
\شے{$d$} دو لامحدود چادروں میں فاصلہ اور
\شے{$\lambda_0$} خالی خلاء میں طول موج ہیں
\end{description}
\انتہا{سوال}
%===========================
\ابتدا{سوال}
لامحدود جسامت کے تانبے کے دو چادروں کے درمیان \عددی{\SI{18}{\milli\meter}} کا فاصلہ ہے۔اس میں \عددی{\SI{10}{\mega\hertz}} تعدد کی \عددی{\TEM} موج کا تضعیفی مستقل اور \عددی{\TE{10}} موج  کا تضعیفی مستقل دریافت کریں۔ 

جواب:\عددی{\alpha=\SI{3.85}{\milli\neper\per\meter}}، \عددی{\alpha=\SI{9.67}{\milli\neper\per\meter}}
\انتہا{سوال}
%==========================
\ابتدا{سوال}
ثابت کریں کہ متوازی دو عدد لامحدود موصل چادروں کے مویج میں انقطاعی طول موج سے کم طول موج \عددی{\lambda_0} پر \عددی{\TM{10}} موج کی تضعیفی مستقل
\begin{align*}
\alpha'=\frac{2\alpha}{\sqrt{1-\left(\frac{\lambda_0}{2d}\right)^2}}
\end{align*}
ہے جہاں 
\begin{description}
\شے{$\alpha$} مویج میں \عددی{\TEM} موج کی تضعیفی مستقل اور
\شے{$d$} چادروں کے درمیان فاصلہ ہے۔
\end{description}
\انتہا{سوال}
%==================
\ابتدا{سوال}
ثابت کریں کہ ایسے مستطیل مویج جس کی چوڑائی \عددی{z_1} اور اونچائی \عددی{y_1} ہو میں انقطاعی تعدد سے زیادہ تعدد پر \عددی{\TE{m0}} موج کی تضعیفی مستقل مندرجہ ذیل ہے۔
\begin{align}
\alpha=\frac{2 Z_{c,h}}{z_1 Z_{d,h}} \frac{\left[\left(\frac{\lambda_0}{\lambda_{0c}}\right)^2+\frac{z_1}{2y_1}\right]}{\sqrt{1-\left(\frac{\lambda_0}{\lambda_{0c}}\right)^2}}
\end{align}
\انتہا{سوال}
%==================
\ابتدا{سوال}
تانبے کی \عددی{\SI{1}{\centi\meter}} چوڑی پٹی تانبے کی وسیع چادر کے متوازی \عددی{\SI{1.2}{\milli\meter}} فاصلے پر پائی جاتی ہے۔ ان کے درمیان \عددی{\epsilon_R=2.6} کا ذو برقی بھرا گیا ہے۔اس مویج میں \عددی{\SI{450}{\mega\hertz}} تعدد کی \عددی{\TEM} موج حرکت کرتی ہے۔مویج میں برقی میدان کا حیطہ \عددی{\SI{300}{\milli\volt\per\meter}} ہے۔ الف) مویج کتنی طاقت منتقل کر رہا ہے۔ ب) فی میٹر مویج میں طاقت کا ضیاع حاصل کریں۔ پ) مویج کا تضعیفی مستقل دریافت کریں۔

جوابات:\عددی{\SI{2.3}{\nano\watt}}، \عددی{\SI{91}{\pico\watt\per\meter}}، \عددی{\SI{39.5}{\milli\neper\per\meter}} یا \عددی{\SI{0.343}{\deci\bel\per\meter}}
\انتہا{سوال}
%==================
\ابتدا{سوال}
موصل چادر \عددی{\sigma=\SI{e6}{\siemens\per\meter}} اور \عددی{\SI{3}{\centi\meter}} چوڑی موصل پٹی \عددی{\sigma=\SI{e6}{\siemens\per\meter}} کے مابین \عددی{\SI{2}{\milli\meter}} موٹائی کا ذو برق \عددی{\epsilon_R=7} پایا جاتا ہے۔برقی میدان \عددی{E_{\text{موثر}}=\SI{3.2}{\volt\per\meter}} اور تعدد  \عددی{\SI{500}{\mega\hertz}} ہے۔ منتقل طاقت اور مویج کا تضعیفی مستقل حاصل کریں۔

جوابات:\عددی{\SI{4.3}{\micro\watt}}، \عددی{\SI{0.312}{\neper\per\meter}} یا \عددی{\SI{2.7}{\deci\bel\per\meter}}
\انتہا{سوال}
%====================
\ابتدا{سوال}
کامل موصل سے بنے مستطیل مویج میں \عددی{\TE{10}} کے لئے ثابت کریں کہ اوسط منتقل طاقت مندرجہ ذیل ہے۔
\begin{align}
P_{\text{اوسط}}=\frac{\omega \mu \beta \abs{H_0}^2 y_1 z_1^3}{4\pi^2}
\end{align}
\انتہا{سوال}
%====================
\ابتدا{سوال}
ہوا اور تانبے کے سرحد پر \عددیء{\SI{1}{\giga\hertz}} تعدد کے موج کا جھکاو حاصل کریں۔

جواب:\عددیء{0.00177^{\circ}}
\انتہا{سوال}
%==============================

\ابتدا{سوال}
ہوا اور پانی \عددیء{\epsilon_R=78} کے سرحد پر \عددیء{\SI{1}{\giga\hertz}} تعدد کے موج کا جھکاو حاصل کریں۔

جواب:\عددیء{6.46^{\circ}}
\انتہا{سوال}
%=====================
\ابتدا{سوال}
تانبے کی چادر کے متوازی \عددی{\SI{150}{\mega\hertz}} تعدد کی موج حرکت کر رہی ہے۔برقی میدان \عددی{E_{\text{موثر}}=\SI{50}{\volt\per\meter}} چادر کی سطح کے عمودی ہے۔ الف) چادر کے متوازی منتقل طاقت کا پوئنٹنگ سمتیہ دریافت کریں۔ ب) مقناطیسی میدان کی موثر قیمت حاصل کریں۔ پ) چادر کی سطح پر برقی میدان حاصل کریں۔ ت) چادر میں داخل طاقت کا پوئنٹنگ سمتیہ دریافت کریں۔

جوابات: \عددی{\SI{6.636}{\watt\per\meter\squared}}، \عددی{\SI{0.133}{\ampere\per\meter}}،  \عددی{\SI{56.3}{\micro\watt\per\meter\squared}}
\انتہا{سوال}
%=======================
\ابتدا{سوال}
موصل کی لامحدود سطح کے متوازی موج حرکت کر رہی ہے۔سطح کے عمودی برقی میدان \عددی{E_{\text{موثر}}=\SI{150}{\volt\per\meter}} ہے۔ موصل کی قدرتی رکاوٹ کی حتمی قیمت \عددی{\abs{Z_c}=\SI{0.012}{\ohm}} ہے۔ الف) سرحد کے متوازی فی میٹر رقبے سے گزرتی طاقت دریافت کریں۔ ب) موصل سطح کے فی میٹر رقبے میں داخل طاقت دریافت کریں۔

جوابات: \عددی{\SI{59.7}{\watt\per\meter\squared}}، \عددی{\SI{1.9}{\milli\watt\per\meter\squared}}
\انتہا{سوال}
%=========================
\ابتدا{سوال}
موصل \عددی{\sigma=\SI{e7}{\siemens\per\meter}} کی سطح  کے متوازی خالی خلاء میں \عددی{\SI{1.2}{\giga\hertz}} تعدد
 اور \عددی{E_{\text{موثر}}=\SI{50}{\milli\volt\per\meter}} کی موج  حرکت کر رہی ہے۔مقناطیسی میدان سرحد کے متوازی ہے جبکہ برقی میدان سرحد کے عمودی ہے۔ فی مربع میٹر موصل رقبے میں طاقت کا ضیاع حاصل کریں۔

جواب:\عددی{\SI{0.38}{\nano\watt\per\meter\squared}} 
\انتہا{سوال}
%=========================
\ابتدا{سوال}
موصل سطح کے متوازی \عددی{\TEM} موج حرکت کر رہی ہے۔ ثابت کریں کہ \عددی{K=\rho_S v_S} کی صورت میں، جہاں \عددی{K} ایمپیئر فی میٹر میں سطحی کثافت برقی رو، \عددی{\rho_S} کولومب فی مربع میٹر میں سطحی کثافت چارج اور \عددی{v_S} میٹر فی سیکنڈ میں موج کی رفتار ہو، \عددی{K=H} ہو گا جہاں \عددی{H} موج کے مقناطیسی میدان \عددی{\kvec{H}} کا حیطہ ہے۔
\انتہا{سوال}
%==========================
