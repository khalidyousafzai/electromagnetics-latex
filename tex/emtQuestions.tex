\باب{سوالات}

\حصہء{ترسیلی تار}
%===================
\ابتدا{سوال}
ترسیلی تار کے مستقل \عددی{R=\SI{20}{\ohm\per\meter}}، \عددی{L=\SI{4}{\micro\henry\per\meter}}، \عددی{G=\SI{80}{\micro\siemens\per\meter}} اور \عددی{C=\SI{60}{\pico\farad\per\meter}} ہیں۔اس میں \عددی{\SI{200}{\mega\hertz}} تعدد کی برقی موج حرکت کر رہی ہے۔ الف) \عددی{\gamma}، \عددی{\alpha}، \عددی{\beta}، \عددی{\lambda} اور \عددی{Z_0} حاصل کریں۔ب)  \عددی{\SI{12}{\meter}} فاصلہ طے کرنے کے بعد موج کا حیطہ ابتدائی قیمت کی نسبت سے کتنا ہو گا؟ پ) \عددی{\SI{1.6}{\meter}} فاصلہ طے کرنے کے بعد موج کا زاویائی فرق کتنا ہو گا؟ 

جوابات:\عددی{\gamma=0.049+j3.1 \, \si{\meter^{-1}}}، \عددی{\alpha=\SI{0.049}{\neper\per\meter}}، \عددی{\beta=\SI{3.1}{\radian\per\meter}}، \عددی{\lambda=\SI{2.03}{\meter}}، \عددی{Z_0=258-j2.37 \, \si{\ohm}}، \عددی{\SI{55.5}{\percent}}، \عددی{284^{\circ}}
\انتہا{سوال}
%==================
\ابتدا{سوال}
ایک ترسیلی تار جس میں موج کی رفتار \عددی{\SI{3e8}{\meter\per\second}} ہے کی قدرتی رکاوٹ \عددی{Z_0=\SI{50}{\ohm}} ہے۔ تار کے داخلی سروں پر \عددی{\SI{20}{\mega\hertz}} کی موج پیدا کی جا رہی ہے جبکہ اس کا دوسرا سرا کسر دور کیا گیا ہے۔ الف) تار کی لمبائی \عددی{\SI{3.75}{\meter}} ہونے کی صورت میں \عددی{Z_{\text{داخلی}}} حاصل کریں۔ ب) تار کی لمبائی بالترتیب \عددی{\SI{7.5}{\meter}}، \عددی{\SI{1.2}{\meter}} اور \عددی{\SI{9}{\meter}} ہونے کی صورت میں \عددی{Z_{\text{داخلی}}} حاصل کریں۔

جوابات:\عددی{\infty}، \عددی{\SI{0}{\ohm}}، \عددی{27.5j \, \si{\ohm}}، \عددی{36.3j \, \si{\ohm}} 
\انتہا{سوال}
%==================
\ابتدا{سوال}
بے ضیاع ترسیلی تار کی فی میٹر امالہ \عددی{\SI{0.25}{\micro\henry\per\meter}} جبکہ اس کی قدرتی رکاوٹ \عددی{\SI{75}{\ohm}} ہے۔الف) تار کی فی میٹر کپیسٹنس دریافت کریں۔ ب) تار میں موج کی رفتار حاصل کریں۔ پ) موج کی تعدد \عددی{\SI{50}{\mega\hertz}} ہونے کی صورت میں \عددی{\beta} حاصل کریں۔ ت) تار کے ساتھ \عددی{\SI{55}{\ohm}} کا بار منسلک ہے۔ \عددی{\Gamma} اور \عددی{s} حاصل کریں۔

جوابات:\عددی{\SI{44.4}{\pico\farad\per\meter}}، \عددی{\SI{3e8}{\meter\per\second}}، \عددی{\beta=\SI{1.05}{\radian\per\meter}}، \عددی{\Gamma=-\tfrac{2}{13}}، \عددی{s=\tfrac{15}{11}}
\انتہا{سوال}
%====================
\ابتدا{سوال}
ترسیلی تار کی قدرتی رکاوٹ \عددی{\SI{300}{\ohm}} ہے۔موج کی تعدد \عددی{\SI{6e8}{\radian\per\second}} جبکہ اس کی رفتار \عددی{\SI{2.8e8}{\meter\per\second}} ہے۔ الف) تار کی فی میٹر امالہ اور کپیسٹنس حاصل کریں۔ ب) تار پر سلسلہ وار جڑی \عددی{\SI{150}{\ohm}} اور \عددی{\SI{0.8}{\micro\henry}} کا بار ڈالا جاتا ہے۔ \عددی{\Gamma} اور \عددی{s} حاصل کریں۔

جوابات:\عددی{L=\SI{1.07}{\micro\henry\per\meter}}، \عددی{C=\SI{11.9}{\pico\farad\per\meter}}، \عددی{\Gamma=0.38+j0.67}، \عددی{s=7.49}
\انتہا{سوال}
%======================
\ابتدا{سوال}
بے ضیاع ترسیلی تار کی \عددی{\SI{80}{\mega\hertz}} تعدد پر قدرتی رکاوٹ \عددی{\SI{75}{\ohm}} اور \عددی{\beta=0.25 \pi \, \si{\radian\per\meter}} ہیں۔ الف) تار کی \عددی{L} اور \عددی{C} حاصل کریں۔ ب) تار پر \عددی{Z_L=80+j100 \, \si{\ohm}} بار لادا جاتا ہے۔بار سے کتنے فاصلے پر تار کی داخلی رکاوٹ \عددی{Z_{\text{داخلی}}} حقیقی  یعنی \عددی{Z_{\text{داخلی}}=R+j0} ہو گا۔

جوابات:\عددی{L=\SI{117}{\nano\henry\per\meter}}، \عددی{C=\SI{20.8}{\pico\farad\per\meter}}، \عددی{\SI{60.34}{\centi\meter}} 
\انتہا{سوال}
%======================
\ابتدا{سوال}
تعدد \عددی{\SI{1}{\mega\radian\per\second}} پر ضیاع کار ترسیلی تار کی قدرتی رکاوٹ  \عددی{Z_0=40+j0 \, \si{\ohm}} اور حرکی مستقل \عددی{\gamma=2+j6 \, \si{\meter^{-1}}} ہیں۔ الف) \عددی{G}، \عددی{C}، \عددی{R} اور \عددی{L} حاصل کریں۔

جوابات:\عددی{G=\SI{0.05}{\siemens\per\meter}}، \عددی{C=\SI{150}{\nano\farad\per\meter}}، \عددی{R=\SI{80}{\ohm\per\meter}}، \عددی{L=\SI{0.24}{\milli\henry\per\meter}}
\انتہا{سوال}
%======================
\ابتدا{سوال}
بے ضیاع ترسیلی تار کی \عددی{\SI{150}{\mega\hertz}} تعدد پر \عددی{Z_0=\SI{80}{\ohm}} اور \عددی{\beta=\SI{6}{\radian\per\meter}} ہیں۔تار پر متوازی جڑے \عددی{\SI{200}{\ohm}} کی مزاحمت اور \عددی{\SI{10}{\pico\farad}} کی کپیسٹر کا بار لادا جاتا ہے۔ الف) \عددی{L} اور \عددی{C} حاصل کریں۔ ب) شرح ساکن موج حاصل کریں۔

جوابات:\عددی{L=\SI{0.51}{\micro\henry\per\meter}}، \عددی{C=\SI{79.6}{\pico\farad\per\meter}}، \عددی{s=4.07}
\انتہا{سوال}
%=====================
\ابتدا{سوال}
منبع برقی دباو سلسلہ وار جڑی رکاوٹ \عددی{Z=300-j300 \, \si{\ohm}} اور بے ضیاع ترسیلی تار کے ساتھ منسلک ہے۔ترسیلی تار کا دوسرا سرا کسے دور ہے۔ترسیلی تار میں طول موج \عددی{\lambda} ہے۔ الف) منبع برقی دباو پر کل \عددی{\SI{300}{\ohm}} رکاوٹ مہیا کرنے کی خاطر ترسیلی تار کی لمبائی کتنی رکھی جائے گی۔ ب) ترسیلی تار کی لمبائی کے  تمام ممکنہ جواب حاصل کریں۔

جوابات:\عددی{\text{لمبائی}=\tfrac{\lambda}{8}}، \عددی{\text{لمبائی}=\tfrac{\lambda}{8}+\tfrac{m \lambda}{2}}
\انتہا{سوال}
%========================
\ابتدا{سوال}
تعدد \عددی{\SI{50}{\mega\hertz}} کے منبع برقی دباو کے ساتھ رکاوٹ \عددی{Z_g=50+j50 \, \si{\ohm}} اور بے ضیاع ترسیلی تار سلسلہ وار جڑے ہیں۔ ترسیلی تار کی قدرتی رکاوٹ \عددی{Z_0=\SI{100}{\ohm}}، لمبائی \عددی{\tfrac{\lambda}{4}} ہے اور یہ بار \عددی{Z_L} کو طاقت فراہم کر رہی ہے۔ الف) بار کی وہ قیمت دریافت کریں جس پر منبع برقی دباو کو کل \عددی{\SI{100}{\ohm}} رکاوٹ نظر آتی ہے۔ ب) ترسیلی تار کی فی میٹر امالہ \عددی{L=\SI{1.5}{\micro\henry\per\meter}} ہونے کی صورت میں ترسیلی تار میں موج کی رفتار اور ترسیلی تار کی لمبائی دریافت کریں۔ 

جوابات:\عددی{Z_L=100+j100 \, \si{\ohm}}، \عددی{\SI{6.6737}{\meter\per\second}}، \عددی{\SI{0.333}{\meter}}
\انتہا{سوال}
%=======================
\ابتدا{سوال}
تیس میٹر لمبی بے ضیاع ترسیلی تار کے دونوں سرے آزاد رکھنے کی صورت میں اس کی کل کپیسٹنس \عددی{C=\SI{1.5}{\nano\farad}} ناپی جاتی ہے۔اس کا ایک سرا کسر دور کرتے ہوئے دوسرے سرے پر نہایت کم دورانیے کا مستطیلی برقی دباو کا جھٹکا دیا جاتا ہے جو کسر دور سرے سے ٹکرا کر واپس لوٹتا ہے۔تار میں دو طرفہ فاصلہ کل \عددی{\SI{0.4}{\micro\second}} میں طے پاتا ہے۔ترسیلی تار کی قدرتی رکاوٹ حاصل کریں۔

جواب:\عددی{Z_0=\SI{133.3}{\ohm}}  
\انتہا{سوال}
%=======================
\ابتدا{سوال}
ترسیلی تار کی قدرتی رکاوٹ \عددی{Z_0=\SI{60}{\ohm}} جبکہ اس پر موج کی رفتار \عددی{\SI{2.8e8}{\meter\per\second}} ہے۔تار پر آمدی موج کی مساوات  \عددی{{V_s^{+}(z,t)=100 \cos(\omega t -\pi z) \, \si{\volt}}} ہے۔ الف) موج کی زاویائی تعدد حاصل کریں۔ ب) آمدی برقی رو کے موج کی مساوات لکھیں۔ پ) ترسیلی تار کا \عددی{z>0} حصہ ہٹا کر \عددی{z=0} پر \عددی{Z_L=60+j40\,\si{\ohm}} رکاوٹ نسب کرنے کی صورت میں \عددی{\Gamma} حاصل کریں۔انعکاسی موج \عددی{V_s^{-}(z,t)} کی مساوات لکھیں اور \عددی{z=\SI{-2.25}{\meter}} پر \عددی{V_s} حاصل کریں۔

جوابات:\عددی{\omega=\SI{879.6}{\mega \radian\per\second}}، \عددی{I^+(z,t)=\tfrac{5}{3}\cos(\omega t -\pi z) \, \si{\ampere}}،
 \عددی{\Gamma=0.1+j0.3= 0.316 \phase{71.6^{\circ}}}، \\
\عددی{V_s^{-}(z,t)=31.6 e^{j(\pi z+1.249)} \, \si{\volt}}، \عددی{V_s(z=\SI{-2.5}{\meter})=130.4 e^{j 0.71} =130.4\phase{40.6^{\circ}}}
\انتہا{سوال}
%========================
\ابتدا{سوال}
\عددی{\SI{300}{\ohm}} قدرتی رکاوٹ کی ترسیلی تار پر متوازی جڑے \عددی{\SI{400}{\ohm}} اور \عددی{\SI{600}{\ohm}} کا بار لادا جاتا ہے۔تار کی لمبائی \عددی{\tfrac{5 \lambda}{8}} ہے جبکہ اسے داخلی جانب \عددی{v(t)=310\cos(2\times 10^9 t) \, \si{\volt}} برقی دباو مہیا کی جاتی ہے۔بار بردار ترسیلی تار کی داخلی رکاوٹ \عددی{Z_{\text{داخلی}}} حاصل کرتے ہوئے  بالترتیب دونوں مزاحمتوں کو مہیا اوسط طاقت حاصل کریں۔

جوابات:\عددی{Z_{\text{داخلی}}=292.7+j65.9 \, \si{\ohm}}، \عددی{\SI{93.8}{\watt}}، \عددی{\SI{62.5}{\watt}}
\انتہا{سوال}
%========================
\ابتدا{سوال}

\انتہا{سوال}
%==========================
