\باب{سوالات}
\حصہ{موصل}
%=====================================
\ابتدا{سوال}
نلکی محدد میں کثافت برقی رو \عددی{\kvec{J}=50 e^{-1.5z} (\rho^2\arho+\az) \, \si{\ampere\per\meter\squared}} دی گئی ہے۔الف) سطح \عددی{z=0} اور \عددی{z=1} پر رداس \عددی{0<\rho<1} کی ٹکیا سے \عددی{\az} سمت میں گزرتی برقی رو دریافت کریں۔ ب) بند بیلن \عددی{0 \le z \le 1}، \عددی{0 \le \rho \le 1} سے خارج کل برقی رو حاصل کریں۔

جوابات:\عددی{\SI{157}{\ampere}}، \عددی{\SI{35}{\ampere}}، \عددی{\SI{41}{\ampere}}
\انتہا{سوال}
%===================================
\ابتدا{سوال}
کثافت برقی رو \عددی{\kvec{J}=\frac{550\sin 2\theta}{r^2+6}\, \si{\ampere\per\meter\squared}} دی گئی ہے۔کروی سطح \عددی{r=0.5}، \عددی{0.1\pi \le \theta \le 0.25\pi}، \عددی{0 \le \phi \le 2\pi} سے رداسی سمت میں خارج کل برقی رو حاصل کریں۔اس سطح پر اوسط کثافت برقی رو دریافت کریں۔

جوابات:\عددی{\SI{29.86}{\ampere}}، \عددی{\SI{77.9}{\ampere\per\meter\squared}}
\انتہا{سوال}
%==================================
\ابتدا{سوال}
دو متوازی سطحیں \عددی{z=0} اور \عددی{z=\SI{10}{\milli\meter}} پر پائے جاتے ہیں جن کے درمیان \عددی{\SI{1000}{\volt}} کا برقی دباو ہے۔نچلی سطح سے الیکٹران صفر ابتدائی رفتار کے ساتھ خارج ہو کر بالائی سطح کی جانب اسراع پذیر ہوتے ہیں جہاں انہیں وصول کیا جاتا ہے۔ الیکٹران کا چارج \عددی{\SI{-1.6e-19}{\coulomb}} جبکہ اس کی کمیت \عددی{\SI{9.1e-31}{\kilo\gram}} ہے۔نچلی سطح سے کل \عددی{\SI{50}{\micro\ampere}} برقی رو \عددی{\SI{0.1}{\milli\meter}} رداس کے شعاع کی صورت میں خارج ہوتی ہے۔چادروں کے درمیان \عددی{E} حاصل کریں۔اسراع پذیر الیکٹران کی رفتار \عددی{v(t)} کی مساوات حاصل کریں۔اسی طرح \عددی{z(t)} اور \عددی{v(z)} بھی حاصل کریں۔کثافت برقی رو اور حجمی کثافت چارج کے مساوات حاصل کریں۔

جوابات:\عددی{a(t)=-10^5 \az \, \si{\volt\per\meter}}، \عددی{v(t)=\num{1.758e16}t \, \si{\meter\per\second}}، \عددی{z(t)=\num{8.792e15}t^2\, \si{\meter}}، \عددی{v(z)=\num{1.87e8} \sqrt{z} \, \si{\meter\per\second}}، \عددی{\SI{1592}{\ampere\per\meter\squared}}، \عددی{\tfrac{\num{-8.51e-6}}{\sqrt{z}} \,\si{\coulomb\per\meter^3}} 
\انتہا{سوال}
%==================================
\ابتدا{سوال}
کثافت برقی رو \عددی{\kvec{J}=\tfrac{J_0e^{-\alpha z}}{(y+1)(x+1)}\az} کی صورت میں سطح \عددی{z=0}، \عددی{0 < x <1} ،\عددی{0 < y <1} سے گزرتی برقی رو حاصل کریں۔

جواب:\عددی{0.48 J_0}
\انتہا{سوال}
%=================================
\ابتدا{سوال}
کثافت برقی رو \عددی{\kvec{J}=3xyz\ax+2x^2yz^2\ay-5xy^2z^2} کی صورت میں نقطہ \عددی{N(7,-4,2)} پر \عددی{\SI{1}{\meter}} لمبائی اطراف کے مکعب سے کتنی برقی رو خارج ہوتی ہے؟ مکعب کے اطراف کارتیسی محدد کے متوازی ہیں جبکہ نقطہ \عددی{N} اس کا وسطی نقطہ ہے۔مکعب میں چارج کس شرح سے بڑھ رہی ہے؟

جواب:\عددی{\SI{1875}{\ampere}}، \عددی{\SI{-1875}{\coulomb\per\second}}
\انتہا{سوال}
%================================
\ابتدا{سوال}\شناخت{سوال_کپیسٹر_سلیکان_پچر}
سلیکان کی موصلیت \عددی{\SI{1200}{\siemens\per\meter}} ہے۔سلیکان کا پچر \عددی{0 \le z \le 0.08}، \عددی{0.02 \le \rho \le 0.07}، \عددی{0 \le \phi \le 0.3\pi} شکل رکھتا ہے۔برقی شدت  \عددی{\kvec{E}=\tfrac{0.005}{\rho}\aphi\, \si{\volt\per\meter}} کی صورت میں پچر میں برقی رو اور اس کی مزاحمت حاصل کریں۔

جوابات:\عددی{\SI{0.6}{\ampere}}، \عددی{\SI{6}{\milli \ohm}}
\انتہا{سوال}
%================================
\ابتدا{سوال}
برقی شدت \عددی{\kvec{E}=\tfrac{0.005}{\rho}\arho\, \si{\volt\per\meter}} ہونے کی صورت میں سوال \حوالہ{سوال_کپیسٹر_سلیکان_پچر} کو دوبارہ حل کریں۔

جوابات:\عددی{\SI{0.3468}{\ampere}}، \عددی{\SI{18}{\milli\ohm}}
\انتہا{سوال}
%==============================
\ابتدا{سوال}
پاکستان میں برقی طاقت کی پیداوار اور منتقلی واپڈا\فرہنگ{واپڈا}\حاشیہب{WAPDA} کے ذمہ ہے۔اگرچہ تانبہ کی موصلیت نہایت عمدہ ہے لیکن تانبہ مہنگا عنصر ہے لہٰذا برقی طاقت کے منتقلی کے لئے المونیم کی تار استعمال کی جاتی ہے۔المونیم ازخود کمزور عنصر ہے لہٰذا المونیم کے تاروں کو سٹیل کے تار کے گرد لپیٹا جاتا ہے۔فرض کریں کہ \عددی{\SI{2}{\milli\meter}}  رداس کے لوہے کی تار پر \عددی{\SI{3}{\milli\meter}} المونیم کی تہہ چڑھائی جاتی ہے۔ایسی ایک کلو میٹر لمبی تار کی مزاحمت حاصل کریں۔المونیم اور لوہے کی موصلیت بالترتیب \عددی{\SI{3.82e7}{\siemens\per\meter}} اور \عددی{\SI{1.03e7}{\siemens\per\meter}} ہیں۔
   
جواب:\عددی{\SI{377.4}{\milli\ohm}}
\انتہا{سوال}
%===============================
\ابتدا{سوال}
ایک خطے میں کثافت برقی رو کو \عددی{\kvec{J}=0.02re^{-1000t}\ar \, \si{\ampere\per\meter \squared}} لکھا جا سکتا ہے۔لمحہ \عددی{t=\SI{1}{\milli\second}} پر رداس \عددی{r=3} کے کرہ سے کتنی برقی رو خارج ہو گی۔استمراری مساوات \عددی{\nabla \cdot \kvec{J}=-\frac{\partial \rho_h}{\partial t}} استعمال کرتے ہوئے  \عددی{\rho_h} حاصل کریں۔ایسا کرتے ہوئے \عددی{t \to \infty} پر \عددی{\rho_h \to 0} لیں۔حجمی کثافت چارج \عددی{\rho_h} کی رفتار کی مساوات حاصل کریں۔

جوابات:\عددی{\SI{2.5}{\ampere}}، \عددی{\rho_h=60e^{-1000t} \, \si{\micro\coulomb\per\meter^3}}، \عددی{333 r \ar \, \si{\meter\per\second}}
\انتہا{سوال}
%==============================
\ابتدا{سوال}
برقی ہیٹر\فرہنگ{ہیٹر}\حاشیہب{heater}\فرہنگ{heater} عموماً نائیکروم کی تار سے بنائے جاتے ہیں۔گھریلو \عددی{\SI{220}{\volt}} اور \عددی{\SI{50}{\hertz}} پر چلنے والے \عددی{\SI{1}{\kilo\watt}} طاقت کے ہیٹر کے تار کا قطر حاصل کریں اگر تار کی لمبائی \عددی{\SI{4}{\meter}} ہو۔اس طاقت پر تار میں کثافت برقی رو حاصل کریں۔صفحہ \حوالہ{جدول_جدول_موصلیت_کے_مستقل} پر جدول \حوالہ{جدول_جدول_موصلیت_کے_مستقل} کی مدد لیں۔   

جواب: \عددی{\SI{0.324}{\milli\meter}}، \عددی{\SI{55}{\mega\ampere\per\meter\squared}}
\انتہا{سوال}
%=====================


\ابتدا{سوال}\شناخت{سوال_موصل_لکیری_چارج_اور_بہاو}
\عددیء{N(0,0,2)} سے گزرتی \عددیء{y} محدد کے متوازی لکیری چارج کثافت
\begin{align*}
\rho_L=\SI{5}{\nano \coulomb \per \meter}  \quad \quad (-\infty < y < \infty, x=0,z=2)
\end{align*} 
سے \عددیء{M(5,3,1)} پر \عددیء{\kvec{D}} حاصل کریں۔

جواب:\عددیء{\kvec{D}=\tfrac{5\times 10^{-9}(5\ax-1\az)}{2 \pi \times 26}}
\انتہا{سوال}
%==========
\ابتدا{سوال}
لامحدود موصل زمینی سطح \عددیء{z=0}  رکھتے ہوئے  سوال \حوالہ{سوال_موصل_لکیری_چارج_اور_بہاو} کو دوبارہ حل کریں۔

جواب:\عددیء{\kvec{D}=\tfrac{5\times 10^{-9}(40\ax-112\az)}{2 \pi \times 884}}
\انتہا{سوال}
%====
\ابتدا{سوال}
\عددیء{N(0,0,2)} سے گزرتی \عددیء{y} محدد کے متوازی لکیری چارج کثافت
\begin{align*}
\rho_L=\SI{5}{\nano \coulomb \per \meter}  \quad \quad (-\infty < y < \infty, x=0,z=2)
\end{align*} 
پایا جاتا ہے جبکہ \عددیء{z=0} پر لامحدود موصل زمینی سطح موجود ہے۔سطح کے \عددیء{M(5,3,0)} مقام پر سطحی چارج کثافت حاصل کریں۔

جواب: \عددیء{\SI{-0.1097}{\nano \coulomb \per \meter \squared}}
\انتہا{سوال}
%=======
\ابتدا{سوال}
مشق \حوالہ{مشق_کپیسٹر_نیم_موصل_موصلیت} میں \عددیء{\SI{300}{\kelvin}} درجہ حرارت پر  سلیکان اور جرمینیم کے مستقل دئے گئے ہیں۔اگر سلیکان میں المونیم کا ایک ایٹم فی  \عددیء{\num{1e7}} سلیکان ایٹم  ملاوٹ شامل کی جائے تو سیلکان کی موصلیت کیا ہو گی۔سلیکان کی تعدادی کثافت \عددیء{\num{5e28}} ایٹم فی مربع میٹر ہے۔(ہر ملاوٹی المونیم کا  ایٹم ایک عدد آزاد خول پیدا کرتا ہے جن  کی تعداد مشق میں دئے خالص سلیکان میں آزاد خول کی تعداد سے بہت زیادہ ہوتی ہے لہٰذا ایسی صورت میں موصلیت صرف ملاوٹی ایٹموں کے پیدا کردہ آزاد خول ہی تعین کرتے ہیں۔) 

جواب: \عددیء{\SI{800}{\siemens \per \meter}}
\انتہا{سوال}
%======
\ابتدا{سوال}
صفحہ \حوالہصفحہ{مثال_کپیسٹر_نقطہ_چارج_سے_لامحدود_سطح_میں_پیدا_کثافت} پر مثال \حوالہ{مثال_کپیسٹر_نقطہ_چارج_سے_لامحدود_سطح_میں_پیدا_کثافت} میں لامحدود موصل سطح \عددیء{z=0} میں \عددیء{(0,0,z)} پر پائے جانے والے نقطہ چارج \عددیء{Q} سے پیدا سطحی چارج کثافت \عددیء{\rho_S} حاصل کیا گیا۔موصل سطح میں پائے جانے والا کل چارج سطحی تکمل سے حاصل کریں۔

جواب: \عددیء{-Q} 
\انتہا{سوال}
%========================
\ابتدا{سوال}
صفحہ \حوالہصفحہ{مساوات_کپیسٹر_موصل_آزاد_چارج_کثافت} پر تانبے کے ایک مربع میٹر میں کل آزاد چارج مساوات \حوالہ{مساوات_کپیسٹر_موصل_آزاد_چارج_کثافت} میں حاصل کیا گیا۔ایک ایمپئیر کی برقی رو کتنے وقت میں اتنے چارج کا اخراج کرے گا۔

جواب: چار سو اکتیس \عددیء{(431)} سال۔ 
\انتہا{سوال}
%=========================

\ابتدا{سوال}
مساوات \حوالہ{مساوات_کپیسٹر_نلکی_زمین_کپیسٹنس} میں  \عددیء{\ln \frac{h+\sqrt{h^2-b^2}}{b}=\cosh^{-1} \frac{h}{b}} لکھا گیا ہے۔اسے ثابت کریں۔
\انتہا{سوال}
%===========

\ابتدا{سوال}
پانچ میٹر رداس کی موصل نلکی کا محور برقی زمین سے تیرہ میٹر پر ہے۔نلکی پر ایک سو وولٹ کا برقی دباو ہے۔
\begin{itemize}
\item
ایسی لکیری چارج کثافت کا زمین سے فاصلہ اور اس کا \عددیء{\rho_L} حاصل کریں جو ایسی ہم قوہ سطح  پیدا کرے۔
\item
موصل نلکی سے پیدا پچاس وولٹ کے ہم قوہ سطح کا رداس اور اس کے محور کا زمین سے فاصلہ دریافت کریں۔
\item
نلکی پر زمین کے قریب اور اس سے دور سطحی چارج کثافت حاصل کریں۔
\end{itemize} 

جوابات:\عددیء{\SI{12}{\meter}}، \عددیء{\SI{3.46}{\nano \coulomb \per \meter}}، \عددیء{\SI{13.4}{\meter}}، \عددیء{\SI{18}{\meter}}، \عددیء{\SI{1.65}{\pico \farad \per \meter \squared}} اور \عددیء{\SI{0.73}{\pico \farad \per \meter \squared}}

\انتہا{سوال}
%=============
\ابتدا{سوال}
مندرجہ ذیل صورتوں میں موصل میں \عددی{\abs{\kvec{J}}} حاصل کریں۔ الف) حرکت پذیری \عددی{\SI{5.2e-3}{\meter\squared\per\volt\per\second}}، حجمی کثافت چارج \عددی{\SI{-4e9}{\coulomb\per\meter^3}} اور برقی شدت \عددی{\SI{72}{\milli\volt\per\meter}} ہے۔ ب) رفتار بہاو \عددی{\SI{30}{\micro\meter\per\second}} ہے جبکہ الیکٹران کی تعدادی کثافت \عددی{\num{5.5e28}} فی مربع میٹر ہے۔ پ) موصلیت \عددی{\SI{2.5e6}{\siemens\per\meter}} جبکہ برقی شدت \عددی{\SI{50}{\milli\volt}} ہے۔

جوابات:\عددی{\SI{1.5}{\mega\ampere\per\meter\squared}}، \عددی{\SI{0.26}{\mega\ampere\per\meter\squared}}، \عددی{\SI{0.125}{\mega\ampere\per\meter\squared}}
\انتہا{سوال}
%======================
\ابتدا{سوال}
نلکی محدد میں رداس \عددی{\rho=0.2} اور \عددی{\rho=0.5} پر موصل نلکی چادر پائے جاتے ہیں جبکہ ان چادروں کے درمیان خالی خلاء میں \عددی{V=150 \rho^3} برقی دباو پایا جاتا ہے۔ الف) اندرونی اور بیرونی نلکی چادر پر سطحی کثافت چارج دریافت کریں۔ دونوں چادروں کے درمیان خطہ \عددی{0<z<1}، \عددی{0.2<\rho<0.5} میں کل چارج حاصل کریں۔ پ) خطہ \عددی{0<z<1} میں کل چارج حاصل کریں۔یہاں چادروں پر اور خلاء میں پائے جانے والے تمام کو شامل کریں۔یاد رہے کہ مثبت موصل چادر سے برقی بہاو کا اخراج ہوتا ہے۔

جوابات:\عددی{\SI{-0.159}{\nano\coulomb\per\meter\squared}}، \عددی{\SI{0.996}{\nano\coulomb\per\meter\squared}}، \عددی{\SI{-2.93}{\nano\coulomb}}، \عددی{0}
\انتہا{سوال}
%===================
\ابتدا{سوال}
گریفائٹ سے بنی  نلکی جس کی لمبائی \عددی{\SI{4}{\centi\meter}}، اندرونی رداس \عددی{\rho=\SI{5}{\centi\meter}} اور بیرونی رداس \عددی{\rho=\SI{7}{\centi\meter}} ہے میں \عددی{\SI{2}{\ampere}} کی برقی رو رداسی سمت میں  گزر رہی ہے۔گریفائٹ  کی موصلیت \عددی{\SI{7e4}{\siemens\per\meter}} ہے۔نلکی میں \عددی{\kvec{J}}، \عددی{\kvec{E}} حاصل کریں۔نلکی کے اندرونی اور بیرونی گول سطحوں کے درمیان برقی دباو \عددی{V} حاصل کرتے ہوئے ان کے درمیان مزاحمت \عددی{R} حاصل کریں۔نلکی میں طاقت کا ضیاع \عددی{V\times I} اور حجمی تکمل \عددی{\iiint \kvec{J}\cdot \kvec{E} \dif h} سے حاصل کریں۔

جوابات:\عددی{\kvec{J}=\tfrac{7.96}{\rho}\arho}، \عددی{\kvec{E}=\tfrac{1.14e-4}{\rho}\arho}، \عددی{V=\SI{38.25}{\micro\volt}}، \عددی{R=\SI{19.12}{\micro\ohm}}، \عددی{\SI{76}{\micro\watt}}، \عددی{\SI{76}{\micro\watt}}
\انتہا{سوال}
%=====================
\ابتدا{سوال}
رقبہ \عددی{A} کی ایک موصل چادر \عددی{z=0} اور دوسری \عددی{z=d} پر رکھی گئی ہے۔چادروں کے درمیانی خطے میں موصلیت \عددی{\sigma(z)=\sigma_0e^{-z}} ہے جہاں \عددی{\sigma_0} مستقل ہے۔چادر \عددی{z=0} کو صفر وولٹ پر رکھا جاتا ہے جبکہ چادر \عددی{z=d} کو \عددی{V_0} برقی دباو پر رکھا جاتا ہے۔ الف) دونوں چادروں کے درمیان مزاحمت حاصل کریں۔ ب) چادروں کے مابین برقی رو حاصل کریں۔ پ) چادروں کے درمیان کثافت برقی رو اور برقی میدان کی شدت حاصل کریں۔

جوابات:\عددی{R=\tfrac{e^d-1}{A\sigma_0}}، \عددی{I=\tfrac{A\sigma_0 V_0}{e^d-1}}، \عددی{\kvec{J}=-\tfrac{V_0 \sigma_0}{e^d-1}\az}،
 \عددی{\kvec{E}=-\tfrac{V_0e^z}{e^d-1}\az}
\انتہا{سوال}
%======================
\ابتدا{سوال}
خالی خلاء میں برقی دباو \عددی{V=50(\rho^2+1)z\sin \phi \,\si{\volt}} دی گئی ہے۔ہم قوہ سطح \عددی{V=\SI{100}{\volt}} موصل سطح کو ظاہر کرتی ہے۔اس سطح کی مساوات حاصل کریں۔اس سطح پر نقطہ \عددی{(0.65,0.25\pi,z)} دیا گیا ہے۔اس نقطے پر \عددی{\kvec{E}} حاصل کرتے ہوئے سطحی کثافت چارج \عددی{\abs{\rho_S}} حاصل کریں۔

جوابات:\عددی{(\rho^2+1)z\sin\phi=2}، \عددی{\kvec{E}=-91\arho-154\aphi-50\az}،
 \عددی{\abs{\rho_S}=\SI{1.65}{\nano\coulomb\per\meter\squared}}
\انتہا{سوال}
%=====================
\ابتدا{سوال}
مثبت \عددی{z} خطے میں برقی دباو \عددی{V=\tfrac{50z(x+y)}{x^2+9} \, \si{\volt}} دی گئی ہے۔موصل سطح \عددی{z=0} پر  \عددی{\kvec{E}} حاصل کریں۔موصل سطح پر سطحی کثافت چارج حاصل کریں۔سطح پر \عددی{2<x<5}، \عددی{3<y<6} کل چارج حاصل کریں۔

جوابات:\عددی{\kvec{E}=-\tfrac{50(x+y)}{x^2+9}\az \, \si{\volt\per\meter}}، \عددی{\rho_S=-\tfrac{50\epsilon_0(x+y)}{x^2+9} \, \si{\coulomb\per\meter\squared}}، \عددی{\SI{-1.52}{\nano\coulomb}}
\انتہا{سوال}
%=====================
\ابتدا{سوال}
خالی خلاء میں \عددی{V=50 \ln \tfrac{(x+1)^2+(y+1)^2}{x^2+(y-1)^2} \, \si{\volt}} پایا جاتا ہے۔نقطہ \عددی{N(3,1,2)} پر موصل سطح ہے۔اس سطح پر \عددی{\kvec{E}} اور \عددی{\rho_S} حاصل کریں۔سطح پر عمودی اکائی سمتیہ \عددی{\aN} حاصل کریں۔

جوابات:\عددی{\kvec{E}=13.33\ax-10\ay \, \si{\volt\per\meter}}، \عددی{\rho_S=\SI{148}{\pico\coulomb\per\meter\squared}}،
 \عددی{\aN=\tfrac{4}{5}\ax-\tfrac{3}{5}\ay}؛ چونکہ ہمیں یہ نہیں معلوم کہ موصل سطح نقطہ \عددی{N}  کے کس جانب ہے لہٰذا سطحی کثافت چارج مثبت یا منفی ہو سکتا ہے۔یوں اکائی سمتیہ کی سمت الٹ بھی ممکن ہے۔
\انتہا{سوال}
%===================
\ابتدا{سوال}
نقطہ چارج \عددی{Q=8\pi \, \si{\micro\coulomb}} موصل زمین \عددی{z=0} کے قریب نقطہ \عددی{(2,0,4)} اور \عددی{(-2,0,4)} پر پائے جاتے ہیں۔الف) محدد \عددی{y} پر \عددی{\kvec{D}} کی مساوات حاصل کریں۔یاد رہے کہ برقی زمین میں چارج کے عکس بھی کردار ادا کریں گے۔ ب) نقطہ \عددی{(0,0,0)} پر سطحی کثافت چارج حاصل کریں۔

جوابات:\عددی{\kvec{D}=-16\left[\tfrac{1}{[(x+2)^2+16]^{3/2}}+\tfrac{1}{[(x-2)^2+16]^{3/2}} \right]\az \, \si{\micro \coulomb \per\meter\squared}}، \عددی{\rho_S=\SI{-0.36}{\micro\coulomb\per\meter\squared}}
\انتہا{سوال}
%===================
\ابتدا{سوال}
سطح \عددی{x=0} برقی زمین ہے۔لکیری کثافت چارج \عددی{\SI{2}{\nano\coulomb\per\meter}} مقام \عددی{y=3}، \عددی{x=2} پر \عددی{z} محدد کے متوازی پایا جاتا ہے جبکہ لکیری کثافت چارج \عددی{\SI{-5}{\nano\coulomb\per\meter}} مقام \عددی{y=5}، \عددی{x=4} پر \عددی{z} محدد کے متوازی پایا جاتا ہے۔نقطہ \عددی{N(7,2,3)} پر برقی دباو حاصل کریں۔

جواب:\عددی{\SI{46.4}{\volt}}
\انتہا{سوال}
%=======================
\ابتدا{سوال}
نیم موصل ٹکڑے کی لمبائی \عددی{\SI{15}{\milli\meter}}، چوڑائی \عددی{\SI{5}{\milli\meter}} اور موٹائی \عددی{\SI{2}{\milli\meter}} ہے۔اگر الیکٹران اور خول کی تعدادی کثافت بالترتیب \عددی{\SI{1.5e18}{\per\meter^3}} اور \عددی{\SI{2e15}{\per\meter^3}} ہوں جبکہ ان کی حرکت پذیری \عددی{\mu_e=\SI{0.08}{\meter\squared\per\volt\per\second}} اور \عددی{\mu_h=\SI{0.0025}{\meter\squared\per\volt \per\second}} ہوں تب لمبائی جانب نیم موصل ٹکڑے کی مزاحمت حاصل کریں۔

جواب:\عددی{\SI{78}{\kilo\ohm}}
\انتہا{سوال}
%=======================
\ابتدا{سوال}
کوارٹس میں \عددی{\kvec{E}=-20\ax+35\ay+15\az \, \si{\volt\per\meter}} میدان پایا جاتا ہے۔الف) کتاب کے آخر میں جدول \حوالہ{جدول_جدول_جزوی_برقی_مستقل_زاویہ_زیاع} سے کوارٹس کا مستقل \عددی{\epsilon_R} دریافت کرتے ہوئے  \عددی{\chi_e} حاصل کریں۔ ب) کوارٹس میں \عددی{\kvec{D}} اور \عددی{\kvec{P}} حاصل کریں۔

جوابات:\عددی{\epsilon_R=3.8}، \عددی{\chi_e=2.8}، \عددی{\kvec{D}=-0.67\ax+1.18\ay+0.50\az\, \si{\nano\coulomb\per\meter\squared}}، \\ \عددی{\kvec{P}=-0.50\ax+0.87\ay+0.37\az\, \si{\nano\coulomb\per\meter\squared}}
\انتہا{سوال}
%=======================
\ابتدا{سوال}
کسی مخصوص حرارت اور میکانی دباو پر ہایڈروجن  گیس کی تعددی کثافت \عددی{\SI{6e25}{\per\meter^3}} ہے۔اگر اسے \عددی{\SI{2400}{\volt\per\meter}} کے برقی میدان میں رکھا جائے تو ہر قطبی ایٹم  میں مثبت اور منفی خطوں کے درمیان اوسطاً \عددی{\SI{6.6e-19}{\meter}} فاصلہ پایا جاتا ہے۔ایسی صورت میں ہایڈروجن کے مستقل \عددی{\chi_e} اور \عددی{\epsilon_R} حاصل کریں۔

جوابات:\عددی{\chi_e=0.000298}، \عددی{\epsilon_R=1.000298}
\انتہا{سوال}
%=====================
\ابتدا{سوال}
خطہ-الف میں \عددی{\kvec{E}_1=25\ax+10\ay-15\az\, \si{\volt\per\meter}} پایا جاتا ہے۔یہ خطہ محدد کے مرکز کو چھوتا ہے۔برقی مستقل \عددی{\epsilon_{R1}=1} اور \عددی{\epsilon_{R2}=2.5} ہیں۔مرکز پر  سمتیہ \عددی{\kvec{A}=45\ax-20\ay+10\az} خطہ-الف کا عمودی سمتیہ ہے جس کی سمت خطہ-الف سے خطہ-ب کی جانب ہے۔سمتیہ \عددی{\kvec{A}} کا \عددی{\kvec{E}_1}اور  \عددی{\kvec{E}_2}  کے ساتھ زاویہ حاصل کریں۔دونوں خطوں کے سرحدی سطح کے عمودی اور متوازی، \عددی{\kvec{E}_1} کے  اجزاء حاصل کریں۔خطہ-ب میں \عددی{\kvec{E}_2} بھی حاصل کریں۔

جوابات:\عددی{60^{\circ}}، \عددی{77^{\circ}}، \عددی{13.8\ax-6.1\ay+3.1\az}، \عددی{11.2\ax+16.1\ay-18\az}، \عددی{\kvec{E}_2=16.7\ax+13.7\ay-16.8\az \, \si{\volt\per\meter}} 
\انتہا{سوال}
%====================
\ابتدا{سوال}
ہم محوری تار کے اندرونی تار کا بیرونی رداس \عددی{\SI{6}{\milli\meter}} ہے جبکہ بیرونی تار کا اندرونی رداس \عددی{\SI{15}{\milli\meter}} ہے۔موصل تاروں کے درمیان دو مختلف ذو برق استعمال کئے جاتے ہیں۔ اندرونی تار پر \عددی{\SI{4}{\milli\meter}} کی پہلی تہہ کا \عددی{\epsilon_{R1}=1.5} ہے جبکہ بقایا حصے میں ذو برق کا \عددی{\epsilon_{R2}=2.5} ہے۔ اندرونی تار پر \عددی{\SI{15}{\nano\coulomb\per\meter\squared}} ہونے کی صورت میں دونوں تاروں کے درمیان برقی دباو حاصل کریں۔

جواب:\عددی{\SI{5.11}{\volt}}
\انتہا{سوال}
%=====================
\ابتدا{سوال}
رداس \عددی{\SI{6}{\milli\meter}} کے موصل کرہ پر \عددی{\epsilon_{R1}=1.5} برقی مستقل کے ذو برق کی \عددی{\SI{4}{\milli\meter}} موٹی تہہ پہلی چڑھائی۔اس کے اوپر \عددی{\epsilon_{R2}=2.5} برقی مستقل کے ذو برق کی \عددی{\SI{5}{\milli\meter}} موٹی دوسری تہہ چڑھائی جاتی ہے۔بیرونی موصل کرہ کا اندرونی رداس \عددی{\SI{15}{\milli\meter}} ہے۔ اندرونی کرہ پر \عددی{\SI{15}{\nano\coulomb\per\meter\squared}} ہونے کی صورت میں دونوں موصل کرہ کے درمیان برقی دباو حاصل کریں۔

جواب:\عددی{\SI{3.52}{\volt}}
\انتہا{سوال}
%=====================
\ابتدا{سوال}
متوازی چادر کپیسٹر کے چادروں کے درمیان فاصلہ \عددی{\SI{1}{\milli\meter}} ہے جبکہ چادر کا رقبہ \عددی{\SI{100}{\centi\meter\squared}} ہے۔چادروں کے درمیان ذو برق کا \عددی{\epsilon_R=4} ہے۔کپیسٹر کا \عددی{C} حاصل کریں۔کپیسٹر کے چادروں کے مابین \عددی{\SI{50}{\volt}} کی صورت میں \عددی{E}، \عددی{D}، \عددی{Q} اور کپیسٹر میں موجود توانائی \عددی{W} حاصل کریں۔اب اگر چادروں کے درمیان سے ذو برق ہٹا دیا جائے تب \عددی{Q} کتنا ہو گا۔\عددی{E}، \عددی{D}، \عددی{Q} اور کپیسٹر میں موجود توانائی \عددی{W} دوبارہ حاصل کریں۔ساتھ ہی ساتھ برقی دباو بھی حاصل کریں۔

جوابات:\عددی{}
\انتہا{سوال}
%======================
