\باب{سوالات}
\section*{گاؤس}
%====================
\ابتدا{سوال}
محدد کے مرکز پر \عددی{\SI{20}{\nano\coulomb}} چارج پایا جاتا ہے۔اس کے علاوہ \عددی{z=0} سطح پر \عددیء{\SI{5}{\nano\coulomb\per\meter}} کے لکیری چارج \عددی{y=-1} اور \عددی{y=-3} پر پائے جاتے ہیں۔نقطہ \عددی{(0,-2,0)} پر \عددی{\kvec{D}} حاصل کریں۔\عددی{(0,1,0)} پر رداس \عددی{r=1.5} کے کرہ کی سطح پر کل برقی بہاو  حاصل کریں۔ 

جوابات:\عددی{-\tfrac{5}{4\pi}\ay \,\si{\coulomb\per\meter\squared}}، \عددی{\SI{20}{\nano\coulomb}}
\انتہا{سوال}
%==============================
\ابتدا{سوال}
رداس \عددی{\rho=\SI{10}{\centi\meter}} کے نلکی سطح کے \عددی{z>0} حصے  پر سطحی کثافت چارج \عددی{\rho_S=2ze^{-z^2} \, \si{\nano\coulomb\per\meter\squared}} پائی جاتی ہے۔سطح پر کل چارج دریافت کریں۔اس سطح سے \عددی{z=1} تا \عددی{z=2} زاویہ \عددی{\phi=45^{\circ}} تا \عددی{\phi=75^{\circ}} کتنی برقی بہاو خارج ہوتی ہے۔

جوابات:\عددی{0.2\pi\, \si{\nano\coulomb}}، \عددی{\SI{18.3}{\pico\coulomb}}
\انتہا{سوال}
%==============================
\ابتدا{سوال}
رداس \عددی{\rho=2}، \عددی{\rho=4} اور \عددی{\rho=5} پر بالترتیب سطحی کثافت چارج \عددی{\SI{-3}{\nano\coulomb\per\meter\squared}}، \عددی{\SI{1.5}{\nano\coulomb\per\meter\squared}} اور \عددی{\SI{0.25}{\nano\coulomb\per\meter\squared}} پائی جاتی ہے۔ \عددی{z=3} تا \عددی{z=6}  پر رداس \عددی{\rho=4.5} نلکی سطح سے کل کتنی برقی بہاو ہوتی ہے۔\عددی{z=3} تا \عددی{z=6}  پر رداس \عددی{\rho=6} نلکی سطح سے کل کتنی برقی بہاو ہوتی ہے۔نقطہ \عددی{(6,8,2)} پر \عددی{\kvec{D}} حاصل کریں۔

\عددی{\SI{0}{\coulomb}}،\عددی{\SI{28.27}{\nano\coulomb}}،\عددی{\kvec{D}=0.09\ax+0.15\ay \, \si{\nano\coulomb\per\meter\squared}}
\انتہا{سوال}
%============================
\ابتدا{سوال}
بند خطہ \عددی{0\le x \le 2, \, 0\le y \le 2, \, 0 \le z \le 2} میں \عددی{\kvec{D}=xy^2\ax+xyz\ay+z(x+y)\az \,\si{\micro\coulomb\per\meter\squared}} ہے۔اس خطے سے کل برقی بہاو کتنی ہے۔

\عددی{\SI{28}{\micro\coulomb}}
\انتہا{سوال}
%==========================
\ابتدا{سوال}
محدد \عددی{z} پر لکیری کثافت چارج \عددی{\SI{50}{\nano\coulomb\per\meter}} پایا جاتا ہے۔محدد کے مرکز پر رداس \عددی{r=\SI{5}{\meter}} کی کرہ سے خارج کل برقی بہاو حاصل کریں۔اگر کرہ کے مرکز کو نقطہ \عددی{(0,2,2)} منتقل کیا جائے تب جواب کیا ہو گا۔

جوابات:\عددی{\SI{500}{\nano\coulomb}}، \عددی{\SI{458}{\nano\coulomb}}
\انتہا{سوال}
%=========================
\ابتدا{سوال}
رداس \عددی{r=\SI{1.1}{\meter}} کی کرہ کے اندر حجمی کثافت چارج \عددی{\rho_h=30e^{-r^3} \,\si{\nano\coulomb\per\meter^3}} پائی جاتی ہے۔کرہ کے اندر کل چارج حاصل کریں۔گاؤس کے قانون سے کرہ کی سطح پر برقی بہاو کی کثافت حاصل کریں۔

جوابات:\عددی{\SI{92.46}{\nano\coulomb}}، \عددی{\SI{6.08}{\nano\coulomb\per\meter\squared}}
\انتہا{سوال}
%=============================
\ابتدا{سوال}
نلکی محدد میں کثافت برقی بہاو \عددی{\kvec{D}=\tfrac{\rho\arho+z\az}{4\pi(\rho^2+z^2)^{3/2}}} دیا گیا ہے۔لامحدود لمبائی کی نلکی جس کا رداس \عددی{\rho=5} ہے سے کل کتنی برقی بہاو خارج ہو گی۔

جواب:\عددی{\SI{1}{\coulomb}}
\انتہا{سوال}
%============================
\ابتدا{سوال}
مرکز پر رداس \عددی{5}، \عددی{9} اور \عددی{14} کے کرہ پر بالترتیب سطحی کثافت چارج \عددی{\SI{20}{\micro\coulomb\per\meter\squared}}، 
\عددی{\SI{-8}{\micro\coulomb\per\meter\squared}} اور \عددی{\rho_S \, \si{\coulomb\per\meter\squared}} پائے جاتے ہیں۔نقطہ \عددی{(20,0,0)} پر صفر \عددی{\kvec{D}} حاصل کرنے کے لئے \عددی{\rho_S} دریافت کریں۔تمام خطوں میں \عددیء{D} کی مساوات حاصل کریں۔  

جوابات:\عددی{\SI{0.7551}{\micro\coulomb\per\meter\squared}}، \عددی{r<5} پر \عددی{D_r=0} ہے، \عددی{5< r < 9} پر \عددی{D_r=\tfrac{500}{r^2}\, \si{\micro\coulomb\per\meter\squared}} ہے،\عددی{9<r<14} پر \عددی{D_r=-\tfrac{148}{r^2}\,\si{\coulomb\per\meter\squared}} ہے جبکہ \عددی{r>14} پر \عددی{D_r=0} ہے۔
\انتہا{سوال}
%===========================
\ابتدا{سوال}
لامحدود سطح \عددی{z=4} پر \عددی{\rho_S=\SI{2}{\nano\coulomb\per\meter\squared}} سطحی کثافت پائی جاتی ہے۔محدد کے مرکز پر \عددی{r=5} رداس کا کرہ رکھا جاتا ہے۔کرہ کتنے چارج کو گھیرے گا۔کرے سے کتنی برقی بہاو خارج ہو گی۔

جوابات:\عددی{\SI{56.549}{\nano\coulomb}}، \عددی{\SI{56.549}{\nano\coulomb}}
\انتہا{سوال}
%==========================
\ابتدا{سوال}
محدد کے مرکز پر \عددی{r=5} رداس کا کرہ جبکہ \عددی{z=4} پر لامحدود سطح پائی جاتی ہے۔لامحدود سطح کے بالائی جانب کرہ کے اندر  حجمی کثافت
 چارج \عددی{\rho_h=\SI{25}{\nano\coulomb\per \meter^3 }} پائی جاتی ہے۔کرہ سے کل خارج برقی بہاو حاصل کریں۔

جواب:\عددی{\SI{1.1812}{\micro\coulomb}} 
\انتہا{سوال}
%===========================
\ابتدا{سوال}
خطہ \عددی{\rho < \SI{3}{\milli\meter}} میں حجمی کثافت چارج \عددی{\rho_h=\tfrac{\rho^2}{1000}\,\si{\coulomb\per\meter^3}} جبکہ
 خطہ \عددی{\SI{3}{\milli\meter} < \rho < \SI{5}{\milli\meter}} میں \عددی{\rho_h=\SI{2}{\micro\coulomb\per\meter^3}} پائی جاتی ہے۔موزوں گاوسی سطحیں چنتے ہوئے رداس \عددی{\rho=0}، \عددی{\rho={\SI{2}{\milli\meter}}}، \عددی{\rho=\SI{4}{\milli\meter}} اور \عددی{\SI{6}{\milli\meter}} پر \عددی{D_{\rho}} حاصل کریں۔

جوابات:\عددی{\SI{0}{\coulomb\per\meter\squared}}، \عددی{\SI{2}{\pico\coulomb\per\meter\squared}}، \عددی{\SI{1.756}{\nano\coulomb\per\meter\squared}}، \عددی{\SI{2.67}{\nano\coulomb\per\meter\squared}}
\انتہا{سوال}
%==============================
\ابتدا{سوال}
خطہ \عددی{r<\SI{3}{\milli\meter}} میں \عددی{\rho_h=\SI{22}{\micro\coulomb\per\meter^3}} جبکہ \عددی{\SI{5}{\milli\meter}<r<\SI{7}{\milli\meter}} خطے میں \عددی{\rho_h=\tfrac{55}{r}\,\si{\nano\coulomb\per\meter^3}} حجمی کثافت چارج پایا جاتا ہے۔ موزوں گاوسی سطحیں چنتے
 ہوئے  \عددی{r=\SI{5}{\milli\meter}} اور \عددی{r=\SI{10}{\milli\meter}} پر \عددی{D_r} دریافت کریں۔

جوابات:\عددی{\SI{22}{\nano\coulomb\per\meter\squared}}، \عددی{\SI{8.58}{\nano\coulomb\per\meter\squared}}
\انتہا{سوال}
%===========================
\ابتدا{سوال}
تفاعل \عددی{\kvec{D}=2x^2\ax+(x+z)\ay+z\az} مکعب \عددی{0 < x,y,z, <a} میں پایا جاتا ہے۔تفاعل کی پھیلاو \عددی{\nabla \cdot \kvec{D}} حاصل کریں۔مکعب کے تمام سطحوں پر تفاعل کے سطحی تکمل کا مجموعہ حاصل کرتے ہوئے مکعب میں کل چارج حاصل کریں۔یہی جواب مسئلہ پھیلاو  کی مدد سے حاصل کریں۔

جوابات:\عددی{\nabla \cdot \kvec{D}=4x+1}، \عددی{2a^4+a^3}
\انتہا{سوال}
%==============================
\ابتدا{سوال}
مکعب \عددی{2<x,y,z<5} میں \عددی{\kvec{G}=\frac{5x^2 y}{z}\ay} ہے۔مسئلہ پھیلاو کے دونوں اطراف کو مکعب کے لئے حل کریں۔

جواب:\عددی{536.03}
\انتہا{سوال}
%===============================
\ابتدا{سوال}
مندرجہ ذیل تفاعل کے پھیلاو حاصل کرتے ہوئے پھیلاو کی قیمت نقطہ \عددی{N(3,4,6)} پر حاصل کریں۔
\begin{align*}
\kvec{D}&=10(xy-\frac{y}{\sqrt{z}})\ax+y^2(x+2)\ay-(6z^2+3x^2y)\az\\
\kvec{D}&=8\rho \sin \phi \arho+4\rho \cos \phi \aphi+z^2\az\\
\kvec{D}&=2 r \sin \theta \cos \phi\ar +r \cos \theta \cos \phi \atheta+r \cos \phi \aphi
\end{align*}

جوابات:\عددی{10y+2y(x+2)-12z}، \عددی{12 \sin \phi}، \عددی{{6 \sin \theta \sin \phi+\frac{\cos 2\theta \sin \phi}{\sin \theta}-\frac{\sin \phi}{\sin \theta}}}، \عددی{80}، \عددی{9.6}، \عددی{2.0486}
\انتہا{سوال}
%===============================
\ابتدا{سوال}
مندرجہ ذیل تفاعل کی پھیلاو نقطہ \عددی{N(3,5-2)} پر حاصل کریں۔
\begin{align*}
\kvec{D}&=(x+yz)(3x\ax-5z\ay+2y^2z\az)\\
\kvec{D}&=\frac{x\ax+y\ay+z\az}{\sqrt{x^2+y^2+z^2}}\\
\kvec{D}&=0.24\ax-0.55\ay+0.12\az\\
\kvec{D}&=x^2yz^3(2\ax-3\ay+\az)
\end{align*}

جوابات:\عددی{-882}، \عددی{0.324}، \عددی{0}، \عددی{276}
\انتہا{سوال}
%===========================
\ابتدا{سوال}
مندرجہ ذیل تفاعل کی پھیلاو نقطہ \عددی{N(3,45^{\circ},30^{\circ})} پر حاصل کریں۔
\begin{align*}
\kvec{D}&=(2r\sin\theta\cos\phi+\cos\theta)\ar+(r\cos\theta\cos\phi-\sin\theta)\atheta-r\sin\phi\aphi\\
\kvec{D}&=\sin^2\theta\sin\phi\ar+\sin 2\theta \sin \phi\atheta+\sin\theta\cos\phi\aphi\\
\kvec{D}&=0.2\ar-0.15\atheta+0.23\aphi\\
\kvec{D}&=0.2r^3 \phi \sin^2 \theta (\ar+\atheta+\aphi)
\end{align*}

جوابات:\عددی{4.899}، \عددی{0.1667}، \عددی{0}، \عددی{5.043}
\انتہا{سوال}
%===================

