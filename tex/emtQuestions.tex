\باب{سوالات}
\section*{گاؤس}
%====================
\ابتدا{سوال}
محدد کے مرکز پر \عددی{\SI{20}{\nano\coulomb}} چارج پایا جاتا ہے۔اس کے علاوہ \عددی{z=0} سطح پر \عددیء{\SI{5}{\nano\coulomb\per\meter}} کے لکیری چارج \عددی{y=-1} اور \عددی{y=-3} پر پائے جاتے ہیں۔نقطہ \عددی{(0,-2,0)} پر \عددی{\kvec{D}} حاصل کریں۔\عددی{(0,1,0)} پر رداس \عددی{r=1.5} کے کرہ کی سطح پر کل برقی بہاو  حاصل کریں۔ 

جوابات:\عددی{-\tfrac{5}{4\pi}\ay \,\si{\coulomb\per\meter\squared}}، \عددی{\SI{20}{\nano\coulomb}}
\انتہا{سوال}
%==============================
\ابتدا{سوال}
رداس \عددی{\rho=\SI{10}{\centi\meter}} کے نلکی سطح کے \عددی{z>0} حصے  پر سطحی کثافت چارج \عددی{\rho_S=2ze^{-z^2} \, \si{\nano\coulomb\per\meter\squared}} پائی جاتی ہے۔سطح پر کل چارج دریافت کریں۔اس سطح سے \عددی{z=1} تا \عددی{z=2} زاویہ \عددی{\phi=45^{\circ}} تا \عددی{\phi=75^{\circ}} کتنی برقی بہاو خارج ہوتی ہے۔

جوابات:\عددی{0.2\pi\, \si{\nano\coulomb}}، \عددی{\SI{18.3}{\pico\coulomb}}
\انتہا{سوال}
%==============================
\ابتدا{سوال}
رداس \عددی{\rho=2}، \عددی{\rho=4} اور \عددی{\rho=5} پر بالترتیب سطحی کثافت چارج \عددی{\SI{-3}{\nano\coulomb\per\meter\squared}}، \عددی{\SI{1.5}{\nano\coulomb\per\meter\squared}} اور \عددی{\SI{0.25}{\nano\coulomb\per\meter\squared}} پائی جاتی ہے۔ \عددی{z=3} تا \عددی{z=6}  پر رداس \عددی{\rho=4.5} نلکی سطح سے کل کتنی برقی بہاو ہوتی ہے۔\عددی{z=3} تا \عددی{z=6}  پر رداس \عددی{\rho=6} نلکی سطح سے کل کتنی برقی بہاو ہوتی ہے۔نقطہ \عددی{(6,8,2)} پر \عددی{\kvec{D}} حاصل کریں۔

\عددی{\SI{0}{\coulomb}}،\عددی{\SI{28.27}{\nano\coulomb}}،\عددی{\kvec{D}=0.09\ax+0.15\ay \, \si{\nano\coulomb\per\meter\squared}}
\انتہا{سوال}
%============================
\ابتدا{سوال}
بند خطہ \عددی{0\le x \le 2, \, 0\le y \le 2, \, 0 \le z \le 2} میں \عددی{\kvec{D}=xy^2\ax+xyz\ay+z(x+y)\az \,\si{\micro\coulomb\per\meter\squared}} ہے۔اس خطے سے کل برقی بہاو کتنی ہے۔

\عددی{\SI{28}{\micro\coulomb}}
\انتہا{سوال}
%==========================
\ابتدا{سوال}
محدد \عددی{z} پر لکیری کثافت چارج \عددی{\SI{50}{\nano\coulomb\per\meter}} پایا جاتا ہے۔محدد کے مرکز پر رداس \عددی{r=\SI{5}{\meter}} کی کرہ سے خارج کل برقی بہاو حاصل کریں۔اگر کرہ کے مرکز کو نقطہ \عددی{(0,2,2)} منتقل کیا جائے تب جواب کیا ہو گا۔

جوابات:\عددی{\SI{500}{\nano\coulomb}}، \عددی{\SI{458}{\nano\coulomb}}
\انتہا{سوال}
%=========================
\ابتدا{سوال}
رداس \عددی{r=\SI{1.1}{\meter}} کی کرہ کے اندر حجمی کثافت چارج \عددی{\rho_h=30e^{-r^3} \,\si{\nano\coulomb\per\meter^3}} پائی جاتی ہے۔کرہ کے اندر کل چارج حاصل کریں۔گاؤس کے قانون سے کرہ کی سطح پر برقی بہاو کی کثافت حاصل کریں۔

جوابات:\عددی{\SI{92.46}{\nano\coulomb}}، \عددی{\SI{6.08}{\nano\coulomb\per\meter\squared}}
\انتہا{سوال}
%=============================
\ابتدا{سوال}
نلکی محدد میں کثافت برقی بہاو \عددی{\kvec{D}=\tfrac{\rho\arho+z\az}{4\pi(\rho^2+z^2)^{3/2}}} دیا گیا ہے۔لامحدود لمبائی کی نلکی جس کا رداس \عددی{\rho=5} ہے سے کل کتنی برقی بہاو خارج ہو گی۔

جواب:\عددی{\SI{1}{\coulomb}}
\انتہا{سوال}
%============================
\ابتدا{سوال}
مرکز پر رداس \عددی{5}، \عددی{9} اور \عددی{14} کے کرہ پر بالترتیب سطحی کثافت چارج \عددی{\SI{20}{\micro\coulomb\per\meter\squared}}، 
\عددی{\SI{-8}{\micro\coulomb\per\meter\squared}} اور \عددی{\rho_S \, \si{\coulomb\per\meter\squared}} پائے جاتے ہیں۔نقطہ \عددی{(20,0,0)} پر صفر \عددی{\kvec{D}} حاصل کرنے کے لئے \عددی{\rho_S} دریافت کریں۔تمام خطوں میں \عددیء{D} کی مساوات حاصل کریں۔  

جوابات:\عددی{\SI{0.7551}{\micro\coulomb\per\meter\squared}}، \عددی{r<5} پر \عددی{D_r=0} ہے، \عددی{5< r < 9} پر \عددی{D_r=\tfrac{500}{r^2}\, \si{\micro\coulomb\per\meter\squared}} ہے،\عددی{9<r<14} پر \عددی{D_r=-\tfrac{148}{r^2}\,\si{\coulomb\per\meter\squared}} ہے جبکہ \عددی{r>14} پر \عددی{D_r=0} ہے۔
\انتہا{سوال}
%===========================
\ابتدا{سوال}
لامحدود سطح \عددی{z=4} پر \عددی{\rho_S=\SI{2}{\nano\coulomb\per\meter\squared}} سطحی کثافت پائی جاتی ہے۔محدد کے مرکز پر \عددی{r=5} رداس کا کرہ رکھا جاتا ہے۔کرہ کتنے چارج کو گھیرے گا۔کرے سے کتنی برقی بہاو خارج ہو گی۔

جوابات:\عددی{\SI{56.549}{\nano\coulomb}}، \عددی{\SI{56.549}{\nano\coulomb}}
\انتہا{سوال}
%==========================
\ابتدا{سوال}
محدد کے مرکز پر \عددی{r=5} رداس کا کرہ جبکہ \عددی{z=4} پر لامحدود سطح پائی جاتی ہے۔لامحدود سطح کے بالائی جانب کرہ کے اندر  حجمی کثافت
 چارج \عددی{\rho_h=\SI{25}{\nano\coulomb\per \meter^3 }} پائی جاتی ہے۔کرہ سے کل خارج برقی بہاو حاصل کریں۔

جواب:\عددی{\SI{1.1812}{\micro\coulomb}} 
\انتہا{سوال}
%===========================
