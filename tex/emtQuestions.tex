\باب{سوالات}
%=====================
\ابتدا{سوال}
میدان \عددی{\kvec{E}=1.5\az \, \si{\volt\per\meter}} میں الیکٹران  حرکت کرتا ہے۔لمحہ \عددی{t=0} پر الیکٹران نقطہ \عددی{(0,0,0)} پر پایا جاتا ہے جبکہ اس کی سمتی رفتار \عددی{\kvec{v}=3\times 10^{5}\ax \, \si{\meter\per\second}} ہے۔الیکٹران کا چارج \عددی{\SI{-1.6e-19}{\coulomb}} اور اس کی کمیت \عددی{\SI{3.1e-31}{\kilo\gram}} ہے۔نیوٹن کے قوانین حرکت سے تفرقی مساوات لکھ کر اسے حل کرتے ہوئے لمحہ \عددی{t=\SI{150}{\nano\second}} پر الیکٹران کی سمتی رفتار، مقام اور حرکی توانائی دریافت کریں۔

جوابات:\عددی{\kvec{v}=\num{300000}\ax-\num{116129}\az \, \si{\meter\per\second}}، \عددی{(0.045,0,-3.48)}، \عددی{\SI{1.63e-20}{\joule}}
\انتہا{سوال}
%=====================
\ابتدا{سوال}
مقناطیسی میدان \عددی{\kvec{B}=0.3\ax-0.2\ay-0.4\az \, \si{\tesla}} میں لمحہ \عددی{t=0} پر الیکٹران کی سمتی رفتار \عددی{\kvec{v}=10^6 \az \si{\meter\per\second}} ہے۔الیکٹران پر قوت دریافت کریں۔ایسا برقی میدان حاصل کریں جس کی موجودگی میں مقناطیسی اور برقی میدان مل کر اس الیکٹران پر صفر قوت  پیدا کرتے ہیں۔

جواب:\عددی{\kvec{F}=-32\ax-48\ay \, \si{\femto\newton}}، \عددی{\kvec{E}=-200\ax-300\ay \, \si{\volt\per\meter}}
\انتہا{سوال}
%====================
\ابتدا{سوال}
میدان \عددی{\kvec{B}=2\ax-1\ay+3\az \, \si{\tesla}} اور \عددی{\kvec{E}=3\ax+2\ay-1\az \, \si{\volt\per\meter}} میں چارج \عددی{\SI{1.2}{\micro\coulomb}} حرکت کر رہا ہے۔لمحہ \عددی{t=0} پر اس کی رفتار \عددی{\kvec{v}=10\ax-30\ay+20\az \, \si{\kilo\meter\per\second}} ہے۔یہ چارج \عددی{\SI{5}{\micro\gram}} کے کمیت پر پایا جاتا ہے۔لمحہ \عددی{t=0} پر چارج کی اسراع حاصل کریں۔

جواب:\عددی{\kvec{a}=-16.8\ax+2.4\ay+12\az \, \si{\mega\meter\per\second\squared}}
\انتہا{سوال}
%======================
\ابتدا{سوال}
محدد \عددی{z} پر پڑی لامحدود لمبائی کے تار میں \عددی{5\az \, \si{\ampere}} برقی رو گزر رہی ہے۔اس کے قریب سطح \عددی{x=0} پر موصل تار \عددی{N_1(0,1,0)}، \عددی{N_2(0,4,0)}، \عددی{N_3(0,4,2)} اور \عددی{N_4(0,1,2)} نقطوں کو جوڑ کر مستطیل بناتی ہے جس میں \عددی{N_1} سے \عددی{N_2} جانب \عددی{\SI{2}{\ampere}} برقی رو چکر  لگا رہی ہے۔چکور کے چاروں اطراف پر قوت دریافت کرتے ہوئے پورے چکور پر قوت حاصل کریں۔

جوابات:تار \عددی{N_1(0,1,0)} تا \عددی{N_2(0,4,0)} پر قوت \عددی{2.77\az \, \si{\micro\newton}} ہے۔گھڑی کے الٹ سمت چلتے ہوئے بقایا قوت  \عددی{-1\ay\, \si{\micro\newton}}، \عددی{-2.77\az \, \si{\micro\newton}} اور \عددی{4\ay\, \si{\micro\newton}} ہیں۔یوں مستطیل پر کل قوت \عددی{3\ay\, \si{\micro\newton}} ہے۔
\انتہا{سوال}
%======================
\ابتدا{سوال}
محدد \عددی{z} پر پڑی لامحدود لمبائی کے تار میں \عددی{10\az \, \si{\ampere}} برقی رو گزر رہی ہے۔اس کے قریب نقطہ \عددی{N_1(2,1,3)} سے \عددی{N_2(5,4,7)} تک سیدھی موصل تار میں \عددی{N_1} سے \عددی{N_2} جانب \عددی{\SI{4}{\ampere}} برقی رو گزر رہی ہے۔چھوٹی تار پر قوت حاصل کریں۔

جواب:\عددی{\kvec{F}=-6.74\ax-4.49\ay+8.42\az \, \si{\micro\newton}}
\انتہا{سوال}
%========================
\ابتدا{سوال}
سطح \عددی{x=0} پر مقناطیسی میدان کا \عددی{z} جزو \عددی{B_z=\tfrac{200}{z^2+1} \, \si{\micro \tesla}} پایا جاتا ہے۔اس مقناطیسی جزو سے خطہ \عددی{1<y<3}، \عددی{-\infty <z< \infty} میں کثافت \عددی{\kvec{K}=0.2\ay \si{\ampere\per\meter}} پر قوت حاصل کریں۔

جواب:\عددی{251\ax \, \si{\micro\newton}}
\انتہا{سوال}
%=======================
\ابتدا{سوال}
\عددی{z} محدد پر پڑی لامحدود لمبائی کے تار میں \عددی{\SI{2.2}{\ampere}} برقی رو پائی جاتی ہے۔سطح \عددی{y=0} پر خطہ \عددی{\SI{1}{\milli\meter} < x < \SI{5}{\milli\meter}} پر \عددی{\az} سمت میں کل \عددی{\SI{8}{\ampere}} برقی رو گزر رہی ہے۔اس خطے کی فی میٹر لمبائی پر مقناطیسی قوت حاصل کریں۔محدد \عددی{z} پر پڑی تار پر بھی فی میٹر قوت حاصل کریں۔

جواب:\عددی{-1.4\ax \, \si{\milli\newton}}، \عددی{1.4\ax \, \si{\milli\newton}}
\انتہا{سوال}
%=======================
\ابتدا{سوال}
محدد \عددی{z} پر پڑی لامحدود لمبائی کی تار میں \عددی{I_1} برقی رو \عددی{\az} جانب گزر رہی ہے۔اس کے قریب سطح \عددی{z=0} پر تار \عددی{y=a}، \عددی{-b<x<b} میں \عددی{I_2} برقی رو \عددی{\ax} سمت میں گزر رہی ہے۔نقطہ \عددی{(0,0,0)} کو محور لیتے ہوئے چھوٹی تار پر مروڑ حاصل کریں۔صفحہ \حوالہصفحہ{شکل_امالہ_مروڑ_لمبی_اور_چھوٹی_تار} پر شکل \حوالہ{شکل_امالہ_مروڑ_لمبی_اور_چھوٹی_تار} میں صورت حال دکھایا گیا ہے۔

جواب:\عددی{-\frac{I_1 I_2 \mu_0}{\pi} \left(b-a \tan^{-1} \frac{b}{a}\right)\ay \,\si{\newton \meter}}
\انتہا{سوال}
%=====================
\ابتدا{سوال}\شناخت{سوال_امالہ_مستطیل}
موصل تار نقطہ \عددی{N_1(2,0,0)}، \عددی{N_2(5,0,0)}، \عددی{N_3(5,0,4)} اور \عددی{N_4(2,0,4)} کو جوڑ کر مستطیل بناتی ہے۔مثبت \عددی{y} محدد کی جانب سے دیکھتے ہوئے، اس مستطیل میں \عددی{\SI{6}{\ampere}} برقی رو سمت گھڑی گردش کر رہی ہے۔الف)  یکساں میدان \عددی{\kvec{B}=5\ax \, \si{\tesla}} کی صورت میں \عددی{z} محدد  کو محور لیتے ہوئے مستطیل کے چاروں اطراف پر علیحدہ علیحدہ  مروڑ حاصل کرتے ہوئے کل مروڑ حاصل کریں۔ ب) سطح \عددی{y=0} پر لکیر \عددی{x=3} کو محور لیتے ہوئے اسی یکساں میدان میں دوبارہ مروڑ حاصل کریں۔

جوابات:(الف) اور (ب):مستطیل کے چار حصوں پر مروڑ \عددی{0}، \عددی{600\az \, \si{\newton \meter}}، \عددی{0} اور \عددی{-240\az\, \si{\newton\meter}} ہے جس سے  کل مروڑ \عددی{360\az \, \si{\newton\meter}} حاصل ہوتا ہے۔
\انتہا{سوال}
%======================
\ابتدا{سوال}\شناخت{سوال_امالہ_مستطیل_بذریعہ_جفت_قطب}
سوال \حوالہ{سوال_امالہ_مستطیل} میں میدان یکساں ہے لہٰذا اس میں محور کا مروڑ پر کوئی اثر نہیں ہوتا۔ایسی صورت میں مروڑ صفحہ \حوالہصفحہ{مساوات_امالہ_یکساں_میدان_مروڑ_بذریعہ_مقناطیسی_جفت_قطب} پر دئے  مساوات \حوالہ{مساوات_امالہ_یکساں_میدان_مروڑ_بذریعہ_مقناطیسی_جفت_قطب} کی مدد سے حاصل کی جا سکتی ہے۔ایسا ہی کریں۔

جواب:\عددی{360\az \, \si{\newton\meter}}
\انتہا{سوال}
%================================
\ابتدا{سوال}
سوال \حوالہ{سوال_امالہ_مستطیل} میں یکساں میدان کی جگہ اگر \عددی{z} محدد پر لامحدود لمبائی کے تار میں \عددی{\az} جانب \عددی{\SI{25}{\ampere}} برقی رو میدان پیدا کرے تب محدد کے مرکز \عددی{(0,0,0)} کو محور لیتے ہوئے مروڑ حاصل کریں۔یاد رہے کہ یہ میدان غیر یکساں ہے لہٰذا مساوات  \حوالہ{مساوات_امالہ_یکساں_میدان_مروڑ_بذریعہ_مقناطیسی_جفت_قطب} قابل استعمال نہیں ہے۔

جواب:مستطیل کے چار حصوں پر مروڑ \عددی{-90\ay \, \si{\micro \newton \meter}}، \عددی{-48\ay \, \si{\micro \newton \meter}}، \عددی{90\ay \, \si{\micro \newton \meter}} اور \عددی{120\ay \, \si{\micro \newton \meter}}  ہے جس سے کل مروڑ  \عددی{72\ay \, \si{\micro\newton\meter}} حاصل ہوتا ہے۔
\انتہا{سوال}
%=============================
\ابتدا{سوال}
دو سنٹی میٹر رداس اور پانچ سو چکر کے پیچ دار لچھے میں \عددی{\SI{3}{\ampere}} کی برقی رو گزر رہی ہے۔یہ لچھا \عددی{\SI{1.5}{\tesla}} کے میدان میں پایا جاتا ہے۔میدان اور لچھے کے محور آپس میں عمودی ہیں۔ لچھے پر مروڑ حاصل کریں۔

جواب:\عددی{\SI{2.83}{\newton \meter}}
\انتہا{سوال}
%========================
\ابتدا{سوال}
ایک مادہ  میدان \عددی{\kvec{B}=0.15z\ay \, \si{\tesla}} میں پایا جاتا ہے۔اس مادے کی \عددی{\chi=2.5} ہے۔آپ سے گزارش ہے کہ \عددی{\mu_R}، \عددی{\kvec{H}}، \عددی{\kvec{M}}، \عددی{\kvec{J}} ،\عددی{\kvec{J}_m} اور \عددی{\kvec{J}_T} حاصل کریں۔

جوابات: \عددی{\mu_R=3.5}، \عددی{\kvec{H}=34.1z\ay \, \si{\kilo\ampere\per\meter}}، \عددی{\kvec{M}=85.3 z \ay \, \si{\kilo\ampere\per\meter}}، \عددی{\kvec{J}=-34.1 \ax \, \si{\kilo\ampere\per\meter\squared}}، \عددی{\kvec{J}_m=-85.3 \ax \, \si{\kilo\ampere\per\meter\squared}} اور \عددی{\kvec{J}_T=-119 \ax \, \si{\kilo\ampere\per\meter\squared}}
\انتہا{سوال}
%========================
\ابتدا{سوال}
مندرجہ ذیل مادوں میں \عددی{\kvec{H}} حاصل کریں۔ الف) جزوی مقناطیسی مستقل \عددی{\mu_R=2.2}، ایٹم کی تعددی کثافت \عددی{\num{1.5e29}} ایٹم فی مکعب میٹر جبکہ ہر ایٹم کا مقناطیسی جفت قطب \عددی{1.9\times 10^{-30} \ax \, \si{\ampere\per\meter \squared}} ہے۔ ب)مادہ میں \عددی{\kvec{M}=160\az \, \si{\ampere\per\meter}} اور  اس کا مقناطیسی مستقل \عددیء{\mu=\SI{2.25}{\micro\henry\per\meter}} ہے۔ پ) مادے کا \عددی{\chi_m=0.65} ہے جبکہ \عددی{\kvec{B}=1.7\ay \, \si{\tesla}} ہے۔ ت) مساوات \عددی{\oint \kvec{M} \cdot \dif \kvec{L}=I_m} کا استعمال کرتے ہوئے  ایسے خطے میں \عددی{\kvec{M}} حاصل کریں جس میں نلکی سطح \عددی{\rho=\SI{0.5}{\meter}} پر \عددی{5\az \, \si{\ampere\per\meter}} اور نلکی سطح \عددی{\rho=\SI{2.5}{\meter}} پر \عددی{-1\az \, \si{\ampere\per\meter}} کثافت برقی رو پائی جاتی ہو۔

جوابات:\عددی{0.24\ax \, \si{\ampere\per\meter}}، \عددی{202\az \, \si{\ampere\per\meter}}، \عددی{820\ay \, \si{\kilo\ampere\per\meter}}، \عددی{\rho<\SI{0.5}{\meter}} اور \عددی{\rho>\SI{1}{\meter}} خطوں میں \عددی{\kvec{M}=0} ہے جبکہ \عددی{0.5<\rho<2.5} میں \عددی{\kvec{M}=\tfrac{2.5}{\rho}\aphi \, \si{\ampere\per\meter}} ہو گا۔
\انتہا{سوال}
%=======================
\ابتدا{سوال}
مندرجہ ذیل صورتوں میں مقناطیسیت \عددی{M} کی قیمت  حاصل کریں۔ الف) میدان \عددی{B=\SI{0.015}{\tesla}} اور \عددی{\chi_m=0.002} ہیں۔ ب) مقناطیسی 
شدت \عددی{H=\SI{1600}{\ampere\per\meter}} جبکہ مقناطیسی جزوی مستقل \عددی{\mu_R=1.004} ہے۔ پ) ایٹم کی تعدادی کثافت \عددی{6.5\times 10^{28}} ایٹم فی مکعب میٹر ہے جبکہ ایک ایٹم کی مقناطیسی جفت قطب \عددی{3\times 10^{-30}} ہے۔ تمام جفت قطب ایک ہی سمت میں ہیں۔

جوابات:\عددی{M=\SI{23.8}{\ampere\per\meter}}، \عددی{\SI{6.4}{\ampere\per\meter}}، \عددی{\SI{0.195}{\ampere\per\meter}}
\انتہا{سوال}
%==========================
\ابتدا{سوال}
خطہ-1 کو مساوات \عددی{2x^2+3y-4xz<3} ظاہر کرتی ہے جبکہ اس کی دوسری جانب خطہ-2 پایا جاتا ہے۔ان کے جزوی مقناطیسی مستقل \عددی{\mu_{R1}=1} اور \عددی{\mu_{R2}=2.2} ہیں۔نقطہ \عددی{N(2,1,1)} پر پہلے خطے سے دوسرے خطے کی جانب اکائی سمتیہ حاصل کریں۔اس نقطے پر پہلے خطے میں میدان
 \عددی{\kvec{H}=15\ax-5\ay-10\az} ہے۔دونوں خطوں میں اس نقطے پر میدان کے عمودی اور متوازی اجزاء حاصل کریں۔سرحد کے عمود کے ساتھ دونوں خطوں میں میدان کا زاویہ حاصل کریں۔

جوابات:\عددی{\aN=0.42\ax+0.32\ay-0.85\az}، \عددی{\kvec{H}_{n1}=5.6\ax+4.2\ay-11.2\az}، \\ 
 \عددی{\kvec{H}_{m1}=9.4\ax-9.2\ay+1.2\az}، \عددی{\kvec{H}_{m2}=9.4\ax-9.2\ay+1.2\az}، \\
 \عددی{\kvec{H}_{n2}=2.6\ax+1.9\ay-5.1\az}، \عددی{\kvec{H}_2=11.9\ax-7.3\ay-3.9\az}، \\
\عددی{\theta_1=44.9^{\circ}}، \عددی{\theta_2=65.5^{\circ}} 
\انتہا{سوال}
%==========================
\ابتدا{سوال}
\عددی{z<0} کو خطہ-الف، \عددی{0<z<2} کو خطہ-ب، \عددی{2<z<3} کو خطہ-پ جبکہ \عددی{3<z} کو خطہ-ت تصور کریں۔خطہ-الف اور خطہ-ت خالی خلاء ہیں۔خطہ-ب کا  \عددی{\mu_{R}=2.5} جبکہ خطہ-پ کا \عددی{\mu_R=1.5} ہے۔خطہ-الف میں میدان \عددی{{\kvec{H}_1=3\ax-2\ay+5\az}} پایا جاتا ہے۔خطہ-الف، ب، پ اور ت میں میدان اور \عددی{z} محدد کے مابین زاویے حاصل کریں۔

جوابات:\عددی{35.8^{\circ}}، \عددی{61^{\circ}}، \عددی{47.2^{\circ}}، \عددی{35.8^{\circ}}
\انتہا{سوال}
%=========================
\ابتدا{سوال}
ایک لمبے پیچ دار لچھے کا رداس \عددی{\SI{5}{\centi\meter}} اور فی میٹر چکر \عددی{4000} ہیں۔لچھے میں \عددی{\SI{100}{\milli\ampere}} برقی رو گزر رہی ہے۔خطہ \عددی{\rho<a} کا \عددی{\mu_R=2.5} ہے جبکہ بقایا خطے  کا \عددی{\mu_R=4.5} ہے۔الف) لچھے میں کل مقناطیسی بہاو \عددی{\SI{10}{\micro\weber}} ہونے کی صورت میں \عددی{a} کی قیمت حاصل کریں۔ ب) دونوں خطوں میں برابر مقناطیسی بہاو کی صورت میں \عددی{a} کی قیمت اور کل بہاو حاصل کریں۔ 

جوابات:\عددی{\SI{4.96}{\centi\meter}}، \عددی{\SI{4}{\centi\meter}}، \عددی{\SI{12.7}{\micro\weber}}
\انتہا{سوال}
%=======================
\ابتدا{سوال}
ایک مقناطیسی دور اندرسے کی شکل کا ہے۔اندرسے کا رقبہ عمودی تراش مستطیل ہے۔
\انتہا{سوال}
%======================
