\باب{سوالات}
\section*{کولومب}
%====================
\ابتدا{سوال}
تکون کے تینوں کونوں پر \عددی{\SI{25}{\micro\coulomb}} کا چارج پایا جاتا ہے جبکہ تینوں کونوں سے \عددیء{\SI{15}{\centi\meter}} فاصلے پر \عددی{\SI{20}{\micro\coulomb}} چارج پایا جاتا ہے۔تکون کے اطراف \عددی{\SI{10}{\centi\meter}} ہونے کی صورت میں چوتھے چارج پر قوت دفع  کی مقدار حاصل کریں۔

 جواب:\عددی{\SI{0.553}{\newton}}
\انتہا{سوال}
%=====================
\ابتدا{سوال}
\عددی{z=0} پر \عددی{\SI{4}{\nano\farad}} اور \عددی{z=\SI{1}{\centi\meter}} پر \عددی{\SI{-3}{\nano\farad}} چارج پائے جاتے ہیں۔\عددی{z} محدد پر وہ نقطے دریافت کریں جہاں مثبت چارج پر صفر قوت پائی جائے گی۔

جوابات: \عددی{z=\SI{0.92}{\centi\meter}}، \عددی{z=\SI{7.08}{\centi\meter}}
\انتہا{سوال}
%====================
\ابتدا{سوال}
ایک چکور کے اطراف \عددی{\SI{25}{\centi\meter}} ہیں جبکہ اس کے چاروں کونوں پر \عددی{\SI{30}{\nano\coulomb}} چارج پایا جاتا ہے۔کسی ایک کونے کے چارج پر کتنی قوت عمل کرے گی۔

جواب:\عددی{\SI{0.248}{\milli \newton}}
\انتہا{سوال}
%===================
\ابتدا{سوال}
نقطہ \عددی{(2,1,-3)} پر \عددی{\SI{15}{\nano\coulomb}} اور نقطہ \عددی{(-3,-5,4)} پر \عددی{\SI{-6}{\nano\coulomb}} چارج پایا جاتا ہے۔نقطہ \عددی{(2,1,-3)} پر برقی شدت \عددی{\kvec{E}} حاصل کریں۔

جواب:\عددی{ -0.191\ax+1.057\ay+2.195\az}
\انتہا{سوال}
%===================
\ابتدا{سوال}
نقطہ \عددی{(0,0,3)} اور \عددی{(0,0,-3)} پر \عددی{\SI{20}{\micro\coulomb}} چارج پائے جاتے ہیں۔نقطہ \عددی{N(2,0,0)} پر برقی شدت \عددی{\kvec{E}} حاصل کریں۔محدد کے مرکز پر کتنا چارج نقطہ \عددی{N} پر اتنی ہی برقی شدت پیدا کرے گا۔

جوابات:\عددی{\kvec{E}=\num{15339}\ax \, \si{\volt\per\meter}}، \عددی{\SI{6.827}{\micro\coulomb}}
\انتہا{سوال}
%===============
\ابتدا{سوال}
نقطہ \عددی{(4,-2,7)} پر \عددی{\SI{5}{\micro\coulomb}} اور \عددی{(-3,4,-2)} پر \عددی{\SI{12}{\micro\coulomb}} چارج پایا جاتا ہے۔\عددی{y} محدد پر کہاں \عددی{E_x=0} ہو گا۔

جواب:\عددی{y=-6.89}، \عددی{y=-22.11}
\انتہا{سوال}
%==============
\ابتدا{سوال}
نقطہ \عددی{P(6,3,7)} پر \عددی{\SI{6}{\micro\coulomb}} پایا جاتا ہے۔نقطہ \عددی{N(5,4,2)} پر کارتیسی، نلکی اور کروی  محدد میں \عددی{\kvec{E}} حاصل کریں۔نقطہ \عددی{N} کے اکائی سمتیات استعمال کریں۔
-
جوابات:\عددی{\kvec{E}=-384.4\ax+384.4\ay-1922\az}، \عددی{\kvec{E}=-60\arho+540\aphi-1922\az}، \\
\عددی{\kvec{E}=-630\ar+1817\atheta+540\aphi}
\انتہا{سوال}
%====================
\ابتدا{سوال}
نقطہ \عددی{(0,0,0.25)} اور \عددی{(0,0,-0.25)} پر \عددی{\SI{50}{\nano\coulomb}} جبکہ \عددی{(0,0,0)} پر \عددی{\SI{-35}{\nano\coulomb}} پایا جاتا ہے۔نقطہ \عددی{N(3,1,2)} پر کارتیسی اور کروی محدد میں \عددی{\kvec{E}} حاصل کریں۔

جواب:\عددی{34\ax+11\ay+22\az} ، \عددی{42\ar+0.39\atheta}
\انتہا{سوال}
%===================
\ابتدا{سوال}
محدد کے مرکز پر \عددی{\SI{1}{\nano\coulomb}} چارج پایا جاتا ہے۔ سطح \عددی{z=0} پر اس خط کی مساوات حاصل کریں جس پر \عددی{E_y=\SI{1}{\volt\per\meter}} ہو گا۔

جواب:\عددی{80.8y^2=(x^2+y^2)^3}، \عددی{\rho^2=8.987\sin\phi}
\انتہا{سوال}
%====================
\ابتدا{سوال}
محدد کے مرکز پر پڑے چکور کے چاروں کونوں پر \عددی{\SI{5}{\nano\coulomb}} نقطہ چارج پائے جاتے ہیں۔چکور \عددی{z=0} سطح پر پایا جاتا ہے جبکہ اس کے اطراف \عددی{\SI{1}{\meter}} لمبے ہیں۔نقطہ \عددی{(0,a,0)} اور نقطہ \عددی{(0,2a,0)} پر برقی شدت کی شرح \عددی{a=2}، \عددی{a=10} اور \عددی{a=\infty} کی صورت میں حاصل کریں۔

جوابات:\عددی{4.15}، \عددی{4.01}، \عددی{4}
\انتہا{سوال}
%===================================
\ابتدا{سوال}
نقطہ \عددی{(0,0,0)} پر \عددی{Q_1} اور نقطہ \عددی{(1,0,0)} پر \عددی{Q_2} نقطہ چارج پائے جاتے ہیں۔نقطہ \عددی{(2,1,0)} پر \عددی{E_x=0} ہونے کی صورت میں چارجوں کا تعلق دریافت کریں۔ 

جواب:\عددی{Q_1=-1.976Q_2}
\انتہا{سوال}
%===============================
\ابتدا{سوال}
کارتیسی محدد کے پہلے آٹھویں حصے \عددی{(x>0,y>0,z>0)} میں حجمی کثافت چارج \عددی{\rho_h=10e^{-2z}(x^2+2y^2)} ہے جبکہ بقایا سات حصوں میں کوئی چارج نہیں پایا جاتا۔خطہ \عددی{(0\le x\le1,\, 0\le y\le1, \,0\le z\le1)} میں کل چارج حاصل کریں۔اسی طرح خطہ \عددی{{(0\le x+2y\le1, \, 0 \le z\le1)}} میں کل چارج حاصل کریں۔

جوابات:پہلا جواب \عددی{\SI{4.32}{\coulomb}} ہے۔دوسرا تکمل \عددی{\int_{0}^{1\!/\!2}\int_{0}^{1-2y}\int_{0}^{1}\rho_h \dif z \dif x \dif y} لکھتے ہوئے \عددی{\SI{0.27}{\coulomb}} حاصل ہو گا۔ 
\انتہا{سوال}
%============================
\ابتدا{سوال}
حجمی کثافت چارج \عددی{\rho_h=(\rho+0.002)z^2\tan\phi \, \si{\coulomb \per \meter^3}} خطہ \عددیء{0\le \rho \le 0.008}، \عددی{30^{\circ} \le \phi \le 75^{\circ}}، \عددیء{2 \le z \le 5} میں پایا جاتا ہے۔ کثافت چارج کی زیادہ سے زیادہ قیمت دریافت کریں۔اس خطے میں کل چارج حاصل کریں۔

جوابات:\عددی{\SI{0.933}{\coulomb\per\meter^3}}، \عددی{\SI{11.05}{\micro\coulomb}}
\انتہا{سوال}
%====================
\ابتدا{سوال}
نلکی محدد میں \عددی{z} محدد کے گرد یکساں حجمی کثافت چارج \عددیء{e^{-\rho^2}} پائی جاتی ہے۔\عددی{z=0} تا \عددی{z=1} کل چارج حاصل کریں۔\عددی{z} محدد کے گرد کتنے رداس کے اندر کل چارج کا آدھا پایا جاتا ہے۔

جوابات:\عددی{\SI{3.142}{\coulomb}}، \عددی{\SI{0.832}{\meter}} 
\انتہا{سوال}
%=================
\ابتدا{سوال}
کروی محدد میں رداس کے ساتھ بدلتی حجمی کثافت چارج \عددی{\rho_h=\sqrt{r}} پائی جاتی ہے۔اکائی رداس کے کرہ میں کل چارج حاصل کریں۔اسی طرح خطہ \عددی{{(r \le 0.5, \theta \le 25^{\circ}, \phi \le \tfrac{\pi}{3})}} میں کل چارج حاصل کریں۔ 

جوابات:\عددی{\SI{3.59}{\coulomb}}، \عددی{\SI{0.028}{\coulomb}}
\انتہا{سوال}
%=================
