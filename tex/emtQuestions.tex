\باب{سوالات}
\section*{کولومب}
%====================
\ابتدا{سوال}
تکون کے تینوں کونوں پر \عددی{\SI{25}{\micro\coulomb}} کا چارج پایا جاتا ہے جبکہ تینوں کونوں سے \عددیء{\SI{15}{\centi\meter}} فاصلے پر \عددی{\SI{20}{\micro\coulomb}} چارج پایا جاتا ہے۔تکون کے اطراف \عددی{\SI{10}{\centi\meter}} ہونے کی صورت میں چوتھے چارج پر قوت دفع  کی مقدار حاصل کریں۔

 جواب:\عددی{\SI{0.554}{\newton}}
\انتہا{سوال}
%=====================
\ابتدا{سوال}
\عددی{z=0} پر \عددی{\SI{4}{\nano\farad}} اور \عددی{z=\SI{1}{\centi\meter}} پر \عددی{\SI{-3}{\nano\farad}} چارج پائے جاتے ہیں۔\عددی{z} محدد پر وہ نقطے دریافت کریں جہاں مثبت چارج پر صفر قوت پائی جائے گی۔

جوابات: \عددی{z=\SI{0.92}{\centi\meter}}، \عددی{z=\SI{7.08}{\centi\meter}}
\انتہا{سوال}
%====================
\ابتدا{سوال}
ایک چکور کے اطراف \عددی{\SI{25}{\centi\meter}} ہیں جبکہ اس کے چاروں کونوں پر \عددی{\SI{30}{\nano\coulomb}} چارج پایا جاتا ہے۔کسی ایک کونے کے چارج پر کتنی قوت عمل کرے گی۔

جواب:\عددی{\SI{0.248}{\milli \newton}}
\انتہا{سوال}
%===================
\ابتدا{سوال}
نقطہ \عددی{(2,1,-3)} پر \عددی{\SI{15}{\nano\coulomb}} اور نقطہ \عددی{(-3,-5,4)} پر \عددی{\SI{-6}{\nano\coulomb}} چارج پایا جاتا ہے۔نقطہ \عددی{(2,1,-3)} پر برقی شدت \عددی{\kvec{E}} حاصل کریں۔

جواب:\عددی{ -0.191\ax+1.058\ay+2.198\az}
\انتہا{سوال}
%===================
\ابتدا{سوال}
نقطہ \عددی{(0,0,3)} اور \عددی{(0,0,-3)} پر \عددی{\SI{20}{\micro\coulomb}} چارج پائے جاتے ہیں۔نقطہ \عددی{N(2,0,0)} پر برقی شدت \عددی{\kvec{E}} حاصل کریں۔محدد کے مرکز پر کتنا چارج \عددی{N} پر اتنی ہی برقی شدت پیدا کرے گا۔

جوابات:\عددی{\kvec{E}=\num{15360}\ax \, \si{\volt\per\meter}}، \عددی{\SI{6.827}{\micro\coulomb}}
\انتہا{سوال}
%===============
\ابتدا{سوال}
نقطہ \عددی{(4,-2,7)} پر \عددی{\SI{5}{\micro\coulomb}} اور \عددی{(-3,4,-2)} پر \عددی{\SI{12}{\micro\coulomb}} چارج پایا جاتا ہے۔\عددی{y} محدد پر کہاں \عددی{E_x=0} ہو گا۔

جواب:\عددی{y=-6.89}، \عددی{y=-22.11}
\انتہا{سوال}
%==============
\ابتدا{سوال}
نقطہ \عددی{P(6,3,7)} پر \عددی{\SI{6}{\micro\coulomb}} پایا جاتا ہے۔نقطہ \عددی{N(5,4,2)} پر کارتیسی اور نلکی محدد میں \عددی{\kvec{E}} حاصل کریں۔ایسا کرتے ہوئے نقطہ \عددی{N} کے نلکی اکائی سمتیات استعمال کریں۔

جوابات:\عددی{\kvec{E}=-384.9\ax-384.9\ay-1924\az}، \عددی{\kvec{E}=-60.11\arho+541\aphi-1924\az}
\انتہا{سوال}
%====================
\ابتدا{سوال}
نقطہ \عددی{(0,0,0.25)} اور \عددی{(0,0,-0.25)} پر \عددی{\SI{50}{\nano\coulomb}} جبکہ \عددی{(0,0,0)} پر \عددی{\SI{-35}{\nano\coulomb}} پایا جاتا ہے۔نقطہ \عددی{N(3,1,2)} پر کارتیسی اور کروی محدد میں \عددی{\kvec{E}} حاصل کریں۔

جواب:\عددی{33.65\ax+11.22\ay+21.97\az} ، \عددی{41.7220547009928\ar+0.387493462435703\atheta}
\انتہا{سوال}
%===================
