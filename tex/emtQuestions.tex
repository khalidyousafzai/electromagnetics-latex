\باب{سوالات}
\حصہ{توانائی}
%================================
\ابتدا{سوال}
میدان \عددی{\kvec{E}=5\arho-3\aphi+2\az \, \si{\volt\per\meter}} میں \عددی{\SI{1}{\micro\coulomb}} چارج کو نقطہ \عددی{N(5,45^{\circ},4)} سے
 نقطہ \عددی{M(5,45^{\circ},6)} کی جانب نہایت کم فاصلہ \عددی{\SI{1}{\micro\meter}} منتقل کرنے کے لئے درکار توانائی دریافت کریں۔اسی طرح \عددی{(5,45.3^{\circ},4)}، \عددی{(7,45^{\circ},4)} اور \عددی{(10,66^{\circ},12)} کی جانب منتقل کرنے کے لئے درکار توانائی بھی حاصل کریں۔

جوابات:\عددی{\SI{-2}{\pico\joule}}، \عددی{\SI{3.013}{\pico\joule}}، \عددی{\SI{-2.753}{\pico\joule}}، {\SI{3.013}{\pico\joule}}
\انتہا{سوال}
%================================
\ابتدا{سوال}
میدان \عددی{\kvec{E}=100\ax-250\ay+50\az) \, \si{\volt\per\meter}} دیا گیا ہے۔چارج \عددی{\SI{25}{\coulomb}} کو نہایت کم فاصلہ \عددی{\SI{1}{\milli\meter}} نقطہ \عددی{(3,4,6)} سے نقطہ \عددی{(5,10,-2)} کی سمت میں منتقل کرنے کے لئے کتنی توانائی درکار ہے۔اسی طرح \عددی{\ax}، \عددی{\ay} اور \عددی{\ax+\ay+2\az} سمت میں منتقل کرنے کے لئے درکار توانائی بھی حاصل کریں۔

جوابات:\عددی{\SI{-0.2}{\joule}}، \عددی{\SI{0.5}{\joule}}، \عددی{\SI{0.115}{\joule}}
\انتہا{سوال}
%===================================
\ابتدا{سوال}
میدان \عددی{\kvec{E}=0.2x(\sin 0.1 z \ax-2\cos 0.15 x \ay+0.02 z\az) \, \si{\volt\per\meter}} دیا گیا ہے۔نقطہ \عددی{N(3,2,4)} پر \عددی{\kvec{E}} حاصل کریں۔اس نقطے سے \عددی{(5,6,-2)} جانب \عددی{\SI{12}{\coulomb}} چارج نہایت کم فاصلہ \عددی{\SI{2.5}{\micro\meter}} منتقل کرنے کے لئے درکار توانائی حاصل کریں۔
 
جوابات:\عددی{0.177\ax-1.081\ay+0.048\az}، \عددی{\SI{17.06}{\micro\joule}}
\انتہا{سوال}
%===================================
\ابتدا{سوال}
میدان \عددی{\kvec{E}=(2x^3+yz^2)\ax-3z^2\ay+xy^2z\az \, \si{\volt\per\meter}} کا لکیری تکمل \عددی{\int \kvec{E} \cdot \dif \kvec{L}} نقطہ \عددی{N(1,2,3)} تا نقطہ \عددی{P(6,1,2)} مندرجہ ذیل دو راستوں پر حاصل کریں۔(الف) پہلے \عددی{x} محدد کے متوازی چلیں، اس کے بعد \عددی{y} محدد کے متوازی چلیں اور آخر میں \عددی{z} محدد کے متوازی چلیں۔(ب) پہلے نقطہ سے بالکل سیدھا دوسرے نقطے کی طرف چلتے ہوئے تکمل حاصل کریں۔ایسا مساوات \عددی{z=y+1} اور مساوات \عددی{x=11-5y} پر بیک وقت چلتے ہوئے ممکن ہو گا۔ 

جوابات:\عددی{749.5}، \عددی{698.9}
\انتہا{سوال}
%==================================
\ابتدا{سوال}
میدان \عددی{\kvec{E}=2x\ax-3z\ay+2\az \, \si{\volt\per\meter}} میں \عددی{\SI{5}{\micro\coulomb}} کا چارج نقطہ \عددی{(0,2,-4)} تا نقطہ \عددی{(2,4,-12)} منتقل کرنے کے لئے درکار توانائی حاصل کریں۔(الف) باری باری \عددی{x}، \عددی{y} اور \عددی{z} محدد کے متوازی چلیں۔ (ب) ابتدائی نقطے سے اختتامی نقطے تک بہ راستہ \عددی{{z=x^2-y^2}}، \عددی{{y=x+2}} چلیں۔

جوابات:\عددی{\SI{-60}{\micro\joule}}، \عددی{\SI{-180}{\micro\joule}}
\انتہا{سوال}
%===================================
