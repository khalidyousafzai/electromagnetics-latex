\باب{سوالات}

\حصہء{مستوی امواج}
%====================
\ابتدا{سوال}
خالی خلاء میں \عددی{\az} سمت میں حرکت کرتی، \عددی{\SI{600}{\mega \hertz}} تعدد  کے مستوی برقی موج \عددی{\kvec{E}} کی  چوٹی لمحہ \عددی{t=\SI{1}{\nano\second}} پر \عددی{z=\SI{0.3}{\meter}}  پر پائی جاتی ہے۔یہ چوٹی \عددی{\SI{310}{\volt\per\meter}} کے برابر ہے۔الف) برقی میدان \عددی{\ax} سمت میں ہونے کی صورت میں سائن نما  \عددی{\kvec{E}} اور \عددی{\kvec{H}} امواج کے مساوات لکھیں۔ ب) میدان سمتیہ \عددی{5\ax-2\ay} کی سمت میں ہونے کی صورت
 میں سائن نما  \عددی{\kvec{E}_s} اور \عددی{\kvec{H}_s}  امواج کی مساوات لکھیں۔

جواب:\عددی{\kvec{E}=310 \ax \cos (12\pi \times 10^8 t -4\pi z)}، \عددی{\kvec{H}=\tfrac{31}{12\pi} \ay \cos (12\pi \times 10^8 t -4\pi z)}،\\ \عددی{\kvec{E}_s=310\left[\tfrac{5}{\sqrt{29}}\ax-\tfrac{2}{\sqrt{29}}\ay\right]e^{-j4\pi z}}،  \عددی{\kvec{H}_s=\tfrac{31}{12\pi}\left[\tfrac{2}{\sqrt{29}}\ax+\tfrac{5}{\sqrt{29}}\ay\right]e^{-j4\pi z}}
\انتہا{سوال}
%===================
\ابتدا{سوال}
خالی خلاء میں  نقطہ \عددی{N(3,-2,5)} پر \عددی{\az} جانب حرکت کرتی، \عددی{\SI{200}{\mega\hertz}} تعدد کے برقی میدان کی سائن نما مستوی موج کی چوٹی لمحہ \عددی{t=0} پر \عددی{\kvec{E}_0=150\ax+210\ay \, \si{\volt\per\meter}} پائی جاتی ہے۔ الف) \عددی{\lambda}، \عددی{\beta}، \عددی{\kvec{a}_E}، \عددی{\kvec{a}_H}، \عددی{H_0} اور مقناطیسی موج \عددی{\kvec{H}_s} حاصل کریں۔  ب) لمحہ \عددی{t=0} پر نقطہ \عددی{N} پہ برقی میدان کی شدت  حاصل کریں۔ پ) لمحہ \عددی{t=\SI{1.5}{\nano\second}} پر نقطہ \عددی{N} پہ برقی میدان کی شدت  حاصل کریں۔ ت) نقطہ \عددی{P(5,3,7)} پہ لمحہ \عددی{t=\SI{2}{\nano\second}} پر برقی میدان کی شدت حاصل کریں۔ 

جوابات:\عددی{\lambda=\tfrac{3}{2}\, \si{\meter}}، \عددی{\beta=4.2 \, \si{\radian \per \meter}}، \عددی{\kvec{a}_E=0.51\ax+0.86\ay}، \عددی{\kvec{a}_H=-0.86\ax+0.51\ay}، \\
\عددی{H_0=\SI{0.7733}{\ampere\per\meter}}،  \عددی{\kvec{H}_s=0.7733(-0.86\ax+0.51\ay)e^{-j4.2 z}}، \عددی{\SI{292}{\volt\per\meter}}، \عددی{\SI{-90}{\volt\per\meter}}، \عددی{\SI{266}{\volt\per\meter}}
\انتہا{سوال}
%==================

\ابتدا{سوال}
خالی خلاء میں مستوی موج \عددی{\kvec{E}_s=\kvec{E}_0 e^{-j 6z}} دی گئی ہے۔ الف) موج کی تعدد \عددی{\omega} حاصل کریں۔ ب) برقی میدان کا حیطہ بالترتیب \عددی{\kvec{E}_0=50\ax}، \عددی{\kvec{E}_0=(5+j10)\ax}، \عددی{\kvec{E}_0=50\ax+80\ay} اور \عددی{\kvec{E}_0=(30\phase{45^{\circ}})\ax} ہونے کی صورت میں لمحہ \عددی{t=0} پر نقطہ \عددی{N(0,0,0)} پہ \عددی{\abs{\kvec{E}}} حاصل کریں۔


جوابات: \عددی{\SI{1.8}{\giga \radian\per\second}}، \عددی{\SI{50}{\volt\per\meter}}، \عددی{\SI{11.18}{\volt\per\meter}}،  \عددی{\SI{94.3}{\volt\per\meter}}، \عددی{\SI{11.18}{\volt\per\meter}}،  \عددی{\SI{21.2}{\volt\per\meter}}،
\انتہا{سوال}
%====================
\ابتدا{سوال}
خالی خلاء میں \عددی{\SI{350}{\mega\hertz}} تعدد کی مستوی موج \عددی{\kvec{E}_s=(5+j2)(3\ax-j4\ay)e^{j \beta z} \, \si{\volt\per\meter}} پائی جاتی ہے۔ \عددی{\lambda} اور \عددی{\beta} کی قیمتیں دریافت کریں۔لمحہ \عددی{t=\SI{1.4}{\nano\second}} پر نقطہ \عددی{z=\SI{40}{\centi\meter}} پہ \عددی{\kvec{E}} حاصل کریں۔موج کا حیطہ حاصل کریں۔

جواب:\عددی{\lambda=\tfrac{6}{7} \, \si{\meter}}، \عددی{\beta=\tfrac{7\pi}{3}\,\si{\radian \per \meter}}، \عددی{\kvec{E}(z=\si{40}{\centi\meter},t=\SI{1.4}{\nano\second})=13.96\ax-10.84\ay \, \si{\volt\per\meter}}، \عددی{\abs{\kvec{E}}_{\textrm{بلندتر}}=\SI{26.9}{\volt\per\meter}}
\انتہا{سوال}
%====================
\ابتدا{سوال}
ایسا خطہ جس کے مستقل \عددی{\mu_R=1}، \عددی{\epsilon_R=4.4} اور \عددی{\sigma=0} ہیں میں بڑھتے \عددی{x} محدد کی جانب حرکت کرتی، \عددی{\SI{250}{\mega\hertz}} تعدد کی مستوی برقی موج پائی جاتی ہے۔برقی میدان \عددی{\ay} سمت میں ہے۔مندرجہ ذیل حاصل کریں۔ \عددی{v_p}، \عددی{\beta}، \عددی{\lambda}، \عددی{\eta} \عددی{\kvec{E}_s}، \عددی{\kvec{H}_s} اور \عددی{\pmb{\mathscr{P}}_{\text{اوسط}}}؛   

جوابات:\عددی{v_p=\SI{1.429e8}{\meter\per\second}}، \عددی{\beta=\SI{10.99}{\radian\per\meter}}، \عددی{\lambda=\SI{57.2}{\centi\meter}}، \عددی{\eta=\SI{179.6}{\ohm}}،  \عددی{\kvec{E}_s=E_0 e^{-j 10.99 x} \ay \, \si{\volt\per\meter}}، 
\عددی{\kvec{H}_s=\tfrac{E_0}{179.6} e^{-j 10.99 x}\az \, \si{\ampere\per\meter}}،
 \عددی{\pmb{\mathscr{P}}_{\text{اوسط}}=\tfrac{E_0^2}{359.2} \ax \, \si{\watt\per\meter\squared}}
\انتہا{سوال}
%===================
\ابتدا{سوال}
مستوی برقی موج \عددی{\kvec{E}=E_0 e^{-\alpha z}\sin(\omega t - \beta z)\ay \, \si{\volt\per\meter}} اور \عددی{\eta = \abs{\eta_0} e^{j \phi}} دئے گئے ہیں۔ الف) دوری سمتیات \عددی{\kvec{E}_s} اور \عددی{\kvec{H}_s} حاصل کریں۔ ب) \عددی{\pmb{\mathscr{P}}_{\text{اوسط}}} حاصل کریں۔

جوابات:\عددی{\kvec{E}_s=E_0 e^{-\alpha z} e^{-j (\beta z +\pi)} \ay \, \si{\volt\per\meter}}،
 \عددی{\kvec{H}_s=-\tfrac{E_0}{\abs{\eta_0}} e^{-\alpha z }e^{-j(\beta z +\pi+\phi)}\ax \, \si{\ampere\per\meter}}،
 \عددی{\pmb{\mathscr{P}}_{\text{اوسط}}=\tfrac{E_0^2}{2\abs{\eta_0}}e^{-2\alpha z} \cos \phi  \az \, \si{\watt\per\meter\squared}}
\انتہا{سوال}
%================
\ابتدا{سوال}
خالی خلاء میں  \عددی{\kvec{E}=(30\ay+22\az)\cos(\omega t -60 x) \, \si{\volt\per\meter}} پایا جاتا ہے۔ الف) \عددی{\lambda} اور \عددی{\omega} حاصل کریں۔ ب) دوری سمتیات \عددی{\kvec{E}_s} اور \عددی{\kvec{H}_s} لکھیں۔ پ) \عددی{\pmb{\mathscr{P}}_{\text{اوسط}}} حاصل کریں۔

جوابات:\عددی{\lambda=\tfrac{\pi}{30} \, \si{\meter}}، \عددی{\omega=\SI{1.8e10}{\radian\per\second}}، \عددی{\kvec{E}_s=(30\ay+22\az)e^{-j 60 x} \, \si{\volt \per\meter}}، \\
 \عددی{\kvec{H}_s=\tfrac{1}{120 \pi}(-22\ay+30\az) e^{-j 60 x} \, \si{\ampere\per\meter}}، 
\عددی{\pmb{\mathscr{P}}_{\text{اوسط}}=\tfrac{173}{30\pi} \ax \, \si{\watt\per\meter\squared}}
\انتہا{سوال}
%================
\ابتدا{سوال}
مستوی مقناطیسی موج کا دوری سمتیہ \عددی{\kvec{H}_s=(5\ax+j4\az)e^{j20y} \, \si{\volt\per\meter}} اور تعدد \عددی{\SI{200}{\mega\hertz}} ہے۔ برقی موج کا زیادہ سے زیادہ حیطہ \عددی{\SI{1200}{\volt\per\meter}} ہے۔حاصل کریں \عددی{\beta}، \عددی{\lambda}، \عددی{\eta}، \عددی{v_p}، \عددی{\epsilon_R}، \عددی{\mu_R} اور \عددی{\kvec{H}(x,y,z,t)}؛

جوابات:\عددی{\beta=\SI{20}{\radian\per\meter}}، \عددی{\lambda=\tfrac{\pi}{10} \, \si{\meter}}، \عددی{\eta=\SI{187.4}{\ohm}}،
 \عددی{v_p=\SI{6.28e7}{\meter\per\second}}، \عددی{\epsilon_R=9.6}، \عددی{\mu_R=2.4}، \عددی{\kvec{H}=5\cos(2\pi\times 200\times 10^6 t +20y)\ax-4\sin(2\pi\times 200 \times 10^6 t +20y)\az \, \si{\ampere\per\meter}}
\انتہا{سوال}
%===================
\ابتدا{سوال}
میدان \عددی{\kvec{E}(y,t)=700\cos(2.5\times 10^7  t -\beta y) \ax \, \si{\volt\per\meter}} اور \عددی{\kvec{H}(y,t)=1.5\cos(2.5 \times 10^7  t -\beta y) \ay \, \si{\ampere\per\meter}} مستوی موج کو ظاہر کرتے ہیں۔یہ موج \عددی{\SI{1.7e8}{\meter\per\second}} رفتار سے حرکت کر رہی ہے۔حاصل کریں \عددی{\beta}، \عددی{\lambda}، \عددی{\eta}، \عددی{\epsilon_R} اور \عددی{\mu_R}؛

جوابات:\عددی{\beta=\SI{0.147}{\radian\per\meter}}، \عددی{\lambda=\SI{42.7}{\meter}}، \عددی{\eta=\SI{467}{\ohm}}،
 \عددی{\epsilon_R=1.4}، \عددی{\mu_R=2.2}
\انتہا{سوال}
%===================
\ابتدا{سوال}
بے ضیاع خطے کے مستقل \عددی{\mu_R=1.2} اور \عددی{\epsilon_R=5.4} ہیں۔لمحہ \عددی{t=\SI{10}{\nano\second}} پر نقطہ \عددی{N(2,0.5,1.5)} پہ \عددی{\SI{15}{\mega\hertz}} تعدد اور \عددی{E_{x}=\SI{350}{\volt\per\meter}}، \عددی{E_{y}=0} ،\عددی{E_{z}=0} کی خطی قطبی موج \عددی{\ay} سمت میں حرکت کر رہی ہے۔حاصل کریں \عددی{\beta}، \عددی{\lambda}، \عددی{v_p}، \عددی{\eta}، \عددی{E_0}، \عددی{\kvec{E}(x,y,z,t)}؛

جوابات:\عددی{v_p=\SI{1.18e8}{\meter\per\second}}، \عددی{\lambda=\SI{7.85}{\meter}}، \عددی{\beta=0.25\pi \, \si{\radian\per\meter}}، \عددی{\eta=\SI{178}{\ohm}}، \عددی{E_0=\SI{408.6}{\volt\per\meter}}، \عددی{\kvec{E}(x,y,z,t)=408.6\cos(3\pi\times 10^7t-0.25\pi y) \ax}
\انتہا{سوال}
%====================
\ابتدا{سوال}
خطی قطبی موج \عددی{\kvec{E}_s=(E_{y0}\ay+E_{z0}\az)e^{\alpha x}e^{j \beta x} \, \si{\volt\per\meter}} ایسے ضیاع کار خطے میں پائی جاتی ہے جہاں \عددی{\eta=\abs{\eta_0}e^{j \phi}} ہے۔ \عددی{\kvec{H}_s}، \عددی{\kvec{E}(x,y,z,t)}، \عددی{\kvec{H}(x,y,z,t)} اور \عددی{\pmb{\mathscr{P}}_{\text{اوسط}}}  کے مساوات لکھیں۔

جوابات:\عددی{\kvec{H}_s=\tfrac{1}{\abs{\eta_0}}(E_{z0}\ay-E_{y0}\az)e^{\alpha x}e^{e^{j(\beta x -\phi)}}\, \si{\ampere\per\meter}}،
 \عددی{\kvec{E}(x,y,z,t)=(E_{y0}\ay+E_{z0}\az)e^{\alpha x}\cos(\omega t +\beta x) \, \si{\volt\per\meter}}،
  \عددی{\kvec{H}(x,y,z,t)=\tfrac{1}{\abs{\eta_0}}(E_{z0}\ay-E_{y0}\az)e^{\alpha x}\cos(\omega t +\beta x-\phi) \, \si{\ampere\per\meter}}،
\عددی{\pmb{\mathscr{P}}_{\text{اوسط}}=\tfrac{1}{2\abs{\eta_0}}(E^2_{y0}+E^2_{z0})e^{2\alpha x}\cos \phi  \ax \, \si{\watt\per\meter\squared}}
\انتہا{سوال}
%========================
\ابتدا{سوال}
کامل موصل سے بنی \عددی{\rho=\SI{5}{\milli\meter}} اور \عددی{\rho=\SI{12}{\milli\meter}} رداس کے نلکیوں کا محور \عددی{z} محدد ہے۔دو نلکیوں کے درمیان ذو برق کے مستقل \عددی{\mu_R=1} اور \عددی{\epsilon_R=3.2} ہیں۔اس ذو برق میں میدان \عددی{\kvec{E}=\tfrac{1200}{\rho}\cos(\omega t -5 z) \arho \, \si{\volt\per\meter}} پایا جاتا ہے۔ الف) میکس ویل کے مساوات استعمال کرتے ہوئے \عددی{\omega} حاصل کریں۔ ب) \عددی{\kvec{H}}  کی مساوات حاصل کریں۔
 پ) \عددی{\pmb{\mathscr{P}}{اوسط}} اور  \عددی{\pmb{\mathscr{P}}_{\text{اوسط}}}  حاصل کریں۔ ت) دونوں نلکیوں کے درمیانی خطے میں \عددی{\az} جانب کتنی طاقت منتقل ہو رہی ہے۔

جوابات:\عددی{\omega=\SI{8.38e8}{\radian\per\second}}، \عددی{\kvec{H}=\tfrac{5.7}{\rho}\cos (8.38 \times 10^8 t -5z)\aphi \, \si{\ampere\per\meter}}، \\
\عددی{\pmb{\mathscr{P}}{اوسط}=\tfrac{6837}{\rho^2}\cos^2(8.38\times 10^ 8 t-5 z) \az \, \si{\watt\per\meter\squared}}، \عددی{\pmb{\mathscr{P}}_{\text{اوسط}}=\tfrac{3418.6}{\rho^2} \, \si{\watt \per\meter\squared}}، \عددی{\SI{2.5}{\mega\watt}}
\انتہا{سوال}
%=======================
\ابتدا{سوال}
کروی محدد میں \عددی{\kvec{E}_s=\tfrac{60}{r}\sin \theta e^{-j 2 r}\atheta \, \si{\volt\per\meter}} اور
 \عددی{\kvec{H}_s=\tfrac{1}{4\pi r}\sin \theta e^{-j 2 r}\aphi \, \si{\ampere\per\meter}} دیے گئے ہیں۔ الف) \عددی{\pmb{\mathscr{P}}_{\text{اوسط}}}  حاصل کریں۔ ب) رداس \عددی{r=\SI{5}{\centi\meter}} پر سطح \عددی{0<\theta<\tfrac{\pi}{3}}، \عددی{0<\phi<2\pi} سے خارج طاقت حاصل کریں۔

جوابات:\عددی{\pmb{\mathscr{P}}_{\text{اوسط}}=\tfrac{15\sin^2 \theta}{2\pi r^2} \ar \, \si{\watt\per\meter\squared}}، \عددی{\SI{3.13}{\watt}}
\انتہا{سوال}
%=======================
\ابتدا{سوال}
\عددی{\SI{12}{\giga\hertz}} تعدد پر ایک فیرائٹ کے مستقل \عددی{\mu_R=5}، \عددی{\epsilon_R=8} اور \عددی{\sigma=\SI{15}{\milli\siemens\per\meter}} ہیں۔ آپ سے گزارش ہے کہ \عددی{\alpha}، \عددی{\beta}، \عددی{v}،  \عددی{\lambda} اور \عددی{\eta} حاصل کریں۔

جوابات:\عددی{\alpha=\SI{2.23}{\neper\per\meter}}، \عددی{\beta=\SI{1590}{\radian\per\meter}}، \عددی{v=\SI{4.74e7}{\meter\per\second}}، \عددی{\lambda=\SI{3.95}{\milli\meter}}، \عددی{\eta=297.83+j0.418 \, \si{\ohm}}
\انتہا{سوال}
%======================
\ابتدا{سوال}
ایسے خطے کے مستقل \عددی{\mu_R}، \عددی{\epsilon_R} اور \عددی{\sigma} حاصل کریں جس میں  \عددی{\SI{100}{\mega\hertz}} تعدد پر طول موج \عددی{\SI{1}{\meter}}، قدرتی رکاوٹ کی حتمی قیمت \عددی{\SI{200}{\ohm}} اور  تضعیفی مستقل \عددی{\SI{2}{\neper\per\meter}} ہو۔

جوابات:\عددی{\mu_R=1.67}، \عددی{\epsilon_R=4.84} ،\عددی{\sigma=\SI{19.06}{\milli\siemens\per\meter}}
\انتہا{سوال}
%======================
\ابتدا{سوال}
\عددی{\SI{330}{\mega\hertz}} تعدد کی مستوی موج ایسے  غیر مقناطیسی خطے میں حرکت کر رہی ہے جس کے مستقل \عددی{\epsilon_R=2.8} اور \عددی{\tfrac{\sigma}{\omega \epsilon}=3.6\times 10^{-4}} ہیں۔ الف) اس خطے کی \عددی{\sigma} حاصل کریں۔ ب) \عددی{\alpha}، \عددی{\beta} اور \عددی{\lambda} حاصل کریں۔ پ) موج کی چوٹی کتنا فاصلہ طے کرنے کے بعد آدھی رہ جائے گی؟  ت) موج کی طاقت کتنا فاصلہ طے کرنے کے بعد آدھا رہ جائے گا؟   ٹ) کتنے فاصلے پر موج کے زاویے میں \عددی{30^{\circ}} تبدیلی رونما ہو گی؟

جوابات:\عددی{\sigma=\SI{1.85e-5}{\siemens\per\meter}}، \عددی{\alpha=\SI{0.04}{\neper\per\meter}}، \عددی{\beta=\SI{11.57}{\radian\per\meter}}، \عددی{\lambda=\SI{0.54}{\meter}}، \عددی{\SI{17.1}{\meter}}، \عددی{\SI{8.55}{\meter}}، \عددی{\SI{4.52}{\centi\meter}}
\انتہا{سوال}
%======================
\ابتدا{سوال}
کپیسٹر \عددی{C} میں طاقت کے ضیاع کو کپیسٹر کے متوازی مزاحمت \عددی{R} سے ظاہر کیا جاتا ہے۔ ایسے متوازی دور کی برقی رکاوٹ \عددی{Z} ہے۔برقی رکاوٹ کے زاویہ \عددی{\theta} کا کوسائن، یعنی \عددی{\cos \theta}، جزو ضربی طاقت کہلاتا ہے جبکہ کپیسٹر کی خاصیت \عددی{Q} سے مراد \عددی{\omega R C} ہے۔ متوازی چادر کپیسٹر  جس کے مستقل \عددی{\sigma}، \عددی{\epsilon} اور \عددی{\mu} ہیں کے جزو ضربی طاقت اور \عددی{Q} کے مساوات کو مماس ضیاع \عددی{\tfrac{\sigma}{\omega \epsilon}} استعمال کرتے ہوئے لکھیں۔

جوابات:\عددی{\cos \theta=\tfrac{1}{\sqrt{1+\left(\tfrac{\sigma}{\omega \epsilon}\right)^{-2}}}}، \عددی{Q=\left(\tfrac{\sigma}{\omega \epsilon}\right)^{-1}}
\انتہا{سوال}
%======================
\ابتدا{سوال}
تانبے کی ہم محوری تار کے اندرونی تار کا رداس \عددی{\SI{5}{\milli\meter}} اور  بیرونی تار کا اندرونی رداس \عددی{\SI{8}{\milli\meter}} ہیں۔دونوں تار گہرائی جلد \عددی{\delta} سے بہت زیادہ موٹائی رکھتے ہیں جبکہ ذو برق بے ضیاع ہے۔\عددی{\SI{550}{\mega\hertz}} تعدد پر فی میٹر اندرونی تار، فی میٹر بیرونی تار اور فی میٹر مکمل ترسیلی تار کی مزاحمت دریافت کریں۔تانبے کے مستقل کتاب کے آخر میں جدول \حوالہ{جدول_جدول_موصلیت_کے_مستقل} سے حاصل کئے جا سکتے ہیں۔

جوابات:\عددی{\SI{195}{\milli\ohm\per\meter}}، \عددی{\SI{122}{\milli\ohm\per\meter}}، \عددی{\SI{316}{\milli\ohm\per\meter}}
\انتہا{سوال}
%=======================
\ابتدا{سوال}
المونیم  سے نلکی نما تار بنائی جاتی ہے جس کا اندرونی رداس \عددی{\SI{5}{\milli\meter}} اور بیرونی رداس \عددی{\SI{6}{\milli\meter}} ہیں۔ایک کلو میٹر تار کی مزاحمت مندرجہ ذیل تعدد پر حاصل کریں۔ الف) یک سمتی رو۔ ب) \عددی{\SI{30}{\mega\hertz}}پ) \عددی{\SI{1.2}{\giga\hertz}}

جوابات:\عددی{\SI{758}{\milli\ohm}}، \عددی{\SI{46.7}{\ohm}}، \عددی{\SI{295}{\ohm}}
\انتہا{سوال}
%=========================
\ابتدا{سوال}
کھانا جلد گرم کرنے کی خاطر عموماً برقی \اصطلاح{خرد موج چولھا}\فرہنگ{خرد موج چولھا}\حاشیہب{micro wave oven}\فرہنگ{micro wave oven} (مائیکرو ویو اون)  استعمال کیا جاتا ہے جو عموماً \عددی{\SI{2.45}{\giga\hertz}} کے تعدد پر کام کرتا ہے۔اس چولھے کے دیوار سٹینلس سٹیل کے بنے ہوتے ہیں۔ سٹینلس سٹیل کے مستقل \عددی{\sigma=\SI{1.1e6}{\siemens\per\meter}}، \عددی{\mu_R=1} اور \عددی{\epsilon_R=1} لیتے ہوئے گہرائی جلد \عددی{\delta} حاصل کریں۔سٹینلس سٹیل چادر کی سطح پر
 \عددی{E_s=64\phase{0^{\circ}} \, \si{\volt\per\meter}} لیتے ہوئے چادر کے اندر میدان کی مساوات لکھیں۔

جوابات:\عددی{\delta=\SI{9.69}{\micro\meter}}، \عددی{E_s(z)=64 e^{-1.03\times 10^{-7}z(1+j)} \, \si{\volt\per\meter}}   
\انتہا{سوال}
%========================
\ابتدا{سوال}
ایک غیر مقناطیسی موصل میں رفتار موج \عددی{\SI{4.5e5}{\meter\per\second}} اور طول موج \عددی{\SI{0.25}{\milli\meter}} ہے۔تعدد \عددی{f}، گہرائی جلد \عددی{\delta}  اور موصل کی موصلیت \عددی{\sigma} حاصل کریں۔

جوابات:\عددی{f=\SI{1.8}{\giga\hertz}}، \عددی{\delta=\SI{39.8}{\micro\meter}}، \عددی{\sigma=\SI{8.89e4}{\siemens\per\meter}}  
\انتہا{سوال}
%=================
\ابتدا{سوال}
برقی موج \عددی{\kvec{E}=\tfrac{270}{r}\sin \theta \cos[\omega( t -\tfrac{r}{c})] \atheta \, \si{\volt\per\meter}} دی گئی ہے۔ رداس \عددی{r} کے کرہ سے کتنی طاقت خارج ہو رہی ہے۔

جواب:\عددی{\SI{810}{\watt}}
\انتہا{سوال}
%==================
\ابتدا{سوال}
برقی موج \عددی{\kvec{E}_s=3\ax-5\ay+2\az \, \si{\kilo \volt\per\meter}} اور مقناطیسی موج
 \عددی{\kvec{H}_s=14\ax+13\ay-16\az \, \si{\ampere\per\meter}} ہیں۔ الف) حرکت موج کی سمت میں اکائی سمتیہ حاصل کریں۔ ب) موج کی اوسط کثافت طاقت حاصل کریں۔ پ) \عددی{\mu_R=1} کی صورت میں \عددی{\epsilon_R} حاصل کریں۔

جوابات:\عددی{\kvec{a}=0.38\ax0.53\ay +0.76\az}، \عددی{\SI{71.7}{\kilo\watt\per\meter\squared}}، \عددی{\epsilon_R=2.32}
\انتہا{سوال}
%=================
\ابتدا{سوال}
ضیاع کار خطہ \عددی{x<0} کے مستقل \عددی{\mu_R=1}، \عددی{\epsilon_R=1} اور \عددی{\sigma=\SI{1500}{\siemens\per\meter}} ہیں  جبکہ \عددی{x>0} خالی خلاء ہے۔ خلاء میں نقطہ \عددی{N(0^+,0,0)} پر مقناطیسی میدان \عددی{\kvec{H}=300\cos 5\times 10^8 t \ay \, \si{\ampere\per\meter}} پایا جاتا ہے۔ الف) نقطہ \عددی{(0^-,0,0)} پر \عددی{\kvec{H}} حاصل کریں۔ ب) خالی خلاء میں \عددی{\az} سمت حرکت کرتی موج تصور کرتے ہوئے  نقطہ \عددی{(0^+,0,0,)} پر
 \عددی{\kvec{E}} حاصل کریں۔ خطہ \عددی{z<0} میں \عددی{-\ax} جانب حرکت کرتی موج تصور کرتے ہوئے  نقطہ \عددی{(0^-,0,0)} پر \عددی{\kvec{E}} حاصل کریں۔

جوابات:\عددی{\kvec{H}=300\cos 5\times 10^8 t \ay \, \si{\ampere\per\meter}}، 
\عددی{\kvec{E}=113\cos 5\times 10^8 t \ax \, \si{\kilo \volt\per\meter}}،
 \عددی{\kvec{E}=238\cos (5\times 10^8 t-45^{\circ}) \az \, \si{\volt\per\meter}}
\انتہا{سوال}
%================
\ابتدا{سوال}
آمدی مستوی موج جس کی تعدد \عددی{\omega=\SI{4.2e8}{\radian\per\second}} ہے خطہ-ا، \عددی{z<0}، \عددی{\sigma_1=0}، \عددی{\mu_{R1}=1}، \عددی{\epsilon_{R1}=3.2} سے  خطہ-2، \عددی{z>0}، \عددی{\sigma_2=0}، \عددی{\mu_{R2}=2.6}، \عددی{\epsilon_{R2}=12} میں داخل ہوتی ہے۔آمدی برقی موج کا حیطہ \عددی{z=0}، \عددی{t=0}  پر \عددی{\SI{5.6}{\volt\per\meter}} ہے۔ الف ) \عددی{\eta_1}، \عددی{\eta_2}، \عددی{\beta_1}  اور \عددی{\beta_2} حاصل کریں۔ ب) \عددی{\Gamma} اور \عددی{\tau} حاصل کریں۔ پ) \عددی{E_1(t)} اور \عددی{E_2(t)} حاصل کریں۔ ت) \عددی{H_1(t)} حاصل کریں۔ ٹ) لمحہ \عددی{t=\SI{4}{\nano\second}} پر نقطہ \عددی{(0,0,-1.5)} پہ \عددی{H_1} حاصل کریں۔

جوابات:\عددی{\eta_1=\SI{211}{\ohm}}، \عددی{\eta_2=\SI{175}{\ohm}}، \عددی{\beta_1=\SI{2.5}{\radian\per\meter}}،
 \عددی{\beta_2=\SI{7.8}{\radian\per\meter}}، \عددی{\Gamma=-0.0913}، \عددی{\tau=0.9087}،
 \عددی{E_1=5.6\cos(4.2\times 10^8 t -2.5z)-0.511\cos(4.2 \times 10^8 t +2.5 z) \, \si{\volt\per\meter}}،
 \عددی{E_2=5.09\cos(4.2\times 10^8 t -7.8z) \, \si{\volt\per\meter}}،\\
 \عددی{H_1=26.59 \cos (4.2\times 10^ 8 t -2.5z) +2.43\cos(4.2\times 10^8 t +2.5 z) \, \si{\milli\ampere\per\meter}} ،\عددی{H_1=\SI{16.49}{\milli\ampere\per\meter}}
\انتہا{سوال}
%===================
\ابتدا{سوال}
تھیلا بنانے والے پلاسٹک میں \عددی{\SI{14}{\giga\hertz}} تعدد کی مستوی موج \عددی{\ax} سمت میں حرکت کرتے ہوئے  \عددی{x=\SI{0.3}{\centi\meter}} پر پائے جانے والے کامل موصل سطح سے انعکاس پذیر ہوتی ہے۔ الف) وہ سطحیں دریافت کریں جن پر \عددی{\kvec{E}=0} ہو گا۔ ب) اس پلاسٹک  میں بلند تر برقی چوٹی اور بلند تر مقناطیسی چوٹی کی شرح حاصل کریں۔ 

جوابات:\عددی{x=0.3-0.71n \, \si{\centi\meter}} جہاں \عددیء{n=0,1,2,\cdots} ہے، \عددی{\eta=\SI{251}{\ohm}}
\انتہا{سوال}
%==================
\ابتدا{سوال}
خطہ \عددی{z<0} بے ضیاع خالی خلاء ہے جبکہ ضیاع کار خطہ \عددی{z>0} کے مستقل \عددی{\epsilon=\SI{30}{\pico\farad\per\meter}}، \عددی{\mu=\SI{4.2}{\micro\henry\per\meter}} اور \عددی{\sigma=\SI{4.6}{\milli\siemens\per\meter}} ہیں۔ خالی خلاء سے سرحد پر آمدی موج کی مساوات
 \عددی{E_{x1}^+=340 e^{-\alpha_1 z} \cos(2\times 10^8 t -\beta_1 z) \, \si{\volt\per\meter}} ہے۔ الف) \عددی{\alpha_1} اور \عددی{\beta_1} حاصل کریں۔ ب) انعکاسی مستقل حاصل کریں۔ پ) انعکاسی موج \عددی{E_{x1}^-} کی مساوات حاصل کریں۔ ت) ترسیلی موج \عددی{E_{x2}^+} کی مساوات حاصل کریں۔

جوابات:\عددی{\alpha_1=0}، \عددی{\beta_1=\SI{0.667}{\radian\per\meter}}، \عددی{\Gamma=0.176\phase{111^{\circ}}}،
 \عددی{E_{x1}^-=59.8\cos(2\times 10^8 t +0.667 z +111^{\circ}) \, \si{\volt\per\meter}}، \عددی{E_{x2}^+=324e^{-0.81 z}\cos(2\times 10^8 t -2.39z +9.9^{\circ})\, \si{\volt\per\meter}}
\انتہا{سوال}
%===================
\ابتدا{سوال}
المونیم کی سطح \عددی{y=0} پر خالی خلاء سے عمودی آمدی موج \عددی{E_{x1}^+=E_{x10}^+ \cos(4\times 10^8 t -\beta y) \, \si{\volt\per\meter}} ہے۔آمدی طاقت کا کتنا فی صد سطح سے انعکاس پذیر ہوتا ہے۔

جواب:\عددی{\SI{99.997}{\percent}}
\انتہا{سوال}
%====================
\ابتدا{سوال}
مستوی موج خطہ-1 سے خطہ-2 پر عمودی پڑتی ہے۔ان خطوں کے مستقل \عددی{\sigma_1=\sigma_2=0}، \عددی{\epsilon_{R1}=\mu_{R1}^3}، اور
 \عددی{\epsilon_{R2}=\mu_{R2}^3} ہیں۔آمدی طاقت کا \عددی{\SI{40}{\percent}} سرحد سے واپس لوٹتا ہے۔\عددی{\tfrac{\mu_{R2}}{\mu_{R1}}} حاصل کریں۔

جوابات:\عددی{\tfrac{\mu_{R2}}{\mu_{R1}}=0.225} اور \عددی{\tfrac{\mu_{R2}}{\mu_{R1}}=4.442}
\انتہا{سوال}
%==================
\ابتدا{سوال}
خالی خلاء سے مستوی موج ضیاع کار خطہ \عددی{\sigma=\SI{0.002}{\siemens\per\meter}}، \عددی{\epsilon_R=8.2} اور \عددی{\mu_R=1.8} پر عمودی پڑتی ہے۔آمد موج کی تعدد \عددی{\SI{100}{\mega\hertz}} اور کثافت طاقت \عددی{\SI{12}{\watt\per\meter\squared}} ہے۔الف) ابتدائی ترسیلی کثافت طاقت حاصل کریں۔ ب) ضیاع کار خطے میں \عددی{} کی قیمت حاصل کریں۔ پ) دوسرے خطے میں کتنا فاصلہ طے کرنے کے بعد ترسیلی کثافت طاقت \عددی{\SI{0.2}{\watt\per\meter\squared}} رہ جائے گی۔

جوابات:\عددی{\SI{10.42}{\watt\per\meter\squared}}، \عددی{\alpha_2=\SI{0.1765}{\neper\per\meter}}، \عددی{\SI{11.2}{\meter}}
\انتہا{سوال}
%=========================
\ابتدا{سوال}
خالی خلاء \عددی{z<0} میں برقی موج \عددی{\kvec{E}_s=100e^{-j 15 z} \ay+28\phase{30^{\circ}}e^{j 15 z}\ay \, \si{\volt\per\meter}} پائی جاتی ہے۔ الف) موج کی تعدد حاصل کریں۔ ب) خطہ \عددی{z>0}  کی قدرتی رکاوٹ حاصل کریں ۔ پ) دو خطوں کے سرحد کے قریب کس مقام پر برقی موج کی چوٹی پائی جاتی ہے؟

جوابات:\عددی{\SI{715.7}{\mega\hertz}}، \عددی{\eta=585+j178}، \عددی{z=\SI{-1.75}{\centi\meter}}
\انتہا{سوال}
%============================
\ابتدا{سوال}
بے ضیاع خطہ \عددی{z<0} کے مستقل \عددی{\sigma_1=0}، \عددی{\mu_1=\SI{30}{\micro\henry\per\meter}} اور \عددی{\epsilon_1=\SI{120}{\pico\farad\per\meter}} ہیں جبکہ ضیاع کار خطہ \عددی{z>0} کے مستقل \عددی{\sigma_1=\SI{0.02}{\siemens\per\meter}}، \عددی{\mu_1=\SI{50}{\micro\henry\per\meter}} اور \عددی{\epsilon_1=\SI{260}{\pico\farad\per\meter}} ہیں۔آمدی موج \عددی{\kvec{E}_s=10e^{-\alpha_1 z} \cos(9\times 10^8 t -\beta_1 z) \, \si{\volt\per\meter}} ہے۔ الف) \عددی{\alpha_1} اور \عددی{\beta_1} حاصل کریں۔ ب) \عددی{\pmb{\mathscr{P}}_{1\text{اوسط}}^+} اور
 \عددی{\pmb{\mathscr{P}}_{1\text{اوسط}}^-} حاصل کریں۔ پ) \عددی{\pmb{\mathscr{P}}_{2\text{اوسط}}^+} کی مساوات حاصل کریں۔

جوابات:\عددی{\alpha_1=\SI{0}{\neper\per\meter}}، \عددی{\beta_1=\SI{54}{\radian\per\meter}}، \عددی{\pmb{\mathscr{P}}_{1\text{اوسط}}^+=100\az \, \si{\milli\watt\per\meter\squared}}، \عددی{\pmb{\mathscr{P}}_{1\text{اوسط}}^-=-0.486\az \, \si{\milli\watt\per\meter\squared}}،\\
 \عددی{\pmb{\mathscr{P}}_{2\text{اوسط}}^+=99.514 e^{-8.76z}\az \, \si{\milli\watt\per\meter\squared}}
\انتہا{سوال}
%==========================
\ابتدا{سوال}
خطہ \عددی{0<z<\SI{1.5}{\meter}} میں بے ضیاع ذو برق پایا جاتا ہے جس  کے مستقل \عددی{\sigma_2=0}، \عددی{\mu_{R2}=1} اور \عددی{\epsilon_{R2}=6} ہیں۔اس خطے کو دونوں جانب خالی خلاء پائی جاتی ہے۔مستوی موج جس کی تعدد \عددی{\omega=\SI{6e8}{\radian\per\meter}} ہے سرحد \عددی{z=0} کی جانب \عددی{\az} سمت میں حرکت کر رہی ہے۔ الف)  ذو برق میں \عددی{\beta_2} حاصل کرتے ہوئے سرحد \عددی{z=0} پر \عددی{\eta_{\text{داخلی}}} حاصل کریں۔ ب) خطہ \عددی{z<0} میں \عددی{\Gamma_{1}} اور \عددی{s_1} حاصل کریں۔ پ) ذو برقی میں \عددی{z=\SI{1.5}{\meter}} پر سرحد سے منعکس موج کو استعمال کرتے ہوئے \عددی{\Gamma_2} اور \عددی{s_2} حاصل کریں۔ ت) خطہ \عددی{z>\SI{1.5}{\meter}} میں \عددی{s_3} حاصل کریں۔ ٹ) خطہ \عددی{z<0} میں سرحد کے قریب ترین ایسا نقطہ حاصل کریں جہاں بلند تر برقی میدان پایا جاتا ہے۔ 

جوابات:\عددی{\beta_2=\SI{2}{\radian\per\meter}}، \عددی{\eta_{\text{داخلی}}=77.69-j66.76 \, \si{\ohm}}، \عددی{\Gamma_1=-0.623-j0.238=0.667e^{-j2.776}}، \عددی{s_1=5}، \عددی{\Gamma_2=0.42}، \عددی{s_2=2.45}، \عددی{s_3=1}، \عددی{z=\SI{-0.924}{\meter}}
\انتہا{سوال}
%=========================
\ابتدا{سوال}
ضیاع کار خطہ جہاں \عددی{\alpha=\SI{0.4}{\neper\per\meter}} ہو میں موج \عددی{\SI{100}{\meter}} چلنے کے بعد سرحد سے منعکس ہو کر واپس اسی ابتدائی نقطے تک پہنچتی ہے۔انعکاسی مستقل \عددی{\Gamma=0.4-j0.5} ہے۔واپس آتی موج اور ابتدائی موج کے طاقت کی شرح حاصل کریں۔

جواب:\عددی{\num{1.33e-70}}
\انتہا{سوال}
%=========================
\ابتدا{سوال}
خطہ \عددی{z<0} اور خطہ \عددی{z>0} کامل ذو برق پر مشتمل ہیں جہاں \عددی{\sigma=0} اور \عددی{\mu_R=1} ہیں۔تعدد \عددی{\SI{2e10}{\radian\per\second}} کی موج \عددی{\az} سمت میں حرکت کرتے ہوئے دونوں خطوں سے گزرتی ہے۔ان خطوں میں طول موج بالترتیب \عددی{\SI{8}{\centi\meter}} اور \عددی{\SI{6}{\centi\meter}} ہیں۔الف) \عددی{\Gamma} حاصل کریں۔ ب) کتنی فی صد طاقت منعکس پذیر ہوتی ہے۔ پ) کتنی فی صد طاقت ترسیل ہوتی ہے۔ ت) شرح ساکن موج \عددی{s} حاصل کریں۔

جوابات:\عددی{\Gamma=0.143e^{j\pi}}، \عددی{\SI{2.04}{\percent}}، \عددی{\SI{97.96}{\percent}}، \عددی{s=1.333}
\انتہا{سوال}
%===========================
\ابتدا{سوال}
کامل ذو برقی \عددی{\sigma=0} سے خالی خلاء میں موج داخل ہوتی ہے۔مندرجہ ذیل صورتوں میں ذو برق کی جزوی برقی مستقل \عددی{\epsilon_R} حاصل کریں۔ الف) منعکس موج کی چوٹی آمدی موج کے چوٹی کی آدھی ہے۔ ب) منعکس موج کا طاقت آمدی موج کے طاقت کا آدھا ہے۔ پ) ذو برقی میں \عددی{\abs{\kvec{E}}_{\text{کمتر}}} کی قیمت \عددی{\abs{\kvec{E}}_{\text{بلندتر}}} کی آدھی ہے۔

جوابات:\عددی{\epsilon_R=9}، \عددی{\epsilon_R=34}، \عددی{\epsilon_R=4}
\انتہا{سوال}
