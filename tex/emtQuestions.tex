\باب{سوالات}
%=====================
\ابتدا{سوال}
لامحدود لمبائی کی سیدھی تار \عددی{y } محدد پر پڑی ہے۔اس میں \عددی{\ay} جانب \عددی{\SI{5}{\milli\ampere}} برقی رو گزر رہی ہے۔ نقطہ \عددی{N(2,5,3)} پر مقناطیسی میدان \عددی{\kvec{H}} اور \عددی{\abs{\kvec{H}}} حاصل کریں۔اگر تار \عددی{x=3}، \عددی{z=-1} پر ہو تب جوابات کیا ہوں گے۔دونوں تاروں کی موجودگی میں جوابات حاصل کریں۔ 

جوابات: \عددی{\kvec{H}=184\ax-122\az \, \si{\micro\ampere\per\meter}}، \عددی{\abs{\kvec{H}}=\SI{221}{\micro\ampere\per\meter}}، \عددی{\kvec{H}=187\ax+47\az \, \si{\micro\ampere\per\meter}}، \عددی{\abs{\kvec{H}}=\SI{193}{\micro\ampere\per\meter}}، \عددی{{\kvec{H}=371\ax-75\az \, \si{\micro\ampere\per\meter}}}، \عددی{\abs{\kvec{H}}=\SI{378}{\micro\ampere\per\meter}}
\انتہا{سوال}
%======================
\ابتدا{سوال}
سطح \عددی{z=0} پر \عددی{y} محدد کے متوازی لامحدود لمبائی کے آٹھ عدد تار پڑے ہیں جن میں \عددی{\ay} جانب \عددی{\SI{1}{\ampere}} برقی رو گزر رہی ہے۔یہ تار \عددی{y=-3.5}، \عددی{y=-2.5}، \عددی{\cdots}، \عددی{y=2.5}، \عددی{y=3.5} پر پائے جاتے ہیں۔نقطہ \عددی{(0,0,1)} اور \عددی{(0,0,50)} پر \عددی{\kvec{H}} حاصل کریں۔

جوابات: \عددی{0.421\ax}، \عددی{0.0254\ax}؛ محدد پر پچاس گنا دور میدان صرف سترہ گنا کم ہے۔
\انتہا{سوال}
%======================
\ابتدا{سوال}
چار میٹر لمبے تار کو چکور کی شکل دی جاتی ہے جس کا رقبہ \عددی{\SI{1}{\meter\squared}} ہے۔اس چکور کو \عددی{z=0} سطح پر رکھا جاتا ہے۔تار میں برقی رو \عددی{\SI{10}{\milli\ampere}} گزرنے کی صورت میں چکور کے وسط \عددی{N(0,0,0)} میں مقناطیسی میدان \عددی{H} حاصل کریں۔نقطہ \عددی{P(0,0,1)} پر بھی میدان حاصل کریں۔

جوابات:\عددی{\SI{9}{\milli\ampere\per\meter}}، \عددی{\SI{2.3}{\milli\ampere\per\meter}}
\انتہا{سوال}
%=======================
\ابتدا{سوال}
ایک تار کو دائری شکل دے کر سطح \عددی{z=0} پر رکھا جاتا ہے۔دائرے کا رقبہ \عددی{\SI{1}{\meter\squared}} ہے۔تار میں \عددی{\SI{10}{\milli\ampere}} گزرنے کی صورت میں دائرے کے وسط \عددی{N(0,0,0)} میں مقناطیسی میدان \عددی{H} حاصل کریں۔نقطہ \عددی{P(0,0,1)} پر بھی میدان حاصل کریں۔

جوابات: \عددی{\SI{2.82}{\milli\ampere\per\meter}}، \عددی{\SI{1.86}{\milli\ampere\per\meter}}
\انتہا{سوال}
%========================
\ابتدا{سوال}
محدد \عددی{x} اور \عددی{y} میں بڑھتے جانب \عددی{\SI{55}{\milli\ampere}} برقی رو گزر رہی ہے۔نقطہ \عددی{N(5,6,4)} پر \عددی{\kvec{H}} حاصل کریں۔

جواب:\عددی{854\ax-673\ay-57\az \, \si{\micro\ampere\per\meter}}
\انتہا{سوال}
%=======================
\ابتدا{سوال}
سطحی رو \عددی{\kvec{K}=\tfrac{8}{\rho}\aphi \, \si{\ampere\per\meter}} خطہ \عددی{\rho=\SI{3}{\meter}} تا \عددی{\rho=\SI{7}{\meter}} میں پائی جاتی ہے۔سطح \عددی{\phi=0} سے گزرتی کل برقی رو حاصل کریں۔نقطہ \عددی{N(0,0,z)} پر \عددی{\kvec{H}} حاصل کرتے ہوئے میدان کی قیمت \عددی{z=10} پر دریافت کریں۔

جوابات: \عددی{I=8\ln \tfrac{7}{3} \, \si{\ampere}}، \عددی{\kvec{H}=\left[\tfrac{4}{\sqrt{z^2+3^2}}-\tfrac{4}{\sqrt{z^2+7^2}}\right]\, \az}، \عددی{\SI{55.4}{\milli\ampere\per\meter}}
\انتہا{سوال}
%======================
\ابتدا{سوال}
سطحی رو \عددی{\kvec{K}=8\rho\aphi \, \si{\ampere\per\meter}} خطہ \عددی{\rho=\SI{3}{\meter}} تا \عددی{\rho=\SI{7}{\meter}} میں پائی جاتی ہے۔سطح \عددی{\phi=0} سے گزرتی کل برقی رو حاصل کریں۔نقطہ \عددی{N(0,0,z)} پر \عددی{\kvec{H}} حاصل کرتے ہوئے میدان کی قیمت \عددی{z=10} پر دریافت کریں۔

جوابات:\عددی{I=\SI{160}{\ampere}}، \عددی{\kvec{H}=4\left[\tfrac{2z^2+49}{\sqrt(z^2+49)}-\tfrac{2z^2+9}{\sqrt{z^2+9}}\right]\, \az}، \عددی{\SI{1.52}{\ampere\per\meter}}
\انتہا{سوال}
%==========================
\ابتدا{سوال}
رداس \عددی{a} کے دائری چادر پر یکساں سطحی کثافت چارج \عددی{\rho_S} پائی جاتی ہے۔چادر کے محور کو محدد کے مرکز پر رکھتے ہوئے چادر کو سطح \عددی{z=0} پر رکھا جاتا ہے۔اگر چادر محور کے گرد زاویائی رفتار \عددی{\omega} سے گھوم رہی ہو تب نقطہ \عددی{N(0,0,z)} پر مقناطیسی میدان \عددی{H} کیا ہو گا؟ میدان کی قیمت \عددی{\rho_S=\SI{5}{\micro\coulomb\per\meter\squared}} اور \عددی{\omega=100\pi \, \si{\radian\per\second}} کی صورت میں \عددی{(0,0,0.1)} پر حاصل کریں۔

جوابات: \عددی{\tfrac{\omega \rho_S}{2}\left[\tfrac{2z^2+4}{\sqrt{z^2+4}}-2z\right]}، \عددی{\SI{1.42}{\milli\ampere\per\meter}}
\انتہا{سوال}
%=======================
\ابتدا{سوال}
سطح \عددی{z=0} پر خطہ \عددی{x=\SI{-3}{\meter}} تا \عددی{x=\SI{3}{\meter}} پر سطحی برقی رو \عددی{\kvec{K}=4\ay \, \si{\ampere\per\meter}} پائی جاتی ہے۔نقطہ \عددی{N(0,0,5)} پر مقناطیسی میدان حاصل کریں۔

جواب:\عددی{0.688 \ax \, \si{\ampere\per\meter}}
\انتہا{سوال}
%========================
\ابتدا{سوال}
سطح \عددی{x=0} پر سطحی برقی رو \عددی{1200\az \, \si{\ampere\per\meter}} پائی جاتی ہے۔خطہ \عددی{0<z<\infty}، \عددی{5<y<15} پر برقی رو سے نقطہ \عددی{N(10,0,0)} پر پیدا مقناطیسی میدان \عددی{\kvec{H}} حاصل کریں۔

جواب: \عددی{\kvec{H}=45.6\ax+49.6\ay \, \si{\ampere\per\meter}}
\انتہا{سوال}
%=======================
\ابتدا{سوال}
خطہ \عددی{0<z<5} میں یکساں کثافت برقی رو \عددی{15\ay \, \si{\ampere\per\meter\squared}} پائی جاتی ہے۔ایمپیئر  کے دوری قانون کی مدد سے ثابت کریں کہ \عددی{{\kvec{H}_{z<0}=-\kvec{H}_{z>5}}} کے برابر ہے۔نقطہ \عددی{(2,5,7)} اور نقطہ \عددی{(4,12,2)} پر \عددی{\kvec{H}} حاصل کریں۔

جوابات:\عددی{37.5 \ax \, \si{\ampere\per\meter}}، \عددی{-7.5\ax \, \si{\ampere\per\meter}}
\انتہا{سوال}
%=======================
\ابتدا{سوال}
محدد کے مرکز پر رداس \عددی{a} کا موصل کرہ پایا جاتا ہے۔منفی \عددی{z} محدد پر \عددی{10\az \, \si{\ampere}} کی برقی رو، کرہ کی سطح پر نقطہ \عددی{(0,0,-a)} تک پہنچتی ہے جہاں سے یہ کرہ کے سطح پر یکساں پھیل کر نقطہ \عددی{(0,0,a)} تک پہنچتی ہے اور اس کے بعد مثبت \عددی{z} محدد پر بڑھتے جانب چلے جاتی ہے۔کرہ کے اندر اور اس کے باہر مقناطیسی میدان حاصل کریں۔

جوابات:\عددی{\SI{0}{\ampere\per\meter}}، \عددی{\tfrac{10}{2\pi \rho} \aphi \, \si{\ampere\per\meter}}   
\انتہا{سوال}
%========================
\ابتدا{سوال}
منفی \عددی{z} محدد سے برقی رو \عددی{I} موصل  \عددی{\theta=30^{\circ}} سطح تک پہنچ کر سطح پر یکساں پھیل کر چلے جاتی ہے۔نقطہ \عددی{(0,0,z)} اور نقطہ \عددی{(5,5,5)} پر مقناطیسی میدان \عددی{H} حاصل کریں۔

جوابات: \عددی{\SI{0}{\ampere\per\meter}}، \عددی{\tfrac{I}{2\pi \sqrt{50}} \, \si{\ampere\per\meter}}
\انتہا{سوال}
%=========================
\ابتدا{سوال}\شناخت{سوال_مقناطیسی_میدان_گردش_جزو}
تفاعل \عددی{\kvec{G}=(5x+yz)\ax+3xyz\ay+\tfrac{x^2 y}{z}\az} نقطہ \عددی{N(0.6,0.4,0.2)} اور اس کے قریب پایا جاتا ہے۔سطح \عددی{z=0.2} پر \عددی{2a} لمبائی کے اطراف کے مربع لکیر پر \عددی{\oint \kvec{G} \cdot \dif \kvec{L}} حاصل کریں جہاں مربع کا مرکز نقطہ \عددی{N} پر ہے۔لکیری تکمل کو مربع کے رقبے سے تقسیم کریں اور \عددی{a \to 0} لیتے ہوئے \عددی{\nabla \times \kvec{G}_z} حاصل کریں۔

جوابات:چاروں اطراف کے لکیری تکمل \عددی{0.48a^2+0.288a}، \عددی{0.48a^2-0.288a}، \عددی{-0.4a^2+6.16a}  اور \عددی{-0.4a^2-6.16a} ہیں۔یوں \عددی{\nabla \times \kvec{G}_z=0.04} ہے۔
\انتہا{سوال}
%========================
\ابتدا{سوال}
مساوات \حوالہ{مساوات_مقناطیسی_کارتیسی_گردش_ب} استعمال کرتے ہوئے تفاعل \عددی{\kvec{G}=(5x+yz)\ax+3xyz\ay+\tfrac{x^2 y}{z}\az} کا نقطہ \عددی{N(0.6,0.4,0.2)} پر \عددی{\nabla \times \kvec{G}} حاصل کریں۔سوال \حوالہ{سوال_مقناطیسی_میدان_گردش_جزو} میں حاصل کئے گئے \عددی{\nabla \times \kvec{G}_z} کے ساتھ موازنہ کریں۔

جواب: \عددی{1.08\ax-2\ay+0.04\az}
\انتہا{سوال}
%====================

\ابتدا{سوال}
ہم محوری تار میں \عددی{\kvec{E}=3000 \rho^{1.3} \cos(\omega t -0.3z) \arho \, \si{\volt\per\meter}} پایا جاتا ہے۔تار میں \عددی{\nabla \times \kvec{E}} حاصل کریں۔

جواب:\عددی{900 \rho^{1.3}\sin(\omega t -0.3 z) \aphi}
\انتہا{سوال}
%=========================
\ابتدا{سوال}
میدان \عددی{V=5(x^2+y^2)} اور \عددی{V=10x^3+y^2+xz^4} کے لئے \عددی{\nabla^2 V} اور \عددی{\nabla \times \kvec{E}} حاصل کریں۔

جوابات:\عددی{20}، \عددی{0}، \عددی{60x+2+12xz^2}، \عددی{0}
\انتہا{سوال}
%=========================
\ابتدا{سوال}
میدان \عددی{\kvec{H}=x^2y^2z\ax-xy^2z^2\az} دیا گیا ہے۔مسئلہ سٹوکس کے دونوں اطراف باری باری استعمال کرتے ہوئے،  سطح \عددی{y=1} میں خطہ \عددی{{1<x<2}}، \عددی{{1<z<3}} سے \عددی{\ay} جانب گزرتی برقی رو حاصل کریں۔

جواب:\عددی{\SI{13.3}{\ampere}}
\انتہا{سوال}
%========================

\ابتدا{سوال}
میدان \عددی{\kvec{H}=\tfrac{2xy}{z^2}\ax-\tfrac{y^2}{z^2}\ay+x^2y^2\az} دیا گیا ہے۔سطح \عددی{x=0.5} میں خطہ \عددی{{1<y<2}}، \عددی{{2<z<3}} سے \عددی{\ax} جانب گزرتی برقی رو درکار ہے۔الف) برقی رو کو بذریعہ سطحی تکمل حاصل کریں۔ ب) برقی رو کو بذریعہ لکیری تکمل حاصل کریں

جواب:\عددی{\SI{0.426}{\ampere}}
\انتہا{سوال}
%========================
\ابتدا{سوال}
کروی محدد میں میدان \عددی{\kvec{H}=\tfrac{50r}{\sin\theta} \aphi} دیا گیا ہے۔مسئلہ سٹوکس کے دونوں اطراف باری باری استعمال کرتے ہوئے کروی خطہ \عددی{{r=0.2}}، \عددی{0<\phi<2\pi}، \عددی{0<\theta<60^{\circ}} سے گزرتی برقی رو حاصل کریں۔

جواب:\عددی{\SI{5.44}{\ampere}}
\انتہا{سوال}
%=======================
\ابتدا{سوال}
میدان \عددی{\kvec{H}=\tfrac{4r^2}{\sin \theta}\atheta+50r\sin\theta\aphi} دیا گیا ہے۔سطح \عددی{\theta=45^{\circ}} میں خطہ \عددی{0<\phi<2\pi}، \عددی{0<r<3} سے \عددی{\atheta} جانب گزرتی برقی رو حاصل کریں۔ 

جواب:\عددی{\SI{-1414}{\ampere}}
\انتہا{سوال}
%=======================
\ابتدا{سوال}
پاکستان میں کل زمینی مقناطیسی میدان \عددی{\SI{45}{\micro\tesla}} تا \عددی{\SI{50}{\micro\tesla}} پایا جاتا ہے جس کا افقی جزو اوسطاً \عددی{\SI{30}{\micro\tesla}} کے لگ بھگ ہے۔ایک تار جس میں \عددی{\SI{1}{\ampere}} کی برقی رو گزر رہی ہو، کتنے فاصلے پر زمینی میدان کے افقی جزو برابر مقناطیسی میدان پیدا کرے گی۔

جواب:\عددی{\SI{4.2}{\centi\meter}} 
\انتہا{سوال}
%=======================
\ابتدا{سوال}
بیس سنٹی میٹر لمبائی اور پانچ سنٹی میٹر رداس کا پیچدار لچھا جس میں برقی رو گزر رہی ہو میں مقناطیسی میدان \عددی{\kvec{H}=200\az \, \si{\ampere\per\meter}} پایا جاتا ہے۔نقطہ \عددی{N_1(0.02,0^{\circ},0.02)} اور \عددی{N_2(0.04,50^{\circ},0.06)} کے درمیان غیر سمتی مقناطیسی دباو \عددی{V_{m21}} حاصل کریں۔سمتی مقناطیسی دباو \عددی{\nabla \times \kvec{B}=\kvec{A}} سے حاصل کرتے ہوئے \عددی{\rho=0} پر \عددی{\kvec{A}=0} لیتے ہوئے انہیں دو نقطوں کے مابین \عددی{\kvec{A}_{21}} حاصل کریں۔یہ مساوات استعمال کرتے وقت \عددی{\kvec{A}} کا درست جزو اپنے علم سے چنیں۔

جوابات: \عددی{\SI{-8}{\ampere}}، \عددی{2.5\aphi \, \si{\micro\weber\per\meter}}
\انتہا{سوال}
%======================
\ابتدا{سوال}
نلکی کثافت برقی رو \عددی{50\az \, \si{\ampere\per\meter}} رداس \عددی{\rho=\SI{2}{\meter}} پر پائی جاتی ہے جبکہ رداس \عددی{\rho=\SI{4}{\meter}} پر \عددی{25\az \, \si{\ampere\per\meter}} اور رداس \عددی{\rho=\SI{5}{\meter}} پر \عددی{-40\az \, \si{\ampere\per\meter}} پائے جاتے ہیں۔زاویہ \عددی{\phi=0} پر \عددی{V_m=0} لیتے اور \عددی{\phi=180^{\circ}} کو رکاوٹ تصور کرتے ہوئے  نقطہ \عددی{N_1(3.5,60^{\circ},0)}  غیر سمتی مقناطیسی دباو \عددی{V_m} حاصل کریں۔

جوابات: \عددی{\SI{-66.6}{\ampere}}
\انتہا{سوال}
%=======================
\ابتدا{سوال}
سطح \عددی{z=0} پر تار \عددی{x=4} میں \عددی{\SI{0.2}{\ampere}} کی برقی رو \عددی{\ay} جانب پائی جاتی ہے جبکہ تار \عددی{x=-4} میں \عددی{\SI{0.2}{\ampere}} برقی رو \عددی{-\ay} جانب پائی جاتی ہے۔محدد کے مرکز پر \عددی{V_m=0} لیتے ہوئے \عددی{z} محدد پر غیر سمتی مقناطیسی دباو \عددی{V_m} حاصل کریں۔

جواب:\عددی{-\tfrac{0.1}{\pi}\tan^{-1} \tfrac{z}{4}\, \si{\ampere}}
\انتہا{سوال}
%=======================
