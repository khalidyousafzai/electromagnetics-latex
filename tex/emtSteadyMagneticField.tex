\باب{برقرار مقناطیسی میدان}
برقی میدان کا منبع برقی چارج ہے جس پر باب \حوالہ{باب_کولومب_قانون} میں تفصیلی غور کیا گیا۔مقناطیسی میدان کا منبع یا تو مقناطیس ہو سکتا ہے، یا وقت کے ساتھ بدلتا برقی میدان اور یا پھر برقی رو۔اس کتاب میں مقناطیس سے پیدا مقناطیسی میدان پر غور نہیں کیا جائے گا۔وقت کے ساتھ بدلتے برقی میدان سے پیدا مقناطیسی میدان پر ایک اور باب میں غور کیا جائے گا جبکہ اس باب میں برقی رو سے پیدا مقناطیسی میدان پر غور کیا جائے گا۔

\حصہ{بایوٹ-سیوارٹ کا قانون}
برقی رو اور اس سے پیدا مقناطیسی میدان کا تعلق \اصطلاح{بایوٹ-سیوارٹ}\فرہنگ{بایوٹ-سیوارٹ}\حاشیہب{Biot-Savart law}\فرہنگ{Biot-Savart law} کا قانون\حاشیہد{یہ قانون فرانس کے  بایوٹ اور سیوارٹ نے 1820 میں پیش کیا۔یہ دونوں ایمپیئر کے ساتھی تھے۔}
\begin{align}\label{مساوات_بایوٹ_سیوارٹ_تفرق_شکل}
\dif \kvec{H}=\frac{I \dif \kvec{L} \times \aR}{4 \pi R^2}
\end{align}
 بیان کرتا ہے جہاں سے مقناطیسی شدت \عددیء{\kvec{H}} کی اکائی ایمپیئر فی میٹر \عددیء{(\si{\ampere \per \meter})} حاصل ہوتی ہے۔آئیں اس قانون کا مطلب سمجھیں۔

یہ قانون باریک تار کے انتہائی چھوٹے حصے \عددیء{\dif \kvec{L}} جس میں \عددیء{I} برقی رو گزر رہا ہو سے  نقطہ \عددیء{P} پر پیدا سمتی برقی میدان \عددیء{\kvec{H}} دیتا ہے۔نقطہ \عددیء{P} باریک تار کے چھوٹی لمبائی سے \عددیء{\kvec{R}} فاصلے پر ہے۔باریک تار سے مراد ایسی ٹھوس نلکی نما موصل تار ہے جس کے رقبہ عمودی تراش کا رداس اتنا کم کر دیا جائے کہ یہ صفر کے قریب تر ہو۔\عددیء{\dif \kvec{L}} کی سمت برقی رو کی سمت میں ہے۔

مقناطیسی شدت کی قیمت برقی رو ضرب باریک تار کی لمبائی ضرب \عددیء{\kvec{R}} اور \عددیء{\dif \kvec{L}} کے مابین زاویہ کے سائن کے برائے راست تناسب جبکہ ان کے مابین فاصلہ \عددیء{R} کے مربع کے بالعکس تناسب رکھتی ہے۔تناسب کا مستقل \عددیء{\tfrac{1}{4\pi}} ہے۔

بایوٹ-سیوارٹ کے  قانون کا موازنہ کولومب کے قانون کے ساتھ کرنے کی غرض سے دونوں مساوات کو ایک ساتھ لکھتے ہیں۔
\begin{align*}
\dif \kvec{H}_2&=\frac{I_1 \dif \kvec{L}_1 \times \kvec{a}_{R21}}{4 \pi R_{21}^2}\\
\dif \kvec{E}_2&=\frac{\dif Q_1 \kvec{a}_{R21}}{4\pi \epsilon_0 R_{21}^2}
\end{align*}
ان مساوات میں زیر نوشت میں \عددیء{2} اس مقام کو ظاہر کرتی ہے جہاں میدان کی قیمت حاصل کی گئی ہے جبکہ زیر نوشت میں \عددیء{1} میدان کے منبع کے مقام کو ظاہر کرتی ہے۔دونوں میدان فاصلے کے مربع کا بالعکس تناسب رکھتے ہیں۔دونوں میں میدان اور اس کے منبع کا خطی یا راست تناسب پایا جاتا ہے۔دونوں میں فرق میدان کی سمت کا ہے۔برقی میدان کی سمت منبع سے اس نقطہ کی جانب ہے جہاں میدان حاصل کیا جا رہے ہو۔مقناطیسی میدان کی سمت سمتی ضرب کے دائیں ہاتھ کے قانون سے حاصل ہوتی ہے۔

بایوٹ-سیوارٹ کے قانون کو مساوات \حوالہ{مساوات_بایوٹ_سیوارٹ_تفرق_شکل} کی شکل میں تجرباتی طور پر ثابت نہیں کیا جا سکتا چونکہ باریک تار کے چھوٹی لمبائی میں برقی رو تب گزرے گی جب یہ اس تک پہنچائی جائے۔جو تار اس تک برقی رو پہنچائے گا، وہ بھی مقناطیسی میدان پیدا کرے گا۔انہیں علیحدہ علیحدہ نہیں کیا جا سکتا۔یوں بایوٹ-سیوارٹ قانون کی تکمل شکل
\begin{align}\label{مساوات_بایوٹ_سیوارٹ_تکمل_شکل}
\kvec{H}=\oint \frac{I \dif \kvec{L} \times \aR}{4 \pi R^2}
\end{align}
ہی تجرباتی طور ثابت کی جا سکتی ہے۔

مساوات \حوالہ{مساوات_بایوٹ_سیوارٹ_تفرق_شکل} سے مساوات \حوالہ{مساوات_بایوٹ_سیوارٹ_تکمل_شکل} لکھی جا سکتی ہے۔البتہ مساوات \حوالہ{مساوات_بایوٹ_سیوارٹ_تکمل_شکل} میں تکمل کے اندر کوئی بھی ایسی اضافی تفاعل شامل کیا جا سکتا ہے جس کا بند تکمل صفر کے برابر ہو۔مقداری میدان کا ڈھلان ہر صورت بقائی میدان ہوتا ہے لہٰذا مساوات \حوالہ{مساوات_بایوٹ_سیوارٹ_تکمل_شکل} میں \عددیء{\nabla G} کے شمول سے اس کے جواب میں کوئی فرق نہیں پڑے گا۔\عددیء{G} کوئی بھی مقداری میدان ہو سکتا ہے۔اس حقیقت کا تذکرہ اس لئے کیا جا رہا ہے کہ اگر ہم ایک چھوٹے برقی رو گزارتے تار پر دوسرے  چھوٹے برقی رو گزارتے تار سے پیدا قوت دریافت کرنا چاہیں جہاں تجرباتی طور پر ان کا میدان قابل دریافت نہ ہو تب ہمیں احمقانہ جوابات ہی حاصل ہوں گے۔

بایوٹ-سیوارٹ کے قانون کو سطحی کثافت برقی رو \عددیء{\kvec{K}} یا کثافت برقی رو \عددیء{\kvec{J}} کی صورت میں بھی لکھا جا سکتا ہے جہاں
\begin{align}
I \dif \kvec{L}=\kvec{K} \dif S=\kvec{J} \dif v
\end{align}
لکھا جائے گا۔ یوں بایوٹ-سیوارٹ کے قانون کو
\begin{align}
\kvec{H}=\int_S \frac{\kvec{K} \times \aR \dif S}{4\pi R^2}
\end{align}
یا
\begin{align}
\kvec{H}=\int_h \frac{\kvec{J} \times \aR \dif h}{4\pi R^2}
\end{align}
لکھا جا سکتا ہے۔
