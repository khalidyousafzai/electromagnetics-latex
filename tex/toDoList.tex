\begin{otherlanguage}{english}
complex permitivity\\
dispersion\\
try to resolve the issue of using k and $\gamma$ as propagation constants\\
kraus p473 fig 10-60 sec 10.18 and 10.19, 10.17,10.16,10.15,10.11 or do the entire chap9 and 10  transmission from kraus too\\
in the continuous aperture i have said Hy=Jx where as it is Hy=-Jx. have been unable to correct this. need to speak to someone in engg deptt knowing this topic\\
fourier transform sec 14.8 of kraus must draw the figures\\
charge is barqi bar and let the reader figure out which bar is meant\\
degree angle and degree celcius, ohm, micro etc not showing\\
kraus p581 mentions three types of impedances: intrinsic, characteristic and transverse. ensure that i too have these distinctions\\
kraus fig-13.28 and fig 13.29 and table 13.3 (on p577) are v. impt\\
Huygens improvements\\
figTransmissionSmithFromInternet.tex is not giving the figure of the book\\
the answers should be at the end of the book\\
handle all side notes (حاشیہط) and remove the corresponding text\\
read chapter 9 onwards (proof reading)\\
energy travels along the wire and not in the wire.\\
antenna chapter, 3D figure at start and complete the start section.\\
house completion certificate.\\
zaryab fish\\
F=-dW/dT to include in inductance chapter plus a question or two\\
magnetizartion curve and an iteration example. fig 8.10, 11 of hayt.\\
add questions to machine book too.\\
\\
when giving fields always remember the following rules:\\
always ensure that divergence of magnetic field is zero.\\
moving waves must be of the form $E=E0 \cos(wt-kz)$ where $c=(\mu*\epsilon)^{-0.5}$ and $k=2*\pi/\lambda$\\
include complex permitivity  (7th ed Q12.18 says sigma=omega*epsilon")\\
include 4th ed fig 11.11 of page 422
\end{otherlanguage}
