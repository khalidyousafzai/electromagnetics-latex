\باب{گاوس کا قانون اور پھیلاو}
\حصہ{ساکن برقی بار}

\حصہ{فیراڈے کا تجربہ}
اس باب کا آغاز \ترچھا{مائکل فیراڈے}\حاشیہب{Michael Faraday} کے ایک تجربہ سے کرتے ہیں جس کا نتیجہ کچھ اس طرح بیان کیا جا سکتا ہے۔برقی بار \عددیء{Q} کو بے بار موصل سطح میں مکمل طور پر یوں بند کرنے کے بعد، کہ بار اور سطح  کہیں بھی ایک دوسرے کو نہ چھوئیں، موصل سطح کو زمین کے ساتھ ایک لمحے کے لئے ملانے سے موصل سطح پر \عددیء{-Q} بار پیدا ہو جاتا ہے۔دیکھا گیا ہے کہ بار اور بیرونی سطح کے درمیان فاصلہ کم یا زیادہ کرنے سے نتیجہ اثر انداز نہیں ہوتا۔اسی طرح بار اور سطح کے درمیان مختلف غیر موصل مواد بھرنے سے بھی نتیجہ اثر انداز نہیں ہوتا۔مزید یہ کہ سطح کی شکل سے بھی نتیجہ اثر انداز نہیں ہوتا۔اسی طرح جس چیز پر بار \عددیء{Q} رکھا گیا ہو، اس کی شکل سے بھی نتیجہ اثر انداز نہیں ہوتا۔ 

ایسا معلوم ہوتا ہے جیسے اندرونی بار سے بیرونی سطح تک بار کی مقدار اور قطب کی خبر پہنچتی ہے۔اس حقیقت کو تصوراتی جامہ یوں پہنایا جا سکتا ہے کہ ہم سمجھیں کہ مثبت بار سے ہر جانب یکساں طور پر کچھ خارج ہوتا ہے۔اس چیز کو ہم \اصطلاح{برقی بہاو}\فرہنگ{برقی بہاو}\حاشیہب{electric flux}\فرہنگ{electric flux}  کہیں گے اور اس کو \عددیء{\psi} سے ظاہر کریں گے۔برقی بہاو کو بار کے برابر تصور کیا جاتا ہے۔
\begin{align}
\psi=Q
\end{align}
برقی بہاو کی اکائی  کولمب \عددیء{\si{\coulomb}} ہی تصور کی جاتی ہے۔منفی بار کی صورت میں برقی بہاو کی سمت الٹی ہو گی اور یہ بار میں داخل ہو گا۔

تصور کریں کہ اندرونی بار  \عددیء{r_1} رداس کے کرہ پر پایا جاتا ہے جبکہ اسے  \عددیء{r_2} رداس کے کرہ نے گھیرا ہوا ہے۔کرہ کی سطح \عددیء{4\pi r^2} کے برابر ہوتی ہے۔اندرونی کرہ سے  \عددیء{\psi} برقی بہاو خارج ہوتا ہے۔یوں اندرونی کرہ سے \عددیء{\tfrac{\psi}{4\pi r_1^2}} برقی بہاو فی اکائی رقبہ خارج ہوتا ہے  جسے \عددیء{\tfrac{Q}{4\pi r_1^2}} لکھا جا سکتا ہے۔ اسی طرح بیرونی کرہ پر \عددیء{\tfrac{Q}{4\pi r_2^2}} برقی بہاو فی اکائی رقبہ پہنچتا ہے۔برقی بہاو فی اکائی رقبے کو \اصطلاح{کثافت برقی بہاو}\فرہنگ{کثافت!برقی بہاو}\حاشیہب{electric flux density}\فرہنگ{density!electric flux} \عددیء{D} کہا جائے گا۔یوں اگر اندرونی کرہ کے رداس کو اتنا کم کر دیا جائے کہ اس کو نقطہ تصور کرنا ممکن ہو اور اس نقطہ بار کو رداس \عددیء{r} کے کرہ کے مبدا پر رکھا جائے تو کرہ پر
\begin{align}\label{مساوات_گاوس_نقطہ_بار_کے_بہاو_کی_کثافت}
\kvec{D}=\frac{Q}{4\pi r^2} \ar
\end{align}
سمتی کثافت برقی بہاو پائی جائے گی۔صفحہ \حوالہصفحہ{مساوات_کولمب_نقطہ_بار_کروی_رداس_پر_تبدیل_نہیں_ہوتا} پر مساوات \حوالہ{مساوات_کولمب_نقطہ_بار_کروی_رداس_پر_تبدیل_نہیں_ہوتا} سے موازنہ کرنے سے ثابت ہوتا ہے کہ خلاء میں
\begin{align}\label{مساوات_گاوس_میدان_اور_کثافت_کا_تعلق}
\kvec{D}=\epsilon_0 \kvec{E} \hspace{2cm} \textrm{خلاء}
\end{align}
کے برابر ہے۔اگر نقطہ بار کو کروی محدد کے مبدا پر نہ رکھا جائے تب کسی بھی مقام پر کثافت برقی بہاو حاصل کرنے کی خاطر مساوات \حوالہ{مساوات_گاوس_نقطہ_بار_کے_بہاو_کی_کثافت} یوں لکھی جائے گی
\begin{align}\label{مساوات_گاوس_کسی_بھی_جگہ_نقطہ_بار_کے_بہاو_کی_کثافت}
\kvec{D}=\frac{Q}{4\pi R^2} \kvec{a}_R
\end{align}
جہاں \عددیء{\kvec{a}_R} بار  سے اس مقام کی جانب اکائی سمتیہ ہے اور \عددیء{R} ان کے درمیان فاصلہ ہے۔

ایسا حجم جس میں تغیر پذیر  بار کی کثافت پائی جائے میں مقام \سمتیہ{r'} پر \عددیء{\Delta h'} حجم میں \عددیء{\rho_h' \Delta h'}  بار پایا جائے گا جو مقام \سمتیہ{r} پر
\begin{align*}
\Delta \kvec{D}(\kvec{r})=\frac{\rho_h' \Delta h'}{4\pi \abs{\kvec{r}-\kvec{r'}}^2} \frac{\kvec{r}-\kvec{r'}}{\abs{\kvec{r}-\kvec{r'}}}
\end{align*} 
کثافت برقی بہاو پیدا کرے گی۔قانون کولمب خطی ہونے کی بنا پر \عددیء{\kvec{D}} بھی خطی نوعیت کا ہوتا ہے لہٰذا حجم کے تمام باروں سے
\begin{align}\label{مساوات_گاوس_حجم_بار_کے_بہاو_کی_کثافت}
\kvec{D}(\kvec{r})=\int\limits_{h}\frac{\rho_h' \dif h'}{4\pi \abs{\kvec{r}-\kvec{r'}}^2} \frac{\kvec{r}-\kvec{r'}}{\abs{\kvec{r}-\kvec{r'}}}
\end{align} 
حاصل ہو گا۔مساوات \حوالہ{مساوات_گاوس_حجم_بار_کے_بہاو_کی_کثافت} کا صفحہ \حوالہصفحہ{مساوات_کولمب_حجم_بار_کا_میدان} پر مساوات \حوالہ{مساوات_کولمب_حجم_بار_کا_میدان} کے ساتھ موازنہ کرنے سے ثابت ہوتا ہے کہ حجمی کثافت کے لئے بھی مساوات \حوالہ{مساوات_گاوس_میدان_اور_کثافت_کا_تعلق} خلاء میں \سمتیہ{D} اور \سمتیہ{E} کا تعلق بیان کرتا ہے۔اسی طرح \عددیء{\rho_S} اور \عددیء{\rho_L} سے پیدا \عددیء{\kvec{D}} اور \عددیء{\kvec{E}} کا خلاء میں تعلق بھی مساوات \حوالہ{مساوات_گاوس_میدان_اور_کثافت_کا_تعلق}  ہی بیان کرتا ہے۔یوں  مساوات \حوالہ{مساوات_گاوس_میدان_اور_کثافت_کا_تعلق} ایک عمومی مساوات ہے۔

\حصہ{گاوس کا قانون}
فیراڈے کے تجربے کو قانون کی شکل میں یوں پیش کیا جا سکتا ہے جسے \اصطلاح{گاوس کا قانون}\فرہنگ{گاوس کا قانون}\حاشیہب{Gauss's law}\فرہنگ{Gauss's law} کہتے ہیں۔

\ابتدا{قانون}
کسی بھی مکمل بند سطح سے  کل گزرتی برقی بہاو سطح میں گھیرے بار کے برابر ہوتی ہے۔
\انتہا{قانون}
گاوس\حاشیہب{Carl Friedrich Gauss} نے اس قانون کو ریاضیاتی شکل دی جس کی بنا پر یہ قانون انہیں کے نام سے منسوب ہے۔آئیں گاوس کے قانون کی ریاضیاتی شکل حاصل کریں۔
\begin{figure}
\centering
\includegraphics{figGaussBalloon}
\caption{مکمل بند سطح سے گزرتی برقی بہاو سطح میں گھیرے کل بار کے برابر ہے۔}
\label{شکل_گاوس_کا_قانون}
\end{figure}

شکل \حوالہ{شکل_گاوس_کا_قانون} میں  بند سطح دکھائی گئی ہے جس کی کوئی مخصوص شکل نہیں ہے۔اس سطح کے اندر یعنی سطح کے گھیرے ہوئے حجم میں کل \عددیء{Q} بار  پایا جاتا ہے۔سطح پر کسی بھی مقام سے گزرتا برقی بہاو اس مقام پر سطح کی عمودی سمت میں کثافت برقی بہاو اور اس مقام کے رقبہ کے حاصل ضرب کے برابر ہو گا۔یوں شکل کو دیکھتے ہوئے  چھوٹے سے رقبہ \عددیء{\Delta S} پر سطح کے عمودی سمت میں برقی بہاو کی کثافت  کی قیمت \عددیء{D_S \cos \alpha} ہو گی لہٰذا
\begin{align*}
\Delta \psi =D_S \cos \alpha  \Delta S
\end{align*}
ہو گا۔کثافتِ برقی بہاو \عددیء{D_S} لکھتے ہوئے زیرنوشت میں \عددیء{S} اس حقیقت کی یاد دہانی کراتا ہے کہ سطح پر کثافتِ برقی بہاو کی قیمت کی بات کی جا رہی ہے۔اس مساوات کو ضرب نقطہ کے استعمال سے
\begin{align*}
\Delta \psi =\kvec{D}_S \cdot \Delta \kvec{S}
\end{align*}
لکھا جا سکتا ہے۔مکمل سطح سے گزرتے ہوئے کل برقی بہاو تکملہ سے حاصل ہو گا جو گاوس کے قانون کے مطابق گھیرے ہوئے بار \عددیء{Q} کے برابر ہے۔یوں
\begin{align}\label{مساوات_گاوس_قانون_گاوس_بنیادی_شکل}
\psi=\oint\limits_{S} \kvec{D}_S \cdot \Delta \kvec{S}=Q
\end{align}
لکھا جا سکتا ہے۔یہ تکملہ دراصل دو درجی  تکملہ ہے جسے ہم عموماً ایک درجی تکملہ سے ہی ظاہر کریں گے۔تکملہ کے نشان پر گول دائرہ بند تکملہ\فرہنگ{بند تکملہ}\حاشیہب{closed integral} کو ظاہر کرتا ہے جبکہ بند تکملہ  کے  نیچے \عددیء{S} اس بند سطح کو ظاہر کرتا ہے جس پر بند تکملہ حاصل کیا جا رہا ہو۔اس بند سطح کو عموماً \اصطلاح{گاوس سطح}\فرہنگ{گاوس سطح}\حاشیہب{gaussian surface}\فرہنگ{gaussian surface} کہتے ہیں۔

جس مقام پر بار کی کثافت \عددیء{\rho_h} ہو، وہاں  چھوٹا سا حجم \عددیء{\Delta h} میں کل بار \عددیء{\rho_h \Delta h}  پایا جاتا ہے۔یوں کسی بھی حجم کو چھوٹے چھوٹے حصوں میں تقسیم کرتے ہوئے تمام حصوں میں پائے جانے والے باروں کا مجموعہ پورے حجم میں بار کے برابر ہو گا یعنی
\begin{align}\label{مساوات_گاوس_کل_حجمی_بار}
Q=\int\limits_{h} \rho_h \dif h
 \end{align}
جہاں تین درجی حجم کے تکملہ کو ایک درجی تکملہ کے نشان سے ظاہر کیا گیا ہے۔

مندرجہ بالا دو مساوات سے
\begin{align}\label{مساوات_گاوس_کے_قانون_کی_تکملہ_شکل}
\oint\limits_{S} \kvec{D}_S \cdot \Delta \kvec{S}=\int\limits_{h} \rho_h \dif h
\end{align}
حاصل ہوتا ہے جو گاوس کے قانون کی تکملہ شکل ہے۔اس مساوات کو یوں پڑھا جاتا ہے کہ کسی بھی بند سطح سے گزرتی کل برقی بہاو اس سطح کے اندر گھیرے ہوئے کل بار کے برابر  ہے۔

یہ ضروری نہیں کہ گھیرے ہوئے حجم یعنی بند حجم میں حجمی کثافت ہی پائی جائے۔بند حجم کے اندر  سطحی کثافت، لکیری کثافت، علیحدہ علیحدہ نقطہ بار یا ان تینوں اقسام کا مجموعہ پایا جا سکتا ہے۔حجم گھیرنے والے بند بیرونی سطح کے اندر کسی بھی سطح پر  سطحی کثافت کی صورت میں مساوات \حوالہ{مساوات_گاوس_کل_حجمی_بار} کی جگہ
\begin{align}
Q&=\int\limits_{S} \rho_S \dif S
\end{align}
لکھا جائے گا جہاں بار بردار سطح ازخود بند یا کھلی سطح ہو سکتی ہے۔لکیری کثافت کی صورت میں
\begin{align}
Q&=\int\limits_{L} \rho_L \dif L
\end{align}
جبکہ  \عددیء{n} عدد نقطہ بار کی صورت میں
\begin{align}
Q=\sum_n Q= Q_1+Q_2+Q_3+\cdots+Q_n
\end{align}
لکھا جائے گا، وغیرہ وغیرہ۔بہرحال مساوات \حوالہ{مساوات_گاوس_کل_حجمی_بار} سے مراد یہ تمام صورتیں لی جاتی ہیں اور یوں ان تمام صورتوں کے لئے گاوس کے  قانون کی تکملہ شکل مساوات \حوالہ{مساوات_گاوس_کے_قانون_کی_تکملہ_شکل} ہی استعمال ہو ہے۔


%=============
\حصہ{گاوس کے قانون کا استعمال}
گزشتہ باب میں ہم نے کولمب کے قانون سے نقطہ بار، لامحدود  لکیری بار اور لامحدود سطحی بار  سے پیدا برقی میدان حاصل کئے۔آئیے اب ان کو  گاوس کے قانون کی مدد سے بھی حاصل کریں۔ آپ دیکھیں گے کہ ان تینوں صورتوں میں گاوس کے قانون کا استعمال بے حد سادہ ثابت ہو گا۔ایسے  مسائل جن میں گاوس کے قانون کا استعمال کیا جا سکے  کی تعداد  بہت کم ہے۔

\جزوحصہ{نقطہ بار}
شکل \حوالہ{شکل_گاوس_کرہ_کی_سطح_پر_مرکز_کے_نقطہ_بار_کا_میدان} میں کرہ کے مبدا پر نقطہ بار دکھایا گیا ہے۔ نقطہ بار کو کروی محدد\حاشیہب{spherical coordinates} کے مبدا پر رکھتے ہوئے ہم  نے مختلف مقامات سے دیکھتے ہوئے مسئلے کی مشابہت کی بنا پر اخذ کیا تھا کہ کثافتِ برقی میدان صرف رداس کی سمت میں ممکن ہے اور اس  کی حتمی قیمت صرف اور صرف رداس \عددیء{r} تبدیل کرنے سے تبدیل ہو گی۔اس کا مطلب ہے کہ کروی محدد کے مبدا کے گرد رداس \عددیء{r} کے کرہ  پر \سمتیہ{D} تبدیل نہیں ہو گا۔
\begin{figure}
\centering
\includegraphics{figGaussSphere}
\caption{کرہ کے مبدا پر نقطہ بار کا کرہ کے سطح پر کثافتِ برقی بہاو}
\label{شکل_گاوس_کرہ_کی_سطح_پر_مرکز_کے_نقطہ_بار_کا_میدان}
\end{figure}

کروی محدد استعمال کرتے ہوئے کرہ پر چھوٹی سی سطح
\begin{align*}
\dif S=r^2 \sin \theta \dif \theta \dif \phi
\end{align*}
لکھی جا سکتی ہے۔اسی کی سمتی شکل
\begin{align*}
\dif \kvec{S}=r^2 \sin \theta \dif \theta \dif \phi \ar
\end{align*}
ہو گی۔اس سطح پر کثافتِ برقی بہاو کی قیمت \عددیء{D_S} اور سمت \عددیء{\ar} ہو گی لہٰذا سمتی کثافتِ برقی بہاو 
\begin{align*}
\kvec{D}_S=D_S \ar
\end{align*}
لکھی جائے گی۔یوں اس چھوٹی سی سطح سے گزرتا برقی بہاو
\begin{align*}
\dif \psi&=\kvec{D}_S \cdot \dif \kvec{S}\\
&=\left( D_S \ar \right) \cdot \left(r^2 \sin \theta \dif \theta \dif \phi \ar \right)\\
&= D_S r^2 \sin \theta \dif \theta \dif \phi
\end{align*}
ہو  گا۔اس طرح پورے کرہ سے گزرتا برقی بہاو تکملہ سے یوں حاصل ہو گا۔
\begin{align*}
\psi&=D_S r^2  \int_{\phi=0}^{\phi=2\pi}\int_{\theta=0}^{\theta=\pi} \sin \theta \dif \theta \dif \phi\\
&=D_S r^2 \int_{\phi=0}^{\phi=2\pi} \eval{-\cos \theta}_0^{\pi} \dif \phi\\
&=D_S r^2 \int_{\phi=0}^{\phi=2\pi} 2 \dif \phi\\
&=4 \pi r^2 D_S
\end{align*}
گاوس کے قانون کے تحت یہ برقی بہاو گھیرے گئے بار \عددیء{Q} کے برابر ہے لہٰذا
\begin{align*}
4 \pi r^2 D_S=Q
\end{align*}
ہو گا جس سے
\begin{align*}
D_S=\frac{Q}{4 \pi r^2}
\end{align*}
حاصل ہوتا ہے۔یہی جواب بغیر زیادہ حساب و کتاب کے یوں حاصل کیا جا سکتا ہے۔

کرہ پر کثافتِ برقی بہاو \عددیء{D_S} عمودی ہے اور اس کی قیمت کرہ پر تبدیل نہیں ہوتی۔کرہ کی سطح \عددیء{4\pi r^2} کے برابر ہے لہٰذا پوری سطح سے \عددیء{4\pi r^2 D_S} برقی بہاو گزرے گا جو گاوس کے قانون کے تحت \عددیء{Q} کے برابر ہے لہٰذا \عددیء{4\pi r^2 D_S=Q} ہو گا جس سے \عددیء{D_S=\tfrac{Q}{4\pi r^2}} حاصل ہوتا ہے۔اس کی سمتی شکل
\begin{align}
\kvec{D}_S=\frac{Q}{4\pi r^2} \ar
\end{align}
اور \عددیء{\kvec{D}=\epsilon_0 \kvec{E}} سے
\begin{align}
\kvec{E}=\frac{Q}{4\pi \epsilon_0 r^2} \ar
\end{align}
حاصل ہوتا ہے۔اس مساوات کا صفحہ \حوالہصفحہ{مساوات_کولمب_نقطہ_بار_کروی_رداس_پر_تبدیل_نہیں_ہوتا} پر مساوات \حوالہ{مساوات_کولمب_نقطہ_بار_کروی_رداس_پر_تبدیل_نہیں_ہوتا} کے ساتھ موازنہ کریں اور دیکھیں کہ موجودہ جواب کتنی آسانی سے حاصل ہوا۔

\جزوحصہ{یکساں بار بردار کروی سطح}\شناخت{حصہ_گاوس_کروی_بار_بردار_سطح_کا_میدان}
صفحہ \حوالہصفحہ{مثال_کولمب_کرہ_بار_کا_میدان} پر حصہ \حوالہ{مثال_کولمب_کرہ_بار_کا_میدان} میں کروی محدد کے مبدا پر  \عددیء{a} رداس کی کروی سطح جس پر یکساں \عددیء{\rho_S} کثافتِ بار پائی جائے کا میدان بیرونِ کرہ اور اندرونِ کرہ  سے حاصل کیا گیا۔آئیے گاوس کے قانون سے انہیں جوابات کو دوبارہ حاصل کریں۔

کرہ کے اندر \عددیء{r} رداس کا کرہ لیتے ہیں۔یوں \عددیء{r<a} رداس کے کرہ میں صفر بار پایا جائے گا۔یوں اس کی سطح پر صفر میدان ہو گا۔اس کے برعکس \عددیء{r>a} رداس کا کرہ \عددیء{a} رداس کے کرہ کو گھیرتا ہے لہٰذا یہ \عددیء{4\pi a^2\rho_S} بار کو گھیرے گا لہٰذا یہاں
\begin{align*}
\kvec{D}=\frac{4\pi a^2\rho_S}{4\pi r^2 }\ar
\end{align*}
ہو گا جس سے
\begin{align*}
\kvec{E}=\frac{Q}{4\pi \epsilon_0 r^2 }\ar
\end{align*}
حاصل ہوتا ہے جہاں کُل بار کو \عددیء{Q} لکھا گیا ہے۔یہ نتائج گاوس کے قانون کے استعمال سے حاصل کئے گئے۔ساتھ ہی ساتھ اس حقیقت کو مدنظر رکھا گیا کہ میدان صرف رداسی سمت میں ممکن ہے۔

آپ دیکھ سکتے ہیں کہ اس مسئلے کو حل کرنے کا موجودہ طریقہ نہایت آسان ہے۔

\جزوحصہ{یکساں بار بردار سیدھی لامحدود لکیر}
ایسی لامحدود لکیر جس پر بار کی یکساں کثافت پائی جائے کے گرد رداس پر گھومتے ہوئے  صورت حال میں کوئی تبدیلی نظر نہیں آتی۔اسی طرح اس لکیر کے ساتھ ساتھ چلتے ہوئے بھی صورت حال میں کسی قسم کی تبدیلی پیدا نہیں ہوتی۔لامحدود لکیر کو نلکی محدد کی \عددیء{z} محدد تصور کرتے  ہوئے  ان حقائق کی روشنی میں ہم توقع کرتے ہیں کہ برقی میدان صرف  رداس تبدیل کرنے سے ہی تبدیل ہو گا۔مزید، جیسا کہ پچھلے باب میں بتلایا گیا، کسی بھی نقطے کے ایک جانب لکیر پر بار سے پیدا برقی میدان کا وہ حصہ جو \عددیء{\az} کی سمت میں ہو کو لکیر پر نقطے کی دوسری جانب برابر فاصلے  پر بار سے پیدا برقی میدان کا وہ حصہ جو \عددیء{\az} کی سمت میں ہو  ختم کرتا ہے۔یوں یہ اخذ کیا جا سکتا ہے کہ کثافتِ برقی بہاو صرف رداس کی سمت میں ہی پایا جائے گا۔آئیں ان معلومات کی روشنی میں گاوس کے قانون کی مدد سے کثافتِ برقی بہاو حاصل کریں۔

بار بردار لکیر جس پر یکساں کثافتِ بار \عددیء{\rho_L} پایا جائے  کی لمبائی \عددیء{L} میں کل بار \عددیء{\rho_L L} ہو گا۔اس لمبائی کے گرد \عددیء{\rho} رداس کی نلکی سطح تصور کرتے ہیں جس کے دونوں آخری سرے\حاشیہد{آخری سروں کو غیر موصل چادر سے بند کیا جا سکتا ہے۔یوں اس نلکی سطح تک بار نہیں پہنچ پائے گا۔} بند تصور کریں۔چونکہ برقی بہاو صرف رداس کی سمت میں ہے لہٰذا ان دونوں آخری سروں سے کوئی برقی بہاو نہیں ہو گا۔نلکی سطح کا رقبہ \عددیء{2\pi \rho L} ہے جبکہ اس سطح پر ہر جگہ کثافتِ برقی بہاو  \عددیء{D_{\rho}} ہے لہٰذا پوری سطح سے \عددیء{2\pi \rho L D_{\rho}} برقی بہاو ہو گا جو گاوس کے قانون کے تحت گھیرے گئے بار \عددیء{\rho_L L} کے برابر ہو گا۔اس طرح
\begin{align*}
2 \pi \rho D_{\rho} = \rho_L L
\end{align*}
لکھتے ہوئے
\begin{align*}
D_{\rho}=\frac{\rho_L}{2\pi \rho}
\end{align*}
حاصل ہوتا ہے جس کی سمتی شکل
\begin{align}\label{مساوات_گاوس_لکیری_کثافت_کا_میدان}
\kvec{D}_\rho=\frac{\rho_L}{2\pi \rho} \arho
\end{align}
سے 
\begin{align}\label{مساوات_گاوس_لکیری_بار_کا_میدان}
\kvec{E}_\rho=\frac{\rho_L}{2\pi \epsilon_0\rho} \arho
\end{align}
حاصل ہوتا ہے۔صفحہ \حوالہصفحہ{مساوات_کولمب_لامحدود_لکیر_رداسی_میدان_پیدا_کرتا_ہے} پر مساوات \حوالہ{مساوات_کولمب_لامحدود_لکیر_رداسی_میدان_پیدا_کرتا_ہے} کے ساتھ موازنہ کریں اور دیکھیں کہ موجودہ طریقہ کتنا سادہ ہے۔
%=============
\حصہ{ہم محوری تار}
یکساں بار بردار سیدھی لامحدود لکیر کے قصے کو آگے بڑھاتے ہوئے تصور کریں کہ اس تار کا رداس \عددیء{\rho_1} ہے۔اگر تار پر کسی بھی  جگہ \عددیء{L} لمبائی میں \عددیء{Q} بار پایا جائے تو تار پر بار کی لکیری کثافت  \عددیء{\rho_L=\tfrac{Q}{L}} ہو گی جبکہ اس پر بار کی سطحی کثافت \عددیء{\tfrac{Q}{2\pi \rho_1 L}} ہو گی۔جیسا آپ جانتے ہیں ہیں ٹھوس موصل میں باروں کے مابین قوت دفع کی وجہ سے تمام  بار موصل کے بیرونی سطح پر دھکیلے جاتے ہیں۔یوں بار \عددیء{Q} تار  کے  بیرونی سطح، محور سے \عددیء{\rho_1} فاصلے، پر پایا جائے گا۔

\begin{figure}
\centering
\includegraphics{figGaussCoaxial}
\caption{ہم محوری تار}
\label{شکل_گاوس_ہم_محوری_تار}
\end{figure}

اب تصور کریں کہ پہلی تار کے اوپر نلکی نما دوسری تار چڑھائی جائے جس کا اندرونی رداس \عددیء{\rho_2} ہو جہاں \عددیء{\rho_2 > \rho_1} ہو گا۔ایسی تار جسے \اصطلاح{ہم محوری تار}\فرہنگ{ہم محوری تار}\حاشیہب{coaxial cable}\فرہنگ{coaxial cable} کہتے ہیں کو شکل \حوالہ{شکل_گاوس_ہم_محوری_تار} میں دکھایا گیا ہے۔تصور کریں کہ بیرونی تار پر کسی بھی جگہ \عددیء{L} لمبائی پر \عددیء{-Q} بار پایا جاتا ہے۔دونوں تاروں پر الٹ اقسام کے بار ہیں جن میں قوت کشش پائی جائے گی۔یوں بیرونی تار پر بار تار کے اندرونی سطح یعنی محور سے \عددیء{\rho_2} رداس پر پایا جائے گا۔بیرونی تار پر \عددیء{\rho_L=\tfrac{-Q}{L}} جبکہ \عددیء{\rho_S=\tfrac{-Q}{2\pi \rho_2 L}} ہو گی۔

دونوں تاروں کے درمیانی فاصلے میں رداس \عددیء{\rho} کی فرضی نلکی سطح صرف اندرونی تار کے بار کو گھیرتی ہے لہٰذا \عددیء{L} لمبائی کی ایسی نلکی  پر مساوات \حوالہ{مساوات_گاوس_لکیری_کثافت_کا_میدان} کی طرح
\begin{gather}
\begin{aligned}\label{مساوات_گاوس_ہم_محوری_تار_میدان}
\kvec{D}&=\frac{\rho_L}{2\pi \rho} \arho\\
&=\frac{Q}{2\pi \rho L} \arho
\end{aligned}
\end{gather}  
پایا جائے گا۔آپ دیکھ سکتے ہیں کہ بیرونی تار  پر بار  کا اس میدان پر کوئی اثر نہیں پایا جاتا۔یوں اندرونی تار کے بیرونی سطح پر
\begin{align}
\kvec{D}_1=\frac{Q}{2\pi \rho_1 L} \arho
\end{align}
جبکہ بیرونی تار کے اندرونی سطح پر
\begin{align}
\kvec{D}_2=\frac{Q}{2\pi \rho_2 L} \arho
\end{align}
پایا جائے گا۔بیرونی تار کے باہر فرضی نلکی سطح میں کل صفر بار پایا جاتا ہے لہٰذا ہم محوری تار کے باہر (یعنی بیرونی تار کے باہر)
\begin{align}\label{مساوات_گاوس_ہم_محوری_تار_سپردار_ہے}
\kvec{D}_{\textrm{تار کے باہر}}=0
\end{align}
ہو گا۔مساوات \حوالہ{مساوات_گاوس_لکیری_کثافت_کا_میدان} انتہائی اہم نتیجہ ہے۔اس کے مطابق ہم محوری تار کے باہر کسی قسم کا برقی میدان نہیں پایا جاتا لہٰذا تار کے باہر سے کسی طرح بھی یہ معلوم نہیں کیا جا سکتا کہ تار پر کس قسم  کا بار پائے جاتے ہیں۔یوں ہم محوری تار  کے ذریعہ اشارات کی منتقلی محفوظ ہوتی ہے۔ہم محوری تار میں بیرونی تار اندرونی تار کو پناہ دیتا ہے۔ لہٰذا ہم محوری تار کو \اصطلاح{پناہ دار تار}\فرہنگ{پناہ دار تار}\حاشیہب{shielded}\فرہنگ{shielded} بھی کہا جائے گا۔

%==================
\ابتدا{مثال}
ہم محوری تار کے اندرونی تار کا رداس \عددیء{\SI{1}{\milli \meter}} اور بیرونی تار کا اندرونی رداس \عددیء{\SI{5}{\milli \meter}} ہے۔\عددیء{\SI{3}{\milli \meter}} رداس پر کثافتِ برقی بہاو  \عددیء{\SI{-5}{\micro\weber \per \meter \squared}} ہے جبکہ تار کے باہر کوئی برقی میدان نہیں پایا جاتا۔دونوں تاروں پر بار کی سطحی کثافت حاصل کریں۔

حل:تار کے گرد برقی میدان صرف رداس کی سمت میں پایا جاتا ہے۔اگر تار پر بار کی لکیری کثافت \عددیء{\rho_L} ہو تب مساوات
\begin{align*}
-5 \times 10^{-6}&=\frac{\rho_L}{2 \pi  \times 0.003}
\end{align*}
سے \عددیء{\rho_L=\SI{-94.26}{\nano \coulomb \per \meter}} حاصل ہوتا ہے۔یوں اندرونی تار کے ایک میٹر لمبائی پر \عددیء{\SI{-94.26}{\nano \coulomb}} بار پایا جائے گا جس سے اس کی سطحی کثافت
\begin{align*}
\rho_{S1}&=\frac{-0.09426 \times 10^{-9}}{2\pi \times 0.001 \times 1}=\SI{-15}{\micro \coulomb \per \meter \squared}
\end{align*}
حاصل ہوتی ہے۔بیرونی تار کے ایک میٹر فاصلے پر \عددیء{\SI{94.26}{\nano \coulomb}} بار پایا جائے گا جس سے یہاں
\begin{align*}
\rho_{S2}=\frac{94.26 \times 10^{-9}}{2\pi \times 0.005 \times 1}=\SI{3}{\micro \coulomb \per \meter \squared}
\end{align*}
حاصل ہوتا ہے۔
\انتہا{مثال}
%==================
\حصہ{یکساں بار بردار ہموار لامحدود سطح}
اگر بار بردار ہموار لامحدود سطح  سے برابر فاصلے پر کسی بھی  مقام سے دیکھا جائے تو صورت حال بالکل یکساں معلوم ہو گی۔کسی بھی نقطے کے ایک جانب باروں سے پیدا برقی میدان کا وہ حصہ جو بار بردار سطح  کے متوازی ہو کو نقطے کے دوسری جانب باروں سے پیدا برقی میدان کا وہ حصہ جو بار بردار سطح  کے متوازی ہو کو ختم کرتا ہے۔ان حقائق سے ہم اخذ کر سکتے ہیں کہ ایسی سطح کا برقی میدان سطح کی عمودی سمت میں ہو گا اور سطح سے یکساں فاصلے پر برقی میدان کی حتمی قیمت برابر ہو گی۔صفحہ \حوالہصفحہ{شکل_کولمب_لامحدود_سطح_پر_بار} پر ایسی لامحدود سطح شکل \حوالہ{شکل_کولمب_لامحدود_سطح_پر_بار} میں دکھائی گئی ہے۔


اس شکل میں بار بردار سطح کے  متوازی دونوں اطراف  برابر فاصلے پر تصوراتی لامحدود سطح تصور کرتے ہیں۔ان سطحوں پر آمنے سامنے رقبہ \عددیء{S} لیتے ہوئے انہیں عمودی سطحوں سے بند کرتے ہوئے حجم گھیرتے ہیں۔سامنے سطح پر \عددیء{D \ax}  جبکہ پیچھے سطح پر \عددیء{-D\ax} ہو گا جبکہ ان رقبوں کو \عددیء{S\ax} اور \عددیء{-S\ax} لکھا جا سکتا ہے۔چونکہ برقی میدان سطحوں کے عمودی ہے لہٰذا دونوں سطحوں کو ملانے والے عمودی سطحوں میں سے کوئی برقی بہاو نہیں ہو گا۔یوں حجم سے برقی بہاو صرف ان آمنے سامنے رقبوں سے  یعنی
\begin{align*}
\psi_{\textrm{سامنے}} &= D \ax \cdot S \ax =S D\\
\psi_{\textrm{پیچھے}}&=(-D\ax) \cdot (-S\ax)=S D
\end{align*} 
جو گھیرے گئی بار کے برابر ہو گا۔اگر  بار بردار سطح پر \عددیء{\rho_S} ہو تب حجم میں \عددیء{\rho_S S} بار پایا جائے گا۔یوں
\begin{align*}
\psi_{\textrm{سامنے}}+\psi_{\textrm{پیچھے}} = 2 DS = \rho_S S
\end{align*}
لکھتے ہوئے
\begin{align*}
D&=\frac{\rho_S}{2}
\end{align*}
حاصل ہوتا ہے جس کی سمتی شکل
\begin{align}
\kvec{D}=\frac{\rho_S}{2} \kvec{a}_N
\end{align}
لکھی جا سکتی ہے جہاں \عددیء{\aN} سے مراد سطح کی اکائی سمتیہ ہے۔یوں
\begin{align}
\kvec{E}=\frac{\rho_S}{2 \epsilon_0} \kvec{a}_N
\end{align}
حاصل ہوتا ہے۔آپ دیکھ سکتے ہیں کہ مساوات \حوالہ{مساوات_کولمب_لامحدود_سطح_کی_برقی-میدان} کے حصول کا موجودہ طریقہ زیادہ آسان ہے۔

\حصہ{انتہائی چھوٹی حجم پر گاوس کے قانون کا اطلاق}\شناخت{حصہ_گاوس_چھوٹی_حجم_گاوس_کا_اطلاق}
شکل \حوالہ{شکل_گاوس_چھوٹی_حجم_پر_اطلاق} میں کارتیسی محدد کے نقطہ \عددیء{N(x_0,y_0,z_0)} پر چھوٹا مستطیلی ڈبہ دکھایا گیا ہے جس کے اطراف \عددیء{\Delta x}، \عددیء{\Delta y} اور \عددیء{\Delta z} ہیں۔یہ ڈبہ برقی میدان \عددیء{{\kvec{D}=D_x\ax+D_y\ay+D_z\az}} میں ہے۔اس چھوٹی ڈبیہ پر گاوس کے قانون
\begin{align}\label{مساوات_گاوس_چھوٹی_ڈبیہ_کارتیسی_محدد}
\oint\limits_{S} \kvec{D} \cdot \dif \kvec{S}=Q=\int\limits_{h} \rho_h \dif h
\end{align}
 کا اطلاق کرتے ہیں۔ڈبیہ کے چھ  اطراف ہیں۔یوں مندرجہ بالا تکملہ کے بائیں بازو کو
\begin{align*}
\oint\limits_S \kvec{D} \cdot \dif \kvec{S}=\int\limits_{\textrm{سامنے}}+\int\limits_{\textrm{پیچھے}}+\int\limits_{\textrm{بائیں}}+\int\limits_{\textrm{دائیں}}+\int\limits_{\textrm{اوپر}}+\int\limits_{\textrm{نیچے}}
\end{align*}
لکھا جا سکتا ہے  جہاں
\begin{align*}
\int\limits_{\textrm{سامنے}} &\overset{.}{=} \kvec{D}_{\textrm{سامنے}} \cdot \Delta \kvec{S}_{\textrm{سامنے}}\\
&\overset{.}{=}\left(D_x \ax+D_y\ay+D_z\az \right)_{\textrm{سامنے}} \cdot \Delta y \Delta z \ax\\
&\overset{.}{=} D_{x,\textrm{سامنے}} \Delta y \Delta z
\end{align*}
کے برابر ہے۔
\begin{figure}
\centering
\includegraphics{figGaussDifferentialVolumeCartesianCoordinates}
\caption{انتہائی چھوٹی حجم پر گاوس کے قانون کا اطلاق}
\label{شکل_گاوس_چھوٹی_حجم_پر_اطلاق}
\end{figure}

ہمیں \عددیء{\kvec{D}} کی قیمت ڈبیہ کے  وسط میں معلوم ہے۔\اصطلاح{ٹیلر تسلسل}\فرہنگ{ٹیلر تسلسل}\حاشیہب{Taylor series}\فرہنگ{Taylor series} کے مطابق کسی بھی تفاعل جس کی قیمت نقطہ \عددیء{a} پر معلوم ہو کو اس نقطے کے قریبی نقطوں پر
\begin{align*}
f(x+a)=f(a)+\frac{1}{1!}(x-a)f'(a)+\frac{1}{2!}(x-a)^2 f''(a)+\cdots
\end{align*}
سے حاصل کیا جا سکتا ہے۔ڈبیہ کے وسط میں نقطہ \عددیء{N(x_0,y_0,z_0)} پر 
\begin{align*}
\kvec{D}(x_0,y_0,z_0)=D_{x0}\ax+D_{y0}\ay+D_{z0}\az
\end{align*}
کی قیمت سے وسط سے \عددیء{+\tfrac{\Delta x}{2}} فاصلے پر  ڈبیہ کے سامنے سطح پر  \عددیء{D_x} ٹیلر تسلسل سے یوں حاصل کیا جا سکتا ہے۔
\begin{align*}
D_{x,\textrm{سامنے}}&=D_{x0}+\frac{1}{1!}\frac{\Delta x}{2} \frac{\partial D_x}{\partial x}+\frac{1}{2!}\left[\frac{\Delta x}{2}\right]^2 \frac{\partial^2 D_x}{\partial x^2}\cdots\\
&\overset{.}{=}D_{x0}+\frac{\Delta x}{2} \frac{\partial D_x}{\partial x}
\end{align*}
جہاں دوسرے قدم پر تسلسل کے صرف پہلے دو اجزاء لئے گئے ہیں۔تفاعل  کے ایک سے زیادہ متغیرات \عددیء{x}، \عددیء{y} اور \عددیء{z} ہیں لہٰذا تسلسل میں جزوی تفرق\حاشیہب{partial differential} کا استعمال کیا گیا ۔

یوں
\begin{align*}
\int\limits_{\textrm{سامنے}} &\overset{.}{=} \left(D_{x0}+\frac{\Delta x}{2} \frac{\partial D_x}{\partial x} \right) \Delta y \Delta z
\end{align*}
حاصل ہوتا ہے۔

بند سطح کی سمت باہر جانب ہوتی ہے لہٰذا  پچھلی سطح \عددیء{-\Delta y \Delta z \ax} ہے اور یوں ڈبیہ کی  پچھلی سطح کے  لئے
\begin{align*}
\int\limits_{\textrm{پیچھے}} &\overset{.}{=} \kvec{D}_{\textrm{پیچھے}} \cdot \Delta \kvec{S}_{\textrm{پیچھے}}\\
&\overset{.}{=}\left(D_x \ax+D_y\ay+D_z\az \right)_{\textrm{پیچھے}} \cdot \left(-\Delta y \Delta z \ax\right)\\
&\overset{.}{=} -D_{x,\textrm{پیچھے}} \Delta y \Delta z
\end{align*}
لکھا جا سکتا ہے جہاں وسط سے \عددیء{-\tfrac{\Delta x}{2}} فاصلے پر  ڈبیہ کی  پچھلی  سطح پر  \عددیء{D_x} ٹیلر تسلسل سے
\begin{align*}
D_{x,\textrm{پیچھے}}&=D_{x0}-\frac{\Delta x}{2} \frac{\partial D_x}{\partial x}
\end{align*}
حاصل ہوتا ہے۔یوں
\begin{align*}
\int\limits_{\textrm{پیچھے}} &\overset{.}{=} -\left(D_{x0}-\frac{\Delta x}{2} \frac{\partial D_x}{\partial x} \right) \Delta y \Delta z
\end{align*}
اور
\begin{align*}
\int\limits_{\textrm{سامنے}}+\int\limits_{\textrm{پیچھے}}&\overset{.}{=}  \left(D_{x0}+\frac{\Delta x}{2} \frac{\partial D_x}{\partial x} \right) \Delta y \Delta z -\left(D_{x0}-\frac{\Delta x}{2} \frac{\partial D_x}{\partial x} \right) \Delta y \Delta z\\
&\overset{.}{=} \frac{\partial D_x}{\partial x} \Delta x \Delta y \Delta z
\end{align*}

حاصل ہوتا ہے۔اسی عمل کو دہراتے ہوئے بائیں اور دائیں سطحوں کے لئے
\begin{align*}
\int\limits_{\textrm{بائیں}}+\int\limits_{\textrm{دائیں}}\overset{.}{=}  \frac{\partial D_y}{\partial y} \Delta x \Delta y \Delta z
\end{align*}
اور اوپر، نیچے سطحوں کے لئے
\begin{align*}
\int\limits_{\textrm{اوپر}}+\int\limits_{\textrm{نیچے}}\overset{.}{=}  \frac{\partial D_z}{\partial z} \Delta x \Delta y \Delta z
\end{align*}
حاصل ہوتا ہے۔اس طرح 
\begin{align}\label{مساوات_گاوس_کارتیسی_محدد_چھوٹی_حجم}
\oint\limits_{S}\kvec{D}\cdot \dif \kvec{S}=Q\overset{.}{=} \left( \frac{\partial D_x}{\partial x}+\frac{\partial D_y}{\partial y}+\frac{\partial D_z}{\partial z} \right)\Delta x \Delta y \Delta z 
\end{align}
حاصل ہوتا ہے۔

اس مساوات کے تحت کسی بھی نقطے پر انتہائی چھوٹا حجم \عددیء{\Delta h} میں بار تقریباً
\begin{align}
Q\overset{.}{=} \left( \frac{\partial D_x}{\partial x}+\frac{\partial D_y}{\partial y}+\frac{\partial D_z}{\partial z} \right)\Delta h
\end{align}
کے برابر ہے۔حجم کی جسامت جتنی کم کی جائے جواب اتنا زیادہ درست ہو گا۔اگلے حصے میں حجم کو کم کرتے کرتے نقطہ نما بنا دیا جائے گا۔ایسی صورت میں مندرجہ بالا مساوات مکمل طور صحیح جواب مہیا کرے گا۔
%==========
\ابتدا{مثال}
اگر \عددیء{\kvec{D}=2x\ax+3y\ay+5\az \, \si{\coulomb / \meter\squared}} ہو تب کارتیسی محدد کے مبدا پر \عددیء{\SI{e-9}{\meter \cubed}} کے انتہائی چھوٹی حجم میں بار حاصل کریں۔

حل:
\begin{align*}
\frac{\partial D_x}{\partial x}+\frac{\partial D_y}{\partial y}+\frac{\partial D_z}{\partial z} =2+3+0
\end{align*}
سے اس حجم میں \عددیء{5 \times 10^{-9}=\SI{5}{\nano \coulomb}} بار  پایا جائے گا۔
\انتہا{مثال}
%========================

\حصہ{پھیلاو}
مساوات \حوالہ{مساوات_گاوس_کارتیسی_محدد_چھوٹی_حجم} میں حجم  کو اتنا کم کرتے ہوئے کہ اس کو صفر تصور کرنا ممکن ہو
\begin{align*}
\frac{\partial D_x}{\partial x}+\frac{\partial D_y}{\partial y}+\frac{\partial D_z}{\partial z}=\lim_{\Delta h \to 0}\frac{\oint\limits_{S}\kvec{D}\cdot \dif \kvec{S}}{\Delta h}=\lim_{\Delta h \to 0} \frac{Q}{\Delta h}
\end{align*}
لکھا جا سکتا ہے۔بار فی حجم کو حجمی کثافت کہتے ہیں۔یوں مساوات کا دایاں بازو نقطے پر حجمی کثافت \عددیء{\rho_h} دیتا ہے۔اس طرح اس مساوات سے دو مساوات حاصل کئے جا سکتے ہیں یعنی
\begin{align}\label{مساوات_گاوس_میکسویل_پھیلاو_مساوات}
\frac{\partial D_x}{\partial x}+\frac{\partial D_y}{\partial y}+\frac{\partial D_z}{\partial z}=\rho_h
\end{align}
اور 
\begin{align}\label{مساوات_گاوس_پھیلاو_کی_تعریف}
\frac{\partial D_x}{\partial x}+\frac{\partial D_y}{\partial y}+\frac{\partial D_z}{\partial z}=\lim_{\Delta h \to 0}\frac{\oint\limits_{S}\kvec{D}\cdot \dif \kvec{S}}{\Delta h}
\end{align}
مساوات \حوالہ{مساوات_گاوس_میکسویل_پھیلاو_مساوات} \اصطلاح{میکس ویل}\فرہنگ{میکس ویل مساوات}\حاشیہب{Maxwell equation}\فرہنگ{Maxwell equation}\حاشیہد{جناب جیمس کلارک میکس ویل (1831-1879) کے مساوات میکس ویل مساوات کہلاتے ہیں۔} کی پہلی مساوات ہے جبکہ مساوات \حوالہ{مساوات_گاوس_پھیلاو_کی_تعریف} سمتیہ \عددیء{\kvec{D}} کا \اصطلاح{پھیلاو}\فرہنگ{پھیلاو}\حاشیہب{divergence}\فرہنگ{divergence} بیان کرتا ہے۔اس مساوات کا دایاں بازو پھیلاو کی تعریف جبکہ اس کا بایاں بازو پھیلاو حاصل کرنے کا طریقہ دیتا ہے۔یوں کارتیسی محدد میں 
\begin{align}\label{مساوات_گاوس_کارتیسی_محدد_پھیلاو_کی_مساوات}
\frac{\partial D_x}{\partial x}+\frac{\partial D_y}{\partial y}+\frac{\partial D_z}{\partial z} \quad \textrm{کارتیسی محدد میں پھیلاو کی مساوات}
\end{align}
سے سمتیہ \عددیء{\kvec{D}} کا پھیلاو حاصل کیا جاتا ہے۔

انجنیئرنگ  کے شعبے میں ایسے کئی مسئلے پائے جاتے ہیں جن  میں چھوٹے سے حجم  کو گھیرنے والے بند سطح پر کسی سمتیہ \عددیء{\kvec{K}} کا  \عددیء{\oint_{S}\kvec{K}\cdot \dif \kvec{S}} درکار ہو۔گزشتہ حصے میں سمتیہ \عددیء{\kvec{D}} کے لئے ایسا ہی کیا گیا۔غور کرنے سے معلوم ہوتا ہے کہ ایسا کرتے ہوئے  \عددیء{\kvec{D}} کی جگہ \عددیء{\kvec{K}} لکھا جا سکتا ہے جس سے 
\begin{align}\label{مساوات_گاوس_کارتیسی_پھیلاو_کی_تعریف}
\frac{\partial K_x}{\partial x}+\frac{\partial K_y}{\partial y}+\frac{\partial K_z}{\partial z}=\lim_{\Delta h \to 0}\frac{\oint\limits_{S}\kvec{K}\cdot \dif \kvec{S}}{\Delta h}
\end{align}
حاصل ہوتا۔سمتیہ \عددیء{\kvec{K}} پانی کا بہاو، ایٹموں کی رفتار یا سلیکان کی پتری میں درجہ حرارت ہو سکتا ہے۔ہم \عددیء{\kvec{K}} کو سمتی بہاو کی کثافت تصور کریں گے۔مندرجہ بالا مساوات  \عددیء{\kvec{K}} کا پھیلاو بیان کرتی ہے۔پھیلاو کے عمل سے مراد مساوات کے بائیں بازو کا عمل ہے  جبکہ مساوات کا دایاں بازو اس کی تعریف بیان کرتا ہے جس کے تحت

\ابتدا{قانون}
کسی بھی سمتی کثافتی بہاو کے \اصطلاح{پھیلاو} سے مراد کسی  چھوٹے حجم کو صفر کرتے ہوئے اس  سے خارج کُل بہاو فی اکائی حجم ہے۔ 
\انتہا{قانون}

یہ ضروری ہے کہ آپ کو پھیلاو کی تعریف کی سمجھ ہو۔یاد رہے کہ پھیلاو کا عمل سمتیہ پر کیا جاتا ہے جبکہ اس سے حاصل جواب غیر سمتی مقدار ہوتا ہے۔کسی نقطے پر چھوٹے حجم سے باہر کی  جانب کل بہاو فی چھوٹے حجم کو پھیلاو کہتے ہیں۔پھیلاو کی کوئی سمت نہیں ہوتی۔پھیلاو کی تعریف جانتے ہوئے کئی مرتبہ بغیر قلم اٹھائے جواب حاصل کیا جا سکتا ہے۔اسی نوعیت کے چند مسئلوں پر اب غور کرتے ہیں۔

پانی سے بھری بالٹی میں پانی میں ڈوبے کسی بھی نقطے پر پانی کی رفتار کا پھیلاو صفر ہو گا چونکہ اس نقطے سے نہ پانی باہر نکل رہا ہے اور نا ہی اس میں داخل ہو رہا ہے۔اسی طرح دریا میں پانی میں ڈوبے نقطے پر بھی پانی کی رفتار کا پھیلاو صفر ہو گا چونکہ ایسے نقطے سے جتنا پانی نکلتا ہے، اتنا ہی پانی اس میں داخل ہوتا ہے۔البتہ اگر بھری بالٹی کے تہہ میں سوراخ کر دیا جائے  تو جب تک نقطہ پانی میں ڈوبا رہے اس وقت تک یہاں پھیلاو صفر رہے  گا البتہ جیسے ہی نقطہ پانی سے نمودار ہونے لگے یہاں مثبت پھیلاو پایا جائے گا اور جب نقطہ پانی سے مکمل طور پر باہر آ جائے تب ایک بار پھر یہاں پھیلاو صفر ہو جائے گا۔جتنی دیر نقطہ پانی کی سطح سے باہر نمودار ہو رہا ہوتا ہے اتنی دیر اس نقطے سے پانی کی انخلا پائی جاتی ہے جس کی وجہ سے یہاں پھیلاو پایا جاتا ہے۔

ایک اور دلچسپ مثال سائیکل کے ٹائر میں ہوا کی ہے۔اگر ٹائر پنکچر ہو جائے اور اس سے ہوا نکلنی شروع ہو جائے تو ٹائر میں کسی بھی نقطے پر سمتی رفتار کا پھیلاو پایا جائے گا چونکہ کسی بھی نقطے پر دیکھا جائے تو یہاں سے ہوا پھیلتے ہوئے خارج ہو گی۔یوں مثبت پھیلاو سے مراد نقطے سے انخلا جبکہ منفی پھیلاو سے مراد نقطے میں داخل ہونا ہے۔ 

ریاضیاتی عمل کو بیان کرنے کے لئے عموماً علامت استعمال کی جاتی ہے۔یوں جمع کے لئے \عددیء{+}، ضرب کے لئے \عددیء{\times} اور تکملہ کے لئے \عددیء{\int} استعمال کئے جاتے ہیں۔آئیے ایک نئی علامت  جسے \اصطلاح{نیبلا}\فرہنگ{نیبلا}\حاشیہب{nabla, del}\فرہنگ{nabla} کہتے اور \عددیء{\nabla} سے ظاہر کرتے ہیں سیکھیں۔نیبلا یونانی حروف تہجی کا حرف ہے۔ تصور کریں کہ
\begin{align}\label{مساوات-گاوس_نیبلا}
\nabla = \frac{\partial }{\partial x}\ax+\frac{\partial }{\partial y}\ay+\frac{\partial }{\partial z}\az
\end{align}
لکھا جاتا ہے جہاں غیر سمتی متغیرہ \عددیء{f} کے سامنے لکھنے سے مراد
\begin{align}\label{مساوات_گاوس_ڈیل_غیر_سمتی}
\nabla f = \frac{\partial f}{\partial x}\ax+\frac{\partial f}{\partial y}\ay+\frac{\partial f}{\partial z}\az
\end{align}
جبکہ سمتیہ \عددیء{\kvec{K}} کے ساتھ نقطہ ضرب سے مراد
\begin{gather}
\begin{aligned}
\nabla \cdot \kvec{K}&=\left(\frac{\partial }{\partial x}\ax+\frac{\partial }{\partial y}\ay+\frac{\partial }{\partial z}\az \right) \cdot \left (K_x \ax+K_y \ay+K_z \az \right)\\
&=\frac{\partial K_x}{\partial x}+\frac{\partial K_y}{\partial y}+\frac{\partial K_z}{\partial z}
\end{aligned}
\end{gather}
لیا جاتا ہے۔یہ علامت انجنیئرنگ  کے شعبے میں انتہائی مقبول ہے۔اسے استعمال کرتے ہوئے پھیلاو کو \عددیء{\nabla \cdot \kvec{D}}  لکھا جا سکتا ہے جہاں
\begin{align}\label{مساوات_گاوس_پھیلاو}
\nabla \cdot \kvec{D}=\frac{\partial D_x}{\partial x}+\frac{\partial D_y}{\partial y}+\frac{\partial D_z}{\partial z}
\end{align}
کے برابر ہے۔\اصطلاح{پھیلاو} کے عمل کو ہم اسی علامت سے ظاہر کریں گے۔مساوات \حوالہ{مساوات_گاوس_میکسویل_پھیلاو_مساوات} یعنی میکس ویل کی پہلی مساوات اب یوں لکھی جا سکتی ہے۔
\begin{align}\label{مساوات_گاوس_میکسویل_پہلی_مساوات_نقطہ_شکل}
\nabla \cdot \kvec{D}=\rho_h \quad \textrm{میکس ویل کی پہلی مساوات}
\end{align}
میکس ویل کی پہلی مساوات درحقیقت گاوس کے قانون کی تفرق\حاشیہب{differential} شکل ہے۔اسی طرح گاوس کا قانون میکس ویل مساوات کی تکمل\حاشیہب{integral} شکل ہے۔

مساوات \حوالہ{مساوات_گاوس_ڈیل_غیر_سمتی} کے طرز پر مساوات صفحہ \حوالہصفحہ{مساوات_توانائی_ڈھلوان_تعریف_پ} پر دیا گیا ہے۔
%=================
\حصہ{نلکی محدد میں پھیلاو کی مساوات}
حصہ \حوالہ{حصہ_گاوس_چھوٹی_حجم_گاوس_کا_اطلاق} میں کارتیسی محدد استعمال کرتے ہوئے چھوٹے حجم پر گاوس کے قانون کے اطلاق سے پھیلاو کی مساوات حاصل کی گئی۔اس حصے میں نلکی محدد استعمال کرتے ہوئے شکل میں دکھائے چھوٹے حجم کو استعمال کرتے ہوئے پھیلاو کی مساوات حاصل کی جائے گی۔شکل کو دیکھتے ہوئے
\begin{align*}
\Delta_{S\textrm{سامنے}}&=-\Delta \rho \Delta z \aphi\\
\Delta_{S\textrm{پیچھے}}&=+\Delta \rho \Delta z \aphi\\
\Delta_{S\textrm{بائیں}}&=-\left(\rho-\frac{\Delta \rho}{2} \right) \Delta \phi \Delta z \arho\\
\Delta_{S\textrm{دائیں}}&=+\left(\rho+\frac{\Delta \rho}{2} \right) \Delta \phi \Delta z \arho\\
\Delta_{S\textrm{اوپر}}&=+\rho \Delta \phi \Delta \rho \az\\
\Delta_{S\textrm{نیچے}}&=-\rho \Delta \phi \Delta \rho \az\\
\end{align*}
لکھا جا سکتا ہے۔کارتیسی محدد میں آمنے سامنے رقبے برابر  تھے۔نلکی محدد میں بائیں اور دائیں رقبے برابر نہیں ہیں۔اس فرق کی بنا پر نلکی محدد میں پھیلاو کی مساوات قدر مختلف حاصل ہو گی۔چھوٹے حجم کے وسط میں
\begin{align}
\kvec{D}=D_{\rho 0} \arho+D_{\phi 0} \aphi+D_{z 0} \az
\end{align}
کے برابر ہے جس سے ٹیلر تسلسل کی مدد سے
\begin{align*}
\kvec{D}_{\textrm{سامنے}}&=\left( D_{\phi 0}-\frac{\Delta \phi}{2}\frac{\partial D_{\phi}}{\partial \phi} \right) \aphi\\
\kvec{D}_{\textrm{پیچھے}}&=\left( D_{\phi 0}+\frac{\Delta \phi}{2}\frac{\partial D_{\phi}}{\partial \phi} \right)\aphi \\
\kvec{D}_{\textrm{بائیں}}&=\left( D_{\rho 0}-\frac{\Delta \rho}{2}\frac{\partial D_{\rho}}{\partial \rho} \right) \arho\\
\kvec{D}_{\textrm{دائیں}}&=\left( D_{\rho 0}+\frac{\Delta \rho}{2}\frac{\partial D_{\rho}}{\partial \rho} \right) \arho\\
\kvec{D}_{\textrm{اوپر}}&=\left( D_{z 0}+\frac{\Delta z}{2}\frac{\partial D_{z}}{\partial z} \right) \az\\
\kvec{D}_{\textrm{نیچے}}&=\left( D_{z 0}-\frac{\Delta z}{2}\frac{\partial D_{z}}{\partial z} \right) \az
\end{align*}
لکھا جا سکتا ہے۔یوں
\begin{align*}
\int \limits_{\textrm{سامنے}}+\int\limits_{\textrm{پیچھے}}=\frac{\partial D_{\phi}}{\partial \phi} \Delta \rho \Delta \phi \Delta z
\end{align*}
حاصل ہوتا ہے۔اسی طرح
\begin{align*}
\int \limits_{\textrm{بائیں}}+\int \limits_{\textrm{دائیں}}=\left(D_{\rho 0} +\rho \frac{\partial D_{\rho}}{\partial \rho} \right) \Delta \rho \Delta \phi \Delta z
\end{align*}
حاصل ہوتا ہے جسے
\begin{align*}
\int \limits_{\textrm{بائیں}}+\int \limits_{\textrm{دائیں}}=\frac{\partial (\rho D_{\rho})}{\partial \rho}  \Delta \rho \Delta \phi \Delta z
\end{align*}
بھی لکھا جا سکتا ہے۔ایسا لکھتے وقت یاد رہے کہ نقطہ \عددیء{N(\rho_0 , \phi_0 , z_0)} پر
\begin{align*}
\eval{\frac{\partial (\rho D_{\rho})}{\partial \rho}}_{N}=\eval{D_{\rho}+\rho \frac{\Delta D_{\rho}}{\Delta \rho}}_{N}=D_{\rho 0} +\rho \frac{\partial D_{\rho}}{\partial \rho}
\end{align*}
کے برابر ہے۔اسی طرح
\begin{align*}
\int \limits_{\textrm{اوپر}}+\int\limits_{\textrm{نیچے}}=\rho \frac{\partial D_{z}}{\partial z} \Delta \rho \Delta \phi \Delta z
\end{align*}
حاصل ہوتا ہے۔ان تمام کو استعمال کرتے ہوئے
\begin{align*}
\oint\limits_{S} \kvec{D}_S \cdot \dif \kvec{S}=\left(\frac{\partial (\rho D_{\rho})}{\partial \rho}+\frac{\partial D_{\phi}}{\partial \phi}  + \rho \frac{\partial D_{z}}{\partial z}   \right)\Delta \rho \Delta \phi \Delta z
\end{align*}
ملتا ہے۔چھوٹے حجم \عددیء{\Delta h = \rho \Delta \rho \Delta \phi \Delta z} کے استعمال سے
\begin{align}\label{مساوات_گاوس_نلکی_پھیلاو_کی_تعریف}
\frac{1}{\rho}\frac{\partial (\rho D_{\rho})}{\partial \rho}+\frac{1}{\rho}\frac{\partial D_{\phi}}{\partial \phi}  +  \frac{\partial D_{z}}{\partial z}=\lim_{\Delta h \to 0} \frac{\oint\limits_{S} \kvec{D}_S \cdot \dif \kvec{S}}{\Delta h}
\end{align}
حاصل ہوتی ہے۔مساوات \حوالہ{مساوات_گاوس_کارتیسی_پھیلاو_کی_تعریف} کا دایاں بازو پھیلاو کی تعریف بیان کرتا ہے جس کے ساتھ موازنہ کرنے سے آپ دیکھ سکتے ہیں کہ مساوات \حوالہ{مساوات_گاوس_نلکی_پھیلاو_کی_تعریف} نلکی محدد میں پھیلاو  دیتا ہے۔

آپ دیکھ سکتے ہیں کہ نلکی محدد میں پھیلاو کی مساوات سادہ شکل نہیں رکھتی۔مساوات \حوالہ{مساوات-گاوس_نیبلا} میں دی گئی \عددیء{\nabla} کو استعمال کرتے ہوئے نلکی محدد میں پھیلاو کی مساوات ہرگز حاصل نہیں کی جا سکتی ہے۔اس کے باوجود نلکی محدد میں بھی پھیلاو کے عمل کو \عددیء{\nabla \cdot \kvec{D}} سے ہی ظاہر کیا جا سکتا ہے جہاں اس سے مراد
\begin{align}\label{مساوات_گاوس_نلکی_پھیلاو}
\nabla \cdot \kvec{D}=\frac{1}{\rho}\frac{\partial (\rho D_{\rho})}{\partial \rho}+\frac{1}{\rho}\frac{\partial D_{\phi}}{\partial \phi}  +  \frac{\partial D_{z}}{\partial z}
\end{align}
لیا جاتا ہے۔مندرجہ بالا مساوات نلکی محدد میں پھیلاو کی مساوات ہے جو کسی بھی سمتیہ کے لئے درست ہے۔یوں سمتیہ \عددیء{\kvec{K}} کے لئے اسے یوں لکھا جا سکتا ہے۔
 \begin{align}\label{مساوات_گاوس_نلکی_عمومی_پھیلاو}
\nabla \cdot \kvec{K}=\frac{1}{\rho}\frac{\partial (\rho K_{\rho})}{\partial \rho}+\frac{1}{\rho}\frac{\partial K_{\phi}}{\partial \phi}  +  \frac{\partial K_{z}}{\partial z}
\end{align}


\حصہ{پھیلاو کی عمومی مساوات}\شناخت{حصہ_گاوس_عمومی_پھیلاو}
کارتیسی محدد میں چھوٹے حجم کے آمنے سامنے اطراف کا رقبہ برابر ہوتا ہے جس سے پھیلاو کی مساوات آسانی سے حاصل ہوتی ہے۔نلکی محدد میں چھوٹے حجم کے رداسی سمت کے آمنے سامنے رقبے مختلف ہوتے ہیں جن کا خصوصی خیال رکھتے ہوئے پھیلاو کی قدر مشکل مساوات گزشتہ حصے میں حاصل کی گئی۔اس حصے میں پھیلاو کی مساوات حاصل کرنے کا ایسا طریقہ دیکھتے ہیں جسے استعمال کرتے ہوئے پھیلاو کی عمومی مساوات حاصل کی جا سکتی ہے جو تمام محدد کے لئے کارآمد ہے۔

کارتیسی محدد کے متغیرات \عددیء{(x,y,z)} جبکہ نلکی محدد کے  \عددیء{(\rho, \phi,z )} اور کروی محدد  کے متغیرات \عددیء{(r,\theta,\phi)} ہیں۔اس حصے میں \اصطلاح{عمومی محدد}\فرہنگ{محدد!عمومی}\حاشیہب{generalized coordinates}\فرہنگ{coordinates!generalized}  استعمال کیا جائے گا جس کے متغیرات \عددیء{(u,v,w)}  اور  عمودی اکائی سمتیات \عددیء{(\au, \av, \aw)} ہیں۔عمومی محدد کسی بھی محدد کے لئے استعمال کیا جا سکتا ہے۔یوں اگر اسے کارتیسی محدد کے لئے استعمال کیا جا رہا ہو تب \عددیء{(u,v,w)}  سے مراد \عددیء{(x,y,z)} ہو گا۔
 
شکل میں عمومی محدد استعمال کرتے ہوئے چھوٹا حجم دکھایا گیا ہے۔عمومی محدد کے تین اطراف
\begin{align*}
\dif L_1 &= k_1 \dif u \\
\dif L_2 &= k_2 \dif v \\
\dif L_3 &= k_3 \dif w 
\end{align*}
ہیں۔کارتیسی محدد میں \عددیء{k_1=k_2=k_3=1} کے برابر لیا جائے گا اور یوں \عددیء{\dif L_1=\dif x } کے برابر ہو گا۔نلکی محدد میں
\begin{gather}
\begin{aligned}\label{مساوات_گاوس_نلکی_اطراف_کے_مستقل}
k_1&=1\\
k_2&=\rho\\
k_3&=1
\end{aligned}
\end{gather}
جبکہ کروی محدد میں
\begin{gather}
\begin{aligned}\label{مساوات_گاوس_کروی_اطراف_کے_مستقل}
k_1&=1\\
k_2&=r\\
k_3&=r \sin \theta
\end{aligned}
\end{gather}
کے برابر ہیں۔اسی طرح تین سمتی رقبے
\begin{align*}
& \dif L_2 \dif L_3 \au\\
&\dif L_1 \dif L_3 \av\\
&\dif L_1 \dif L_2 \aw
\end{align*}
ہوں گے۔

گزشتہ حصوں میں چھوٹے حجم کے آمنے سامنے سطحوں پر بہاو حاصل کرتے وقت پہلے  ان سطحوں پر \عددیء{\kvec{D}} کی قیمت اور ان سطحوں کے رقبے حاصل کئے گئے جن کے نقطہ ضرب سے بہاو حاصل کیا گیا۔یہاں چھوٹے حجم کے وسط میں تین اکائی سمتیات کی سمت میں بہاو سے ٹیلر تسلسل کے استعمال سے حجم کے سطحوں پر بہاو حاصل کیا جائے گا۔حجم کے وسط میں تین اکائی سمتیات کے رخ میں سطحوں پر بہاو
\begin{align*}
& \dif L_2 \dif L_3  D_{u0}\\
& \dif L_1 \dif L_3  D_{v0}\\
& \dif L_1 \dif L_2  D_{w0}
\end{align*}
ہے۔ٹیلر تسلسل سے سامنے اور پیچھے سطحوں پر ان مساوات سے
\begin{align*}
\dif L_2 \dif L_3  D_{u0} &+\frac{1}{2}\frac{\partial }{\partial u} ( \dif L_2 \dif L_3  D_{u})\dif u \quad \textrm{سامنے}\\
-\dif L_2 \dif L_3  D_{u0} &+\frac{1}{2}\frac{\partial }{\partial u} ( \dif L_2 \dif L_3  D_{u})\dif u \quad \textrm{پیچھے}
\end{align*}
یعنی
\begin{align*}
k_2 k_3 \dif v \dif w  D_{u0} &+\frac{1}{2}\frac{\partial }{\partial u} ( k_2 k_3  D_{u})\dif u  \dif v \dif w\quad \textrm{سامنے}\\
-k_2 k_3 \dif v \dif w  D_{u0} &+\frac{1}{2}\frac{\partial }{\partial u} ( k_2 k_3  D_{u})\dif u  \dif v \dif w\quad \textrm{پیچھے}
\end{align*}
لکھتے ہوئے دونوں سطحوں پر بہاو کا مجموعہ
\begin{align*}
\frac{\partial }{\partial u} ( k_2 k_3  D_{u})\dif u  \dif v \dif w
\end{align*}
حاصل ہوتا ہے۔اسی طرح بائیں اور دائیں سطحوں پر کل
\begin{align*}
\frac{\partial }{\partial v} ( k_1 k_3  D_{v})\dif u  \dif v \dif w
\end{align*}
اور اوپر، نیچے کا مجموعہ
\begin{align*}
\frac{\partial }{\partial w} ( k_1 k_2  D_{w})\dif u  \dif v \dif w
\end{align*}
حاصل ہوتا ہے۔چھوٹا حجم
\begin{align*}
\dif h &= \dif L_1 \dif L_2 \dif L_3 \\
&= k_1 k_2 k_3 \dif u \dif v \dif w
\end{align*}
لکھتے ہوئے گاوس کے قانون سے
\begin{align*}
\oint\limits_{S} \kvec{D} \cdot \dif \kvec{S}=\left[\frac{\partial }{\partial u} ( k_2 k_3  D_{u})+\frac{\partial }{\partial v} ( k_1 k_3  D_{v})+\frac{\partial }{\partial w} ( k_1 k_2  D_{w}) \right]\dif u  \dif v \dif w 
\end{align*}
یعنی
\begin{align*}
\frac{1}{k_1 k_2 k_3}\left[\frac{\partial }{\partial u} ( k_2 k_3  D_{u})+\frac{\partial }{\partial v} ( k_1 k_3  D_{v})+\frac{\partial }{\partial w} ( k_1 k_2  D_{w}) \right] =\lim_{\dif h \to 0}\frac{\oint\limits_{S} \kvec{D} \cdot \dif \kvec{S}}{\dif h}
\end{align*}
حاصل ہوتی ہے۔اس مساوات کا دایاں بازو پھیلاو کی تعریف ہے۔یوں پھیلاو کی عمومی مساوات
\begin{align}\label{مساوات_گاوس_پھیلاو_کی_عمومی_مساوات}
\nabla \cdot \kvec{D}=\frac{1}{k_1 k_2 k_3}\left[\frac{\partial }{\partial u} ( k_2 k_3  D_{u})+\frac{\partial }{\partial v} ( k_1 k_3  D_{v})+\frac{\partial }{\partial w} ( k_1 k_2  D_{w}) \right] 
\end{align}
حاصل ہوتی ہے۔
%==============
\ابتدا{مثال}
مساوات \حوالہ{مساوات_گاوس_پھیلاو_کی_عمومی_مساوات} سے  نلکی اور کروی محدد میں پھیلاو کی مساوات حاصل کریں۔

حل: \عددیء{u,v,w} کی جگہ \عددیء{\rho, \phi, z} اور مساوات \حوالہ{مساوات_گاوس_نلکی_اطراف_کے_مستقل} کے استعمال سے  نلکی محدد میں پھیلاو
\begin{gather}
\begin{aligned}
\nabla \cdot \kvec{D}&=\frac{1}{\rho}\left[\frac{\partial }{\partial \rho} ( \rho  D_{\rho})+\frac{\partial }{\partial \phi} (   D_{\phi})+\frac{\partial }{\partial z} ( \rho  D_{z}) \right]\\
&=\frac{1}{\rho}\frac{\partial }{\partial \rho} ( \rho  D_{\rho})+\frac{1}{\rho}\frac{\partial }{\partial \phi} (   D_{\phi})+\frac{\partial }{\partial z} ( D_{z}) \quad \textrm{نلکی محدد میں پھیلاو کی مساوات}
\end{aligned}
\end{gather}
حاصل ہوتی ہے۔اسی طرح  \عددیء{u,v,w} کی جگہ \عددیء{r,\theta,\phi} اور مساوات \حوالہ{مساوات_گاوس_کروی_اطراف_کے_مستقل} کے استعمال سے کروی محدد میں پھیلاو
\begin{gather}
\begin{aligned}
\nabla \cdot \kvec{D}&=\frac{1}{r^2 \sin \theta}\left[\frac{\partial }{\partial r} (r^2 \sin \theta  D_{r})+\frac{\partial }{\partial \theta} ( r \sin \theta  D_{\theta})+\frac{\partial }{\partial \phi} ( r  D_{\phi}) \right]\\
&=\frac{1}{r^2 }\frac{\partial }{\partial r} (r^2   D_{r})+\frac{1}{r \sin \theta }\frac{\partial }{\partial \theta} (\sin \theta  D_{\theta})+\frac{1}{r \sin \theta }\frac{\partial }{\partial \phi} (  D_{\phi})\quad \textrm{کروی محدد میں پھیلاو کی مساوات}
\end{aligned}
\end{gather}
حاصل ہوتا ہے۔
\انتہا{مثال}
%=================
\حصہ{مسئلہ پھیلاو}
صفحہ \حوالہصفحہ{مساوات_گاوس_چھوٹی_ڈبیہ_کارتیسی_محدد} پر مساوات \حوالہ{مساوات_گاوس_چھوٹی_ڈبیہ_کارتیسی_محدد} میں
\begin{align*}
\nabla \cdot \kvec{D}=\rho_h
\end{align*}
لکھتے ہوئے
\begin{align}\label{مساوات_گاوس_مسئلہ_پھیلاو_تکمل_شکل}
\oint\limits_{S} \kvec{D} \cdot \dif \kvec{S}=\int\limits_{h} \nabla \cdot \kvec{D} \dif h
\end{align}
لکھا جا سکتا ہے جو \اصطلاح{مسئلہ پھیلاو}\فرہنگ{مسئلہ پھیلاو}\حاشیہب{divergence theorem}\فرہنگ{divergence theorem} بیان کرتا ہے۔اگرچہ ہم  نے اس مسئلے کو برقی بہاو \عددیء{\kvec{D}} کے لئے حاصل کیا حقیقت میں یہ ایک عمومی نتیجہ ہے جو کسی بھی تین درجی تکملہ کو دو درجی تکملہ اور دو درجی تکملہ کو تین درجی تکملہ میں تبدیل کرتا ہے۔مسئلہ پھیلاو کو یوں بیان کیا جا سکتا ہے

\ابتدا{قانون}
کسی بھی بند سطح پر  سمتیہ کے عمودی حصے کا تکمل بند حجم میں اسی سمتیہ کے پھیلاو کے تکمل کے برابر ہو گا۔
\انتہا{قانون}
\begin{figure}
\centering
\includegraphics{figGaussDiversionAsSurfaceIntegral}
\caption{بند سطح پر سمتیہ کا عمودی حصے کا تکمل بند حجم میں سمتیہ کے تکمل کے برابر ہوتا ہے۔}
\label{شکل_گاوس_پھیلاو_بطور_سطحی_تکلم}
\end{figure}

مسئلہ پھیلاو کی سمجھ شکل \حوالہ{شکل_گاوس_پھیلاو_بطور_سطحی_تکلم} کی مدد سے با آسانی ممکن ہے۔جیسے شکل میں دکھایا گیا ہے کہ کسی بھی چھوٹے حجم سے بہاو قریبی چھوٹے حجم  کی منفی بہاو ثابت ہوتی ہے لہٰذا دونوں کا مجموعی بہاو  حاصل کرتے ہوئے ان کی درمیانی دیوار  پر بہاو رد کیا جائے گا۔یہی سلسلہ تمام حجم پر لاگو کرتے ہوئے ظاہر ہے کہ پورے حجم سے بہاو کے حصول میں اندرونی تمام دیواروں پر بہاو کا کوئی کردار نہیں ہوتا اور صرف بیرونی سطح پر بہاو سے ہی جواب حاصل کیا جا سکتا ہے۔
%==============

\ابتدا{مثال}
نقطہ بار کے \عددیء{\kvec{D}} سے پھیلاو کی مساوات سے مختلف مقامات پر کثافت بار \عددیء{\rho_h} حاصل کریں۔

حل:کروی محدد کے مبدا پر نقطہ بار کا
\begin{align*}
\kvec{D}=\frac{Q}{4\pi r^2} \ar
\end{align*}
ہوتا ہے۔کروی محدد میں پھیلاو کی مساوات کے تحت
\begin{align*}
\nabla \cdot \kvec{D}=\frac{1}{r^2 }\frac{\partial }{\partial r} (r^2   D_{r})+\frac{1}{r \sin \theta }\frac{\partial }{\partial \theta} (\sin \theta  D_{\theta})+\frac{1}{r \sin \theta }\frac{\partial }{\partial \phi} (  D_{\phi}) 
\end{align*}
کے برابر ہے۔چونکہ \عددیء{D_{\theta}} اور \عددیء{D_{\phi}} صفر کے برابر ہیں لہٰذا مندرجہ بالا مساوات سے
\begin{align*}
\nabla \cdot \kvec{D}=\frac{1}{r^2 }\frac{\partial }{\partial r} (r^2 \frac{Q}{4\pi r^2}  )=\left\{
\begin{array}{l  l}
0 &\quad r> 0\\
\infty & \quad r=0
\end{array}
\right.
\end{align*}
حاصل ہوتا ہے جس کے تحت مبدا کے علاوہ تمام خلاء میں کوئی بار نہیں پایا جاتا۔مبدا پر لامحدود کثافت کا بار پایا جاتا ہے۔یاد رہے کہ نقطہ بار سے مراد ایسا بار ہے جس کا حجم صفر ہو۔ایسی صورت میں اس نقطے پر نقطہ بار کی کثافت لا محدود ہی ہو گی۔
\انتہا{مثال}

\newpage
\حصہء{سوالات}
%====================
\ابتدا{سوال}
محدد کے مبدا پر \عددی{\SI{20}{\nano\coulomb}} بار پایا جاتا ہے۔اس کے علاوہ \عددی{z=0} سطح پر \عددیء{\SI{5}{\nano\coulomb\per\meter}} کے لکیری بار \عددی{y=-1} اور \عددی{y=-3} پر پائے جاتے ہیں۔نقطہ \عددی{(0,-2,0)} پر \عددی{\kvec{D}} حاصل کریں۔\عددی{(0,1,0)} پر رداس \عددی{r=1.5} کے کرہ کی سطح پر کل برقی بہاو  حاصل کریں۔ 

جوابات:\عددی{-\tfrac{5}{4\pi}\ay \,\si{\coulomb\per\meter\squared}}، \عددی{\SI{20}{\nano\coulomb}}
\انتہا{سوال}
%==============================
\ابتدا{سوال}
رداس \عددی{\rho=\SI{10}{\centi\meter}} کے نلکی سطح کے \عددی{z>0} حصے  پر سطحی کثافت بار \عددی{\rho_S=2ze^{-z^2} \, \si{\nano\coulomb\per\meter\squared}} پائی جاتی ہے۔سطح پر کل بار دریافت کریں۔اس سطح سے \عددی{z=1} تا \عددی{z=2} زاویہ \عددی{\phi=45^{\circ}} تا \عددی{\phi=75^{\circ}} کتنی برقی بہاو خارج ہوتی ہے۔

جوابات:\عددی{0.2\pi\, \si{\nano\coulomb}}، \عددی{\SI{18.3}{\pico\coulomb}}
\انتہا{سوال}
%==============================
\ابتدا{سوال}
رداس \عددی{\rho=2}، \عددی{\rho=4} اور \عددی{\rho=5} پر بالترتیب سطحی کثافت بار \عددی{\SI{-3}{\nano\coulomb\per\meter\squared}}، \عددی{\SI{1.5}{\nano\coulomb\per\meter\squared}} اور \عددی{\SI{0.25}{\nano\coulomb\per\meter\squared}} پائی جاتی ہے۔ \عددی{z=3} تا \عددی{z=6}  پر رداس \عددی{\rho=4.5} نلکی سطح سے کل کتنی برقی بہاو ہوتی ہے۔\عددی{z=3} تا \عددی{z=6}  پر رداس \عددی{\rho=6} نلکی سطح سے کل کتنی برقی بہاو ہوتی ہے۔نقطہ \عددی{(6,8,2)} پر \عددی{\kvec{D}} حاصل کریں۔

\عددی{\SI{0}{\coulomb}}،\عددی{\SI{28.27}{\nano\coulomb}}،\عددی{\kvec{D}=0.09\ax+0.15\ay \, \si{\nano\coulomb\per\meter\squared}}
\انتہا{سوال}
%============================
\ابتدا{سوال}
بند خطہ \عددی{0\le x \le 2, \, 0\le y \le 2, \, 0 \le z \le 2} میں \عددی{\kvec{D}=xy^2\ax+xyz\ay+z(x+y)\az \,\si{\micro\coulomb\per\meter\squared}} ہے۔اس خطے سے کل برقی بہاو کتنی ہے۔

\عددی{\SI{28}{\micro\coulomb}}
\انتہا{سوال}
%==========================
\ابتدا{سوال}
محدد \عددی{z} پر لکیری کثافت بار \عددی{\SI{50}{\nano\coulomb\per\meter}} پایا جاتا ہے۔محدد کے مبدا پر رداس \عددی{r=\SI{5}{\meter}} کی کرہ سے خارج کل برقی بہاو حاصل کریں۔اگر کرہ کی وسط کو نقطہ \عددی{(0,2,2)} منتقل کیا جائے تب جواب کیا ہو گا۔

جوابات:\عددی{\SI{500}{\nano\coulomb}}، \عددی{\SI{458}{\nano\coulomb}}
\انتہا{سوال}
%=========================
\ابتدا{سوال}
رداس \عددی{r=\SI{1.1}{\meter}} کی کرہ کے اندر حجمی کثافت بار \عددی{\rho_h=30e^{-r^3} \,\si{\nano\coulomb\per\meter^3}} پائی جاتی ہے۔کرہ کے اندر کل بار حاصل کریں۔گاوس کے قانون سے کرہ کی سطح پر برقی بہاو کی کثافت حاصل کریں۔

جوابات:\عددی{\SI{92.46}{\nano\coulomb}}، \عددی{\SI{6.08}{\nano\coulomb\per\meter\squared}}
\انتہا{سوال}
%=============================
\ابتدا{سوال}
نلکی محدد میں کثافت برقی بہاو \عددی{\kvec{D}=\tfrac{\rho\arho+z\az}{4\pi(\rho^2+z^2)^{3/2}}} دیا گیا ہے۔لامحدود لمبائی کی نلکی جس کا رداس \عددی{\rho=5} ہے سے کل کتنی برقی بہاو خارج ہو گی۔

جواب:\عددی{\SI{1}{\coulomb}}
\انتہا{سوال}
%============================
\ابتدا{سوال}
مبدا پر رداس \عددی{5}، \عددی{9} اور \عددی{14} کے کرہ پر بالترتیب سطحی کثافت بار \عددی{\SI{20}{\micro\coulomb\per\meter\squared}}، 
\عددی{\SI{-8}{\micro\coulomb\per\meter\squared}} اور \عددی{\rho_S \, \si{\coulomb\per\meter\squared}} پائے جاتے ہیں۔نقطہ \عددی{(20,0,0)} پر صفر \عددی{\kvec{D}} حاصل کرنے کے لئے \عددی{\rho_S} دریافت کریں۔تمام خطوں میں \عددیء{D} کی مساوات حاصل کریں۔  

جوابات:\عددی{\SI{0.7551}{\micro\coulomb\per\meter\squared}}، \عددی{r<5} پر \عددی{D_r=0} ہے، \عددی{5< r < 9} پر \عددی{D_r=\tfrac{500}{r^2}\, \si{\micro\coulomb\per\meter\squared}} ہے،\عددی{9<r<14} پر \عددی{D_r=-\tfrac{148}{r^2}\,\si{\coulomb\per\meter\squared}} ہے جبکہ \عددی{r>14} پر \عددی{D_r=0} ہے۔
\انتہا{سوال}
%===========================
\ابتدا{سوال}
لامحدود سطح \عددی{z=4} پر \عددی{\rho_S=\SI{2}{\nano\coulomb\per\meter\squared}} سطحی کثافت پائی جاتی ہے۔محدد کے مبدا پر \عددی{r=5} رداس کا کرہ رکھا جاتا ہے۔کرہ کتنے بار کو گھیرے گا۔کرے سے کتنی برقی بہاو خارج ہو گی۔

جوابات:\عددی{\SI{56.549}{\nano\coulomb}}، \عددی{\SI{56.549}{\nano\coulomb}}
\انتہا{سوال}
%==========================
\ابتدا{سوال}
محدد کے مبدا پر \عددی{r=5} رداس کا کرہ جبکہ \عددی{z=4} پر لامحدود سطح پائی جاتی ہے۔لامحدود سطح کے بالائی جانب کرہ کے اندر  حجمی کثافت
 بار \عددی{\rho_h=\SI{25}{\nano\coulomb\per \meter^3 }} پائی جاتی ہے۔کرہ سے کل خارج برقی بہاو حاصل کریں۔

جواب:\عددی{\SI{1.1812}{\micro\coulomb}} 
\انتہا{سوال}
%===========================
\ابتدا{سوال}
خطہ \عددی{\rho < \SI{3}{\milli\meter}} میں حجمی کثافت بار \عددی{\rho_h=\tfrac{\rho^2}{1000}\,\si{\coulomb\per\meter^3}} جبکہ
 خطہ \عددی{\SI{3}{\milli\meter} < \rho < \SI{5}{\milli\meter}} میں \عددی{\rho_h=\SI{2}{\micro\coulomb\per\meter^3}} پائی جاتی ہے۔موزوں گاوسی سطحیں چنتے ہوئے رداس \عددی{\rho=0}، \عددی{\rho={\SI{2}{\milli\meter}}}، \عددی{\rho=\SI{4}{\milli\meter}} اور \عددی{\SI{6}{\milli\meter}} پر \عددی{D_{\rho}} حاصل کریں۔

جوابات:\عددی{\SI{0}{\coulomb\per\meter\squared}}، \عددی{\SI{2}{\pico\coulomb\per\meter\squared}}، \عددی{\SI{1.756}{\nano\coulomb\per\meter\squared}}، \عددی{\SI{2.67}{\nano\coulomb\per\meter\squared}}
\انتہا{سوال}
%==============================
\ابتدا{سوال}
خطہ \عددی{r<\SI{3}{\milli\meter}} میں \عددی{\rho_h=\SI{22}{\micro\coulomb\per\meter^3}} جبکہ \عددی{\SI{5}{\milli\meter}<r<\SI{7}{\milli\meter}} خطے میں \عددی{\rho_h=\tfrac{55}{r}\,\si{\nano\coulomb\per\meter^3}} حجمی کثافت بار پایا جاتا ہے۔ موزوں گاوسی سطحیں چنتے
 ہوئے  \عددی{r=\SI{5}{\milli\meter}} اور \عددی{r=\SI{10}{\milli\meter}} پر \عددی{D_r} دریافت کریں۔

جوابات:\عددی{\SI{22}{\nano\coulomb\per\meter\squared}}، \عددی{\SI{8.58}{\nano\coulomb\per\meter\squared}}
\انتہا{سوال}
%===========================
\ابتدا{سوال}
تفاعل \عددی{\kvec{D}=2x^2\ax+(x+z)\ay+z\az} مکعب \عددی{0 < x,y,z, <a} میں پایا جاتا ہے۔تفاعل کی پھیلاو \عددی{\nabla \cdot \kvec{D}} حاصل کریں۔مکعب کے تمام سطحوں پر تفاعل کے سطحی تکمل کا مجموعہ حاصل کرتے ہوئے مکعب میں کل بار حاصل کریں۔یہی جواب مسئلہ پھیلاو  کی مدد سے حاصل کریں۔

جوابات:\عددی{\nabla \cdot \kvec{D}=4x+1}، \عددی{2a^4+a^3}
\انتہا{سوال}
%==============================
\ابتدا{سوال}
مکعب \عددی{2<x,y,z<5} میں \عددی{\kvec{G}=\frac{5x^2 y}{z}\ay} ہے۔مسئلہ پھیلاو کے دونوں اطراف کو مکعب کے لئے حل کریں۔

جواب:\عددی{536.03}
\انتہا{سوال}
%===============================
\ابتدا{سوال}
مندرجہ ذیل تفاعل کے پھیلاو حاصل کرتے ہوئے پھیلاو کی قیمت نقطہ \عددی{N(3,4,6)} پر حاصل کریں۔
\begin{align*}
\kvec{D}&=10(xy-\frac{y}{\sqrt{z}})\ax+y^2(x+2)\ay-(6z^2+3x^2y)\az\\
\kvec{D}&=8\rho \sin \phi \arho+4\rho \cos \phi \aphi+z^2\az\\
\kvec{D}&=2 r \sin \theta \cos \phi\ar +r \cos \theta \cos \phi \atheta+r \cos \phi \aphi
\end{align*}

جوابات:\عددی{10y+2y(x+2)-12z}، \عددی{12 \sin \phi}، \عددی{{6 \sin \theta \sin \phi+\frac{\cos 2\theta \sin \phi}{\sin \theta}-\frac{\sin \phi}{\sin \theta}}}، \عددی{80}، \عددی{9.6}، \عددی{2.0486}
\انتہا{سوال}
%===============================
\ابتدا{سوال}
مندرجہ ذیل تفاعل کی پھیلاو نقطہ \عددی{N(3,5-2)} پر حاصل کریں۔
\begin{align*}
\kvec{D}&=(x+yz)(3x\ax-5z\ay+2y^2z\az)\\
\kvec{D}&=\frac{x\ax+y\ay+z\az}{\sqrt{x^2+y^2+z^2}}\\
\kvec{D}&=0.24\ax-0.55\ay+0.12\az\\
\kvec{D}&=x^2yz^3(2\ax-3\ay+\az)
\end{align*}

جوابات:\عددی{-882}، \عددی{0.324}، \عددی{0}، \عددی{276}
\انتہا{سوال}
%===========================
\ابتدا{سوال}
مندرجہ ذیل تفاعل کی پھیلاو نقطہ \عددی{N(3,45^{\circ},30^{\circ})} پر حاصل کریں۔
\begin{align*}
\kvec{D}&=(2r\sin\theta\cos\phi+\cos\theta)\ar+(r\cos\theta\cos\phi-\sin\theta)\atheta-r\sin\phi\aphi\\
\kvec{D}&=\sin^2\theta\sin\phi\ar+\sin 2\theta \sin \phi\atheta+\sin\theta\cos\phi\aphi\\
\kvec{D}&=0.2\ar-0.15\atheta+0.23\aphi\\
\kvec{D}&=0.2r^3 \phi \sin^2 \theta (\ar+\atheta+\aphi)
\end{align*}

جوابات:\عددی{4.899}، \عددی{0.1667}، \عددی{0}، \عددی{5.043}
\انتہا{سوال}
%===================
\ابتدا{سوال}
کارتیسی محدد کے مبدا \عددی{(0,0,0)} پر نقطہ بار \عددی{Q} سے پیدا \عددی{\kvec{D}} کی عمومی مساوات کارتیسی اور کروی محدد میں حاصل کرتے ہوئے \عددی{\kvec{D}} کی پھیلاو نقطہ \عددی{(0,0,0)} سے ہٹ کر حاصل کریں۔محدد کے مبدا پر پھیلاو کی قیمت بھی دریافت کریں۔  

جوابات:\عددی{\SI{0}{\coulomb\per\meter\squared}}، \عددی{\SI{0}{\coulomb\per\meter\squared}}،
 \عددی{\infty \, \si{\coulomb\per\meter\squared}}
\انتہا{سوال}
%=======================
\ابتدا{سوال}
\عددی{z} محدد پر لکیری کثافت بار \عددی{\rho_L} پائی جاتی ہے۔ثابت کریں کہ اس محدد سے ہٹ کر تمام خلاء میں \عددی{\nabla \cdot \kvec{D}=0} کے برابر ہے۔اگر لکیری بار کی جگہ حجمی کثافت بار \عددی{\rho_h} خطہ \عددی{0<\rho<a} میں پائی جائے تب پوری خلاء میں \عددی{\nabla \cdot \kvec{D}} کی قیمت کیا ہو گی۔

جوابات:خطہ \عددی{0<\rho<a} میں \عددی{\nabla \cdot \kvec{D}=\rho_h} ہو گا جبکہ بقایا خلاء میں \عددی{\nabla \cdot \kvec{D}=0} ہو گا۔
\انتہا{سوال}
%==========================
\ابتدا{سوال}
اگر \عددی{\kvec{D}=2x^2\ax+x(z-22)\ay+x^2y^3\az} ہو تب حجمی کثافت بار کی مساوات کیا ہو گی۔مکعب \عددی{0<x,y,z<2} میں کل بار کتنا ہو گا۔

جوابات:\عددی{\rho_h=4x}، \عددی{\SI{32}{\coulomb}}
\انتہا{سوال}
%===========================
\ابتدا{سوال}
نلکی \عددی{\rho<\SI{3}{\meter}} میں \عددی{\kvec{D}=3\rho\arho \, \si{\coulomb\per\meter\squared}} ہے۔نقطہ \عددی{(1.5,45^{\circ},3)} پر حجمی کثافت بار اور کثافت برقی بہاو حاصل کریں۔نلکی \عددیء{{\rho<2.5}}، \عددی{{0<z<2}} سے کل کتنی برقی بہاو کا اخراج ہوتا ہے اور اس نلکی میں کل کتنا بار پایا جاتا ہے۔

جوابات:\عددی{\SI{6}{\coulomb\per\meter^3}}، \عددی{\SI{4.5}{\coulomb\per\meter\squared}}، \عددی{\SI{235.62}{\coulomb}}،
 \عددی{235.62\coulomb}
\انتہا{سوال}
%===========================
\ابتدا{سوال}
کرہ \عددی{r<\SI{4}{\meter}} میں کثافت برقی بہاو \عددی{3 r \ar} ہے۔نقطہ \عددی{(3,30^{\circ},70^{\circ})} پر حجمی کثافت بار اور کثافت برقی بہاو کیا ہوں گے؟ کرہ \عددیء{{r=\SI{3}{\meter}}} سے کل کتنی برقی بہاو کا اخراج ہو گا اور اس کرہ میں کل کتنا بار پایا جائے گا؟

جوابات:\عددی{\SI{9}{\coulomb\per\meter^3}}، \عددی{\SI{9}{\coulomb\per\meter\squared}}، \عددی{\SI{1017.9}{\coulomb}}، \عددی{\SI{1017.9}{\coulomb}}
\انتہا{سوال}
%===========================
\ابتدا{سوال}
خطہ \عددی{0\le x\le1}، \عددی{0\le y\le 1}، \عددی{0\le z \le 1} میں \عددی{\kvec{G}=(4x-x^2)\ax-3y^2z^2\ay-(2y^3z^2-z)\az} پر مسئلہ پھیلاو کے دونوں اطراف کے تکمل حاصل کرتے ہوئے ثابت کریں کہ دونوں جوابات برابر ہیں۔

جوابات:\عددی{2.5}، \عددی{2.5} 
\انتہا{سوال}
%====================================
\ابتدا{سوال}
خطہ \عددی{2\le r \le 5}، \عددی{0 \le \theta \le \tfrac{\pi}{4}}، \عددی{0 \le \phi \le 2\pi} میں \عددی{\kvec{D}=\tfrac{0.1}{r}\cos \theta \atheta} پایا جاتا ہو۔اس خطے میں کل بار مسئلہ پھیلاو کے دونوں اطراف کی مدد سے حاصل کریں۔

جوابات:\عددی{\SI{0.942}{\coulomb}}، \عددی{\SI{0.942}{\coulomb}}
\انتہا{سوال}
%=================================
