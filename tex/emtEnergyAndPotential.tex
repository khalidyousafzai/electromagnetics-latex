\باب{توانائی اور برقی دباو}

\حصہ{توانائی اور کام}
قوت \عددیء{F} کی سمت میں فاصلہ \عددیء{\dif L} طے کرنے سے 
\begin{align*}
\dif W= F \dif L
\end{align*}
\ترچہ{کام} ہوتا ہے۔اگر قوت اور طے کرد فاصلہ ایک ہی سمت میں نہ ہوں تب فاصلے اور قوت کا وہ حصہ جو طے کردہ فاصلے کی سمت میں ہو کے حاصل ضرب کو \ترچہ{کام}\فرہنگ{کام}\حاشیہب{work}\فرہنگ{work} کہتے ہیں۔شکل \حوالہ{شکل_دباو_کام_کی_تعریف} کو دیکھتے ہوئے سمتیات کے استعمال سے 
\begin{align*}
\dif W&=F \cos \alpha  \dif L\\
&=\kvec{F} \cdot \dif \kvec{L}
\end{align*}
لکھا جا سکتا ہے جہاں \عددیء{F \cos \alpha \dif L} کو نقطہ ضرب کی مدد سے \عددیء{\kvec{F} \cdot \dif \kvec{L}} لکھا گیا ہے۔
\begin{figure}
\centering
\includegraphics{figVoltageWork}
\caption{طے فاصلہ اور فاصلے کی سمت میں قوت کا حاصل ضرب کام کہلاتا ہے}
\label{شکل_دباو_کام_کی_تعریف}
\end{figure}

زمین اور کمیت \عددیء{m} کے درمیان قوت ثقل \عددیء{\kvec{F}_G=-\tfrac{GMm}{r^2}\ar} پایا جاتا ہے\حاشیہد{\عددیء{\ar}  اکائی سمتیہ ہے۔} جس میں \عددیء{\tfrac{GM}{r^2}=g} لکھتے ہوئے  \عددیء{\kvec{F}_G=-mg \ar} لکھا جا سکتا ہے۔کام کرتے ہوئے کمیت کو \عددیء{\Delta h \ar} اونچائی پر منتقل کرنے  کی خاطر قوت ثقل کے خلاف
\begin{align*}
\kvec{F}_{\textrm{لاگو}}=-\kvec{F}_G
\end{align*}
لاگو کرتے ہوئے
\begin{align*}
\Delta W=\kvec{F}_{\textrm{لاگو}} \cdot \Delta h \ar=mg \Delta h
\end{align*}
 توانائی درکار ہو گی۔کام کرنے کے لئے درکار توانائی کمیت میں منتقل ہو جاتی ہے جسے \ترچہ{مخففی توانائی}\فرہنگ{مخفففی توانائی}\حاشیہب{potential energy}\فرہنگ{potential energy} کہتے ہیں۔اگر \عددیء{\Delta h} کی قیمت  \عددیء{r} کی نسبت سے  بہت کم نہ ہو تب \عددیء{g} کو مستقل تصور کرنا ممکن نہ ہو گا اور مخففی توانائی تکملہ کے ذریعہ حاصل کی جائے گی۔
\begin{align*}
W =-\int_{\textrm{ابتدا}}^{\textrm{اختتام}} \kvec{F}_G \cdot \dif \kvec{r}=\int_{\textrm{ابتدا}}^{\textrm{اختتام}} \frac{GMm }{r^2} dr
\end{align*}
ثقلی میدان میں کمیت کو ابتدائی نقطے سے اختتامی نقطے تک پہنچاتے ہوئے کوئی بھی راستہ اختیار کیا جا سکتا ہے۔اختیار کردہ راستے کا مخففی توانائی پر کسی قسم کا کوئی اثر نہیں ہوتا۔ایسے میدان جن میں دو نقطوں کے مابین مخفففی توانائی کا دارومدار ابتدائی نقطے سے اختتامی نقطے تک پہنچنے کے راستے پر نہیں ہوتا \ترچہ{قائم میدان}\فرہنگ{قائم میدان}\حاشیہب{conservative field}\فرہنگ{conservative field} کہلاتے ہیں۔ 

برقی میدان میں چارجوں کے حرکت کے مسئلے کو بھی اسی طرح حل کیا جاتا ہے۔برقی میدان \سمتیہ{E} میں چارج \عددیء{q} پر قوت \عددیء{\kvec{F}_E=q \kvec{E}} عمل کرتا ہے۔چارج کو فاصلہ \عددیء{\dif \kvec{L}} ہلانے کی خاطر اس قوت کے خلاف بیرونی قوت
\begin{align*}
\kvec{F}_{\textrm{بیرونی}} = -\kvec{F}_E
\end{align*}
لاگو کرتے ہوئے
\begin{align}\label{مساوات_دباو_کام_کی_تعریف}
\dif W=-q \kvec{E} \cdot \dif \kvec{L}
\end{align}
کام\حاشیہب{work} کیا جاتا ہے۔کسی بھی ابتدائی نقطے سے اختتامی نقطے تک یوں
\begin{align}\label{مساوات_دباو_لکیری_تکملہ}
W=-q \int_{\textrm{ابتدا}}^{\textrm{اختتام}} \kvec{E} \cdot \dif\kvec{L}
\end{align}
توانائی درکار ہو گی۔

\حصہ{لکیری تکملہ}
مساوات \حوالہ{مساوات_دباو_لکیری_تکملہ} لکیری تکملہ ہے جس پر مزید غور کرتے ہیں۔شکل \حوالہ{شکل_دباو_تکملہ_بمع_مجموعہ} میں غیر متغیر \عددیء{\kvec{E}} میں  نقطہ \عددیء{O} سے نقطہ \عددیء{N} تک  کسی راستے چارج  کی منتقلی دکھائی گئی ہے۔ 

\begin{figure}
\centering
\includegraphics{figVoltageLineIntegralAsSum}
\caption{تکملہ دراصل چھوٹے حصوں کا مجموعہ ہوتا ہے}
\label{شکل_دباو_تکملہ_بمع_مجموعہ}
\end{figure}
%============
\حصہ{برقی دباو}
اس مساوات میں منتقلی کرتے ہوئے \سمتیہ{E} کو تغیر پذیر تصور کیا گیا ہے۔اس توانائی کا دارومدار مختلف مقامات پر برقی شدت اور چارج \عددیء{q} پر ہے۔ساکن برقی میدان\فرہنگ{ساکن برقی میدان} میں بھی اختیار کردہ راستے کا درکار توانائی پر کسی قسم کا کوئی اثر نہیں ہوتا۔برقی و برقیات میں عموماً اس توانائی سے زیادہ اہم وہ توانائی ہے جو اکائی چارج کو ابتدائی نقطے سے اختتامی نقطے تک منتقل کرنے کی خاطر درکار ہو۔اس توانائی کو \ترچہ{برقی دباو}\فرہنگ{برقی دباو}\حاشیہب{voltage}\فرہنگ{voltage} کہتے اور \عددیء{v} سے ظاہر کرتے ہیں۔چونکہ توانائی غیر سمتی یعنی مقداری ہے لہٰذا برقی دباو بھی مقداری ہے۔مندرجہ بالا مساوات سے برقی دباو یوں حاصل ہوتا ہے۔
\begin{align}
v=\frac{W}{q}=-\int_{\textrm{ابتدا}}^{\textrm{اختتام}} \kvec{E} \cdot \dif\kvec{L}
\end{align}
ہمارے ملک میں \عددیء{\SI{220}{\volt}} برقی دباو مہیا کی جاتی ہے۔یوں منفی تار سے مثبت تار تک \عددیء{\SI{1}{\coulomb}} کا چارج منتقل کرنے کی خاطر \عددیء{\SI{220}{\joule}} توانائی درکار ہو گی۔برقی دباو پر تفصیلاً آگے باب میں غور کیا جائے گا۔ 
%==============================

