\باب{مویج اور گھمکیا}\شناخت{باب_مویج}
اب تک ہم صرف \اصطلاح{عرضی برقی و مقناطیسی}\فرہنگ{عرضی!برقی و مقناطیسی}\حاشیہب{transverse electromagnetic, TEM}\فرہنگ{transverse electromagnetic}\فرہنگ{TEM}  \تحریر{TEM} امواج کی بات کرتے آ رہے ہیں جن میں برقی اور مقناطیسی دونوں میدان سمت حرکت کے عمودی ہوتے ہیں۔اس باب میں ترسیلی تار پر بحث کو آگے بڑھاتے ہوئے ایسے امواج پر غور کیا جائے گا جن میں برقی یا مقناطیسی میدان سمت حرکت کی جانب بھی جزو رکھتے ہوں۔وہ ترسیلی تار جو صرف اس طرح کے امواج کو گزار سیکھیں \اصطلاح{میوج}\فرہنگ{مویج}\حاشیہب{waveguide}\فرہنگ{waveguide} کہلاتے ہیں۔

دو لامحدود جسامت کے مستوی سطحوں کے مویج سے بات شروع کرتے ہوئے کھوکھلے مستطیلی اور نلکی مویج تک بات بڑھائی جائے گی۔ان مویج میں میدان کے اشکال، ان کے منقطع طول موج اور تقلیلی مستقل  حاصل کئے جائیں گے۔اس کے بعد ایک تار پر بیرونی موج اور دیگر اقسام کے مویج پر غور کیا جائے گا۔آخر میں موصل کے بند ڈبوں میں قید امواج پر غور کیا جائے گا جنہیں گھمکیا کہتے ہیں۔

\حصہ{برقی دور، ترسیلی تار اور مویج کا موازنہ}
کم تعدد پر برقی دباو، برقی رو، مزاحمت وغیرہ عملی متغیر ہیں جنہیں استعمال کرتے ہوئے برقی ادوار حل کئے جاتے ہیں۔ان تعدد پر تمام مزاحمت یا رکاوٹ کو نقطہ نما تصور کیا جاتا ہے۔یوں تار کے ایک سرے پر منبع برقی دباو لاگو کرتے ہوئے تار کے دوسرے سرے پر مزاحمت میں برقی رو حاصل کی جا سکتی ہے۔

قدر زیادہ تعدد پر انہیں حقائق کو ترسیلی تار پر لاگو کیا جا سکتا ہے۔ایسا کرتے وقت ترسیلی تار کی مزاحمت یا امالہ تار کی لمبائی پر تقسیم شدہ  تصور کرنا لازم ہے۔ساتھ ہی ساتھ ترسیلی تار پر برقی دباو کی رفتار پر بھی نظر رکھنی ہوتی ہے۔

اب موصل کھوکھلے نلکی یا مستطیلی نالی پر مبنی نظام کی بات کرتے ہیں۔کیا ایسی نالی برقی و مقناطیسی طاقت منتقل کرنے کی صلاحیت رکھتی ہے؟ اگر ہماری معلومات برقی ادوار یا ترسیلی تار تک محدود ہوتی تب اس سوال کا جواب یہ ہے کہ ایسا ممکن نہیں ہے کیونکہ برقی طاقت کے منتقلی کے لئے دو تار ضروری ہیں۔البتہ اگر ہم شعاعوں کا علم رکھتے تب جواب ہوتا کہ ایسا ممکن ہے چونکہ شعاعیں سیدھی کھوکھلے نلکی سے گزر سکتی ہیں اور شعاعیں بلند تعدد \عددیء{(\SI{e16}{\hertz})} کی برقی و مقناطیسی امواج ہی ہیں۔

اصل جواب ہے کہ ایسا امواج کے تعدد پر منحصر ہے۔کم تعدد کے امواج نالی سے نہیں گزر سکتے جبکہ بلند تعدد کے امواج اس سے گزر سکتے ہیں۔تعدد کے ان دو خطوں کے درمیان ایسی تعدد ہو گی جس سے کم تعدد نالی سے نہیں گزرے گی اور جس سے زیادہ تعدد نالی سے گزرے گی۔اس تعدد کو \اصطلاح{پست انقطاعی تعدد}\فرہنگ{تعدد!پست انقطاعی}\فرہنگ{انقطاعی!پست تعدد}\حاشیہب{low cutoff frequency}\فرہنگ{frequency!low cutoff} کہا جاتا ہے۔ 

کھوکھلے نالی سے برقی و مقناطیسی طاقت کی منتقلی برقی ادوار حل کرنے کے علم سے ناقابل سمجھ مسئلہ ہے۔کھوکھلے نالی میں طاقت کی منتقلی، نالی کے کھوکھلے حصے میں برقی اور مقناطیسی میدان پر غور سے سمجھا جا سکتا ہے جنہیں استعمال کرتے ہوئے  پوئنٹنگ سمتیہ سے موج کی طاقت حاصل ہوتی ہے۔دراصل برقی و مقناطیسی طاقت نالی کے کھوکھلے حصے میں برقی اور مقناطیسی امواج سے منتقل ہوتا ہے نا کہ نالی کے موصل حصے میں۔برقی دباو اور برقی رو اس منتقلی کے محض اضافی اثرات ہیں۔ 

\حصہ{دو لامحدود وسعت کے مستوی چادروں کے مویج میں عرضی برقی موج}
شکل \حوالہ{شکل_مویج_لامحدود_متوازی_چادر} میں دو لامحدود وسعت کے متوازی چادروں پر مبنی ترسیلی تار دکھائی گئی ہے جو \عددیء{y} سمتی عرضی برقی و مقناطیسی موج گزار سکتی
 ہے۔ اس تار کی خاص خاصیت یہ ہے کہ ایک مخصوص تعدد کے اوپر یہ دیگر \اصطلاح{بلند درجی انداز}\فرہنگ{انداز!بلند درجی}\حاشیہب{higher order mode}\فرہنگ{mode!higher order} کے امواج بھی گزار سکتی ہے۔یوں ترسیلی تار سے شروع کرتے ہوئے مویج تک بحث کو پہنچانے  کے لئے یہ بہترین مثال ہے۔

\begin{figure}
\centering
\includegraphics{figWaveguidesInfiniteParallelPlates}
\caption{دو لامحدود وسعت کے متوازی موصل چادروں کا نظام۔}
\label{شکل_مویج_لامحدود_متوازی_چادر}
\end{figure}

ایسی بلند درجی انداز کی بات کرتے ہیں جس میں برقی میدان ہر نقطے پر \عددیء{y} سمتی ہے جبکہ سمت حرکت \عددیء{\ax} ہے۔چونکہ برقی میدان سمت حرکت کے عمودی ہے لہٰذا اس انداز کو \اصطلاح{عرضی برقی انداز}\فرہنگ{عرضی!برقی انداز}\فرہنگ{انداز!عرضی برقی}\حاشیہب{transverse electric mode, TE mode}\فرہنگ{transverse electric mode}\فرہنگ{mode, transverse electric, TE} \تحریر{(TE)} کہا جائے گا۔اگرچہ اس موج میں برقی میدان عرضی ہے، مقناطیسی میدان عرضی اور طولی اجزاء پر مشتمل ہے۔کامل موصل چادروں کی صورت میں چادروں پر برقی میدان صفر ہو گا البتہ چادر سے دور  اس کی کچھ بھی قیمت ممکن ہے۔ایسی عرضی برقی انداز موج کے خصوصیات باآسانی یوں حاصل کئے جا سکتے ہیں کہ اسے دو عرضی برقی و مقناطیسی انداز \تحریر{TEM} امواج کا مجموعہ تصور کیا جائے جو موصل چادروں کے درمیان بار بار انعکاس کرتی ہوں۔

آئیں پہلے شکل \حوالہ{شکل_مویج_دو_عرضی_امواج_خالی_خلاء} پر غور کریں جہاں خالی خلاء میں ایک ہی تعدد کے دو سطحی \تحریر{TEM} امواج کے ملاپ کی صورت حال دکھائی گئی ہے۔اس شکل میں امواج خطی قطبی تصور کئے گئے ہیں جن کا برقی میدان صفحہ کے عمودی فرض کیا گیا ہے۔موج الف کی شعاع اوپر بائیں ہاتھ سے نیچے دائیں ہاتھ کی طرف جبکہ موج ب کی شعاع نیچے بائیں ہاتھ سے اوپر دائیں ہاتھ کی جانب گامزن ہے۔یوں ان کا آپس میں ملاپ کسی زاویے پر ہوتا ہے۔شکل میں گہری سیاہی کی ٹھوس لکیر سے موج کی چوٹی جبکہ ہلکی سیاہی کے ٹھوس لکیر سے اس کا نشیب دکھایا گیا ہے۔یوں سطحی موج الف کی چوٹیاں اور نشیب، شعاع الف کے عمودی دکھائے گئے ہیں۔گہری سیاہی کے ٹھوس لکیر کو برقی میدان کی چوٹی تصور کیا جائے۔یوں اس لکیر پر برقی میدان زیادہ سے زیادہ قیمت رکھتا ہے اور اس کی سمت صفحہ سے عمودی باہر جانب کو ہے۔اسی طرح ہلکی ٹھوس لکیر میدان کی نشیب کو ظاہر کرتی ہے لہٰذا یہاں میدان کی قیمت زیادہ سے زیادہ ہو گی البتہ اس کی سمت صفحہ کے عمودی اندر جانب کو ہو گی۔چوٹی اور نشیب کے درمیان فاصلہ \عددیء{\tfrac{\lambda_0}{2}} کے برابر ہے۔ 
%
\begin{figure}
\centering
\includegraphics{figWaveguidesTwoIntersectingTEMwaves}
\caption{دو عرضی برقی و مقناطیسی امواج خلاء میں مختلف سمتوں میں حرکت کر رہی ہیں۔}
\label{شکل_مویج_دو_عرضی_امواج_خالی_خلاء}
\end{figure}

جس نقطے پر ایک موج کی چوٹی اور دوسری موج کا نشیب ملتے ہیں اس نقطے پر کل میدان صفر کے برابر ہو گا۔یوں جہاں گہری سیاہی اور ہلکی سیاہی کے لکیر ملتے ہیں وہاں میدان صفر ہو گا۔شکل میں ہلکی سیاہی میں ایسی دو نقطہ دار لکیریں کھینچی گئی ہیں جن پر میدان صفر کے برابر ہے۔آپ غور کر کے تسلی کر لیں کہ ان لکیروں کے ہر نقطے پر برقی میدان صفر ہی ہے۔مزید آپ ذہن میں دونوں امواج کو حرکت دیتے ہوئے تسلی کر لیں کہ امواج کے حرکت کے باوجود ان دو لکیروں پر میدان صفر ہی رہتا ہے۔اسی طرح جن نقطوں پر دونوں امواج کی چوٹیاں آپس میں ملتی ہوں یا دونوں کے نشیب آپس میں ملتے ہوں وہاں میدان دگنا ہو گا۔شکل میں ہلکی سیاہی اور دو نقطوں والی ایسی ایک عدد  لکیر دکھائی گئی ہے جہاں میدان دگنا پایا جائے گا۔

صفر میدان دکھاتے نقطہ دار لکیر پر برقی میدان صفر کے برابر ہے لہٰذا ان پر موصل سطح کے سرحدی برقی میدان کا شرط پورا اترتا ہے۔یوں ان لکیروں پر، صفحہ کے عمودی  موصل چادر رکھے جا سکتے ہیں۔البتہ ایسا کرنے سے موج کی سیدھی حرکت متاثر ہو گی چونکہ آمدی زاویے کے برابر، موصل سطح پر، انعکاسی زاویے سے موج انعکاس کرے گی۔یوں موج موصل سطح سے گزر نہیں پائے گی۔ہاں اگر دو موصل چادروں کے درمیان ان امواج کو بھیجا جائے، تب یہ دونوں موصل سطحوں کے درمیان بار بار انعکاس کرتی حرکت کریں گی۔شکل \حوالہ{شکل_مویج_شعاع_انعکاس_کرتی_حرکت_کرتی_ہے} میں ایسا دکھایا گیا ہے۔ شکل \حوالہ{شکل_مویج_خالی_خلاء_اور_میوج_طول-موج} میں مویج میں موج کی چوٹی اور نشیب دکھائے گئے ہیں۔خالی خلاء میں طول موج اور مویج میں طول موج کا تعلق بھی دکھایا گیا ہے۔ اس شکل میں موصل چادروں کے درمیان میدان ہوبہو شکل \حوالہ{شکل_مویج_دو_عرضی_امواج_خالی_خلاء} میں دو متوازی نقطہ دار لکیروں کے درمیان میدان ہے۔یہاں بھی گہری سیاہی میں ٹھوس لکیر \عددیء{\kvec{E}} کی چوٹی اور ہلکی سیاہی میں لکیر اس کا نشیب ہے۔موصل چادر پر یہ دونوں مل کر صفر برقی میدان پیدا کرتے ہیں۔   

\begin{figure}
\centering
\includegraphics{figWaveguidesTwoConductingSheetsTwoTEMwaves}
\caption{شعاعیں دو چادروں کے درمیان بار بار انعکاس کرتی حرکت کرتی ہیں۔}
\label{شکل_مویج_شعاع_انعکاس_کرتی_حرکت_کرتی_ہے}
\end{figure}
%
\begin{figure}
\centering
\includegraphics{figWaveguidesTwoConductingSheetsTwoTEMwavesWavefronts}
\caption{موجوں کی چوٹیاں، نشیب، خالی خلاء اور مویج میں طول موج۔}
\label{شکل_مویج_خالی_خلاء_اور_میوج_طول-موج}
\end{figure}

اگرچہ ہم دو عدد عرضی برقی و مقناطیسی \تحریر{TEM} امواج کی بات کرتے آ رہے ہیں، درحقیقت ان کا مجموعہ بلند درجی \تحریر{TE} انداز کی موج ہے۔بلند درجی انداز کے موج کی اہم خصوصیت  یہ ہے کہ اس کا طول موج ایک مخصوص حد سے کم ہونا لازم ہے۔ایسا نہ ہونے کی صورت میں یہ مویج سے نہیں گزر سکتی۔طول کی یہ حد \اصطلاح{انقطاعی طول}\فرہنگ{انقطاعی طول}\فرہنگ{طول!انقطاعی}\حاشیہب{cutoff wavelength}\فرہنگ{cutoff wavelength}\فرہنگ{wavelength!cutoff} پکاری جاتی ہے۔ آئیں انقطاعی طول حاصل کریں۔

 \begin{figure}
\centering
\includegraphics{figWaveguidesTwoConductingSheetsCutoffWavelength}
\caption{متوازی لامحدود وسعت کے چادروں کے مویج میں میدان کے اجزاء۔}
\label{شکل_مویج_متوازی_چادر_مویج_اجزاء_میدان}
\end{figure}

شکل \حوالہ{شکل_مویج_متوازی_چادر_مویج_اجزاء_میدان} میں \تحریر{TE} موج کے دو \تحریر{TEM} اجزاء دکھائے گئے ہیں جو \عددیء{x'} اور \عددیء{x''} سمت میں گامزن ہیں۔دونوں جزو موصل چادر یعنی \عددیء{x} محدد کے ساتھ \عددیء{\theta} زاویہ بناتے ہیں۔برقی میدان صفحہ کے عمودی \عددیء{y} محدد کی سمت میں ہے۔چادروں کے درمیان فاصلہ \عددیء{b} ہے۔نقطہ \عددیء{D} پر موج \عددیء{x'} کی چوٹی ہے لہٰذا یہاں برقی میدان \عددیء{E_y'} مثبت  قیمت رکھتا ہے جو صفحہ کے عمودی باہر کو ہے اور جسے گول دائرے میں بند نقطے سے ظاہر کیا گیا ہے۔ اس نقطے پر لکیر \عددیء{AD} لہر کی چوٹی ظاہر کرتی ہے۔عین اسی لمحہ نقطہ \عددیء{C} پر موج \عددیء{x''} کا نشیب ہے جسے گول دائرے میں بند صلیبی نشان سے ظاہر کیا گیا ہے۔اس لہر کے نشیب کو ہلکی سیاہی میں لکیر \عددیء{AC} سے ظاہر کیا گیا ہے۔ایک لہر کی چوٹی اور دوسرے لہر کا نشیب نقطہ \عددیء{A} پر مل کر صفر میدان پیدا کرتے ہیں۔ہم جانتے ہیں کہ عین دو چادروں کے درمیان دونوں امواج کی چوٹیاں مل کر دگنا میدان پیدا کرتی ہیں۔اس نقطے کو شکل میں \عددیء{B} سے ظاہر کیا گیا ہے۔یوں موج \عددیء{x''} کا نشیب \عددیء{C} پر جبکہ اس کی چوٹی \عددیء{B} پر ہے۔اس طرح ان نقطوں کے درمیان فاصلہ طول موج کا چوتھا حصہ ہو گا۔اسی طرح \عددیء{BD} اور \عددیء{C'B} بھی طول موج کے چوتھائی برابر  ہیں
\begin{align}
BC=BC'=BD=\frac{\lambda_0}{4}
\end{align}
جہاں لامحدود خلاء میں \تحریر{TEM} موج کا طول موج \عددیء{\lambda_0} ہے اور یہ خلاء اسی مادے سے بھری ہے جو دو چادروں کے درمیان پایا جاتا ہے۔موصل چادر پر ایک موج کی کوئی بھی چوٹی اور دوسری موج کا کوئی بھی نشیب مل کر صفر میدان پیدا کر سکتے ہیں۔یوں مندرجہ بالا مساوات کی عمومی شکل
\begin{align}\label{مساوات_مویج_چوٹی_نشیب_ختم_عمومی}
BC=\frac{n \lambda_0}{4}
\end{align}
ہے جہاں \عددیء{n=1,2,3,\cdots} ہو سکتے ہیں۔جفت \عددیء{n} کی صورت میں دو چادروں کے عین درمیان برقی میدان صفر حاصل ہو گا جبکہ طاق \عددیء{n} کی صورت میں یہاں میدان دگنا ہو گا۔ان حقائق ہر تفصیلاً جلد بات کی جائے گی۔شکل \حوالہ{شکل_مویج_متوازی_چادر_مویج_اجزاء_میدان} میں تکون \عددیء{ABC} سے
\begin{align*}
AB \sin \theta = \frac{b}{2}\sin \theta =\frac{n \lambda_0}{4}
\end{align*}
یعنی
\begin{align}\label{مساوات_مویج_طول_اور_درجہ_انداز}
\lambda_0 = \frac{2b}{n} \sin \theta
\end{align}
لکھا جا سکتا ہے جہاں لمبائی \عددیء{BC} کے لئے مساوات \حوالہ{مساوات_مویج_چوٹی_نشیب_ختم_عمومی} استعمال کیا گیا۔اس مساوات کے تحت زیادہ سے زیادہ طول موج \عددیء{\lambda_{0c}} کی قیمت \عددیء{\sin \theta=1} یعنی \عددیء{\theta=90^{\circ}} پر
\begin{align}\label{مساوات_مویج_طول_موج_بالمقابل_درجہ_موج}
\lambda_{0c}=\frac{2b}{n}
\end{align}
 حاصل ہوتی ہے جس سے \عددیء{n} کی ہر قیمت کے مقابل طول کی انقطاعی قیمت حاصل کی جا سکتی ہے۔جب \عددیء{n=1} ہو تب
 \begin{align}
\lambda_{0c}=2b
\end{align}
حاصل ہوتا ہے۔یہ کم تر درجے کی \تحریر{TE} موج کا انقطاعی طول ہے جو ان چادروں کے درمیان صفر کر سکتی ہے۔یہ مساوات کہتا ہے کہ چادروں کے درمیان فاصلہ کم از کم آدھے طول کے برابر ہو گا تو موج چادروں کے درمیان سے گزر پائے گی۔

\عددیء{n=1} کو بلند درجی \تحریر{TE} امواج کا کم تر درجہ کہا جاتا ہے۔\عددیء{n=2} اس سے ایک قدم بلند  درجے کی موج کہلائے گی اور اس کا انقطاعی طول
\begin{align}
\lambda_{0c}=b
\end{align}
ہو گا۔یوں \عددیء{n=2} درجے کی \تحریر{TE} موج کے گزرنے کا لئے چادروں کے درمیان کم از کم فاصل موج کے طول کے برابر ضروری ہے۔اسی طرح \عددیء{n=3} کے لئے \عددیء{\lambda_{0c}=\tfrac{2b}{3}} حاصل ہوتا ہے، وغیرہ وغیرہ۔

مساوات \حوالہ{مساوات_مویج_طول_موج_بالمقابل_درجہ_موج} اور مساوات \حوالہ{مساوات_مویج_طول_اور_درجہ_انداز} کو ملا کر
\begin{align}
\lambda_0=\lambda_{0c} \sin \theta
\end{align}
یا
\begin{align}
\theta=\sin^{-1}\frac{\lambda_0}{\lambda_{0c}}
\end{align}
لکھا جا سکتا ہے۔یوں کسی بھی درجے کی موج کا انقطاعی زاویہ \عددیء{\theta=90^{\circ}} حاصل ہوتا ہے۔اس زاویے پر موج  دونوں چادروں کے مابین، \عددیء{x} تبدیل کئے بغیر،  انعکاس کرتی رہتی ہے۔یوں چادروں کے درمیان ساکن موج پیدا ہوتی ہے جو \عددیء{x} سمت میں طاقت منتقل نہیں کر سکتی۔اگر طول موج \عددیء{\lambda_0} انقطاعی طول موج \عددیء{\lambda_{0c}} سے قدر کم ہو تب \عددیء{\theta} کی قیمت \عددیء{90^{\circ}} سے کم ہو گی اور موج، بار بار انعکاس کرتی ہوئی، چادروں کے درمیان \عددیء{x} سمت میں حرکت کر پائے گی۔جیسے شکل \حوالہ{شکل_مویج_طول_موج_اور_زاویہ_انعاکس} میں دکھایا گیا ہے، طول موج مزید کم کرنے سے زاویہ مزید کم ہوتا ہے۔آخر کار انتہائی کم طول موج پر صورت حال لامحدود خلاء میں موج کے حرکت مانند ہو جاتی ہے اور یہ شعاع کی طرح چادروں کے درمیان سیدھا گزرنے کے قابل ہو جاتی ہے۔

\begin{figure}
\centering
\begin{subfigure}{0.4\textwidth}
\centering
\includegraphics{figWaveguidesTwoConducorsAngleNinty}
\caption{طول موج، عین انقطاعی طول موج کے برابر ہے۔}
\end{subfigure}%
%
\begin{subfigure}{0.4\textwidth}
\centering
\includegraphics{figWaveguidesTwoConducorsAngleAbitLessThanNinty}
\caption{طول موج، انقطاعی طول موج سے قدر کم ہے۔}
\end{subfigure}%

\begin{subfigure}{0.4\textwidth}
\centering
\includegraphics{figWaveguidesTwoConducorsAngleMuchLessThanNinty}
\caption{طول موج مزید کم کرنے سے زاویہ بھی مزید کم ہوتا ہے۔}
\end{subfigure}%
\begin{subfigure}{0.4\textwidth}
\centering
\includegraphics{figWaveguidesTwoConducorsPhaseGroupVelocities}
\caption{مختلف اقسام کے رفتار کا تعلق۔}
\end{subfigure}%
\caption{طول موج اور انعکاس موج کے زاویے۔ مختلف اقسام کے رفتاروں کا آپس میں تعلق۔}
\label{شکل_مویج_طول_موج_اور_زاویہ_انعاکس}
\end{figure}

شکل \حوالہ{شکل_مویج_متوازی_چادر_مویج_اجزاء_میدان} میں \تحریر{TEM} امواج کی \اصطلاح{دوری رفتار}\فرہنگ{دوری!رفتار}\فرہنگ{رفتار!دوری}\حاشیہب{phase velocity}\فرہنگ{phase velocity} \عددیء{v_0} لامحدود خلاء میں آزاد موج کی دوری رفتار
\begin{align}
v_0=\frac{1}{\sqrt{\mu \epsilon}} \quad (\si{\meter \per \second})
\end{align}
ہی ہے جہاں خلاء کا مقناطیسی مستقل \عددیء{\mu} اور اس کا برقی مستقل \عددیء{\epsilon} ہیں۔شکل \حوالہ{شکل_مویج_طول_موج_اور_زاویہ_انعاکس}-د میں \تحریر{TE} موج کی \عددیء{x} سمت میں دوری رفتار \عددیء{v} ہے۔\تحریر{TE} موج کی چوٹی یا نشیب یا کوئی اور زاویائی نقطہ اس رفتار سے \عددیء{x} سمت میں حرکت کرتا نظر آئے گا۔ان دو اقسام کے رفتار کا تعلق شکل \حوالہ{شکل_مویج_طول_موج_اور_زاویہ_انعاکس}-د سے
\begin{align}
\frac{v_0}{v}=\cos \theta
\end{align}
لکھا جا سکتا ہے جس سے
\begin{align}\label{مساوات_مویج_دوری_رفتار_تعلق_الف}
v=\frac{v_0}{\cos \theta}=\frac{1}{\sqrt{\mu \epsilon} \cos \theta} \quad \quad \si{\meter\per\second}
\end{align}
حاصل ہوتا ہے۔اس مساوات کے تحت جیسے جیسے طول موج کو انقطاعی طول موج کے قریب لایا جائے، ویسے ویسے \تحریر{TE} موج کی دوری رفتار کی قیمت بڑھتی ہے حتٰی کہ عین \عددیء{\lambda_{0c}} پر دوری رفتار لامحدود قیمت اختیار کر لیتی ہے۔اس کے برعکس جیسے جیسے طول موج کو کم کیا جائے، یعنی جیسے جیسے \عددیء{\theta} کو کم کیا جائے، ویسے ویسے \تحریر{TE} موج کی دوری رفتار \تحریر{TEM} کے دوری رفتار کے قریب ہو گی حتٰی کہ انتہائی کم طول موج یعنی انتہائی بلند تعدد کے موج کی صورت میں یہ قیمت \عددیء{v_0} کے برابر ہو جائے گی۔ یوں مویج میں بند، بلند درجی موج کا دوری رفتار \تحریر{TEM} موج کے دوری رفتار سے زیادہ یا اس کے برابر ممکن ہے۔طاقت کی منتقلی انعکاس کرتی موج کے \اصطلاح{مجموعی رفتار}\فرہنگ{مجموعی رفتار}\فرہنگ{رفتار!مجموعی}\حاشیہب{group velocity}\فرہنگ{group velocity} سے ہوتی ہے جسے شکل میں \عددیء{u} سے ظاہر کیا گیا ہے۔شکل \حوالہ{شکل_مویج_طول_موج_اور_زاویہ_انعاکس}-د سے
\begin{align}\label{مساوات_مویج_دوری_رفتار_تعلق_ب}
u=v_0 \cos \theta
\end{align}
لکھا جا سکتا ہے لہٰذا طاقت کی منتقلی کی رفتار \تحریر{TEM} کے رفتار سے کم یا اس کے برابر ممکن ہے۔طاقت کسی صورت بھی \تحریر{TEM} موج کی رفتار سے زیادہ رفتار پر منتقل کرنا ممکن نہیں ہے۔یہ حقیقت آئن سٹائن کے قانون کے عین مطابق ہے جس کے تحت کوئی بھی چیز رفتار شعاع سے تجاوز نہیں کر سکتی۔یاد رہے کہ \تحریر{TE} موج کی دوری رفتار درحقیقت کسی چیز کی منتقلی نہیں کرتی لہٰذا اس کی قیمت \عددیء{v_0} سے بڑھ سکتی ہے۔مساوات \حوالہ{مساوات_مویج_دوری_رفتار_تعلق_الف} اور مساوات \حوالہ{مساوات_مویج_دوری_رفتار_تعلق_ب}  کو ملا کر 
\begin{align}\label{مساوات_مویج_دوری_رفتار_تعلق_پ}
u v =v_0^2
\end{align}
حاصل ہوتا ہے۔

دو چادروں میں بند ہونے سے \تحریر{TEM} موج کا تعدد تبدیل نہیں ہوتا۔اسی طرح ایسے دو یکساں تعدد کے امواج سے حاصل \تحریر{TE} موج کا تعدد بھی وہی رہتا ہے۔چونکہ طول موج ضرب تعدد کا حاصل رفتار کے برابر ہوتا ہے لہٰذا مساوات \حوالہ{مساوات_مویج_دوری_رفتار_تعلق_الف} کو
\begin{align*}
f \lambda=\frac{f \lambda_0}{\cos \theta}
\end{align*}
لکھا جا سکتا ہے جس سے
\begin{align*}
\lambda=\frac{\lambda_0}{\cos \theta}
\end{align*}
حاصل ہوتا ہے جو بلند درجہ موج کے طول \عددیء{\lambda} اور آزاد موج کے طول \عددیء{\lambda_0} کا تعلق ہے۔

\begin{figure}
\centering
\includegraphics{figWaveguidesTwoConducorsPhaseGroupVelocitiesGraphicalView}
\caption{دوری اور مجموعی رفتار بالمقابل زاویہ موج۔}
\label{شکل_مویج_دوری_مجموعی_رفتار_بالمقابل_زاویہ}
\end{figure} 

شکل \حوالہ{شکل_مویج_دوری_مجموعی_رفتار_بالمقابل_زاویہ} میں دوری رفتار بالمقابل زاویہ موج اور مجموعی رفتار بالمقابل زاویہ موج دکھائے گئے ہیں۔جیسے جیسے \عددیء{\theta} کی قیمت \عددیء{90^{\circ}} کے قریب آتی ہے ویسے ویسے دوری رفتار کی قیمت لامحدود جبکہ مجموعی رفتار کی قیمت صفر کے قریب تر ہوتی ہے۔

حقیقت میں دو متوازی لامحدود وسعت\حاشیہد{حقیقی دنیا میں لا محدود وسعت کے چادر نہیں پائے جاتے۔} کے چادروں پر مبنی مویج کہیں نہیں پایا جاتا۔حقیقی مویج عموماً کھوکھلے مستطیل یا کھوکھلے نالی کے اشکال رکھتے ہیں۔چونکہ برقی میدان کے عمودی موصل چادر رکھنے سے میدان متاثر نہیں ہوتا لہٰذا دو لامحدود وسعت کے متوازی چادر، جن کے درمیان فاصلہ \عددیء{b} ہو، میں \تحریر{TE} موج کے عمودی دو چادر رکھنے سے  میدان میں کوئی تبدیلی رونما نہیں ہو گی، لیکن ایسا کرنے سے مستطیل مویج حاصل ہوتا ہے۔شکل \حوالہ{شکل_مویج_مستطیل}-الف میں مستطیلی مویج بنتا  دکھایا گیا ہے جہاں \عددیء{d} فاصلے پر دو متوازی چادر رکھے گئے ہیں۔مستطیل شکل کے علاوہ بقایا چادر ہٹانے سے مستطیل مویج حاصل ہوتا ہے جسے شکل \حوالہ{شکل_مویج_مستطیل}-ب میں دکھایا گیا ہے۔اس طرح ہم دیکھتے ہیں کہ اگرچہ دو لامحدود چادروں کا مویج تو استعمال نہیں ہوتا لیکن اس کے \تحریر{TE} امواج جوں کے توں مستطیل مویج کے لئے استعمال کئے جا سکتے ہیں۔اب تک \تحریر{TE} امواج کے نقطہ نظر سے مستطیل کی \عددیء{d} لمبائی کچھ بھی ممکن ہے۔

\begin{figure}
\centering
\begin{subfigure}{0.4\textwidth}
\centering
\includegraphics{figWaveguidesInfiniteParallelPlatesToRectangularWaveguide}
\caption{لامحدود متوازی چادر مویج سے مستطیلی مویج کا حصول۔}
\label{شکل_مویج_حصول_مستطیل}
\end{subfigure}%
%
\begin{subfigure}{0.4\textwidth}
\centering
\includegraphics{figWaveguidesRectangularWaveguide}
\caption{مستطیلی مویج کا رقبہ عمودی تراش۔}
\label{شکل_مویج_رقبہ_عمودی_تراش_مستطیل}
\end{subfigure}
\caption{مستطیلی مویج کا حصول اور اس کا رقبہ عمودی تراش۔}
\label{شکل_مویج_مستطیل}
\end{figure}

لامحدود چادر کے مویج پر غور کرنے سے انقطاعی طول موج کے علاوہ دوری رفتار اور مجموعی رفتار کے مساوات بھی حاصل کئے گئے۔دیگر بلند درجے کے امواج پر معلومات حاصل کرنے کی خاطر میکس ویل کے مساوات حل کرنا لازم ہے۔آئیں  مستطیل مویج کے لئے میکس ویل مساوات حل کرتے ہیں۔

\حصہ{کھوکھلا مستطیلی مویج} 
مستطیل مویج کے اطراف پر برقی اور مقناطیسی سرحدی شرائط، کارتیسی محدد میں نہایت آسانی سے لاگو کئے جا سکتے ہیں۔اسی لئے مستطیلی مویج کو کارتیسی نظام میں حل کیا جائے گا۔ہم کارتیسی نظام میں میکس ویل کے مساوات سے  موج کی مساوات حاصل کرتے ہیں۔مویج کو \عددیء{x} محدد پر رکھتے ہوئے ہم سمت موج کو اسی سمت حرکت کے پابند بناتے ہیں اور ساتھ ہی ساتھ اسے سائن نما تصور کرتے ہیں۔اس کے بعد بلند درجے موج کی قسم کا انتخاب کرتے ہیں۔یوں ہم برقی میدان \عددیء{E} کو سمت موج کے عمودی رہنے کے پابند رکھتے ہوئے \اصطلاح{عرضی برقی}\فرہنگ{عرضی!برقی}\حاشیہب{transverse electric, TE}\فرہنگ{transverse electric, TE} \تحریر{TE} موج پر غور کر سکتے ہیں یا مقناطیسی میدان کو سمت موج کے عمودی رہنے کے پابند رکھتے ہوئے \اصطلاح{عرضی مقناطیسی}\حاشیہب{transverse magnetic, TM} \تحریر{TM} موج پر غور کر سکتے ہیں۔\تحریر{TEM} موج میں برقی اور مقناطیسی میدان سمت حرکت کے عمودی ہوتے ہیں۔بلند درجی موج میں میدان، سمت حرکت کی سمت میں بھی پائے جاتے ہیں۔اب عرضی برقی \تحریر{TE} موج کی صورت میں \عددیء{E_x=0} ہو گا لہٰذا ایسی صورت میں \عددیء{H_x} صفر کے برابر نہیں ہو سکتا۔اگر \عددیء{H_x} بھی صفر کے برابر ہو تب موج \تحریر{TEM} قسم کی ہو گی نا کہ \تحریر{TE} قسم کی۔\تحریر{TE} کی صورت میں تمام مساوات کو \عددیء{H_x} کی صورت میں لکھنا بہتر ثابت ہوتا ہے۔حاصل موج پر سرحدی شرائط لاگو کرتے ہوئے اسے \عددیء{H_x} کے لئے حل کیا جاتا ہے۔حاصل \عددی{H_x} کو بقایا مساوات میں پر کرتے ہوئے \عددیء{E_y}، \عددیء{E_z}، \عددیء{H_y} اور \عددیء{H_z} حاصل کئے جاتے ہیں۔یوں برقی اور مقناطیسی میدان کے تمام کارتیسی اجزاء کی مکمل معلومات حاصل ہوتی ہے۔یہ عمومی طریقہ کار ہے جسے دیگر مسائل حل کرنے کے لئے بھی استعمال کیا جا سکتا ہے۔

اس طریقے کو مستطیلی مویج میں \تحریر{TE} موج کے لئے تفصلیلاً  استعمال کرتے ہیں۔ایسا کرنے کی خاطر مندرجہ ذیل اقدامات سلسلہ وار اٹھائے جائیں گے۔
\begin{itemize}
\item
میکس ویل مساوات سے شروع کریں۔
\item
موج کو وقت کے ساتھ سائن نما رہنے کا پابند بنائیں۔
\item
موج کو \عددیء{x} سمت کے ساتھ سائن نما رہنے کا پابند بناتے ہوئے  حرکی مستقل بروئے کار لائیں۔
\item
بلند درجی موج کا انتخاب کریں۔ہم \تحریر{TE} موج کا انتخاب کرتے ہوئے \عددیء{E_x=0} اور \عددیء{H_x \ne 0} رکھیں گے۔
\item
بقایا چار اجزاء یعنی \عددیء{E_y}، \عددیء{E_z}، \عددیء{H_y} اور \عددیء{H_z} کے مساوات \عددیء{H_x} کی صورت میں لکھیں۔
\item
موج کی مساوات \عددیء{H_x} کی صورت میں حاصل کریں۔
\item
مستطیلی مویج کے اطراف کے سرحدی شرائط لاگو کرتے ہوئے موج کی اس مساوات کو \عددیء{H_x} کے لئے حل کریں۔
\item
\عددیء{E_y}، \عددیء{E_z}، \عددیء{H_y} اور \عددیء{H_z} کے مساوات میں حاصل \عددیء{H_x} پر کرتے ہوئے ان کی مساوات بھی حاصل کریں۔
\end{itemize}  

ان اقدامات سے مکمل حل حاصل ہو گا۔

آئیں پہلے قدم سے شروع کرتے ہوئے میکس ویل کے مساوات کو کارتیسی نظام میں لکھتے ہیں۔صفحہ \حوالہصفحہ{مساوات-میکس_ویل_تفرقی_الف} پر مساوات \حوالہ{مساوات-میکس_ویل_تفرقی_الف} اور  مساوت \حوالہ{مساوات-میکس_ویل_تفرقی_ب}
\begin{align*}
\nabla \times \kvec{E}&=-\frac{\partial \kvec{B}}{\partial t}\\
\nabla \times \kvec{H}&=\kvec{J}+\frac{\partial \kvec{D}}{\partial t}
\end{align*}
کارتیسی محدد میں
\begin{align}
\frac{\partial E_z}{\partial y}-\frac{\partial E_y}{\partial z}+\mu \frac{\partial H_x}{\partial t}&=0  \label{مساوات_مویج_میکس_ویل_الف}\\
\frac{\partial E_x}{\partial z}-\frac{\partial E_z}{\partial x}+\mu \frac{\partial H_y}{\partial t}&=0   \label{مساوات_مویج_میکس_ویل_ب}\\
\frac{\partial E_y}{\partial x}-\frac{\partial E_x}{\partial y}+\mu \frac{\partial H_z}{\partial t}&=0 \label{مساوات_مویج_میکس_ویل_پ}
\end{align}
اور
\begin{align}
\frac{\partial H_z}{\partial y}-\frac{\partial H_y}{\partial z}-\sigma E_x-\epsilon \frac{\partial E_x}{\partial t}&=0  \label{مساوات_مویج_میکس_ویل_ت}\\
\frac{\partial H_x}{\partial z}-\frac{\partial H_z}{\partial x}-\sigma E_y-\epsilon \frac{\partial E_y}{\partial t}&=0  \label{مساوات_مویج_میکس_ویل_ٹ}\\
\frac{\partial H_y}{\partial x}-\frac{\partial H_x}{\partial y}-\sigma E_z-\epsilon \frac{\partial E_z}{\partial t}&=0 \label{مساوات_مویج_میکس_ویل_ث}
\end{align}
لکھے جائیں گے جہاں \عددیء{\kvec{B}=\mu \kvec{H}} اور \عددیء{\kvec{D}=\epsilon \kvec{E}} کا استعمال کیا گیا ہے۔اسی طرح خالی خلاء میں \عددیء{\rho_h=0} لیتے ہوئے  مساوات \حوالہ{مساوات_میکس_ویل_گاوس_قانون_نقطہ} اور مساوات \حوالہ{مساوات_میکس_ویل_مقناطیسی_میدان_دو_قطب} کارتیسی محدد میں
\begin{align}
\frac{\partial E_x}{\partial x}+\frac{\partial E_y}{\partial y}+\frac{\partial E_z}{\partial z}&=0 \label{مساوات_مویج_میکس_ویل_ج}\\
\frac{\partial H_x}{\partial x}+\frac{\partial H_y}{\partial y}+\frac{\partial H_z}{\partial z}&=0 \label{مساوات_مویج_میکس_ویل_چ}
\end{align}
لکھے جائیں گے۔

اب دوسرا قدم کہتا ہے کہ موج وقت کے ساتھ سائن نما تعلق رکھتا ہے جبکہ تیسرا قدم کہتا ہے کہ موج \عددیء{x} فاصلے کے ساتھ بھی سائن نما تعلق رکھتا ہے۔ساتھ ہی ساتھ \عددیء{x} سمت میں حرکی مستقل بھی بروئے کار لانا ہے۔ان دو اقدام کو استعمال کرتے ہوئے میدان کے تمام اجزاء لکھتے ہیں۔یوں \عددیء{E_y} کو مثال بناتے ہوئے
\begin{align}\label{مساوات_مویج_سائن_نما_کی_قید}
E_y=E_1 e^{j \omega t -\gamma x}
\end{align}
لکھا جائے گا جہاں
\begin{align*}
\gamma&=\text{\RL{حرکی مستقل}}=\alpha+j \beta \\
\alpha&=\text{\RL{تقلیلی مستقل}}\\
\beta&=\text{\RL{زاویائی مستقل}}
\end{align*}
ہیں۔مساوات \حوالہ{مساوات_مویج_سائن_نما_کی_قید} کے اطلاق سے مساوات \حوالہ{مساوات_مویج_میکس_ویل_الف} تا مساوات \حوالہ{مساوات_مویج_میکس_ویل_چ}
\begin{align}
\frac{\partial E_z}{\partial y}-\frac{\partial E_y}{\partial z}+j \omega \mu H_x&=0  \\
\frac{\partial E_x}{\partial z}+\gamma E_z+j \omega \mu H_y&=0  \\
-\gamma E_y-\frac{\partial E_x}{\partial y}+j \omega \mu H_z&=0
\end{align}
%
\begin{align}
\frac{\partial H_z}{\partial y}-\frac{\partial H_y}{\partial z}-(\sigma+j \omega \epsilon)E_x&=0 \\
\frac{\partial H_x}{\partial z}+\gamma H_z-(\sigma+j \omega \epsilon)E_y&=0  \\
-\gamma H_y-\frac{\partial H_x}{\partial y}-(\sigma+j \omega \epsilon)E_z&=0 
\end{align}
%
\begin{align}
-\gamma E_x+\frac{\partial E_y}{\partial y}+\frac{\partial E_z}{\partial z}&=0 \\
-\gamma H_x+\frac{\partial H_y}{\partial y}+\frac{\partial H_z}{\partial z}&=0
\end{align}
صورت اختیار کر لیتے ہیں۔مندرجہ بالا آٹھ مساوات میں ترسیلی تار کے برقی رکاوٹ \عددیء{Z} اور برقی فراوانی \عددیء{Y} کی طرز کے مستقل
\begin{align}
Z&=-j \omega \mu  \quad \quad \left(\si{\ohm / \meter} \right) \\
Y&=\sigma +j \omega \epsilon \quad \quad \left(\si{\siemens / \meter} \right)
\end{align}
استعمال کرتے ہوئے انہیں قدر چھوٹا لکھتے ہیں۔
\begin{align}
\frac{\partial E_z}{\partial y}-\frac{\partial E_y}{\partial z}-Z H_x&=0  \\
\frac{\partial E_x}{\partial z}+\gamma E_z-Z H_y&=0  \\
-\gamma E_y-\frac{\partial E_x}{\partial y}-Z H_z&=0
\end{align}
%
\begin{align}
\frac{\partial H_z}{\partial y}-\frac{\partial H_y}{\partial z}-YE_x&=0 \\
\frac{\partial H_x}{\partial z}+\gamma H_z-YE_y&=0  \\
-\gamma H_y-\frac{\partial H_x}{\partial y}-YE_z&=0 
\end{align}
%
\begin{align}
-\gamma E_x+\frac{\partial E_y}{\partial y}+\frac{\partial E_z}{\partial z}&=0 \\
-\gamma H_x+\frac{\partial H_y}{\partial y}+\frac{\partial H_z}{\partial z}&=0
\end{align}
 یہ \عددیء{x} سمت میں حرکت کرتی موج کی عمومی مساوات ہیں۔

ابھی تک نا تو مویج کی شکل اور نا ہی بلند درجی موج  کا انتخاب کیا گیا ہے لہٰذا چوتھے قدم کا اطلاق کرتے ہوئے \تحریر{TE} قسم کا انتخاب کرتے ہیں جس کا مطلب ہے کہ \عددیء{E_x=0} لیا جائے گا۔ایسا کرنے سے مندرجہ بالا مساوات   
\begin{align}
\frac{\partial E_z}{\partial y}-\frac{\partial E_y}{\partial z}-Z H_x&=0  \label{مساوات_میوج_الف}\\
\gamma E_z-Z H_y&=0  \label{مساوات_میوج_ب}\\
-\gamma E_y-Z H_z&=0\label{مساوات_میوج_پ}
\end{align}
%
\begin{align}
\frac{\partial H_z}{\partial y}-\frac{\partial H_y}{\partial z}&=0 \label{مساوات_میوج_ت}\\
\frac{\partial H_x}{\partial z}+\gamma H_z-YE_y&=0  \label{مساوات_میوج_ٹ}\\
-\gamma H_y-\frac{\partial H_x}{\partial y}-YE_z&=0 \label{مساوات_میوج_ث}
\end{align}
%
\begin{align}
\frac{\partial E_y}{\partial y}+\frac{\partial E_z}{\partial z}&=0 \label{مساوات_میوج_ج}\\
-\gamma H_x+\frac{\partial H_y}{\partial y}+\frac{\partial H_z}{\partial z}&=0\label{مساوات_میوج_چ}
\end{align}
صورت اختیار کر لیتے ہیں۔

پانچویں قدم پر تمام مساوات کو \عددیء{H_x} کی صورت میں لکھنا ہو گا۔ایسا کرنے کی خاطر پہلے مساوات \حوالہ{مساوات_میوج_ب} اور \حوالہ{مساوات_میوج_پ} سے
\begin{align}\label{مساوات_میوج_ح}
\frac{E_z}{H_y}=-\frac{E_y}{H_z}=\frac{Z}{\gamma}
\end{align}
لکھتے ہیں۔اب \عددیء{\tfrac{E_z}{H_y}} یا \عددیء{\tfrac{E_y}{H_z}} کی شرح قدرتی رکاوٹ کی مانند ہے۔چونکہ  مساوات \حوالہ{مساوات_میوج_ح} میں صرف عرضی اجزاء پائے جاتے ہیں لہٰذا اس شرح کو \اصطلاح{عرضی-موج کی قدرتی رکاوٹ}\فرہنگ{عرضی-موج!قدرتی رکاوٹ}\فرہنگ{قدرتی رکاوٹ!عرضی موج}\حاشیہب{transverse-wave impedance}\فرہنگ{impedance!transverse-wave}\فرہنگ{transverse!wave impedance} \عددیء{Z_{yz}} کہا جائے گا جہاں
\begin{align}\label{مساوات_میوج_خ}
Z_{yz}=\frac{E_y}{H_z}=-\frac{E_z}{H_y}=-\frac{Z}{\gamma}=\frac{j\omega \mu}{\gamma}  \quad \quad (\si{\ohm})
\end{align}
کے برابر ہے۔مساوات \حوالہ{مساوات_میوج_خ} کو مساوات \حوالہ{مساوات_میوج_ث} میں پر کرتے ہوئے \عددیء{H_y} کے لئے حل کرنے سے
\begin{align}\label{مساوات_میوج_د}
H_y=\frac{-1}{\gamma-Y Z_{yz}} \frac{\partial H_x}{\partial y}
\end{align}
حاصل ہوتا ہے۔اسی طرح  مساوات \حوالہ{مساوات_میوج_خ} کو مساوات \حوالہ{مساوات_میوج_ٹ} میں پر کرتے ہوئے \عددیء{H_z} کے لئے حل کرنے سے
\begin{align}\label{مساوات_میوج_ڈ}
H_z=\frac{-1}{\gamma-Y Z_{yz}} \frac{\partial H_x}{\partial z}
\end{align}
حاصل ہوتا ہے۔اب مساوات \حوالہ{مساوات_میوج_د} کو مساوات \حوالہ{مساوات_میوج_خ} میں پر کرتے ہوئے
\begin{align}\label{مساوات_میوج_ذ}
E_z=\frac{Z_{yz}}{\gamma-Y Z_{yz}}\frac{\partial H_x}{\partial y}
\end{align}
اور مساوات \حوالہ{مساوات_میوج_ڈ} کو مساوات \حوالہ{مساوات_میوج_خ} میں پر کرتے ہوئے
\begin{align}\label{مساوات_میوج_ر}
E_y=\frac{-Z_{yz}}{\gamma-Y Z_{yz}}\frac{\partial H_x}{\partial z}
\end{align}
حاصل ہوتے ہیں۔ مساوات \حوالہ{مساوات_میوج_د} تا مساوات \حوالہ{مساوات_میوج_ر} تمام اجزاء کو \عددیء{H_x} کی صورت میں پیش کرتے ہیں۔ 

چھٹے قدم پر ان مساوات سے موج کی مساوات کا حصول ہے۔ایسا کرنے کی خاطر مساوات \حوالہ{مساوات_میوج_د} کا \عددیء{y} کے ساتھ تفرق اور مساوات \حوالہ{مساوات_میوج_ڈ} کا \عددیء{z} کے ساتھ تفرق لیتے ہوئے دونوں حاصل جواب کو مساوات \حوالہ{مساوات_میوج_چ} میں پر کرتے ہوئے
\begin{align*}
-\gamma H_x-\frac{1}{\gamma -Y Z_{yz}} \left(\frac{\partial^2 H_x}{\partial y^2}+\frac{\partial^2 H_x}{\partial z^2} \right)=0
\end{align*}
یا
\begin{align*}
\frac{\partial^2 H_x}{\partial y^2}+\frac{\partial^2 H_x}{\partial z^2}+\gamma \left(\gamma-Y Z_{yz} \right)H_x =0
\end{align*}
حاصل کرتے ہیں جس میں
\begin{align}\label{مساوات_میوج_ڑ}
k^2=\gamma \left(\gamma-Y Z_{yz}\right)
\end{align}
پر کرتے ہوئے
\begin{align}\label{مساوات_میوج_ز}
\frac{\partial^2 H_x}{\partial y^2}+\frac{\partial^2 H_x}{\partial z^2}+k^2 H_x =0
\end{align}
لکھا جا سکتا ہے۔مساوات \حوالہ{مساوات_میوج_ز} کسی بھی شکل کے مویج کے عرضی برقی موج کی مساوات ہے۔یہاں چھٹا قدم پورا ہوتا ہے۔

ساتویں قدم میں مویج کے اطراف کے سرحدی شرائط لاگو کرتے ہوئے، اس موج کو حل کرنا ہے۔ 


