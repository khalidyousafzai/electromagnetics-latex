\باب{کولومب کا قانون}
\حصہ{قوت کشش یا دفع}
نیوٹن کے \اصطلاح{کائناتی تجاذب کے قانون}\فرہنگ{تجاذب}\حاشیہب{Law of Universal Gravitation} سے آپ بخوبی واقف ہوں گے۔\اصطلاح{کولومب کا قانون}\فرہنگ{کولومب کا قانون}\حاشیہب{Coulomb's law} اس سے قریبی مشابہت رکھتا ہے۔کائناتی تجاذب کے قانون کو مساوات \حوالہ{مساوات_کولوم_کشش_ثقل} میں پیش کیا گیا ہے۔
\begin{align}\label{مساوات_کولوم_کشش_ثقل}
F&=G \frac{M_1 M_2}{R^2}
\end{align}
یہ مساوات کمیت \عددیء{M_1} اور کمیت \عددیء{M_2} کے مابین قوت کشش \عددیء{F} دیتا ہے جہاں ایک کمیت کے مرکز سے دوسری کمیت کے مرکز تک کا فاصلہ \عددیء{R} ہے۔قوت کشش دونوں کمیت کے حاصل ضرب کے  راست متناسب اور ان کے مرکزوں کے درمیانی فاصلے  کے مربع کے بالعکس متناسب ہوتی ہے۔دونوں کمیتوں پر قوت کشش کی مقدار برابر ہوتی ہے اور یہ قوت دونوں کمیتوں کے  مرکزوں پر کھینچی لکیر پر عمل درآمد ہوتی ہے۔\عددیء{M_1} پر قوت کشش کی سمت \عددیء{M_1} کے مرکز سے \عددیء{M_2} کے مرکز کی جانب کو ہوتا ہے جبکہ \عددیء{M_2} پر قوت کشش کی سمت \عددیء{M_2} کے مرکز سے \عددیء{M_1} کے مرکز کی جانب کو ہوتا ہے۔تناسب کے جزو مستقل کو \عددیء{G} لکھا اور \اصطلاح{تجاذبی مستقل}\فرہنگ{تجاذبی مستقل}\حاشیہب{gravitational constant}\فرہنگ{gravitational constant} پکارا جاتا ہے جس کی قیمت تقریباً \عددیء{\SI{6.674e-11}{\meter \cubed \per \kilo \gram \per \second \squared}} کے برابر ہے۔

کولومب کا قانون مساوات \حوالہ{مساوات_کولوم_کولومب_کشش_چارج} میں بیان کیا گیا ہے۔یہ مساوات چارج \عددیء{Q_1} اور چارج \عددیء{Q_2} کے مابین قوت کشش یا قوت دفع \عددیء{F} دیتا ہے جہاں ایک چارج کے مرکز سے دوسری چارج کے مرکز تک کا فاصلہ \عددیء{R} ہے۔ان چارجوں کا حجم صفر تصور کیا جاتا ہے۔یوں اگر چارج کو گیند کی شکل کا تصور کیا جائے تو اس گیند کے رداس  کی لمبائی صفر ہو گی۔ایسے چارج کو \اصطلاح{نقطہ چارج}\فرہنگ{نقطہ چارج}\حاشیہب{point charge}\فرہنگ{point charge} کہا جاتا ہے۔
\begin{align}\label{مساوات_کولوم_کولومب_کشش_چارج}
F&=\frac{1}{4 \pi \epsilon_0}\frac{Q_1 Q_2}{ R^2}
\end{align}

قوت کشش یا دفع دونوں چارجوں کے حاصل ضرب کے  راست متناسب  اور باہمی فاصلہ کے  مربع کے بالعکس متناسب ہوتی ہے۔دونوں چارجوں پر قوت کی مقدار برابر ہوتی ہے اور یہ قوت دونوں چارجوں سے گزرتی لکیر پر عمل درآمد ہوتی ہے۔دو مختلف اقسام کے چارجوں کے مابین قوت کشش پائی جاتی ہے جبکہ دو یکساں چارجوں کے مابین قوت دفع پائی جاتی ہے۔مساوات کے جزو مستقل کو \عددیء{\tfrac{1}{4 \pi \epsilon_0}} لکھا جاتا ہے جہاں \عددیء{\epsilon_0} خالی خلاء  کا \اصطلاح{برقی مستقل}\فرہنگ{برقی مستقل}\حاشیہب{permittivity}\فرہنگ{permittivity}\حاشیہب{electric constant}\فرہنگ{electric constant} ہے جس کی قیمت اٹل ہے۔خالی خلاء کے برقی مستقل کی قیمت
\begin{align}
\epsilon_0=\frac{1}{\mu_0 c^2}
\end{align}
ہے جہاں \عددیء{c} خالی خلاء میں روشنی کی رفتار اور \عددیء{\mu_0} خالی خلاء کی \اصطلاح{مقناطیسی مستقل}\فرہنگ{مقناطیسی مستقل}\حاشیہب{permeability}\فرہنگ{permeability} ہے۔یہ دونوں بھی اٹل مستقل ہیں جن کی قیمتیں
\begin{align}
c&=\SI{299792458}{\meter \per \second}\\
\mu_0&=\SI{4 \numpi e-7}{\henry \per \meter}
\end{align}
 ہیں۔یوں مقناطیسی مستقل  کی قیمت تقریباً
\begin{align}
\epsilon_0 =\num{8.854e-12} \overset{.}{=}\frac{1}{36 \pi} 10^{-9} \si{\farad \per \meter}
\end{align}
 کے برابر ہے۔اس کتاب میں \عددیء{\tfrac{1}{4 \pi \epsilon_0}} بار بار استعمال ہو گا جسے عموماً
\begin{align}
\frac{1}{4 \pi \epsilon_0} \overset{.}{=} 9 \times 10^9
\end{align}
لیا جائے گا۔\عددیء{\epsilon_0}  کی اکائی فیراڈ فی میٹر  \عددیء{\si{\farad \per \meter}} ہے  جس کی وضاحت جلد کر دی جائے گی۔

\ابتدا{مثال}
زمین کی سطح پر زمین اور ایک کلو گرام کمیت کے مابین \عددیء{\SI{9.8}{\newton}} کی قوت کشش پائی جاتی ہے۔زمین کا رداس \عددیء{\SI{6370}{\kilo \meter}} لیتے ہوئے زمین کی کمیت حاصل کریں۔

حل:مساوات \حوالہ{مساوات_کولوم_کشش_ثقل} کی مدد سے
\begin{align*}
9.8=\frac{6.674 \times 10^{-11} \times M \times 1}{\num{6370000} \times \num{6370000}}\\
\end{align*}
لکھتے ہوئے زمین کی کمیت \عددیء{\SI{5.959e24}{\kilo \gram}} حاصل ہوتی ہے۔
\انتہا{مثال}
%===========================
\ابتدا{مثال}
زمین کی مرکز سے تقریباً \عددیء{\SI{42000}{\kilo \meter}} کے فاصلے پر  ذرائع ابلاغ کے سیٹلّائٹ زمین کے گرد مدار میں گردش کرتے ہیں۔اس فاصلے پر ایک کلا گرام کی کمیت اور زمین کے مابین قوت کشش کی مقدار حاصل کریں۔

حل: 
\begin{align*}
F=\frac{6.674 \times 10^{-11} \times 5.959 \times 10^{24} \times 1}{\num{42000000} \times \num{42000000}}=\SI{0.225}{\newton}\\
\end{align*}
\انتہا{مثال}
%============================
\ابتدا{مثال}
ایک ایک کولومب کے دو مثبت چارجوں کے درمیان  ایک میٹر کا فاصلہ ہے۔ان میں قوت دفع حاصل کریں۔

حل: 
\begin{align*}
F&=9 \times 10^9 \frac{1 \times 1}{1 \times 1}=\SI{9e9}{\newton}
\end{align*}
\انتہا{مثال}
%===================

مندرجہ بالا مثال سے  آپ دیکھ سکتے ہیں کہ چارج کی اکائی (کولومب)  انتہائی بڑی مقدار ہے۔

شکل \حوالہ{شکل_سمتیہ_دو_مثبت_چارج_قوت_دفع} میں چارج \عددیء{Q_1} محدد کے مرکز سے سمتی فاصلہ \سمتیہ{r_1} پر جبکہ چارج \عددیء{Q_2} مرکز سے سمتی فاصلہ \سمتیہ{r_2} پر دکھائے گئے ہیں۔چارج \عددیء{Q_1} سے چارج \عددیء{Q_2} تک کا سمتی فاصلہ \سمتیہ{R_{21}} ہے جہاں
\begin{align}
\kvec{R_{21}}=\kvec{r_2}-\kvec{r_1}
\end{align}
کے برابر ہے۔سمتیہ \سمتیہ{R_{21}} کی سمت میں اکائی سمتیہ \سمتیہ{a_{21}} یوں حاصل کیا جاتا ہے
\begin{align}
\kvec{a_{21}}=\frac{\kvec{R_{21}}}{\abs{\kvec{R_{21}}}}=\frac{\kvec{R_{21}}}{R_{21}}=\frac{\kvec{r_2}-\kvec{r_1}}{\abs{\kvec{r_2}-\kvec{r_1}}}
\end{align}
چارج \عددیء{Q_2} پر قوت \سمتیہ{F_2} کی حتمی قیمت مساوات  \حوالہ{مساوات_کولوم_کولومب_کشش_چارج} سے حاصل کی جا سکتی ہے جبکہ اس کی سمت اکائی سمتیہ \سمتیہ{a_{21}} کے سمت میں ہو گی۔اس طرح یہ قوت
\begin{gather}
\begin{aligned}\label{مساوات_کولومب_قوت_کی_مساوات_الف}
\kvec{F_2}&=\frac{1}{4 \pi \epsilon_0}\frac{Q_1 Q_2}{ R_{21}^2} {\kvec{a_{21}}}\\
&=\frac{Q_1 Q_2}{4 \pi \epsilon_0}\frac{\kvec{r_2}-\kvec{r_1}}{\abs{\kvec{r_2}-\kvec{r_1}}^3}
\end{aligned}
\end{gather}
لکھا جائے گا۔مساوات \حوالہ{مساوات_کولومب_قوت_کی_مساوات_الف} کولومب کے قانون کی سمتی شکل ہے۔چونکہ دونوں چارجوں پر برابر مگر الٹ سمت میں قوت عمل کرتا ہے لہٰذا \عددیء{Q_1} پر قوت \سمتیہ{F_1} یوں لکھا جائے گا
 \begin{gather}
\begin{aligned}
\kvec{F_1}=-\kvec{F_2}&=\frac{-1}{4 \pi \epsilon_0}\frac{Q_1 Q_2}{ R_{21}^2} \kvec{a_{21}}\\
&=\frac{1}{4 \pi \epsilon_0}\frac{Q_1 Q_2}{ R^2} {\kvec{a_{12}}}
\end{aligned}
\end{gather}
جہاں دوسری قدم پر \عددیء{R_{21}=R_{12}=R} لکھا گیا ہے اور \عددیء{\kvec{a_{12}}=-\kvec{a_{21}}} کے برابر ہے۔
%
\begin{figure}
\centering
\includegraphics{figCoulombForceBetweenTwoCharges}
\caption{دو مثبت چارجوں کے مابین قوت دفع}
\label{شکل_سمتیہ_دو_مثبت_چارج_قوت_دفع}
\end{figure}
دونوں چارج مثبت یا دونوں چارج منفی ہونے کی صورت میں \عددیء{Q_2} پر مساوات \حوالہ{مساوات_کولومب_قوت_کی_مساوات_الف} سے قوت \سمتیہ{a_{21}} کی سمت میں حاصل ہوتا ہے۔یوں یکساں چارجوں کے مابین قوت دفع پایا جاتا ہے۔دو الٹ اقسام کے چارجوں کی صورت میں \عددیء{Q_2} پر قوت \سمتیہ{-a_{21}} کی سمت میں حاصل ہوتا ہے۔یوں الٹ اقسام کے چارجوں کے مابین قوت کشش پایا جاتا ہے۔  

\ابتدا{مثال}
شکل \حوالہ{شکل_سمتیہ_دو_مثبت_چارج_قوت_دفع} میں نقطہ \عددیء{A(3,2,4)} پر \عددیء{\SI{20}{\micro \coulomb}} کا چارج \عددیء{Q_1} جبکہ نقطہ \عددیء{B(1,5,9)} پر \عددیء{\SI{-50}{\coulomb}} کا چارج \عددیء{Q_2} پایا جاتا ہے۔منفی چارج \عددیء{Q_2} پر سمتی قوت حاصل کریں۔

حل:
\begin{align*}
\kvec{R_{21}}&=(1-3) \ax+(5-2) \ay+(9-4)\az\\
&=-2 \ax+3 \ay+5 \az\\
R_{21}&=\abs{\kvec{R_{21}}}=\sqrt{2^2+3^2+5^2}\\
&=\sqrt{38}\\
&=6.1644
\end{align*}
اور یوں
\begin{align*}
\kvec{a_{21}}&=\frac{{\kvec{R_{21}}}}{R_{21}}=\frac{-2 {\ax}+3 {\ay}+5 \az}{6.1644}\\
&=-0.324 \ax+0.487 \ay+0.811\az
\end{align*}
حاصل ہوتا ہے جس سے
\begin{align*}
\kvec{F_2}&=\frac{36 \pi  \times 10^9}{4 \pi} \frac{\left (-50\times 10^{-6} \times 20 \times 10^{-6} \right )}{38} \left(-0.324 \ax+0.487 \ay+0.811\az\right)\\
&=-0.237  \left(-0.324 \ax+0.487 \ay+0.811\az \right) \, \si{\newton}
\end{align*}
حاصل ہوتا ہے۔آپ دیکھ سکتے ہیں کہ قوت کی سمت \سمتیہ{a_{21}} کے الٹ سمت میں ہے۔یوں منفی چارج پر قوت کی سمت مثبت چارج کی جانب ہے یعنی اس پر قوت کشش پایا جاتا ہے۔
\انتہا{مثال}
%========================
کسی بھی چارج پر ایک سے زیادہ چارجوں سے پیدا مجموعی قوت تمام چارجوں سے پیدا علیحدہ علیحدہ قوتوں کا سمتی مجموعہ ہوتا ہے یعنی
\begin{align}
\kvec{F}=\sum_{i=1}^n \kvec{F}_i
\end{align}


%========================
\حصہ{برقی میدان کی شدت}
نیوٹن کے کائناتی تجاذب کے قانون میں زمین کی کمیت کو \عددیء{M} لکھ کر کمیت \عددیء{m} پر قوت \عددیء{F} حاصل کی جا سکتی ہے۔ایک کلوگرام کمیت پر اس قوت کی مقدار \عددیء{\tfrac{F}{m}} ہو گی جسے \اصطلاح{زمین کی کشش}\فرہنگ{کشش!زمین}\حاشیہب{gravity}\فرہنگ{gravity} یا \اصطلاح{ثقلی اسراع} پکارا اور \عددیء{g} لکھا جاتا ہے۔زمین کی سطح پر \عددیء{g} کی مقدار تقریباً \عددیء{\SI{9.8}{\meter \per \second \squared}} کے برابر ہے۔
\begin{align}\label{مساوات_کولومب_زمین_کی_کشش}
g&=\frac{F}{m}=\frac{GM}{R^2}
\end{align}

کسی بھی کمیت \عددیء{M} کے گرد  \اصطلاح{تجاذبی میدان}\حاشیہب{gravitational field}\فرہنگ{تجاذبی میدان} پایا جاتا ہے۔ کسی بھی نقطے پر  اس تجاذبی میدن کو ناپنے کی خاطر اس نقطے پر پیمائشی کمیت \عددیء{m_p}\حاشیہد{\عددیء{m_p} لکھتے ہوئے  زیرنوشت میں \عددیء{p} لفظ پیمائشی  کے \موٹا{پ} کو ظاہر کرتا ہے، یعنی یہ وہ کمیت ہے جسے قوت  کی پیمائش کی خاطر استعمال کیا جا رہا ہے۔} رکھ کر اس پر قوت ناپی جاتی ہے۔مختلف مقامات پر اس طرح قوت ناپ کر ہم تجاذبی میدان کا جائزہ لے سکتے ہیں۔تجاذبی قوت کی مقدار کا دارومدار پیمائشی کمیت \حاشیہب{test mass} \عددیء{m_p} پر بھی منحصر ہے۔مختلف تجاذبی میدانوں کا آپس میں موازنہ کرتے وقت یہ ضروری ہے کہ تمام تجاذبی میدان جانچتے وقت ایک ہی قیمت کے پیمائشی کمیت استعمال کی جائے۔ماہرین طبیعیات عموماً \عددیء{m_p} کو ایک کلوگرام  رکھتے ہیں۔یہ ضروری نہیں کہ تجاذبی قوت ناپتے وقت ایک کلوگرام کی پیمائشی کمیت ہی استعمال کی جائے البتہ جوابات اکٹھے کرتے وقت \سمتیہ{F} کو \عددیء{m_p} سے تقسیم کرتے ہوئے ایک کلوگرام پر تجاذبی قوت حاصل کی جا سکتی ہے۔زمین کے قریب ایک کلو گرام کمیت پر قوت کشش کو ثقلی اسراع \عددی{g} پکارا جاتا ہے۔

\ابتدا{مثال}
زمین کی سطح پر دو سو گرام پیمائشی کمیت پر \عددیء{\SI{1.96}{\newton}} قوت ناپی جاتی ہے۔ثقلی اسراع حاصل کریں۔

حل:
\begin{align}
g=\frac{1.96}{0.2}=\SI{9.8}{\newton \per \kilogram}
\end{align}   
\انتہا{مثال}

مساوات \حوالہ{مساوات_کولومب_زمین_کی_کشش} سے ہم
\begin{gather}
\begin{aligned}
F&=m g\\
w&=m g
\end{aligned}
\end{gather}
لکھ سکتے ہیں جو زمین کی  سطح پر کمیت \عددیء{m} پر کشش ثقل \عددیء{F} دیتا ہے جسے  وزن پکارا اور  \عددیء{w} لکھا جاتا ہے۔
  
چارجوں پر بھی اسی طرح غور کیا جاتا ہے۔کسی بھی چارج \عددیء{Q} کے گرد برقی میدان پایا جاتا ہے۔اس برقی میدان میں چارج پر قوت اثر انداز ہوتا ہے۔چارج \عددیء{Q} کے برقی میدان کی شدت کے پیمائش کی خاطر اس میدان میں مختلف مقامات پر پیمائشی چارج \حاشیہب{test charge} \عددیء{q_p} پر قوت \سمتیہ{F} ناپ کر برقی میدان کا مطالعہ کیا جا سکتا ہے اور اس کا نقشہ بنایا جا سکتا ہے۔مختلف چارجوں کے برقی میدانوں کا آپس میں موازنہ کرتے وقت یہ ضروری ہے کہ تمام صورتوں میں ایک ہی قیمت کے پیمائشی چارج  استعمال کئے جائیں۔ماہرین طبیعیات \عددیء{q_p} کو ایک کولومب کا مثبت چارج  رکھتے ہیں۔یہ ضروری نہیں کہ قوت ناپتے وقت ایک کولومب کا مثبت پیمائشی چارج ہی استعمال کیا جائے البتہ جوابات اکٹھے کرتے وقت \سمتیہ{F} کو \عددیء{q_p} سے تقسیم کرتے ہوئے ایک مثبت کولومب پر قوت حاصل کی جاتی ہے جسے \اصطلاح{برقی میدان کی شدت}\فرہنگ{برقی میدان کی شدت}\حاشیہب{electric field intensity}\فرہنگ{electric field intensity} یا صرف \اصطلاح{برقی میدان} پکارا اور \سمتیہ{E} لکھا جاتا ہے یعنی
\begin{align}\label{مساوات_کولومب_برقی_شدت_اور_قوت}
\kvec{E}=\frac{\kvec{F}}{q_p}
\end{align}
مختلف مقامات پر موجود مختلف قیمتوں کے چارجوں سے کسی ایک نقطے پر  پیدا برقی میدان  تمام چارجوں کے مجموعی اثر سے پیدا ہوگا۔کسی بھی نقطے پر \عددیء{n} چارجوں کا مجموعی \سمتیہ{E} تمام چارجوں کے علیحدہ علیحدہ پیدا کردہ  \سمتیازیرنوشت{E}{1}،  \سمتیازیرنوشت{E}{2}،\سمتیازیرنوشت{E}{3}،\نقطے  کا سمتی مجموعہ
\begin{align}
\kvec{E}=\sum_{i=1}^n \kvec{E}_i=\kvecsub{E}{1}+\kvecsub{E}{2}+\kvecsub{E}{3}+\cdots 
\end{align}
 ہوتا ہے۔یوں کسی بھی نقطے \عددیء{P}  پر \سمتیہ{E} ناپتے وقت اس نقطے پر ایک کولومب چارج \عددیء{q_p} رکھ کر اس چارج پر قوت ناپی جاتی ہے۔یہ قوت اس نقطے پر تمام چارجوں کا مجموعی  \سمتیہ{E} ہوتا ہے۔یاد رہے کہ کسی بھی نقطے پر  \سمتیہ{E} ناپتے وقت یہاں رکھے پیمائشی چارج \عددیء{q_p} کا اثر شامل نہیں ہوتا۔

مساوات  \حوالہ{مساوات_کولومب_قوت_کی_مساوات_الف} سے  چارج \عددیء{Q} سے \عددیء{\kvec{a}_R} سمت میں \عددیء{R} فاصلے پر برقی میدان کو
\begin{gather}
\begin{aligned}\label{مساوات_کولومب_قوت_کی_عمومی_مساوات}
\kvec{E}&=\frac{Q}{4 \pi \epsilon_0}\frac{\kvec{a_{R}}}{ R^2}\\
&=\frac{Q}{4 \pi \epsilon_0}\frac{\kvec{R}}{ R^3}
\end{aligned}
\end{gather}
لکھا جا سکتا ہے۔چارج کو کروی محدد کے مرکز پر تصور کرتے ہوئے اسی مساوات کو یوں لکھا جا سکتا ہے
\begin{align}\label{مساوات_کولومب_نقطہ_چارج_کروی_رداس_پر_تبدیل_نہیں_ہوتا}
\kvec{E}=\frac{Q}{4 \pi \epsilon_0 r^2} \ar
\end{align}
جہاں  \عددیء{\ar}  کروی محدد کا رداسی سمت میں اکائی سمتیہ ہے۔

نقطہ \عددیء{(x',y',z')} پر موجود چارج \عددیء{Q} سے نقطہ \عددیء{(x,y,z)} پر برقی شدت  یوں حاصل کی جا سکتی ہے۔
\begin{gather}
\begin{aligned} \label{مساوات_کولومب_نقطہ_چارج_سے_میدان_کا_حصول}
\kvec{E}&=\frac{1}{4 \pi \epsilon_0 }\frac{Q}{\abs{\kvec{r}-\kvec{r'}}^2} \frac{\kvec{r}-\kvec{r'}}{\abs{\kvec{r}-\kvec{r'}}}=\frac{Q}{4 \pi \epsilon_0 }\frac{\kvec{r}-\kvec{r'}}{\abs{\kvec{r}-\kvec{r'}}^3}\\
&=\frac{Q}{4 \pi \epsilon_0}\frac{\left[(x-x')\ax+(y-y')\ay+(z-z')\az\right]}{\left [(x-x')^2+(y-y')^2+(z-z')^2\right]^{\frac{3}{2}}}
\end{aligned}
\end{gather} 
جہاں
\begin{align*}
\kvec{r}&=x \ax+y \ay+z\az\\
\kvec{r'}&=x' \ax+y' \ay+z'\az\\
\kvec{R}&=\kvec{r}-\kvec{r'}=(x-x') \ax+(y-y') \ay+(z-z')\az
\end{align*}
کے برابر ہے۔
%===================
\ابتدا{مثال}\حاشیہط{چارج کم کرتے ہوئے جوابات 0 تا 1 کریں}
نقطہ \عددیء{N_1(4,1,1)} پر \عددیء{\SI{100}{\micro \coulomb}} کا چارج \عددیء{Q_1} جبکہ نقطہ \عددیء{N_2(1,4,2)} پر \عددیء{\SI{50}{\micro \coulomb}} کا چارج \عددیء{Q_2} پایا جاتا ہے۔نقطہ \عددیء{N_3(2,2,5)} پر  \عددیء{Q_1} سے پیدا \سمتیازیرنوشت{E}{1}  اور \عددیء{Q_2} سے پیدا \سمتیازیرنوشت{E}{2} حاصل کریں۔اس نقطے پر دونوں چارجوں کا مجموعی \عددیء{\kvec{E}} کیا ہو گا۔
\begin{figure}
\centering
\includegraphics[height=3.5cm]{figCoulombElectricFieldOfTwoCharges}
\caption{دو چارجوں سے پیدا برقی شدت}
\label{شکل_کولومب_دو_چارجوں_سے_پیدا_برقی_شدت}
\end{figure}

حل:شکل \حوالہ{شکل_کولومب_دو_چارجوں_سے_پیدا_برقی_شدت} میں صورت حال دکھایا گیا ہے۔پہلے \عددیء{Q_1} سے پیدا \سمتیازیرنوشت{E}{1} حاصل کرتے ہیں۔\عددیء{N_1}  سے \عددیء{N_3} تک سمتی فاصلہ
\begin{align*}
\kvecsub{R}{31}=\kvecsub{R}{3}-\kvecsub{R}{1}&=(2-4)\ax+(2-1)\ay+(5-1)\az\\
&=-2\ax+1\ay+4\az
\end{align*}
ہے جس سے
\begin{align*}
R_{31}&=\abs{\kvecsub{R}{31}}=\sqrt{2^2+1^2+4^2}\\
&=\sqrt{21}=4.583\\
\kvecsub{a}{31}&=\frac{\kvecsub{R}{31}}{R_{31}}=\frac{-2\ax+1\ay+4\az}{\sqrt{21}}\\
&=-0.436\ax+0.218\ay+0.873\az
\end{align*}
حاصل ہوتے ہیں۔یوں
\begin{align*}
\kvecsub{E}{1}&=9 \times 10^9 \frac{100 \times 10^{-6}}{21} \left(-0.436\ax+0.218\ay+0.873\az\right)\\
&=-\num{18686}\ax+\num{9343}\ay+\num{37414}\az \quad \si{\volt \per \meter}
\end{align*}
ہوتا ہے۔اسی طرح \عددیء{Q_2} کے لئے حل کرتے ہوئے
\begin{align*}
\kvecsub{R}{32}&=(2-1)\ax+(2-4)\ay+(5-2)\az\\
&=1\ax-2\ay+3\az
\end{align*}
اور
\begin{align*}
R_{32}&=\abs{\kvecsub{R}{32}}=\sqrt{1^2+2^2+3^2}=\sqrt{14}\\
\kvecsub{a}{32}&=\frac{1\ax-2\ay+3\az}{\sqrt{14}}\\
&=0.267\ax-0.535\ay+0.802\az
\end{align*}
سے
\begin{align*}
\kvecsub{E}{2}&=9 \times 10^9 \frac{50 \times 10^{-6}}{14} \left(0.267\ax-0.535\ay+0.802\az \right)\\
&=\num{8582}\ax-\num{17196}\ay+\num{25779}\az \quad \si{\volt \per \meter}
\end{align*}
ملتا ہے۔ان دو جوابات کا سمتی مجموعہ لیتے ہوئے کُل \kvec{E} حاصل کرتے ہیں۔
\begin{align*}
\kvec{E}&=\kvecsub{E}{1}+\kvecsub{E}{2}\\
&=\left(\num{-18686} \ax+\num{9343} \ay+\num{37414}\az\right)+\left(\num{8582}\ax-\num{17196}\ay+\num{25779}\az\right)\\
&=-\num{10104}\ax-\num{7853}\ay+\num{63193}\az \quad \si{\volt \per \meter}
\end{align*} 
\انتہا{مثال}
%===================
 
مساوات \حوالہ{مساوات_کولومب_برقی_شدت_اور_قوت} کو
\begin{align}
\kvec{F}=q \kvec{E}
\end{align}
لکھا جا سکتا ہے جو برقی میدان \سمتیہ{E} کے موجودگی میں چارج \عددیء{q} پر قوت \سمتیہ{F} دیتا ہے۔


\حصہ{یکساں چارج بردار سیدھی لامحدود لکیر  کا برقی میدان}
شکل \حوالہ{شکل_کولومب_لامحدود_لکیر_پر_چارج_کا_میدان} میں \عددیء{z} محدد پر \عددیء{z=-\infty} سے \عددیء{z=+\infty} تک یکساں چارج کی کثافت پائی جاتی ہے۔آپ تصور کر سکتے ہیں کہ \عددیء{z} محدد پر انتہائی قریب قریب برابر فاصلے پر یکساں \اصطلاح{نقطہ چارج} رکھے گئے ہیں۔یوں اگر \عددیء{\Delta L} لمبائی میں کُل \عددیء{\Delta Q} چارج پایا جائے تب اکائی لمبائی میں \عددیء{\tfrac{\Delta Q}{\Delta L}} چارج پایا جائے گا جسے \اصطلاح{لکیری چارج کثافت}\فرہنگ{کثافت!لکیری چارج}\حاشیہب{line charge density}\فرہنگ{density!line charge}  \عددیء{\rho_L}\حاشیہد{اس کتاب میں رداس کے لئے بھی \عددیء{\rho} استعمال کیا جاتا ہے۔\عددیء{\rho} کو جب بھی کثافت کے لئے استعمال کیا جائے،اس کے زیر نوشت میں \تحریر{L}، \تحریر{S} یا \تحریر{h} لکھا جائے گا۔} کہا جاتا ہے اور جس کی اکائی \عددیء{\si{\coulomb / \meter}} ہے۔لکیری چارج کثافت کی تعریف
\begin{align}
\rho_L=\lim_{\Delta L \to 0}\frac{\Delta Q}{\Delta L}
\end{align}
ہے۔لکیر پر چھوٹی لمبائی  اتنی کم نہیں کی جاتی کہ چارج بردار الیکٹران علیحدہ علیحدہ نظر آئیں اور لکیری کثافت کی جگہ نقطہ چارج نظر آئیں۔اگر لکیر پر چارج کی تقسیم ہر جگہ یکساں نہ ہو تب لکیری چارج کثافت متغیر ہو گی۔آئیں یکساں لکیری چارج کثافت سے خالی خلاء میں پیدا برقی میدان پر غور کریں۔ 
 \begin{figure}
\centering
\includegraphics[height=3.5cm]{figCoulombElectricFieldOfInfiniteLineCharge}
\caption{یکساں چارج بردار سیدھی لامحدود لکیر  کا برقی میدان}
\label{شکل_کولومب_لامحدود_لکیر_پر_چارج_کا_میدان}
\end{figure}

پہلے بغیر قلم اٹھائے اس مسئلے کی نوعیت پر توجہ دیتے ہیں۔مقام \عددیء{(0,0,z)} پر  چھوٹی سی لمبائی \عددیء{\Delta z} میں \عددیء{\rho_L \Delta z} چارج پایا جاتا ہے جسے نقطہ چارج تصور کرتے ہوئے آگے بڑھتے ہیں۔\عددیء{z} محدد کے گرد \عددیء{z=0} یعنی  \عددیء{xy} سطح پر  شکل \حوالہ{شکل_کولومب_لامحدود_لکیر_پر_چارج_کا_میدان} میں نقطہ دار گول دائرہ بنایا گیا ہے۔نقطہ چارج \عددیء{\rho_L \Delta z} سے دائرے پر کسی بھی مقام پر پیدا برقی میدان پر غور کرتے ہیں۔برقی میدان کی مقدار کا دارومدار میدان پیدا کرنے والے چارج اور چارج سے فاصلے پر ہے۔نقطہ دار لکیر پر پائے جانے والے تمام نقطوں کا \عددیء{(0,0,z)} سے فاصلہ برابر ہے۔یوں ہم توقع کرتے ہیں کہ اس دائرے پر برقی میدان کی شدت کی حتمی قیمت ہر جگہ برابر ہو گی۔اس کو یوں بھی بیان کیا جا سکتا ہے کہ چارج کی نقطہ نظر سے نقطہ دار لکیر پر تمام نقطے بالکل یکساں نظر آتے ہیں۔اس مشابہت سے ہم کہہ سکتے ہیں کہ نقطہ دار دائرے پر ہر جگہ برقی میدان یکساں ہو گا۔ 

آئیں شکل \حوالہ{شکل_کولومب_لامحدود_لکیر_پر_چارج_کا_میدان} کو دیکھتے ہوئے  ایک اور مشابہت پر غور کرتے ہیں۔چونکہ \سمتیہ{E} سمتی فاصلہ \سمتیہ{R} کی سمت میں ہوتا ہے لہٰذا دائرے پر کسی بھی نقطے پر نقطہ چارج \عددیء{\rho_L \Delta z} سے پیدا  \عددیء{\kvec{E}} کے دو اجزاء پائے جائیں گے یعنی
\begin{align}
\kvec{E}=\kvec{E}_\rho+\kvec{E}_z
\end{align}
مثبت \عددیء{\rho_L} کی صورت میں \عددیء{(0,0,z)} پر موجود چارج سے  \عددیء{\kvec{E}_z} کی سمت منفی  \عددیء{z} جانب ہو گی۔اسی طرح \عددیء{(0,0,-z)} پر پائے جانے والے مثبت چارج سے دائرے پر پیدا \عددیء{\kvec{E}} کی سمت مثبت \عددیء{z} جانب ہو گی۔دائرے پر یہ دونوں ارکان ایک دونوں کو ختم کریں گے۔اسی عمل سے دائرے پر کسی بھی نقطے پر مثبت \عددیء{z} محدد پر کسی  بھی  فاصلے پر پائے جانے والے چارج سے پیدا \عددیء{\kvec{E}_z} کے اثر کو منفی \عددیء{z} محدد پر اتنے ہی فاصلے پر چارج سے پیدا  \عددیء{\kvec{E}_z} ختم کرتا ہے۔یوں دائرے پر
\begin{align}\label{مساوات_کولومب_لامحدود_چارج_رداس_کا_میدان_صفر}
\kvec{E}_z=0
\end{align}
ہو گا۔

ایک آخری مشابہت پر اب غور کرتے ہیں۔اگر نقطہ دار دائرے کو \عددیء{z} محدد پر مثبت یا منفی جانب  لے جایا جائے تو کیا ہو گا؟ اب بھی دائرے  کے ایک جانب کسی بھی فاصلے پر چارج کا اثر دائرے کے دوسری جانب اتنے ہی فاصلے پر چارج ختم کرے گا۔یوں دائرے کے ایک جانب یعنی \عددیء{z} محدد پر \عددیء{\infty} تک فاصلے پر چارجوں کے \عددیء{\kvec{E}_z} کو دائرے کی دوسری جانب \عددیء{z} محدد پر \عددیء{-\infty} تک فاصلے پر چارجوں کا \عددیء{\kvec{E}_z} ختم کرے گا اور یوں خلاء میں  ہر جگہ مساوات \حوالہ{مساوات_کولومب_لامحدود_چارج_رداس_کا_میدان_صفر} درست ثابت ہوتا ہے۔اس حقیقت کو یوں بہتر بیان کیا جا سکتا ہے کہ  لامحدود لکیر پر یکساں کثافت چارج  سے خلاء میں برقی میدان صرف رداس کی سمت میں پیدا ہو گا۔آئیں  اس \سمتیہ{E} کو حاصل کریں۔

شکل \حوالہ{شکل_کولومب_لامحدود_لکیر_پر_چارج_کا_میدان} میں مقام \عددیء{z} پر نقطہ چارج \عددیء{\rho_L \Delta z}  دائرے پر \عددیء{\Delta \kvec{E}} پیدا کرتا ہے۔محدد کے مرکز سے نقطہ چارج کا مقام سمتیہ \عددیء{z \az} سے ظاہر کیا جا سکتا ہے جبکہ دائرے پر کسی بھی نقطے \عددیء{N} کو سمتیہ \عددیء{\rho \arho} ظاہر کرتا ہے۔یوں نقطہ چارج سے  \عددیء{N} تک کا سمتی فاصلہ اور اسی سمت میں اکائی سمتیہ یوں حاصل کئے جائیں گے۔
\begin{align*}
\kvec{R}&=\rho \arho-z\az\\
\abs{\kvec{R}}&=R=\sqrt{\rho^2+z^2}\\
\kvec{a}_R&=\frac{\kvec{R}}{\abs{\kvec{R}}}=\frac{\rho \arho-z\az}{\sqrt{\rho^2+z^2}}
\end{align*}
مساوات \حوالہ{مساوات_کولومب_نقطہ_چارج_کروی_رداس_پر_تبدیل_نہیں_ہوتا} سے
\begin{align*}
\Delta \kvec{E}&=\frac{\rho_L \Delta z}{4 \pi \epsilon_0 \left(\rho^2+z^2 \right)} \frac{\rho \arho-z\az}{\sqrt{\rho^2+z^2}}\\
&=\frac{\rho_L \Delta z \left(\rho \arho-z\az \right)}{4 \pi \epsilon_0\left(\rho^2+z^2 \right)^{\frac{3}{2}}}
\end{align*}
حاصل ہوتا ہے۔تمام چارجوں کے اثرات کو یکجا کرنے کی خاطر مندرجہ بالا مساوات کو تکمل کی شکل دے کر مندرجہ ذیل مساوات میں دکھایا گیا ہے۔تکملہ کے حدود \عددیء{-\infty} اور \عددیء{+\infty}  ہیں۔
\begin{align}
\kvec{E}=\int \dif  \kvec{E}=\int\limits_{-\infty}^{+\infty} \left[\frac{\rho_L \left(\rho \arho-z\az \right)}{4 \pi \epsilon_0\left(\rho^2+z^2 \right)^{\frac{3}{2}}}\right] \dif z
\end{align}
اس تکمل کو یوں لکھا جا سکتا ہے
\begin{align}\label{مساوات_کولومب_لامحدود_لکیر_چارج_کا_میدان}
\kvec{E}&=\frac{\rho_L \rho \arho}{4 \pi \epsilon_0}\int\limits_{-\infty}^{+\infty} \frac{\dif z }{\left(\rho^2+z^2 \right)^{\frac{3}{2}}} -\frac{\rho_L \az}{4 \pi \epsilon_0}\int\limits_{-\infty}^{+\infty} \frac{z \dif z}{\left(\rho^2+z^2 \right)^{\frac{3}{2}}}
\end{align}
جہاں مساوات کی نشان کے دائیں جانب پہلا تکمل  \Erho اور دوسرا تکمل \Ez دیتا ہے  یعنی
\begin{gather}
\begin{aligned}\label{مساوات_کولومب_رداسی_اور_عمودی_برقی_میدان}
\Erho&=\frac{\rho_L \rho \arho}{4 \pi \epsilon_0}\int\limits_{-\infty}^{+\infty} \frac{\dif z }{\left(\rho^2+z^2 \right)^{\frac{3}{2}}}\\
\Ez&=-\frac{\rho_L \az}{4 \pi \epsilon_0}\int\limits_{-\infty}^{+\infty} \frac{z \dif z}{\left(\rho^2+z^2 \right)^{\frac{3}{2}}}
\end{aligned}
\end{gather}

مساوات \حوالہ{مساوات_کولومب_لامحدود_چارج_رداس_کا_میدان_صفر} کی مدد سے ہم دیکھ سکتے ہیں کہ دوسرا تکملہ صفر جواب دیگا۔آئیں دونوں تکمل کو باری باری حل کریں۔پہلے \Erho حل کرتے ہیں۔اس مساوات میں
\begin{align*}
z=\rho \tan \alpha
\end{align*}
استعمال کرتے ہیں۔ایسا کرتے ہوئے تکمل کا ابتدائی حد 
\begin{align*}
-\infty &= \rho \tan \alpha_{\textrm{ابتدائی}}\\
\alpha_{\textrm{ابتدائی}} &= -\frac{\pi}{2}
\end{align*}
اور اختتامی حد
\begin{align*}
\infty &= \rho \tan \alpha_{\textrm{اختتامی}}\\
\alpha_{\textrm{اختتامی}} &= \frac{\pi}{2}
\end{align*}
حاصل ہوتے ہیں۔مزید
\begin{align*}
\dif z=\rho \sec^2 \alpha \dif \alpha
\end{align*}
لکھا جائے گا۔یوں
\begin{align*}
\Erho&=\frac{\rho_L \rho \arho}{4 \pi \epsilon_0}\int\limits_{-\frac{\pi}{2}}^{+\frac{\pi}{2}} \frac{\rho \sec^2 \alpha \dif \alpha}{\left(\rho^2+\rho^2 \tan^2 \alpha \right)^{\frac{3}{2}}}\\
&=\frac{\rho_L \rho \arho}{4 \pi \epsilon_0}\int\limits_{-\frac{\pi}{2}}^{+\frac{\pi}{2}} \frac{\rho \sec^2 \alpha \dif \alpha}{\rho^3 \left(1+ \tan^2 \alpha \right)^{\frac{3}{2}}}
\end{align*}
لکھا جائے گا جس میں
\begin{align*}
1+\tan^2 \alpha=\sec^2 \alpha
\end{align*}
استعمال کرتے ہوئے
\begin{gather}
\begin{aligned}\label{مساوات_کولومب_لامحدود_لکیر_رداسی_میدان}
\Erho&=\frac{\rho_L \rho \arho}{4 \pi \epsilon_0}\int\limits_{-\frac{\pi}{2}}^{+\frac{\pi}{2}} \frac{\rho \sec^2 \alpha \dif \alpha}{\rho^3 \sec^3 \alpha}\\
&=\frac{\rho_L  \arho}{4 \pi \epsilon_0 \rho}\int\limits_{-\frac{\pi}{2}}^{+\frac{\pi}{2}} \cos \alpha \dif \alpha\\
&=\frac{\rho_L  \arho}{4 \pi \epsilon_0 \rho} \eval{\sin \alpha}_{-\frac{\pi}{2}}^{\frac{+\pi}{2}}\\
&=\frac{\rho_L }{2 \pi \epsilon_0 \rho} \arho
\end{aligned}
\end{gather}
ملتا ہے جہاں دوسری  قدم پر \عددیء{\sec \alpha=\tfrac{1}{\cos \alpha}} کا استعمال کیا گیا۔

آئیں اب مساوات \حوالہ{مساوات_کولومب_رداسی_اور_عمودی_برقی_میدان} کے   دوسرے جزو کو حل کریں۔اس میں بھی \عددیء{z= \rho \tan \alpha} استعمال کرتے ہیں۔یوں
\begin{align*}
\Ez&=-\frac{\rho_L \az}{4 \pi \epsilon_0}\int\limits_{-\infty}^{+\infty} \frac{z \dif z}{\left(\rho^2+z^2 \right)^{\frac{3}{2}}}\\
&=-\frac{\rho_L \az}{4 \pi \epsilon_0}\int\limits_{-\frac{\pi}{2}}^{+\frac{\pi}{2}} \frac{\rho^2 \tan \alpha  \sec^2 \alpha \dif \alpha}{\left(\rho^2+\rho^2 \tan^2 \alpha \right)^{\frac{3}{2}}}\\
&=-\frac{\rho_L \az}{4 \pi \epsilon_0 \rho}\int\limits_{-\frac{\pi}{2}}^{+\frac{\pi}{2}} \frac{ \tan \alpha  \sec^2 \alpha \dif \alpha}{\left(1+ \tan^2 \alpha \right)^{\frac{3}{2}}}
\end{align*}
سے
\begin{gather}
\begin{aligned}\label{مساوات_کولومب_لامحدود_لکیر_عمودی_میدان}
\Ez&=-\frac{\rho_L \az}{4 \pi \epsilon_0 \rho}\int\limits_{-\frac{\pi}{2}}^{+\frac{\pi}{2}} \frac{ \tan \alpha  \sec^2 \alpha \dif \alpha}{\sec^3 \alpha}\\
&=-\frac{\rho_L \az}{4 \pi \epsilon_0 \rho}\int\limits_{-\frac{\pi}{2}}^{+\frac{\pi}{2}} \sin \alpha \dif \alpha\\
&=\frac{\rho_L \az}{4 \pi \epsilon_0 \rho} \eval{\cos \alpha}_{-\frac{\pi}{2}}^{+\frac{\pi}{2}}\\
&=0
\end{aligned}
\end{gather}
ملتا ہے۔یہی جواب مساوات \حوالہ{مساوات_کولومب_لامحدود_چارج_رداس_کا_میدان_صفر} میں حاصل کیا گیا تھا۔

مساوات \حوالہ{مساوات_کولومب_لامحدود_لکیر_رداسی_میدان} اور مساوات \حوالہ{مساوات_کولومب_لامحدود_لکیر_عمودی_میدان} سے مساوات \حوالہ{مساوات_کولومب_لامحدود_لکیر_چارج_کا_میدان}  کا حل یوں لکھا جائے گا
\begin{align}\label{مساوات_کولومب_لامحدود_لکیر_رداسی_میدان_پیدا_کرتا_ہے}
\kvec{E}=\Erho=\frac{\rho_L }{2 \pi \epsilon_0 \rho} \arho
\end{align}
جس کے مطابق لامحدود سیدھی لکیر پر یکساں چارج سے برقی میدان رداس \عددیء{\rho}  کے بالعکس متناسب ہے۔اس نتیجے  کا مساوات \حوالہ{مساوات_کولومب_نقطہ_چارج_کروی_رداس_پر_تبدیل_نہیں_ہوتا} کے ساتھ موازنہ کریں جو نقطہ چارج کی برقی میدان بیان کرتا ہے۔نقطہ چارج کا برقی میدان کروی رداس کے مربع کے بالعکس متناسب ہے۔یوں اگر لامحدود لکیر کے چارج سے فاصلہ دگنا کر دیا جائے تو برقی میدان آدھا ہو جائے گا جبکہ نقطہ چارج سے فاصلہ دگنا کرنے سے برقی میدان چار گنا کم ہوتا ہے۔

کسی بھی سمت میں لامحدود سیدھی لکیر پر چارج کا برقی میدان مساوات \حوالہ{مساوات_کولومب_لامحدود_لکیر_رداسی_میدان_پیدا_کرتا_ہے} میں بیان خوبیوں پر پورا اترے گا۔ایسی صورت میں کسی بھی نقطے پر \سمتیہ{E} حاصل کرنے کی خاطر اس نقطے سے چارج کے لکیر تک کم سے کم فاصلہ \عددیء{R} حاصل کریں۔یہ فاصلہ نقطے سے لکیر پر عمود کھینچنے سے حاصل ہو گا۔اس فاصلے کو \عددیء{\rho} تصور کریں۔لکیر سے عمودی سمت میں نقطے کی جانب اکائی سمتیہ \عددیء{\kvec{a}_R} کو \arho   \, تصور کریں۔ایسی صورت میں مساوات \حوالہ{مساوات_کولومب_لامحدود_لکیر_رداسی_میدان_پیدا_کرتا_ہے} کو
\begin{align}\label{مساوات_کولومب_کسی_بھِی_سمت_میں_لامحدود_لکیر_رداسی_میدان_پیدا_کرتا_ہے}
\kvec{E}=\frac{\rho_L }{2 \pi \epsilon_0 R} \kvec{a}_R
\end{align}
لکھ سکتے ہیں۔
%===================
\ابتدا{مثال}
\عددیء{y} محدد کے  متوازی اور \عددیء{(x,0,0)} سے گزرتی لامحدود لکیر پر \عددیء{\rho_L} کثافت کا چارج پایا جاتا ہے۔نقطہ \عددیء{(x',y',z')} پر \سمتیہ{E} حاصل کریں۔
\begin{figure}
\centering
\includegraphics[height=3.5cm]{figCoulombElectricFieldOfInfiniteLineChargeAlongYaxis}
\caption{کسی بھی سمت میں لامحدود لکیر پر چارج کی مثال}
\label{شکل_کولومب_کسی_بھی_سمت_لامحدود_لکیر_پر_چارج_کا_میدان}
\end{figure}

حل:شکل \حوالہ{شکل_کولومب_کسی_بھی_سمت_لامحدود_لکیر_پر_چارج_کا_میدان} میں صورت حال دکھایا گیا ہے۔\عددیء{(x',y',z')} سے چارج کے لکیر پر عمود \عددیء{(x,y',0)} پر ٹکراتا ہے۔ان دو نقطوں کا آپس میں فاصلہ \عددیء{\sqrt{(x'-x)^2+z^2}} ہے جبکہ 
\begin{align*}
\kvec{R}&=(x'-x)\ax+z\az\\
\kvec{a}_R&=\frac{(x'-x)\ax+z\az}{\sqrt{(x'-x)^2+z^2}}
\end{align*}
ہیں۔یوں
\begin{align*}
\kvec{E}=\frac{\rho_L}{2 \pi \epsilon_0 \sqrt{(x'-x)^2+z^2}} \kvec{a}_R
\end{align*}
ہو گا۔
\انتہا{مثال}
%================================
\ابتدا{مشق}
\عددیء{y} محدد پر \عددیء{-\infty} سے \عددیء{+\infty} تک \عددیء{\SI{10}{\nano \coulomb \per \meter}} چارج کی کثافت پائی جاتی ہے۔ نقطہ \عددیء{N_1(0,0,6)} اور نقطہ  \عددیء{N_2(0,8,6)}  پر \سمتیہ{E} حاصل کریں۔

جواب: دونوں نقطوں پر \عددی{\kvec{E}=30 \az} کے برابر ہے۔
\انتہا{مشق}
%====================

\ابتدا{مشق}
\عددیء{x} محدد پر \عددیء{-\infty} سے \عددیء{+\infty} تک \عددیء{\SI{5}{\nano \coulomb \per \meter}} چارج کی کثافت پائی جاتی ہے۔ نقطہ \عددیء{N_1(0,5,0)} اور نقطہ
  \عددیء{N_2(7,3,4)}  پر \سمتیہ{E} حاصل کریں۔

جوابات:\عددیء{\kvec{E}_1=18 \az \, \si{\volt \per \meter}} اور \عددیء{ \kvec{E}_2=18 \left(\frac{3\ay+4\az}{5}\right) \,\si{\volt \per \meter}}

\انتہا{مشق}
%=======================
\حصہ{یکساں چارج بردار ہموار لامحدود سطح}
شکل \حوالہ{شکل_کولومب_لامحدود_سطح_پر_چارج} میں  \عددیء{z=0} پر لامحدود \عددیء{x-y} سطح دکھائی گئی ہے۔تصور کریں کہ اس پوری  سطح پر انتہائی قریب قریب نقطہ چارج یوں رکھے گئے ہیں کہ سطح پر کہیں بھی چھوٹی رقبہ \عددیء{\Delta S} پر  یکساں قیمت کا چارج \عددیء{\Delta Q} پایا جاتا ہے۔اس طرح اکائی رقبے پر کل \عددیء{\tfrac{\Delta Q}{\Delta S}} چارج پایا جائے گا جسے \اصطلاح{سطحی چارج  کثافت}\فرہنگ{کثافت!سطحی  چارج}\حاشیہب{surface charge density}\فرہنگ{density!surface charge} \عددیء{\rho_S} کہتے ہیں۔سطحی چارج کثافت کی تعریف
\begin{align}\label{مساوات_کولومب_سطحی_کثافت_تعریف}
\rho_S=\lim_{\Delta S \to 0}\frac{\Delta Q}{\Delta S}
\end{align}
ہے۔چھوٹی سطح اتنی کم نہیں لی جاتی کہ اس پر چارج بردار الیکٹران علیحدہ علیحدہ بطور نقطہ چارج نظر آئیں بلکہ اسے اتنا رکھا جاتا ہے کہ علیحدہ علیحدہ الیکٹران کا اثر قابل نظر انداز ہو۔ سطح پر ہر جگہ چارج کا تقسیم یکساں نہ ہونے کی صورت میں \عددیء{\rho_S} کی قیمت متغیر ہو گی۔آئیں لامحدود سطح پر یکساں چارج کثافت سے خالی خلاء میں  پیدا \سمتیہ{E} حاصل کریں۔

پہلے غور کرتے ہیں کہ آیا مختلف مقامات سے  دیکھتے ہوئے کچھ اخذ کرنا ممکن ہے۔اگر اس چارج بردار سطح کے سامنے ہم کھڑے ہو جائیں تو ہمیں سامنے لامحدود چارج بردار سطح نظر آئے گی۔سطح سے برابر فاصلے پر ہم جہاں بھی جائیں ہمیں صورت حال میں کوئی تبدیلی نظر نہیں آئے گی۔اسی طرح اگر ہم سطح کی دوسری طرف اتنے ہی فاصلے پر چلے جائیں تو ہمیں صورت حال میں کسی قسم کی کوئی تبدیلی نظر نہیں آئے گی۔اس مشابہت سے ہم کہہ سکتے ہیں کہ ایسی سطح سے برابر فاصلے پر تمام نقطوں  پر یکساں برقی میدان پایا جائے گا۔اس کے برعکس اگر ہم اس سطح سے دور ہو جائیں تو ہمیں سطح قدر دور نظر آئے گی اور ہو سکتا ہے کہ اس تبدیلی سے \سمتیہ{E} پر اثر ہو۔آئیں اب مسئلے کو حساب و کتاب سے حل کرتے ہوئے \سمتیہ{E} حاصل کریں۔
\begin{figure}
\centering
\includegraphics[height=3.5cm]{figCoulombElectricFieldOfInfiniteSurfaceCharge}
\caption{یکساں چارج بردار ہموار لامحدود سطح}
\label{شکل_کولومب_لامحدود_سطح_پر_چارج}
\end{figure}

شکل \حوالہ{شکل_کولومب_لامحدود_سطح_پر_چارج}  میں چارج بردار سطح پر  \عددیء{z} محدد کے متوازی دو انتہائی قریب قریب لکیریں کھینچی گئی ہیں جن کے مابین فاصلہ \عددیء{\dif y} ہے۔اس گھیرے گئے رقبے کی چوڑائی \عددیء{\dif y} ہے۔یوں \عددیء{\Delta L} لمبائی اور \عددیء{\dif y} چوڑائی رقبے میں \عددیء{\rho_S \Delta L \dif y} چارج پایا جائے گا۔لکیروں سے گھیرے رقبے کو چارج کی سیدھی لکیر تصور کیا جا سکتا ہے جس پر اکائی لمبائی کے رقبے پر \عددیء{\tfrac{\rho_S \Delta L \dif y}{\Delta L}} چارج پایا جائے گا جسے \عددیء{\rho_L} تصور کیا جا سکتا ہے یعنی
\begin{align}
\rho_L=\rho_S \dif y
\end{align}

لامحدود لکیر پر یکساں چارج کی کثافت سے پیدا برقی میدان پر گزشتہ حصے میں غور کیا گیا۔نقطہ \عددیء{(x,0,0)}  پر \سمتیہ{E} حاصل کرتے ہیں۔شکل میں لامحدود چارج کی لکیر سے اس نقطے تک کا قریبی سمتی فاصلہ \سمتیہ{R} دکھایا گیا ہے جہاں
\begin{align}
\kvec{R}=x \ax-y\ay
\end{align}
کے برابر ہے جس سے
\begin{gather}
\begin{aligned}
R&=\abs{\kvec{R}}=\sqrt{x^2+y^2}\\
\kvec{a}_R&=\frac{x\ax-y\ay}{\sqrt{x^2+y^2}}
\end{aligned}
\end{gather}
حاصل ہوتے ہیں۔یوں چارج بردار لکیر سے \عددیء{(x,0,0)} پر پیدا برقی میدان کو مساوات \حوالہ{مساوات_کولومب_کسی_بھِی_سمت_میں_لامحدود_لکیر_رداسی_میدان_پیدا_کرتا_ہے}  کی مدد سے
\begin{gather}
\begin{aligned}
\dif \kvec{E} &= \frac{\rho_S \dif y}{2 \pi \epsilon_0 \sqrt{x^2+y^2}} \frac{x\ax-y\ay}{\sqrt{x^2+y^2}}\\
&=\frac{\rho_S \dif y \left(x\ax-y\ay \right)}{2 \pi \epsilon_0 \left(x^2+y^2\right)}
\end{aligned}
\end{gather}
لکھا جا سکتا ہے۔اس جواب کو \عددیء{\dif \kvec{E}=\dif \kvec{E}_x+\dif \kvec{E}_y} لکھا جا سکتا ہے جہاں
\begin{gather}\label{مساوات_کولومب_لامحدود_سطح_کی_میدان_کے_اجزاء}
\begin{aligned}
\dif \kvec{E}_x&=\frac{\rho_S x \dif y }{2 \pi \epsilon_0 \left(x^2+y^2\right)}\ax\\
\dif \kvec{E}_y&=-\frac{\rho_S y\dif y }{2 \pi \epsilon_0 \left(x^2+y^2\right)} \ay
\end{aligned}
\end{gather}
کے برابر ہیں۔\عددیء{x} محدد کے ایک جانب چارج بردار لکیر مندرجہ بالا برقی میدان پیدا کرتا ہے۔غور کرنے سے معلوم ہوتا ہے کہ \عددیء{x} محدد کے دوسری جانب اتنے ہی فاصلے پر چارج بردار لکیر سے پیدا برقی میدان مندرجہ بالا \عددیء{\dif \kvec{E}_y} کو ختم کرے گا۔یوں کسی بھی مثبت \عددیء{y} پر کھینچی لکیر کے \عددیء{\dif \kvec{E}_y}  کو منفی \عددیء{y} پر کھینچی لکیر کا \عددیء{\dif \kvec{E}_y} ختم کرے گا۔\عددیء{x} محدد کے دونوں جانب مسئلے کی مشابہت سے یوں ہم توقع کرتے ہیں کہ
\begin{align}\label{مساوات_کولومب_سطحی_صفر_شدت_حصہ}
\kvec{E}_y=0
\end{align} 
ہو گا۔

آئیں اب حساب و کتاب سے مساوات \حوالہ{مساوات_کولومب_لامحدود_سطح_کی_میدان_کے_اجزاء} کو حل کریں۔پہلے \عددیء{\kvec{E}_x} حاصل کرتے ہیں۔مساوات \حوالہ{مساوات_کولومب_لامحدود_سطح_کی_میدان_کے_اجزاء} میں دئے \عددیء{\dif \kvec{E}_x} کا تکمل لیتے ہیں۔ایسا کرنے کی خاطر
\begin{gather}
\begin{aligned}\label{مساوات_کولومب_ٹینجنٹ_کی_مدد_سے_تکملہ_کا_حصول}
y&=x \tan \alpha\\
\dif y &=x \sec^2 \alpha \dif \alpha
\end{aligned}
\end{gather}
کا استعمال کرتے ہیں۔شکل \حوالہ{شکل_کولومب_لامحدود_سطح_پر_چارج} میں \عددیء{\alpha} کی نشاندہی کی گئی ہے۔یوں
\begin{align*}
\kvec{E}_x=\int \dif \kvec{E}_x &=\frac{\rho_S x \ax}{2\pi \epsilon_0}\int_{y=-\infty}^{y=+\infty}\frac{\dif y }{ \left(x^2+y^2\right)}\\
&=\frac{\rho_S x \ax}{2\pi \epsilon_0}\int_{\alpha=-\frac{\pi}{2}}^{\alpha=+\frac{\pi}{2}} \frac{x \sec^2 \alpha \dif \alpha}{x^2\left(1+ \tan^2 \alpha\right)}
\end{align*}
میں \عددیء{\sec^2  \alpha=1+\tan^2 \alpha} کے استعمال سے
\begin{gather}
\begin{aligned}\label{مساوات_کولومب_لامحدود_سطح_عمودی_جزو}
\kvec{E}_x&=\frac{\rho_S\ax}{2\pi \epsilon_0}\int_{\alpha=-\frac{\pi}{2}}^{\alpha=+\frac{\pi}{2}}\dif \alpha\\
&=\frac{\rho_S\ax}{2\pi \epsilon_0} \eval{\alpha}_{-\frac{\pi}{2}}^{+\frac{\pi}{2}}\\
&=\frac{\rho_S}{2 \epsilon_0}\ax
\end{aligned}
\end{gather}
حاصل ہوتا ہے۔آئیں اب \عددیء{\kvec{E}_y} حاصل کریں۔

مساوات \حوالہ{مساوات_کولومب_لامحدود_سطح_کی_میدان_کے_اجزاء} میں دئے \عددیء{\dif \kvec{E}_y} کا تکمل لیتے ہیں۔
\begin{align*}
\kvec{E}_y=\int \dif \kvec{E}_y&=-\frac{\rho_S \ay}{2 \pi \epsilon_0} \int_{y=-\infty}^{y=+\infty}\frac{ y\dif y }{ \left(x^2+y^2\right)}
\end{align*}
تکمل کے نشان کے اندر \عددیء{f(y)=x^2+y^2} لیتے ہوئے  اسے \عددیء{\tfrac{\dif f(y)}{2 f(y)}} لکھا جا سکتا ہے جس کا تکمل \عددیء{\tfrac{\ln f(y)}{2}} ہے۔یوں
\begin{gather}
\begin{aligned}\label{مساوات_کولومب_لامحدود_سطح_افقی_جزو}
\kvec{E}_y&=-\frac{\rho_S \ay}{2 \pi \epsilon_0} \eval{\frac{\ln(x^2+y^2)}{2}}_{y=-\infty}^{y=+\infty}\\
&=0
\end{aligned}
\end{gather}
حاصل ہوتا ہے۔مساوات \حوالہ{مساوات_کولومب_سطحی_صفر_شدت_حصہ} میں یہی جواب حاصل کیا گیا تھا۔

مساوات \حوالہ{مساوات_کولومب_لامحدود_سطح_عمودی_جزو} اور مساوات \حوالہ{مساوات_کولومب_لامحدود_سطح_افقی_جزو} کی مدد سے یکساں چارج بردار لامحدود سطح کی برقی میدان
\begin{align}\label{مساوات_کولومب_لامحدود_سطح_کی_برقی-میدان}
\kvec{E}=\frac{\rho_S}{2 \epsilon_0}\kvec{a}_N
\end{align}
لکھی جا سکتی ہے جہاں \عددیء{\kvec{a}_N} اس سطح کا عمودی اکائی سمتیہ ہے۔آپ دیکھ سکتے ہیں کہ سطح سے فاصلہ کم یا زیادہ کرنے سے برقی میدان کی شدت پر کوئی اثر نہیں ہوتا۔سطح کے دونوں جانب برقی میدان اسی مساوات سے حاصل کی جائے گی۔ظاہر ہے کہ سطح کے دونوں جانب کے اکائی عمودی سمتیہ آپس میں الٹ ہیں۔

اب تصور کریں کہ اس سطح کے متوازی \عددیء{x=x_1} پر  ایک اور لامحدود سطح  رکھی جائے جس پر چارج کی یکساں کثافت \عددیء{-\rho_S} ہو۔ان دو متوازی سطحوں کو دو دھاتی چادروں سے بنایا گیا کپیسٹر\فرہنگ{کپیسٹر}\حاشیہب{capacitor}\فرہنگ{capacitor} سمجھا جا سکتا ہے۔کسی بھی نقطے پر کل \سمتیہ{E} دونوں سطحوں پر چارج سے پیدا برقی میدان کا مجموعہ ہو گا۔پہلے دونوں سطحوں کے دونوں جانب برقی میدان لکھتے ہیں۔ 
\begin{itemize}
\item
\عددیء{x=0} پر \عددیء{+\rho_S} کثافت کی سطح کا برقی میدان۔
\begin{align*}
\kvec{E}_{x>0}^{+}&=+\frac{\rho_S}{2\epsilon_0} \ax &x>0\\
\kvec{E}_{x<0}^{+}&=-\frac{\rho_S}{2\epsilon_0} \ax &x<0
\end{align*} 
\item
\عددیء{x=x_1} پر \عددیء{-\rho_S} کثافت کی سطح کا برقی میدان۔
\begin{align*}
\kvec{E}_{x>x_1}^{-}&=-\frac{\rho_S}{2\epsilon_0} \ax &x>x_1\\
\kvec{E}_{x<x_1}^{-}&=+\frac{\rho_S}{2\epsilon_0} \ax &x<x_1
\end{align*} 

\end{itemize}
ان نتائج کو استعمال کرتے ہوئے \عددیء{x<0}،\عددیء{x>x_1} اور \عددیء{0<x<x_1} خطوں میں برقی میدان حاصل کرتے ہیں۔
\begin{gather}
\begin{aligned}
\kvec{E}_{x<0}&=\kvec{E}_{x<0}^{+} +\kvec{E}_{x<x_1}^-=-\frac{\rho_S}{2\epsilon_0} \ax+\frac{\rho_S}{2\epsilon_0} \ax =0\\
\kvec{E}_{x>x_1}&=\kvec{E}_{x>0}^{+} +\kvec{E}_{x>x_1}^{-}=+\frac{\rho_S}{2\epsilon_0} \ax-\frac{\rho_S}{2\epsilon_0} \ax=0\\
\kvec{E}_{0<x<x_1}&=\kvec{E}_{x>0}^{+}+\kvec{E}_{x<x_1}^{-}=+\frac{\rho_S}{2\epsilon_0} \ax+\frac{\rho_S}{2\epsilon_0} \ax =\frac{\rho_S}{\epsilon_0} \ax
\end{aligned}
\end{gather}

اس نتیجے کے مطابق دو متوازی لامحدود سطحوں جن پر الٹ یکساں کثافت پائی جائے کے باہر کوئی برقی میدان نہیں پایا جاتا جبکہ سطحوں کے درمیانی خطے میں \عددیء{\tfrac{\rho_S}{\epsilon_0}} برقی میدان پایا جاتا ہے۔یہی مساوات ایک ایسے کپیسٹر کے برقی میدان کے لئے بھی استعمال کیا جا سکتا ہے  جس میں دھاتی چادروں  کی لمبائی اور چوڑائی دونوں چادروں کے درمیانی فاصلے سے کئی گنا زیادہ ہو اور چادروں کے درمیان خالی خلاء یا ہوا پائی جائے۔چادروں کے کناروں کے قریب کپیسٹر کے اندر اور باہر صورت حال قدر مختلف ہو گی۔ 
%=================================
\ابتدا{مثال}
خلاء میں تین متوازی لامحدود سطح پائے جاتے ہیں جن پر چارج کی یکساں کثافت پائی جاتی ہے۔پہلی سطح \عددیء{y=2} پر  \عددیء{\SI{2}{\nano \coulomb / \meter \squared}}، دوسری سطح  \عددیء{y=5} پر  \عددیء{\SI{4}{\nano \coulomb / \meter \squared}} اور تیسری سطح \عددیء{y=10} پر  \عددیء{\SI{-6}{\nano \coulomb / \meter \squared}} کثافت پائی جاتی ہے۔\عددیء{N_1(0,0,0)}، \عددیء{N_2(5,3,4)}،\عددیء{N_3(-2,7,11)} اور \عددیء{N_4(-7,30,22)} پر \سمتیہ{E} حاصل کریں۔

جوابات:\عددیء{0}، \عددیء{144 \pi \ay}، \عددیء{216 \pi \ay} اور  \عددیء{0}
\انتہا{مثال}
%=================================

\حصہ{چارج بردار حجم}
ہم نقطہ چارج، لامحدود لکیر پر چارج اور لامحدود سطح پر چارج دیکھ چکے ہیں۔اگلا فطری قدم  چارج بردار حجم بنتا ہے لہٰذا اسی پر غور کرتے ہیں۔لکیر اور سطح کے چارج  پر غور کرتے ہوئے ہر جگہ یکساں کثافت کی بات کی گئی۔حجم میں چارج کی بات کرتے ہوئے اس شرط کو دور کرتے ہوئے کثافت کو متغیرہ تصور کرتے ہیں۔یوں مختلف مقامات پر کثافت کی قیمت مختلف ہو سکتی ہے۔

تصور کریں کہ حجم میں انتہائی قریب قریب نقطہ چارج پائے جاتے ہیں۔یوں اگر کسی نقطے پر \عددیء{\Delta h} حجم میں \عددیء{\Delta Q} چارج پایا جائے تب اس نقطے پر اوسط حجمی چارج کثافت \عددیء{\tfrac{\Delta Q}{\Delta h}} ہو گی۔کسی بھی نقطے پر چارج کی حجمی  کثافت  یوں بیان کی جاتی ہے۔
\begin{align}
\rho_h=\lim_{\Delta h \to 0} \frac{\Delta Q}{\Delta h}
\end{align} 

کسی بھی حجم میں کل چارج   تین درجی تکمل سے حاصل کیا جائے گا۔کارتیسی محدد میں ایسا تکمل یوں لکھا جائے گا۔
\begin{align}
Q=\iiint\limits_{\textrm{h}} \rho_h \dif x \dif y \dif z
\end{align}
جہاں تکمل کے نشان کے نیچے \تحریر{h} حجم کو ظاہر کرتا ہے۔اس طرز کے تکمل کو عموماً ایک درجی تکمل سے ہی ظاہر کیا جاتا ہے یعنی
\begin{align}
Q=\int\limits_{\textrm{h}} \rho_h \dif h
\end{align}


حجم میں \سمتیہ{r'}  نقطے پر چھوٹی سی حجم \عددیء{\Delta h'} میں \عددی{\Delta Q = \rho_h' \Delta h'} چارج پایا جائے گا جسے نقطہ چارج تصور کیا جا سکتا ہے۔نقطہ \سمتیہ{r} پر اس نقطہ چارج کا برقی میدان \عددیء{\dif \kvec{E}} مساوات \حوالہ{مساوات_کولومب_نقطہ_چارج_سے_میدان_کا_حصول} سے یوں حاصل ہوتا ہے۔
\begin{align*}
\dif \kvec{E}=\frac{1}{4 \pi \epsilon_0 }\frac{\rho_h' \Delta h'}{\abs{\kvec{r}-\kvec{r'}}^2} \frac{\kvec{r}-\kvec{r'}}{\abs{\kvec{r}-\kvec{r'}}}
\end{align*}
اس مساوات میں  نقطہ \عددیء{r'} پر چارج کی کثافت \عددیء{\rho_h'} لکھی گئی ہے۔تمام حجم میں پائے جانے والے چارج کا نقطہ \سمتیہ{r} پر میدان مندرجہ بالا مساوات کے تکمل سے یوں حاصل کیا جائے گا۔
\begin{align}\label{مساوات_کولومب_حجم_چارج_کا_میدان}
\kvec{E}(\kvec{r})=\frac{1}{4 \pi \epsilon_0 }\int\limits_{\textrm{h}}\frac{\rho_h'  \dif h'}{\abs{\kvec{r}-\kvec{r'}}^2} \frac{\kvec{r}-\kvec{r'}}{\abs{\kvec{r}-\kvec{r'}}}
\end{align}

اس مساوات کی  شکل قدر خوف ناک  ہے البتہ حقیقت میں ایسا ہرگز نہیں۔سمتیہ \سمتیہ{r} اس نقطے کی نشاندہی کرتا ہے جہاں برقی میدان حاصل کرنا درکار ہو۔اس نقطے پر برقی میدان
 کو  \عددیء{\kvec{E}(\kvec{r})} لکھ کر اس حقیقت کی وضاحت کی گئی ہے کہ نقطے کی تبدیلی سے برقی میدان تبدیل ہو سکتا ہے۔کثافت از خود  متغیرہ ہے جس کی قیمت \سمتیہ{r'} پر منحصر ہے۔ \سمتیہ{r'} پر چھوٹی حجم \عددیء{\dif h'} اور چارج کی کثافت \عددیء{\rho_h'} لکھے گئے ہیں جہاں \عددیء{'} اس بات کی یاد دہانی کراتا ہے کہ یہ متغیرات نقطہ \عددیء{r'} پر پائے جاتے ہیں۔آخر میں یاد رہے کہ کسی بھی نقطے پر \سمتیہ{E} حاصل کرتے وقت اسی نقطے پر موجود چارج کو نظرانداز کیا جاتا ہے۔
%==================================================
\حصہ{مزید مثال}

\ابتدا{مثال}
شکل \حوالہ{شکل_کولومب_محدود_لکیر_پر_چارج} میں \عددیء{z=z_1} سے \عددیء{z=z_2} تک کی سیدھی لکیر پر  یکساں \عددیء{\rho_L} پایا جاتا ہے۔نقطہ دار گول دائرے پر \سمتیہ{E} حاصل کریں۔
\begin{figure}
\begin{subfigure}{0.5\textwidth}
\centering
\includegraphics[height=3.5cm]{figCoulombElectricFieldOfFiniteLineCharge}
\caption{محدود لکیر پر چارج کی یکساں کثافت}
\end{subfigure}
%
\begin{subfigure}{0.5\textwidth}
\centering
\includegraphics[height=3.5cm]{figCoulombElectricFieldOfFiniteLineChargeLimits}
\caption{\عددی{z} اور \عددیء{\alpha} کا تعلق}
\end{subfigure}
\caption{محدود لکیر پر چارج}
\label{شکل_کولومب_محدود_لکیر_پر_چارج}
\end{figure}

حل:شکل \حوالہ{شکل_کولومب_محدود_لکیر_پر_چارج} سے واضح ہے کہ نکتہ دار گول دائرے پر \سمتیہ{E} کی حتمی قیمت \عددیء{\abs{\kvec{E}}} یکساں ہو گی۔یوں ہم لکھ سکتے ہیں
\begin{align*}
\kvec{E}&=\frac{\rho_L}{4\pi\epsilon_0} \int_{z_1}^{z_2} \frac{\dif z}{\abs{\rho^2+z^2}} \frac{\rho\arho-z\az}{\sqrt{\rho^2+z^2}}\\
&=\frac{\rho_L \rho\arho}{4\pi\epsilon_0} \int_{z_1}^{z_2} \frac{\dif z}{\abs{\rho^2+z^2}^{\frac{3}{2}}}-\frac{\rho_L \az}{4\pi\epsilon_0} \int_{z_1}^{z_2} \frac{z\dif z}{\abs{\rho^2+z^2}^{\frac{3}{2}}}\\
&=\kvec{E}_\rho +\kvec{E}_z
\end{align*}
دائیں جانب باری باری تکملہ حل کرتے ہیں۔تکملہ حل کرنے کی خاطر \عددیء{z=\rho \tan \alpha} کا تعلق استعمال کرتے ہیں۔\عددیء{\alpha} کا \عددیء{z} کا تعلق شکل \حوالہ{شکل_کولومب_محدود_لکیر_پر_چارج}-ب میں دکھایا گیا ہے۔
\begin{align*}
\kvec{E}_\rho&=\frac{\rho_L \rho\arho}{4\pi\epsilon_0} \int_{\alpha_1}^{\alpha_2} \frac{\rho \sec^2 \alpha\dif \alpha}{\abs{\rho^2+\rho^2 \tan^2 \alpha}^{\frac{3}{2}}}\\
&=\frac{\rho_L \arho}{4\pi\epsilon_0 \rho} \eval{\sin \alpha}_{\alpha_1}^{\alpha_2}\\
&=\frac{\rho_L \arho}{4\pi\epsilon_0 \rho} \left(\sin \alpha_2-\sin \alpha_1 \right)
\end{align*}
حاصل ہوتا ہے جہاں
\begin{align*}
\alpha_2&=\arctan \frac{z_2}{\rho}\\
\alpha_1&=\arctan \frac{z_1}{\rho}
\end{align*}
کے برابر ہے۔شکل  \حوالہ{شکل_کولومب_محدود_لکیر_پر_چارج}-ب سے \عددیء{\sin \alpha=\tfrac{z}{\sqrt{\rho^2+z^2}}} لکھا جا سکتا ہے۔یوں
\begin{align*}
\kvec{E}_{\rho}=\frac{\rho_L \arho}{4\pi\epsilon_0 \rho} \left(\frac{z_2}{\sqrt{\rho^2+z_2^2}}-\frac{z_1}{\sqrt{\rho^2+z_1^2}}\right)
\end{align*}
حاصل ہوتا ہے۔اسی طرح
\begin{align*}
\kvec{E}_z&=-\frac{\rho_L \az}{4\pi\epsilon_0} \int_{z_1}^{z_2} \frac{z\dif z}{\abs{\rho^2+z^2}^{\frac{3}{2}}}\\
&=-\frac{\rho_L \az}{4\pi\epsilon_0} \int_{\alpha_1}^{\alpha_2} \frac{\rho^2 \tan \alpha \sec^2 \alpha \dif \alpha}{\abs{\rho^2+\rho^2 \tan^2 \alpha}^{\frac{3}{2}}}
\end{align*}
سے
\begin{align*}
\kvec{E}_z&=\frac{\rho_L \az}{4\pi\epsilon_0 \rho} \left(\cos \alpha_2 -\cos \alpha_1 \right)\\
&=\frac{\rho_L \az}{4\pi\epsilon_0 } \left(\frac{1}{\sqrt{\rho^2+z_2^2}}-\frac{1}{\sqrt{\rho^2+z_1^2}} \right)
\end{align*}
حاصل ہوتا ہے۔\عددیء{\kvec{E}_{\rho}} اور \عددیء{\kvec{E}_z} کا مجموعہ لیتے ہوئے کل برقی میدان یوں حاصل ہوتا ہے۔
\begin{align}
\kvec{E}=\frac{\rho_L \arho}{4\pi\epsilon_0 \rho} \left(\frac{z_2}{\sqrt{\rho^2+z_2^2}}-\frac{z_1}{\sqrt{\rho^2+z_1^2}}\right)+\frac{\rho_L \az}{4\pi\epsilon_0 } \left(\frac{1}{\sqrt{\rho^2+z_2^2}}-\frac{1}{\sqrt{\rho^2+z_1^2}} \right)
\end{align}

\انتہا{مثال}
%=============================

\ابتدا{مثال}
شکل \حوالہ{شکل_کولومب_گول_دائرے_پر_چارج} میں \عددیء{z=0} پر گول دائرہ دکھایا گیا ہے جس پر چارج کی یکساں کثافت پائی جاتی ہے۔نقطہ \عددیء{(0,0,z)} پر \سمتیہ{E} حاصل کریں۔

\begin{figure}
\centering
\includegraphics[height=3.5cm]{figCoulombElectricFieldOfCircularLineCharge}
\caption{چارج بردار گول دائرہ}
\label{شکل_کولومب_گول_دائرے_پر_چارج}
\end{figure}

حل:نلکی محدد استعمال کرتے ہوئے اسے حل کرتے ہیں۔کسی بھی زاویہ پر رداس کھینچتے ہوئے دائرے پر کوئی نقطہ حاصل کیا جا سکتا ہے۔زاویہ میں باریک تبدیلی \عددیء{\Delta \phi} سے لمبائی \عددیء{\rho \Delta \phi} حاصل ہوتی ہے جس پر کل چارج \عددیء{\Delta Q=\rho_L \rho \Delta \phi} پایا جائے گا۔یوں چارج \عددیء{\Delta Q} مقام \عددیء{\rho \arho} پر پایا جاتا ہے جبکہ \سمتیہ{E} مقام \عددیء{z \az} پر درکار ہے۔آپ دیکھ سکتے ہیں کہ \سمتیہ{E} رداس کی سمت میں ممکن نہیں۔\عددیء{\Delta Q} سے
\begin{align*}
\Delta \kvec{E}=\frac{\rho_L \rho \Delta \phi}{4\pi\epsilon_0 \left(\rho^2+z^2 \right)} \frac{z\az-\rho \arho}{\sqrt{\rho^2+z^2}}
\end{align*}
پیدا ہو گا۔دائرے پر تمام چارج کے اثر کے لئے تکملہ لینا ہو گا۔
\begin{align*}
\kvec{E}&=\frac{\rho_L \rho}{4\pi\epsilon_0  \left(\rho^2+z^2 \right)^{\frac{3}{2}}}\int_{0}^{2\pi}(z\az-\rho \arho) \dif \phi
\end{align*}
تکملہ کا متغیرہ \عددیء{\phi} ہے جسے تبدیل کرنے سے \عددیء{\rho} اور \عددیء{z} میں کوئی تبدیلی رونما نہیں ہوتی۔اسی لئے انہیں تکملہ کی نشان سے باہر لے جایا گیا ہے۔حاصل تکملہ کو دو حصوں میں لکھا جا سکتا ہے البتہ معاملہ اتنا سیدھا نہیں جتنا معلوم ہوتا ہے۔\عددیء{\kvec{E}_z} لکھتے ہوئے کارتیسی محدد کی اکائی سمتیہ \سمتیہ{\az} کو تکملہ کے باہر لے جایا جا سکتا ہے چونکہ \عددیء{\phi} کی تبدیلی سے \سمتیہ{\az} تبدیل نہیں ہوتا البتہ  \عددیء{\kvec{E}_{\rho}} لکھتے ہوئے نلکی محدد کی اکائی سمتیہ \عددیء{\arho} کو تکملہ کے باہر نہیں لے جایا جا سکتا چونکہ \عددیء{\phi} کی تبدیلی سے  \عددیء{\arho} کی سمت تبدیل ہوتی ہے۔چونکہ \عددیء{\arho} کی سمت تبدیل ہوتی ہے لہٰذا اس کو مستقل تصور کرنا غلط ہے اور یوں یہ تکملہ کے اندر ہی رہے گا۔ 
\begin{gather}
\begin{aligned}
\kvec{E}_z&=\frac{\rho_L \rho z \az}{4\pi\epsilon_0  \left(\rho^2+z^2 \right)^{\frac{3}{2}}}\int_{0}^{2\pi} \dif \phi\\
\kvec{E}_{\rho}&=-\frac{\rho_L \rho^2}{4\pi\epsilon_0  \left(\rho^2+z^2 \right)^{\frac{3}{2}}}\int_{0}^{2\pi} \arho \dif \phi
\end{aligned}
\end{gather}
پہلے تکملہ کا جواب اب دیکھ کر ہی
\begin{align}
\kvec{E}_z&=\frac{2\pi \rho_L \rho z \az}{4\pi\epsilon_0  \left(\rho^2+z^2 \right)^{\frac{3}{2}}}
\end{align}
لکھا جا سکتا ہے جبکہ دوسرے تکملہ میں \عددیء{\arho=\cos \phi \ax+\sin \phi \ay} لکھتے ہوئے حل کرتے ہیں۔
\begin{align*}
\kvec{E}_{\rho}&=-\frac{\rho_L \rho^2}{4\pi\epsilon_0  \left(\rho^2+z^2 \right)^{\frac{3}{2}}}\int_{0}^{2\pi} (\cos \phi \ax+\sin \phi \ay) \dif \phi\\
&=-\frac{\rho_L \rho^2}{4\pi\epsilon_0  \left(\rho^2+z^2 \right)^{\frac{3}{2}}} \eval{ (\sin \phi \ax-\cos \phi \ay) }_{0}^{2\pi}\\
&=0
\end{align*}

یہی جواب اس طرح بھی حاصل کیا جا سکتا ہے کہ گول دائرے پر تمام چارج کو \عددیء{Q=2\pi\rho\rho_L} لکھیں۔یہ چارج نقطہ \عددیء{(0,0,z)} سے \عددیء{\sqrt{\rho^2+z^2}} فاصلے پر ہے۔اگر اس تمام چارج کو ایک ہی نقطے  \عددیء{(\rho,0,0)} پر موجود تصور کیا جائے تو یہ
\begin{align*}
\kvec{E}_R=\frac{2\pi\rho\rho_L}{4\pi\epsilon_0 \left(\rho^2+z^2 \right)} \kvec{a}_R
\end{align*}
برقی میدان پیدا کرے گا۔چارج گول دائرے پر پھیلا ہوا ہے لہٰذا حقیقت میں صرف \عددیء{\az} جانب ہی \kvec{E} پیدا ہوتا ہے۔شکل میں اس تکون کو دیکھتے ہوئے جس کا \عددیء{\kvec{R}} حصہ ہے  آپ دیکھ سکتے ہیں کہ \عددیء{\kvec{E}_R} کا \عددیء{\tfrac{z}{\sqrt{\rho^2+z^2}}} حصہ  حقیقت میں پایا جائے گا۔یوں 
\begin{align*}
\kvec{E}_z=\frac{2\pi\rho\rho_L}{4\pi\epsilon_0 \left(\rho^2+z^2 \right)} \frac{z}{\sqrt{\rho^2+z^2}} \az
\end{align*}
ہی حاصل ہوتا ہے۔

\انتہا{مثال}

