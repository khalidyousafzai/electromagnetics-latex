\باب{موصل، ذو برق اور  کپیسٹر}
اس باب میں ہم برقی رو اور کثافت برقی رو سے شروع ہو کر بنیادی \اصطلاح{استمراری مساوات}\حاشیہب{continuity equation}  حاصل کریں گے۔اس کے بعد  اوہم کے قانون کی نقطہ شکل  اور اس کی بڑی شکل حاصل کریں گے۔دو اجسام کے جوڑ پر \اصطلاح{سرحدی شرائط}\حاشیہب{boundary conditions} حاصل کرتے ہوئے \اصطلاح{عکس}\حاشیہب{images} کے طریقے کا استعمال دیکھیں گے۔

\اصطلاح{ذوبرق}\حاشیہب{dielectric}  کی \اصطلاح{تقطیب}\حاشیہب{polarization} پر غور کرتے ہوئے  جزو برقی مستقل حاصل کریں گے۔اس کے بعد کپیسٹر پر غور کیا جائے گا۔سادہ شکل و صورت رکھنے والے  کپیسٹر کی قیمتیں حاصل کی جائیں گیں۔ایسا گزشتہ بابوں کے نتائج استعمال کرتے ہوئے کیا جائے گا۔ 

\حصہ{برقی رو اور کثافت برقی رو}
جیسے پانی کے حرکت کو پانی کا بہاو کہتے ہیں،  اسی طرح برقی چارج کے حرکت کو برقی رو کہتے ہیں۔برقی رو کو \عددیء{i} اور \عددیء{I} سے ظاہر کیا جاتا ہے۔برقی رو کی اکائی ایمپیئر \عددیء{(A)} ہے۔کسی نقطے یا سطح سے ایک کولمب چارج فی سیکنڈ کے گزر کو ایک ایمپیئر کہتے ہیں۔یوں
\begin{align}
I=\frac{\dif Q}{\dif t}
\end{align}
لکھا جائے گا۔

ایسی موصل تار جس کی ایک سرے سے دوسری سرے تک موٹائی مسلسل کم ہوتی ہو کے بالکل محور پر برقی چارج محوری سمت میں حرکت کرے گا جبکہ محور سے دور چارج کی حرکت تار کی موٹائی کم یا زیادہ ہونے کی وجہ سے قدرِ ترچھی ہو گی۔یوں اگرچہ تار میں ہر مقام پر برقی رو کی مقدار برابر ہے لیکن برقی رو کی سمتیں مختلف ہو سکتی ہیں۔اسی بنا پر ہم برقی رو کو مقداری تصور کریں گے۔اگر تار کی موٹائی انتہائی کم ہو تب برقی رو سمتیہ مانند ہو گا لیکن ایسی صورت میں بھی ہم اسے مقداری ہی تصور کرتے ہوئے تار کی لمبائی کو سمتیہ لیں گے۔

\اصطلاح{کثافت برقی رو}\فرہنگ{برقی رو!کثافت}\حاشیہب{current density}\فرہنگ{density!current} سے مراد برقی رو فی اکائی مربع سطح \عددیء{(\si{\ampere \per \meter \squared})} ہے اور اسے \عددیء{J} سے ظاہر کیا جاتا ہے۔اگر چھوٹی سطح \عددیء{\Delta S} سے عمودی سمت میں \عددیء{\Delta I} برقی رو گزرے تب
\begin{align}
\Delta I=J_n \Delta S
\end{align}
کے برابر ہو گا۔اگر کثافت برقی رو اور سمتی رقبہ کی سمتیں مختلف ہوں تب
\begin{align}
\Delta I = \kvec{J} \cdot \Delta S
\end{align}
لکھا جائے گا اور پوری سطح سے کُل گزرتی برقی رو تکمل کے ذریعہ حاصل کی جائے گی۔
\begin{align} \label{مساوات_کپیسٹر_برقی_رو_کثافت_کا_سطحی_تکمل_ہے}
I=\int_S \kvec{J} \cdot \dif \kvec{S}
\end{align}
%
\ابتدا{مثال}
شکل \حوالہ{شکل_کپیسٹر_سطح_سے_گزرتی_برقی_رو} میں سیدھی سطح \عددیء{\kvec{S}=2\ax} دکھائی گئی ہے جہاں کثافت برقی رو \عددیء{\kvec{J}=1\ax+1\ay} پائی جاتی ہے۔سطح سے گزرتی برقی رو اور اس کی سمت دریافت کریں۔اگر سطح کی دوسری سمت کو سطح کی سمت لی جائے تب برقی رو کی مقدار اور اس کی سمت کیا ہوں گے۔
\begin{figure}
\centering
\includegraphics{figCapacitorCurrentFromCurrentDensityDotArea}
\caption{سطح سے گزرتی برقی رو۔}
\label{شکل_کپیسٹر_سطح_سے_گزرتی_برقی_رو}
\end{figure}

حل:چونکہ  یہاں \عددیء{\kvec{J}} مستقل مقدار ہے لہٰذا اسے مساوات \حوالہ{مساوات_کپیسٹر_برقی_رو_کثافت_کا_سطحی_تکمل_ہے} میں تکمل کے باہر لایا جا سکتا ہے اور یوں اس تکمل سے
\begin{align*}
I=\kvec{J} \cdot \kvec{S}= \SI{2}{\ampere}
\end{align*} 
حاصل ہوتا ہے۔برقی رو چونکہ مثبت ہے لہٰذا یہ سطح کی سمت میں ہی سطح سے گزر رہی ہے۔

اگر سطح کی دوسری طرف کو سطح کی سمت لی جائے تب \عددیء{\kvec{S}=-2\ax} لکھا جائے گا اور یوں
\begin{align*}
I=\kvec{J} \cdot \kvec{S}= \SI{-2}{\ampere}
\end{align*} 
حاصل ہو گا۔برقی رو کی مقدار اب بھی دو ایمپیئر ہی ہے البتہ اس کی علامت منفی ہے جس کا مطلب یہ ہے کہ برقی رو سطح کے سمت کی الٹی سمت میں ہے۔یوں اب بھی برقی رو بائیں سے دائیں ہی  گزر رہی ہے۔

\انتہا{مثال}
%

اس مثال سے آپ دیکھ سکتے ہیں کہ \عددیء{\kvec{S}} کی سمت میں برقی رو کو مثبت برقی رو کہا جاتا ہے۔
\begin{figure}
\centering
\includegraphics{figCapacitorPointFormOfOhmLaw}
\caption{حرکت کرتے چارج کی رفتار اور کثافت برقی رو۔}
\label{شکل_کپیسٹر_حرکت_کرتا_چارج_اور_کثافت_برقی_رو}
\end{figure}

شکل \حوالہ{شکل_کپیسٹر_حرکت_کرتا_چارج_اور_کثافت_برقی_رو} میں \عددیء{a} اور \عددیء{b} اطراف کی تار میں لمبائی کی سمت میں \عددیء{v} رفتار سے چارج حرکت کر رہا ہے۔شکل میں اس تار کا کچھ حصہ دکھایا گیا ہے۔یوں \عددیء{\dif t} دورانیہ میں چارج \عددیء{v \dif t} فاصلہ طے کرے گا۔اس طرح اس دورانیہ میں \عددیء{m} پر لگائی گئی نقطہ دار لکیر \عددیء{n} پہنچ جائے گی۔آپ دیکھ سکتے ہیں کہ اس دورانیہ میں \عددیء{m} اور \عددیء{n} کے درمیان موجود چارج سطح \عددیء{\Delta S} سے گزر جائے گا۔\عددیء{m} سے \عددیء{n} تک حجم \عددیء{a b v \dif t} کے برابر ہے۔اگر تار میں چارج کی حجمی کثافت \عددیء{\rho_h} ہو تب اس حجم میں  کُل چارج \عددیء{\rho_h a b v \dif t } ہو گا۔یوں برقی رو
\begin{align*}
I=\frac{\Delta Q}{\Delta t}=\frac{\rho_h a b v \dif t}{\dif t}=\rho_h \Delta S v
\end{align*}
لکھتے ہوئے  کثافت برقی رو
\begin{align*}
J=\frac{I}{\Delta S}=\rho_h v
\end{align*}
حاصل ہوتی ہے جس کی سمتی شکل
\begin{align}\label{مساوات_کپیسٹر_کثافت_رو_مساوی_کثافت_چارج_ضرب_رفتار}
\kvec{J}=\rho_h \kvec{v}
\end{align}
ہے۔

یہ مساوات کہتا ہے کہ حجمی چارج کثافت بڑھانے سے کثافت برقی رو اسی نسبت سے بڑھتی ہے۔اسی طرح چارج کی رفتار بڑھانے سے کثافت برقی رو اسی نسبت سے بڑھتی ہے۔یہ ایک عمومی نتیجہ ہے۔یوں سڑک پر زیادہ لوگ گزارنے کا ایک طریقہ انہیں تیز چلنے پر مجبور کرنے سے حاصل کیا جا سکتا ہے۔دوسرا طریقہ یہ ہے کہ انہیں قریب قریب کر دیا جائے۔ 


\حصہ{استمراری مساوات}
قانون بقائے چارج کہتا ہے کہ چارج کو نہ تو پیدا  اور نا ہی اسے ختم کیا جا سکتا ہے، اگرچہ برابر مقدار میں مثبت اور منفی چارج کو ملا کی انہیں ختم کیا جا سکتا ہے اور اسی طرح برابر مقدار میں انہیں پیدا بھی کیا جا سکتا ہے۔

یوں اگر ڈبے میں ایک جانب \عددیء{\SI{+5}{\coulomb}} اور دوسری جانب \عددیء{\SI{-3}{\coulomb}} چارج موجود ہو تو اس ڈبے میں کُل \عددیء{\SI{2}{\coulomb}} چارج ہے۔اگر ہم \عددیء{\SI{+3}{\coulomb}}  کو \عددیء{\SI{-3}{\coulomb}}  کے ساتھ ملا کر ختم کر دیں تب  بھی ڈبے میں کُل \عددیء{\SI{2}{\coulomb}} ہی چارج رہے گا۔

%=================
\ابتدا{مثال}
ایک ڈبہ جس کا حجم \عددیء{\SI{5}{\meter^3}} ہے میں حجمی کثافت چارج \عددیء{\SI{3}{\coulomb \per \meter^3}} ہے۔اس ڈبے سے چارج کی نکاسی ہو رہی ہے۔دو سیکنڈ میں حجمی کثافت چارج \عددیء{\SI{1}{\coulomb \per \meter^3}} رہ جاتی ہے۔ان دو سکینڈوں میں ڈبے سے خارج برقی رو کا تخمینہ لگائیں۔

حل:شروع میں ڈبے میں \عددیء{Q_1=3 \times 5=\SI{15}{\coulomb}} چارج ہے جبکہ دو سیکنڈ بعد اس میں \عددیء{Q_1=1 \times 5=\SI{5}{\coulomb}} رہ جاتا ہے۔یوں دو سیکنڈ میں ڈبے سے \عددیء{\SI{10}{\coulomb}} چارج خارج ہوتا ہے۔اس طرح  ڈبے سے خارج برقی رو \عددیء{\tfrac{10}{2}=\SI{5}{\ampere}} ہے۔اسی کو یوں لکھا جا سکتا ہے۔
\begin{align*}
I=-\frac{\Delta Q}{\Delta t}=-\frac{(5-15)}{2}=\SI{5}{\ampere}
\end{align*}
\انتہا{مثال}
%=======================

اس مثال میں آپ نے دیکھا کہ ڈبے میں \عددیء{\Delta Q} منفی ہونے کی صورت میں خارجی برقی رو کی قیمت مثبت ہوتی ہے۔آئیں اس حقیقت کو بہتر شکل دیں۔

حجم کو مکمل طور پر گھیرتی سطح کو بند سطح کہتے ہیں۔کسی بھی مقام پر ایسی سطح کی سمت سطح کے عمودی باہر کو ہوتی ہے۔مساوات \حوالہ{مساوات_کپیسٹر_برقی_رو_کثافت_کا_سطحی_تکمل_ہے} کے تحت برقی رو کو کثافت برقی رو کے سطحی تکمل سے بھی حاصل کیا جا سکتا ہے۔یوں
\begin{align}\label{مساوات_کپیسٹر_استمراری_مساوات_تکمل_شکل}
I=\oint_S \kvec{J} \cdot \dif \kvec{S}=-\frac{\dif Q}{\dif t}
\end{align}
لکھا جا سکتا ہے جہاں حجم کی سطح بند سطح ہونے کی بنا پر بند تکمل کی علامت استعمال کی گئی ہے اور \عددیء{Q} حجم میں کل چارج ہے۔

مساوات \حوالہ{مساوات_کپیسٹر_استمراری_مساوات_تکمل_شکل} \اصطلاح{استمراری مساوات}\فرہنگ{استمراری مساوات}\حاشیہب{continuity equation}\فرہنگ{continuity equation} کی تکمل شکل ہے۔آئیں اب اس کی نقطہ شکل حاصل کریں۔

مسئلہ پھیلاو کو صفحہ \حوالہصفحہ{مساوات_گاوس_مسئلہ_پھیلاو_تکمل_شکل} پر مساوات \حوالہ{مساوات_گاوس_مسئلہ_پھیلاو_تکمل_شکل} میں بیان کیا گیا ہے۔مسئلہ پھیلاو کسی بھی سمتی تفاعل کے لئے درست ہے لہٰذا اسے استعمال کرتے ہوئے مساوات \حوالہ{مساوات_کپیسٹر_استمراری_مساوات_تکمل_شکل} میں بند سطحی تکمل کو حجمی تکمل میں تبدیل کرتے ہیں۔
\begin{align*}
\oint_S \kvec{J} \cdot \dif \kvec{S}=\int_h (\nabla \cdot \kvec{J}) \dif h
\end{align*}
اگر حجم میں حجمی کثافت چارج \عددیء{\rho_h} ہو تب اس میں کل چارج
\begin{align*}
Q=\int_h \rho_h \dif h
\end{align*}
ہو گا۔ان دو نتائج کو استعمال کرتے ہوئے
\begin{align*}
\int_h (\nabla \cdot \kvec{J}) \dif h=-\frac{\dif}{\dif t} \int_h \rho_h \dif h
\end{align*}
لکھا جا سکتا ہے۔اس مساوات میں \عددیء{\tfrac{\dif}{\dif t}} دو متغیرات پر لاگو ہو گا۔یہ متغیرات تکمل کے اندر حجمی چارج کثافت \عددیء{\rho_h} اور حجم \عددیء{h} ہے۔

آپ جانتے ہیں کہ دو متغیرات کے تفرق کو جزوی تفرق کی شکل میں
\begin{align*}
\frac{\dif (uv)}{\dif t}=\frac{\partial u}{\partial t} v+ u  \frac{\partial v}{\partial t}
\end{align*}
لکھا جا سکتا ہے جہاں \عددیء{v} کو مستقل رکھتے ہوئے \عددیء{\tfrac{\partial u}{\partial t}} اور \عددیء{u} کو مستقل رکھتے ہوئے \عددیء{\tfrac{\partial v}{\partial t}} حاصل کیا جاتا ہے۔

اگر ہم یہ شرط لاگو کریں کہ حجم کی سطح تبدیل نہیں ہو گی تب حجم بھی تبدیل نہیں ہو گا اور یوں \عددیء{\tfrac{\dif}{\dif t}} کو جزوی تفرق میں تبدیل کرتے ہوئے تکمل کے اندر لکھتے ہوئے
\begin{align*}
\int_h (\nabla \cdot \kvec{J}) \dif h=\int_h -\frac{\partial \rho_h}{\partial t} \dif h
\end{align*}
حاصل ہوتا ہے۔یہ مساوات ہر ممکنہ حجم کے لئے درست ہے لہٰذا یہ نہایت چھوٹی حجم کے لئے بھی درست ہے۔نہایت چھوٹی حجم \عددیء{\dif h} کے لئے تکمل
\begin{align*}
 (\nabla \cdot \kvec{J}) \dif h= -\frac{\partial \rho_h}{\partial t} \dif h
\end{align*}
ہی ہے جس سے
\begin{align}\label{مساوات_کپیسٹر_استمراری_مساوات_نقطہ_شکل}
\nabla \cdot \kvec{J}=-\frac{\partial \rho_h}{\partial t}
\end{align}
حاصل ہوتا ہے۔مساوات \حوالہ{مساوات_کپیسٹر_استمراری_مساوات_نقطہ_شکل} استمراری مساوات کی نقطہ شکل ہے۔

پھیلاو کی تعریف کو ذہن میں رکھتے ہوئے آپ دیکھ سکتے ہیں کہ مساوات \حوالہ{مساوات_کپیسٹر_استمراری_مساوات_نقطہ_شکل} کہتا ہے کہ ہر نقطے پر چھوٹی سی حجم سے فی سیکنڈ چارج کا اخراج، یعنی برقی رو، فی اکائی حجم مساوی ہے چارج کے گھٹاو فی سیکنڈ فی اکائی حجم۔ 


\حصہ{موصل}
غیر چارج شدہ موصل میں منفی الیکٹران اور مثبت ساکن ایٹموں کی تعداد برابر ہوتی ہے البتہ اس میں برقی رو آزاد الیکٹران کے حرکت سے پیدا ہوتا ہے۔موصل میں الیکٹران آزادی سے بے ترتیب حرکت کرتا رہتا ہے۔یہ حرکت کرتا ہوا لمحہ بہ لمحہ ساکن ایٹم سے ٹکراتا ہے اور ہر ٹکر سے اس کے حرکت کی سمت تبدیل ہو جاتی ہے۔یوں ایسے الیکٹران کی اوسط رفتار صفر کے برابر ہوتی ہے۔آئیں دیکھیں کہ برقی میدان کے موجودگی میں کیا ہوتا ہے۔

برقی میدان \عددیء{\kvec{E}} میں الیکٹران پر قوت
\begin{align}
\kvec{F}=-e\kvec{E}
\end{align}
عمل کرے گی جہاں الیکٹران کا چارج \عددیء{-e} ہے۔ الیکٹران کی رفتار اس قوت کی وجہ سے  اسراع کے ساتھ قوت کی سمت میں بڑھنے شروع ہو جائے گی۔یوں بلا ترتیب رفتار  کے ساتھ ساتھ قوت کے سمت میں الیکٹران رفتار پکڑے گا۔موصل میں پائے جانے والا الیکٹران جلد کسی ایٹم سے ٹکرا جاتا ہے اور یوں اس کی سمت تبدیل ہو جاتی ہے۔جس لمحہ  الیکٹران کسی ایٹم سے ٹکراتا ہے اگر لاگو میدان کو صفر کر دیا جائے تو الیکٹران دوبارہ بلا ترتیب حرکت کرتا رہے گا اور اس کی اوسط رفتار دوبارہ صفر ہی ہو گی،  البتہ اس کی رفتار اب پہلے سے زیادہ ہو گی۔اگر الیکٹران ایٹم سے نہ ٹکراتا تب برقی میدان صفر کرنے کے بعد یہ برقرار قوت کی سمت میں حاصل کردہ رفتار سے حرکت کرتا رہتا۔یوں آپ دیکھ سکتے ہیں کہ ہر ٹکر سے الیکٹران کی اوسط رفتار صفر ہو جاتی ہے۔
اس طرح ہم دیکھتے ہیں کہ  \عددیء{\kvec{E}} کے موجودگی میں موصل میں الیکٹران کی رفتار مسلسل نہیں بڑھتی بلکہ یہ قوت کی سمت میں اوسط رفتار \عددیء{\kvec{v}_d} حاصل کرتا ہے اور جیسے ہی میدان صفر کر دیا جائے الیکٹران کی اوسط رفتار بھی صفر ہو جاتی ہے۔\عددیء{\kvec{v}_d} کو \اصطلاح{رفتار بہاو}\فرہنگ{رفتار بہاو}\حاشیہب{drift velocity}\فرہنگ{drift velocity} کہتے ہیں۔رفتار بہاو کا دارومدار \عددیء{\kvec{E}} کی قیمت پر ہے لہٰذا ہم
\begin{align}\label{مساوات_کپیسٹر_رفتار_بہاو}
\kvec{v}_d=-\mu_e \kvec{E}
\end{align}
لکھ سکتے ہیں جہاں مساوات کے مستقل \عددیء{\mu_e} کو الیکٹران کی \اصطلاح{حرکت پذیری}\فرہنگ{حرکت پذیری!الیکٹران}\حاشیہب{mobility}\فرہنگ{mobility!electron} کہتے ہیں۔حرکت پذیری کی مقدار مثبت ہے ۔چونکہ \عددیء{\kvec{v}_d} کو میٹر فی سیکنڈ اور \عددیء{\kvec{E}} کو وولٹ فی میٹر میں ناپا جاتا ہے لہٰذا حرکت پذیری کو \عددیء{\si{\meter \squared \per \volt \per \second}} میں ناپا جائے گا۔

مساوات \حوالہ{مساوات_کپیسٹر_رفتار_بہاو} کو صفحہ \حوالہصفحہ{مساوات_کپیسٹر_کثافت_رو_مساوی_کثافت_چارج_ضرب_رفتار} پر دئے مساوات \حوالہ{مساوات_کپیسٹر_کثافت_رو_مساوی_کثافت_چارج_ضرب_رفتار} میں پر کرتے ہوئے
\begin{align}
\kvec{J}=-\rho_e \mu_e \kvec{E}
\end{align}
حاصل ہوتا ہے جہاں موصل میں آزاد الیکٹران کی حجمی چارج کثافت کو \عددیء{\rho_e} لکھا گیا ہے۔\عددیء{\rho_e} منفی مقدار ہے۔یاد رہے کہ غیر چارج شدہ موصل میں حجمی کثافت چارج صفر کے برابر ہے چونکہ اس میں منفی الیکٹران  اور مثبت ایٹم کے چارج برابر ہوتے ہیں۔اس مساوات کو عموماً
\begin{align}\label{مساوات_کپیسٹر-اوہم_قانون_نقطہ_شکل}
\kvec{J}=\sigma \kvec{E}
\end{align}
لکھا جاتا ہے جو اوہم کے قانون\فرہنگ{قانون!اوہم}\فرہنگ{Ohm's law} کی نقطہ شکل ہے اور جہاں
\begin{align}\label{مساوات_کپیسٹر_موصلیت_تعریف}
\sigma=-\rho_e \mu_e
\end{align}
لکھا گیا ہے۔\عددیء{\sigma} کو \اصطلاح{موصلیت کا مستقل}\فرہنگ{موصلیت!مستقل}\حاشیہب{conductivity}\فرہنگ{conductivity} کہتے ہیں اور اس کی اکائی\حاشیہد{یہ اکائی جرمنی کے جناب ارنسٹ ورنر وان سیمنز (1816-1892) کے نام ہے جنہوں نے موجودہ سیمنز کمپنی کا بنیاد رکھا۔} سیمنز فی میٹر \عددیء{\si{\siemens \per \meter}} ہے۔سیمنز کو بڑے \عددیء{\si{\siemens}} سے  جبکہ سیکنڈ کو چھوٹے \عددیء{\second} سے ظاہر کیا جاتا ہے۔امید کی جاتی ہے کہ آپ ان میں غلطی نہیں کریں گے۔ اس کتاب کے آخر میں صفحہ  \حوالہصفحہ{جدول_جدول_موصلیت_کے_مستقل} پر جدول \حوالہ{جدول_جدول_موصلیت_کے_مستقل} میں کئی موصل اور غیر موصل اشیاء کی موصلیت پیش کی گئی ہیں۔
%================
\ابتدا{مثال}
تانبے\فرہنگ{تانبا}\حاشیہب{copper}\فرہنگ{copper} کی موصلیت کے مستقل  کی قیمت \عددیء{\SI{5.8e7}{\siemens \per \meter}} ہے جبکہ اس کی کمیتی کثافت \عددیء{\SI{8940}{\kilogram \per \meter^3}} اور ایٹمی کمیت \عددیء{\SI{63.5}{\gram}} ہیں۔اگر ہر ایٹم ایک عدد الیکٹران آزاد کرتا ہو تب تانبے میں الیکٹران کی حرکت پذیری حاصل کریں۔برقی میدان \عددیء{E=\SI{0.1}{\volt \per \meter}} کی صورت میں الیکٹران کا رفتار بہاو حاصل کریں۔

حل:ایٹمی کمیت \عددیء{\num{6.023e23}} یعنی ایک مول\حاشیہب{mole} ایٹم کی کمیت کو کہتے ہیں۔چونکہ ایک مربع میٹر میں \عددیء{\SI{8940}{\kilogram}} ہیں لہٰذا ایک مربع میٹر میں
\begin{align*}
\frac{8940 \times 6.023 \times 10^{23}}{0.0635}=8.48 \times 10^{28}
\end{align*} 
ایٹم پائیں جائیں گے۔ہر ایٹم ایک الیکٹران آزاد کرتا ہے لہٰذا \عددیء{\SI{0.1}{\nano \meter}} اطراف کے مربع میں اوسطاً \عددیء{0.848} یعنی تقریباً ایک عدد آزاد الیکٹران پایا جائے گا۔ اس طرح  ایک مربع میٹر میں کل آزاد الیکٹران چارج یعنی حجمی آزاد چارج کثافت
\begin{align}\label{مساوات_کپیسٹر_موصل_آزاد_چارج_کثافت}
\rho_e=-1.6 \times 10^{-19} \times 8.48 \times 10^{28}=\SI{-1.36e10}{\coulomb \per \meter^3}
\end{align}
ہو گی۔ایک مربع میٹر میں یوں انتہائی زیادہ آزاد چارج پایا جاتا ہے۔ اس طرح مساوات \حوالہ{مساوات_کپیسٹر_موصلیت_تعریف} کی مدد سے
\begin{align*}
\mu_e=-\frac{\sigma}{\rho_e}=\frac{5.8 \times 10^7}{-1.36 \times 10^{10}}=\SI{0.00427}{\meter \squared \per \volt \per \second}
\end{align*}
حاصل ہوتا ہے جہاں \عددیء{\SI{0.00427}{\meter \squared \siemens \per \coulomb}} کو \عددیء{\SI{0.00427}{\meter \squared \per \volt \per \second}} لکھا گیا ہے۔آپ تسلی کر سکتے ہیں کہ یہ برابر مقدار ہیں۔اب مساوات  \حوالہ{مساوات_کپیسٹر_رفتار_بہاو} استعمال کرتے ہوئے الیکٹران کی رفتار بہاو 
\begin{align*}
v_d = -0.00427 \times 0.1=\SI{-0.000427}{\meter \per \second}
\end{align*}
حاصل ہوتی ہے۔منفی رفتار کا مطلب ہے کہ الیکٹران \عددیء{\kvec{E}} کے الٹ سمت حرکت کر رہا ہے۔اس رفتار\حاشیہد{کھودا پہاڑ، نکلا چوہا۔آزاد الیکٹران تو کچھوے سے بھی آہستہ چلتا ہے۔} سے الیکٹران ایک کلو میٹر کا فاصلہ ستائیس دن و رات چل کر طے کرے گا۔یہاں یہ بتلاتا چلوں کہ عام درجہ حرارت مثلاً \عددیء{\SI{300}{\kelvin}} پر تانبے میں حرارتی توانائی سے حرکت کرتے الیکٹران کی رفتار تقریباً \عددیء{\SI{1000}{\kilo \meter \per \second}} ہوتی ہے۔

یوں موصل میں آزاد الیکٹرانوں کو نئی جگہ منتقل ہوتے  شہد کے مکھیوں کا جھنڈ  سمجھا جا سکتا ہے۔ایسے جھنڈ میں کوئی ایک مکھی نہایت تیز رفتار سے آگے پیچھے اڑتی ہے جبکہ پورا جھنڈ  نسبتاً آہستہ رفتار سے ایک سمت میں حرکت کرتا ہے۔موصل میں بھی کوئی ایک الیکٹران نہایت تیز رفتار سے ایٹموں سے ٹکراتا ہوا حرارتی توانائی کی وجہ سے  نہایت تیزی سے  اِدھر اُدھر حرکت کرتا ہے جبکہ بیرونی لاگو میدان کی وجہ سے ایسے تمام الیکٹران نہایت آہستہ رفتار سے میدان کی سمت میں حرکت کرتے ہیں۔

اگر موصل میں آزاد الیکٹران اتنے کم رفتار سے بیرونی لاگو میدان کی سمت میں صفر کرتے ہیں تب بجلی چالو کرتے ہی بلب  کس طرح روشن ہوتا ہے۔اس کو سمجھنے کی خاطر برقی تار کو پانی بھرے ایک لمبے  پائپ مانند سمجھیں۔ایسے پائپ میں جیسے ہی ایک جانب سے مزید پانی داخل کیا جائے، اسی وقت پائپ کے دوسرے سرے سے برابر پانی خارج ہو گا۔امید ہی سمجھ آ گئی ہو گی۔  
\انتہا{مثال}
%=============

مندرجہ بالا مثال میں بتلایا گیا کہ تانبے کا ہر ایٹم ایک عدد الیکٹران آزاد کرتا ہے۔اس حقیقت کو یوں سمجھا جا سکتا ہے کہ تانبے کا ایٹمی عدد \عددیء{29}ہے۔ایٹم کے کسی بھی مدار میں  \عددیء{2 n^2}  الیکٹران ہو سکتے ہیں جہاں پہلے مدار کے لئے \عددیء{n=1}، دوسرے مدار کے لئے \عددیء{n=2} وغیرہ لیا جاتا ہے۔یوں اس کے پہلے مدار میں \عددیء{2}، دوسرے مدار میں \عددیء{8}، تیسرے مدار میں \عددیء{18} اور آخری مدار\حاشیہد{چوتھے مدار میں \عددیء{32} الیکٹران ممکن ہیں لیکن تانبے کے ایٹم میں اس مدار کے لئے صرف ایک عدد الیکٹران بچتا ہے۔} میں \عددیء{1} الیکٹران ہو گا۔ایٹم آخری مدار میں واحد الیکٹران کو آزاد کرتا ہے۔آئیں اب بڑی شکل میں اوہم کا قانون حاصل کریں۔ 
\begin{figure}
\centering
\includegraphics{figCapacitorOhmLawFromPointForm}
\caption{اوہم کے قانون کی بڑی شکل}
\label{شکل_کپیسٹر_اوہم_قانون_بڑی_شکل}
\end{figure}

شکل \حوالہ{شکل_کپیسٹر_اوہم_قانون_بڑی_شکل} میں  موصل سلاخ دکھایا گیا ہے جس کی لمبائی \عددیء{L} اور  رقبہ عمودی تراش \عددیء{S} ہیں۔سلاخ کو \عددیء{\ay} سمت میں لیٹا تصور کریں۔سلاخ میں لمبائی کی سمت میں مستقل اور یکساں برقی میدان \عددیء{\kvec{E}=-E\ay}  اور کثافت برقی رو \عددیء{\kvec{J}=-J\ay} پائے جاتے ہیں۔یوں اگر سلاخ کا بایاں سرا برقی زمین تصور کیا جائے تب اس کے دائیں سرے پر برقی دباو کو صفحہ \حوالہصفحہ{مساوات_توانائی_برقی_دباو_تعریف} پر دئے مساوات \حوالہ{مساوات_توانائی_برقی_دباو_تعریف} سے یوں
\begin{align*}
V= -\int_0^L \kvec{E} \cdot \dif \kvec{L}=\int_0^L E \ay \cdot \dif  y \ay=\int_0^L E \dif y=E\int_0^L \dif y=EL
\end{align*}
حاصل کرتے ہیں۔رقبہ عمودی تراش کو شکل میں گہرے رنگ سے اجاگر کیا گیا ہے۔سمتی رقبہ عمودی تراش بند سطح نہیں ہے لہٰذا اس کے دو ممکنہ رخ ہیں۔سلاخ کے دائیں سرے سے داخل برقی رو حاصل کرنے کی غرض سے رقبہ عمودی تراش کو \عددیء{\kvec{S}=-S\ay} لکھتے ہیں۔یوں دائیں سرے سے داخل برقی رو کی مقدار مثبت ہو گی۔برقی رو
\begin{align*}
I=\int_S \kvec{J} \cdot \dif \kvec{S}=JS
\end{align*}
حاصل ہوتی ہے۔ان معلومات کو شکل \حوالہ{مساوات_کپیسٹر-اوہم_قانون_نقطہ_شکل} میں پُر کرتے ہوئے
\begin{align*}
\frac{I}{S}=\sigma \frac{V}{L}
\end{align*}
  یا
\begin{align*}
V=I \frac{L}{\sigma S}
\end{align*}
حاصل ہوتا ہے جہاں
\begin{align}\label{مساوات_کپیسٹر_سلاخ_کی_مزاحمت}
R= \frac{L}{\sigma S}
\end{align}
کو مزاحمت لکھتے ہوئے
\begin{align}\label{مساوات_کپیسٹر_اوہم_قانون_بڑی_شکل}
V=IR
\end{align}
حاصل ہوتا ہے جو اوہم کے قانون\فرہنگ{اوہم!قانون}\فرہنگ{ohm's law} کی جانی پہچانی شکل ہے۔

مساوات \حوالہ{مساوات_کپیسٹر_سلاخ_کی_مزاحمت} یکساں رقبہ عمودی تراش رکھنے والے موصل سلاخ کی \اصطلاح{مزاحمت}\فرہنگ{مزاحمت}\حاشیہب{resistance}\فرہنگ{resistance} دیتا ہے جہاں مزاحمت کی اکائی \اصطلاح{اوہم}\فرہنگ{اوہم}\حاشیہب{ohm}\فرہنگ{ohm} ہے جسے \عددیء{\si{\ohm}} سے ظاہر کیا جاتا ہے۔یکساں رقبہ عمودی تراش کے سلاخ میں برقی میدان یکساں ہوتا ہے۔اگر سلاخ کا رقبہ عمودی تراش یکساں نہ ہو تب اس میں برقی میدان بھی یکساں نہ ہو گا اور ایسی صورت میں مساوات \حوالہ{مساوات_کپیسٹر_سلاخ_کی_مزاحمت} استعمال نہیں کیا جا سکتا البتہ ایسی صورت میں بھی مزاحمت کو مساوات \حوالہ{مساوات_کپیسٹر_اوہم_قانون_بڑی_شکل} کی مدد سے برقی دباو فی اکائی برقی رو سے بیان کیا جاتا ہے۔یوں مساوات \حوالہ{مساوات_توانائی_برقی_دباو_تعریف} اور مساوات \حوالہ{مساوات_کپیسٹر_برقی_رو_کثافت_کا_سطحی_تکمل_ہے} استعمال کرتے ہوئے سلاخ کے  \عددیء{b} سے \عددیء{a} سرے تک مزاحمت
\begin{align}\label{مساوات_کپیسٹر_مزاحمت_کی_عمومی_مساوات}
R=\frac{V}{I}=\frac{-\int \limits_b^a \kvec{E} \cdot \dif \kvec{L}}{\int \limits_S \kvec{J} \cdot \dif \kvec{S}}=\frac{-\int \limits_b^a \kvec{E} \cdot \dif \kvec{L}}{\int \limits_S \sigma \kvec{E} \cdot \dif \kvec{S}}
\end{align} 
سے حاصل ہو گی جہاں برقی رو سلاخ کے مثبت برقی دباو والے سرے سے سلاخ میں داخل ہوتے برقی رو کو کہتے ہیں۔یوں مندرجہ بالا مساوات میں سطحی تکمل سلاخ کے مثبت سرے پر حاصل کیا جائے گا جہاں سطح عمودی تراش کی سمت سلاخ کی جانب لی جائے گی۔
%=============
\ابتدا{مثال}
تانبے کی ایک کلو میٹر لمبی اور تین ملی میٹر رداس کے تار کی مزاحمت حاصل کریں۔

حل:یہاں \عددیء{L=\SI{1000}{\meter}} جبکہ \عددیء{S=\pi r^2=\SI{2.83e-7}{\meter \squared}} اور \عددیء{\sigma=5.8\times 10^7} ہے لہٰذا
\begin{align*}
R=\frac{1000}{5.8\times 10^7 \times 2.83 \times 10^{-7}}=\SI{0.61}{\ohm}
\end{align*}
حاصل ہوتا ہے۔
\انتہا{مثال}

%=============
\ابتدا{مشق}
المونیم میں کثافت برقی رو مندرجہ ذیل صورتوں میں حاصل کریں۔( الف) برقی میدان کی شدت \عددیء{\SI{50}{\milli \volt \per \meter}} ہے۔ (ب) آزاد الیکٹران کی رفتار بہاو \عددیء{\SI{0.12}{\milli \meter \per \second}} ہے۔ (پ) ایک ملی میٹر موٹی تار جس میں \عددیء{\SI{2}{\ampere}} برقی رو گزر رہی ہے۔

جوابات:\عددیء{\SI{1.91}{\mega \ampere \per \meter \squared}}،  \عددیء{\SI{3.82}{\mega \ampere \per \meter \squared}} اور \عددیء{\SI{2.55}{\mega \ampere \per \meter \squared}}
\انتہا{مشق}
%===============

\حصہ{موصل کے خصوصیات اور سرحدی شرائط}
غیر چارج شدہ موصل میں کُل آزاد الیکٹران اور مثبت ایٹم برابر تعداد میں پائے جاتے ہیں۔یوں اس میں برقی میدان صفر کے برابر ہوتا ہے۔فرض کریں کہ غیر چارج شدہ موصل کے اندر کسی طرح چند الیکٹران نمودار ہو جاتے ہیں۔یہ الیکٹران برقی میدان \عددیء{\kvec{E}} پیدا کریں گے جس کی وجہ سے  موصل میں آزاد الیکٹران  موصل کے سطح کی جانب چل پڑیں گے۔سطح کے باہر غیر موصل خلاء پائی جاتی ہے جس میں الیکٹران حرکت نہیں کر سکتے لہٰذا الیکٹران موصل کے سطح پر پہنچ کر رک جائیں گے۔موصل میں نمودار ہونے والے الیکٹران کے برابر تعداد میں الیکٹران موصل کے سطح پر منتقل ہوں گے جس کے بعد موصل میں دوبارہ منفی الیکٹران اور مثبت ایٹموں کی تعداد برابر ہو جائے گی اور یہ غیر چارج شدہ صورت اختیار کر لے گا۔

آپ نے دیکھا کہ اضافی  چارج موصل میں زیادہ دیر نہیں رہ سکتا اور یہ جلد  سطح پر منتقل ہو جاتا ہے۔یوں اضافی چارج  موصل  کے سطح پر بیرونی جانب چمٹا رہتا ہے۔یہ موصل کی پہلی اہم خاصیت ہے۔

موصل کی دوسری خاصیت \اصطلاح{برقی سکون}\فرہنگ{برقی سکون}\حاشیہب{electrostatic}\فرہنگ{electrostatic} کی حالت کے لئے بیان کرتے ہیں۔برقی سکون سے مراد ایسی صورت ہے جب چارج حرکت نہ کر رہا ہو یعنی جب برقی رو صفر کے برابر ہو۔برقی سکون کی حالت میں موصل کے اندر ساکن برقی میدان صفر رہتا ہے۔اگر ایسا نہ ہوتا تو میدان کی وجہ سے اس میں آزاد الیکٹران حرکت کر کے برقی رو کو جنم دیتے جو غیر ساکن حالت ہے۔

یوں برقی سکون کی حالت میں موصل کے اندر اضافی چارج اور برقی میدان دونوں صفر کے برابر ہوتے ہیں البتہ اس کے سطح پر بیرونی جانب چارج پایا جا سکتا ہے۔آئیں دیکھیں کہ سطح پر پائے جانے والا چارج موصل کے باہر کس قسم کا برقی میدان پیدا کرتا ہے۔

موصل کے سطح پر چارج، موصل کے باہر برقی میدان پیدا کرتا ہے۔سطح پر کسی بھی نقطے پر ایسے میدان کو دو اجزاء کے مجموعے کی شکل میں لکھا جا سکتا ہے۔پہلا جزو سطح کے متوازی اور دوسرا جزو سطح کے عمودی رکھتے ہوئے ہم دیکھتے ہیں کہ سطح کے متوازی جزو صفر ہو گا۔اگر ایسا نہ ہو تو اس میدان کی وجہ سے سطح پر پائے جانے والے آزاد الیکٹران حرکت میں آئیں گے جو غیر ساکن حالت ہو گی۔یوں ہم
\begin{align}
E_{\textup{متوازی}}=0
\end{align}
لکھ سکتے ہیں۔سطح پر عمودی برقی میدان گاوس کے قانون کی مدد سے حاصل کیا جا سکتا ہے جو کہتا ہے کہ کسی بھی بند سطح سے کُل برقی بہاو کا اخراج، سطح میں گھیرے چارج کے برابر ہوتا ہے۔چونکہ سطح پر متوازی برقی میدان صفر ہے اور موصل کے اندر بھی برقی میدان صفر ہے لہٰذا سطح پر چارج سے برقی بہاو کا اخراج صرف عمودی سمت میں ہو سکتا ہے۔یوں \عددیء{\Delta S} سطح سے عمودی اخراج \عددیء{D \Delta S} اسی سطح پر چار \عددیء{\rho_S \Delta S} کے برابر ہو گا جس سے
\begin{align}
D_{\textup{عمودی}}=\rho_S
\end{align} 
حاصل ہوتا ہے۔آئیں اسی بحث کو بہتر جامہ پہنائیں۔ایسا کرتے ہوئے ہم ایک عمومی ترکیب سیکھ لیں گے جو مختلف اقسام کے اشیاء کے سرحد پر میدان کے حصول کے لئے استعمال کیا جاتا ہے۔  

\begin{figure}
\centering
\includegraphics{figCapacitorElectricFieldConductorAirBoundaryCondition}
\caption{موصل اور خلاء کے سرحد پر برقی شرائط۔}
\label{شکل_کپیسٹر_موصل_خلاء_برقی_شرائط}
\end{figure}

شکل \حوالہ{شکل_کپیسٹر_موصل_خلاء_برقی_شرائط} میں موصل اور خالی خلاء کے درمیان سرحد موٹی لکیر سے دکھایا گیا ہے۔اس سرحد پر خلاء کی جانب \عددیء{\kvec{E}_{\textup{خ}}} اور \عددیء{\kvec{D}_\textup{خ}} دکھائے گئے ہیں۔خلاء میں \عددیء{\kvec{E}_{\textup{خ}}} کو \عددیء{\kvec{E}_\textup{م،خ}} اور \عددیء{\kvec{E}_\textup{ع،خ}} کے مجموعے کے طور پر بھی دکھایا گیا ہے جو بالترتیب سرحد کے متوازی اور عمودی \عددیء{\kvec{E}_{\textup{خ}}} کے  اجزاء ہیں۔اسی طرح \عددیء{\kvec{D}_\textup{خ}} کو بھی متوازی اور عمودی اجزاء کے مجموعہ کے طور پر دکھایا گیا ہے۔ہم صرف اس حقیقت کو لے کر آگے بڑھتے ہیں کہ موصل کے اندر \عددیء{\kvec{E}} اور \عددیء{\kvec{D}} دونوں صفر کے برابر ہیں۔آئیں اس حقیقت کی بنا پر خلاء میں   \عددیء{\kvec{E}_\textup{خ}} کی قیمت حاصل کریں۔ہم \عددیء{\kvec{E}_\textup{خ}} کے مجموعے  \عددیء{\kvec{E}_\textup{م،خ}} اور \عددیء{\kvec{E}_\textup{ع،خ}} حاصل کریں گے۔پہلے \عددیء{\kvec{E}_\textup{م،خ}} حاصل کرتے ہیں۔ 

 سرحد پر \عددیء{abcd}  مستطیل بنایا گیا ہے جہاں \عددیء{ab} اور \عددیء{cd} سرحد کے متوازی جبکہ \عددیء{bc} اور \عددیء{da} سرحد کے عمودی ہیں۔\عددیء{ab} خالی خلاء میں سرحد سے \عددیء{\Delta h /2} فاصلے پر  ہے جبکہ \عددیء{cd} موصل میں سرحد سے  \عددیء{\Delta h /2} فاصلے پر  ہے۔\عددیء{ab} اور \عددیء{cd} کی لمبائیاں \عددیء{\Delta w} ہیں جبکہ \عددیء{bc} اور \عددیء{da} کی لمبائیاں \عددیء{\Delta h} ہے۔صفحہ \حوالہصفحہ{مساوات_توانائی_بند_راستا} پر دئے مساوات \حوالہ{مساوات_توانائی_بند_راستا} 
\begin{align*}
\oint \kvec{E} \cdot \dif \kvec{L}=0
\end{align*}
کو \عددیء{abcd} پر  استعمال کرتے ہیں۔اس تکمل کو چار ٹکڑوں کا مجموعہ لکھا جا سکتا ہے۔
   \begin{align*}
\oint \kvec{E} \cdot \dif \kvec{L}=\int_a^b \kvec{E} \cdot \dif \kvec{L}+\int_b^c \kvec{E} \cdot \dif \kvec{L}+\int_c^d \kvec{E} \cdot \dif \kvec{L}+\int_d^a \kvec{E} \cdot \dif \kvec{L}=0
\end{align*}
اب \عددیء{a} سے \عددیء{b} تک
\begin{align*}
\int_a^b\kvec{E} \cdot \dif \kvec{L}=E_{\textup{م،خ}} \Delta w
\end{align*}
حاصل ہوتا ہے۔خلاء میں نقطہ \عددیء{b}  پر عمودی میدان کو \عددیء{E_{b \textup{ع،خ}}} لکھتے ہوئے \عددیء{b} سے \عددیء{c} تک
\begin{align*}
\int_b^c\kvec{E} \cdot \dif \kvec{L}=-E_{b \textup{ع،خ}} \frac{\Delta h}{2}
\end{align*}
حاصل ہوتا ہے۔\عددیء{c} سے \عددیء{d} تک تکمل صفر کے برابر ہے چونکہ یہ راستہ موصل کے اندر ہے جہاں \عددیء{\kvec{E}=0} ہے۔
\begin{align*}
\int_c^d\kvec{E} \cdot \dif \kvec{L}=0
\end{align*}
خلاء میں نقطہ \عددیء{a}  پر عمودی میدان کو \عددیء{E_{a \textup{ع،خ}}} لکھتے ہوئے \عددیء{d} سے \عددیء{a} تک
\begin{align*}
\int_d^a\kvec{E} \cdot \dif \kvec{L}=E_{a \textup{ع،خ}} \frac{\Delta h}{2}
\end{align*}
ان چار نتائج سے
\begin{align*}
\oint \kvec{E} \cdot \dif \kvec{L}=E_{\textup{م،خ}} \Delta w+\left(E_{a\textup{ع،خ}}-E_{b\textup{ع،خ}} \right) \frac{\Delta h}{2}=0
\end{align*}
لکھا جا سکتا ہے۔سرحد کے قریب میدان حاصل کرنے کی خاطر ہمیں سرحد کے قریب تر ہونا ہو گا یعنی \عددیء{\Delta h} کو تقریباً صفر کے برابر کرنا ہو گا۔ہم \عددیء{\Delta w} کو اتنا چھوٹا لیتے ہیں کہ اس کی پوری لمبائی پر میدان کو یکساں تصور کرنا ممکن ہو۔ایسا کرتے ہوئے اس مساوات سے
\begin{align*}
\oint \kvec{E} \cdot \dif \kvec{L}=E_{\textup{م،خ}} \Delta w=0
\end{align*}
یعنی
\begin{align}\label{مساوات_کپیسٹر_سرحد_موصل_خلاء_متوازی_میدان_صفر}
E_{\textup{م،خ}} = 0
\end{align}
حاصل ہوتا ہے۔آئیں اب \عددیء{\kvec{E}_\textup{ع،خ}} حاصل کریں۔\عددیء{\kvec{E}_\textup{ع،خ}} کی بجائے گاوس کے قانون
\begin{align*}
\oint_S \kvec{D} \cdot \dif \kvec{S}=Q
\end{align*}
 کی مدد سے \عددیء{\kvec{D}_\textup{ع،خ}} کا حصول زیادہ آسان ثابت ہوتا ہے لہٰذا ہم اسی کو حاصل کرتے ہیں۔

شکل \حوالہ{شکل_کپیسٹر_موصل_خلاء_برقی_شرائط} میں موصل اور خالی خلاء کے سرحد پر بیلن دکھایا گیا ہے جس کی لمبائی \عددیء{\Delta h} اور سیدھی سطحوں کا رقبہ \عددیء{\Delta S} ہے۔اگر سرحد پر \عددیء{\rho_S} پایا جائے تب بیلن \عددیء{\rho_S \Delta S} چارج کو گھیرے گا۔گاوس کے قانون کے تحت بیلن سے اسی مقدار کے برابر برقی بہاو کا اخراج ہو گا۔برقی بہاو کا اخراج بیلن کے دونوں سروں اور اس کے نلکی نما سطح سے ممکن ہے۔یوں
\begin{align*}
\oint_S \kvec{D} \cdot \dif \kvec{S}=\int_{\textup{نچلا سرا}} \kvec{D} \cdot \dif \kvec{S}+\int_{\textup{اوپر سرا}} \kvec{D} \cdot \dif \kvec{S}+\int_{\textup{نلکی سطح}} \kvec{D} \cdot \dif \kvec{S}=\rho_S \Delta S
\end{align*}
لکھا جا سکتا ہے۔اب بیلن کی نچلی سطح موصل کے اندر ہے جہاں میدان صفر کے برابر ہے لہٰذا
\begin{align*}
\int_{\textup{نچلا سرا}} \kvec{D} \cdot \dif \kvec{S}=0
\end{align*}
ہو گا۔مساوات \حوالہ{مساوات_کپیسٹر_سرحد_موصل_خلاء_متوازی_میدان_صفر} کے تحت سرحد پر خلاء میں متوازی میدان صفر ہوتا ہے۔موصل میں بھی میدان صفر ہوتا ہے لہٰذا  
\begin{align*}
\int_{\textup{نلکی سطح}} \kvec{D} \cdot \dif \kvec{S}=0
\end{align*}
ہو گا۔بیلن کے اوپر والے سرے پر
\begin{align*}
\int_{\textup{اوپر سرا}} \kvec{D} \cdot \dif \kvec{S}=D_{\textup{ع،خ}} \Delta S
\end{align*}
ہو گا۔ان تین نتائج کو استعمال کرتے ہوئے
\begin{align*}
\oint_S \kvec{D} \cdot \dif \kvec{S}=D_{\textup{ع،خ}} \Delta S=\rho_S \Delta S
\end{align*}
یعنی
\begin{align*}
D_{\textup{ع،خ}}=\rho_S
\end{align*}
حاصل ہوتا ہے۔چونکہ \عددیء{D=\epsilon_0 E} ہوتا ہے لہٰذا یوں 
\begin{align}\label{مساوات_کپیسٹر_سرحد_موصل_خلاء_عمودی_میدان}
D_{\textup{ع،خ}}=\epsilon_0 E_{\textup{ع،خ}}=\rho_S
\end{align}
لکھا جا سکتا ہے۔

مساوات \حوالہ{مساوات_کپیسٹر_سرحد_موصل_خلاء_متوازی_میدان_صفر} اور مساوات \حوالہ{مساوات_کپیسٹر_سرحد_موصل_خلاء_عمودی_میدان} موصل اور خالی خلاء کے سرحد پر برقی میدان کے شرائط بیان کرتے ہیں۔موصل اور خلاء کے سرحد پر برقی میدان موصل سے عمودی خارج ہوتا ہے جبکہ اس کے سرحد کے متوازی میدان صفر کے برابر ہوتا ہے۔نتیجتاً موصل کی سطح \اصطلاح{ہم قوہ سطح} ہوتی ہے۔یوں موصل کی سطح پر دو نقطوں کے مابین کسی بھی راستے پر برقی میدان کا تکمل صفر کے برابر ہو گا یعنی \عددیء{\int_a^b \kvec{E} \cdot \dif \kvec{L}=0} ہو گا۔یاد رہے کہ برقی میدان کا تکمل برقی دباو دیتا ہے جو تکمل کے راستے پر منحصر نہیں ہوتا لہٰذا اس راستے کو موصل کی سطح پر ہی رکھا جا سکتا ہے جہاں \عددیء{\kvec{E}_{\textup{متوازی}}=0} ہونے کی وجہ سے تکمل صفر کے برابر ہو گا۔ 
%===============

\ابتدا{مشق}
نقطہ \عددیء{N(2,-3,5)} موصل کی سطح پر پایا جاتا ہے جہاں \عددیء{\kvec{E}=210\ax-350\ay+99\az \, \si{\volt \per \meter}} کے برابر ہے۔اس نقطے  پر \عددیء{E_{\textup{متوازی}}}، \عددیء{E_{\textup{عمودی}}} اور \عددیء{\rho_S} حاصل کریں۔

جوابات:0، \عددیء{\SI{420}{\volt \per \meter}} اور \عددیء{\SI{3.71}{\nano \coulomb \per \meter \squared}}
\انتہا{مشق}
%==================

\حصہ{عکس کی ترکیب}
جفت قطب کے خطوط صفحہ \حوالہصفحہ{شکل_توانائی_جفت_قطب_ہم_قوہ_اور_سمت_بہاو_خط} پر  شکل \حوالہ{شکل_توانائی_جفت_قطب_ہم_قوہ_اور_سمت_بہاو_خط} میں دکھائے گئے ہیں جہاں دونوں چارجوں سے برابر فاصلے پر لامحدود برقی زمینی سطح دکھائی گئی ہے۔برقی زمین پر انتہائی باریک موٹائی کی لامحدود موصل سطح رکھی جا سکتی ہے۔ایسی موصل سطح پر برقی دباو صفر وولٹ ہو گا اور اس پر میدان عمودی ہو گا۔موصل کے اندر برقی میدان صفر رہتا ہے اور اس سے برقی میدان گزر نہیں پاتا۔

اگر اس موصل سطح کے نیچے سے جفت قطب کا منفی چارج ہٹا دیا جائے تب بھی سطح کے اوپر جانب میدان عمودی ہی ہو گا۔یاد رہے برقی زمین صفر وولٹ پر ہوتا ہے۔موصل سطح کے اوپر جانب میدان جوں کا توں رہے گا جبکہ اس سے نیچے میدان صفر ہو جائے گا۔اسی طرح سطح کے اوپر جانب سے جفت قطب کا مثبت چارج ہٹانے سے سطح کے نچلے میدان پر کوئی اثر نہیں پڑتا جبکہ سطح سے اوپر میدان صفر ہو جاتا ہے۔

آئیں ان حقائق کو دوسری نقطہ نظر سے دیکھیں۔فرض کریں کہ لامحدود موصل سطح یا برقی زمین کے  اوپر مثبت نقطہ چارج پایا جاتا ہے۔چونکہ ایسی صورت میں سطح کے اوپر جانب برقی میدان بالکل جفت قطب کے میدان کی طرح ہو گا لہٰذا ہم برقی زمین کے نچلی جانب عین مثبت چارج کے نیچے اور اتنے ہی فاصلے پر برابر مگر منفی چارج رکھتے ہوئے برقی زمین کو ہٹا سکتے ہیں۔اوپر جانب کے میدان پر ان اقدام کا کوئی اثر نہیں ہو گا۔یوں جفت قطب کے تمام مساوات بروئے کار لاتے ہوئے زمین کے اوپر جانب کا میدان حاصل کیا جا سکتا ہے۔یاد رہے کہ سطح کے نیچے برقی زمین کو صفر ہی تصور کیا جائے گا۔اگر برقی زمین کی سطح کو آئینہ تصور کیا جائے تب مثبت چارج کا عکس اس آئینہ میں اسی مقام پر نظر آئے گا جہاں ہم نے تصوراتی منفی چارج رکھا۔یوں اس منفی چارج کو حقیقی چارج کا \اصطلاح{عکس}\فرہنگ{عکس}\حاشیہب{image}\فرہنگ{image} کہتے ہیں۔

ایسی ہی ترکیب لامحدود زمینی سطح کے ایک جانب منفی چارج سے پیدا میدان حاصل کرنے کی خاطر بھی استعمال کیا جاتا ہے۔ایسی صورت میں زمین کی دوسری جانب عین منفی چارج کے سامنے،  اتنے ہی فاصلے پر برابر مقدار مگر مثبت چارج رکھتے ہوئے برقی زمین کو ہٹایا جا سکتا ہے۔

کسی بھی چارج کو نقطہ چارجوں کا مجموعہ تصور کیا جا سکتا ہے۔لہٰذا لامحدود برقی زمین یا لامحدود موصل سطح  کی ایک جانب کسی بھی شکل کے چارجوں  کا میدان، سطح کی دوسری جانب چارجوں کا عکس رکھتے اور زمین کو ہٹاتے ہوئے حاصل کیا جاتا ہے۔اس ترکیب کو \اصطلاح{عکس کی ترکیب}\فرہنگ{عکس کی ترکیب} کہتے ہیں۔یاد رہے کہ کسی بھی لامحدود موصل سطح جس کے ایک جانب چارج پایا جاتا ہو پر سطحی چارج پایا جائے گا۔عموماً مسئلے میں لامحدود سطح اور سطح کے باہر چارج معلوم ہوں گے۔ایسے مسئلے کو حل کرنے کی خاطر سطح پر سطحی چارجوں کا علم بھی ضروری ہوتا ہے۔سطحی چارج دریافت کرنا نسبتاً مشکل کام ہے جس سے چھٹکارا حاصل کرنا عقلمندی ہو گی۔عکس کی ترکیب میں سطحی چارج کا جاننا ضروری نہیں لہٰذا اس ترکیب سے مسئلہ کو حل کرنا عموماً زیادہ آسان ثابت ہوتا ہے۔
\begin{figure}
\centering
\includegraphics{figCapacitorMethodOfImages}
\caption{عکس کی ترکیب۔}
\label{شکل_کپیسٹر_عکس_کی_ترکیب}
\end{figure}

شکل  \حوالہ{شکل_کپیسٹر_عکس_کی_ترکیب} میں لامحدود موصل سطح کے اوپر جانب مختلف اقسام کے چارج دکھائے گئے ہیں۔اسی شکل میں مسئلے کو عکس کے ترکیب کی نقطہ نظر سے بھی دکھایا گیا ہے۔موصل سطح کے مقام پر دونوں صورتوں میں صفر وولٹ ہی رہتے ہیں۔ 
%==============

\ابتدا{مثال}
 لامحدود موصل سطح \عددیء{z=3} کے قریب \عددیء{N(5,7,8)} پر \عددیء{\SI{+5}{\micro \coulomb}} چارج پایا جاتا ہے۔موصل کی سطح پر نقطہ \عددیء{M(2,4,3)} پر \عددیء{\kvec{E}} حاصل کرتے ہوئے اسی مقام پر موصل کی سطحی کثافت چارج حاصل کریں۔

حل:\عددیء{\SI{+5}{\micro \coulomb}} کا عکس \عددیء{\SI{-5}{\micro \coulomb}} لامحدود سطح کے دوسری جانب نقطہ \عددیء{P(5,7,-2)} پر رکھتے ہوئے موصل سطح ہٹاتے ہیں۔اب \عددیء{N} سے \عددیء{M} تک  سمتیہ \عددیء{\kvec{R}_{MN}=-3\ax-3\ay-5\az} ہے جبکہ \عددیء{P} سے \عددیء{M} تک  سمتیہ \عددیء{\kvec{R}_{MP}=-3\ax-3\ay+5\az} ہے۔یوں \عددیء{\SI{+5}{\micro \coulomb}} نقطہ \عددیء{M} پر 
\begin{align*}
\kvec{E}_+=\frac{5 \times 10^{-6} (-3\ax-3\ay-5\az)}{4 \pi \epsilon_0 (3^2+3^2+5^2)^{\frac{3}{2}}}=\frac{5\times 10^{-6} (-3\ax-3\ay-5\az)}{4 \pi \epsilon_0 (43)^{\frac{3}{2}}}
\end{align*}
پیدا کرے گا۔اسی طرح \عددیء{\SI{-5}{\micro \coulomb}} چارج  نقطہ \عددیء{M} پر
\begin{align*}
\kvec{E}_-=\frac{-5 \times 10^{-6}(-3\ax-3\ay+5\az)}{4 \pi \epsilon_0 (3^2+3^2+5^2)^{\frac{3}{2}}}=\frac{-5 \times 10^{-6}(-3\ax-3\ay+5\az)}{4 \pi \epsilon_0 (43)^{\frac{3}{2}}}
\end{align*}
میدان پیدا کرے گا۔چونکہ برقی میدان خطی نوعیت کا ہوتا ہے لہٰذا کسی بھی نقطے پر مختلف چارجوں کے پیدا کردہ میدان جمع کرتے ہوئے کُل میدان حاصل کیا جا سکتا ہے۔یوں نقطہ \عددیء{M} پر کُل میدان
\begin{align*}
\kvec{E}_{\textup{کل}}=\kvec{E}_+ +\kvec{E}_-=\frac{-50 \times 10^{-6} \az}{4\pi \epsilon_0 (43) ^{\frac{3}{2}}}
\end{align*}
ہو گا۔موصل کی سطح پر میدان عمودی ہوتا ہے۔موجودہ جواب اس حقیقت کی تصدیق کرتا ہے۔یوں موصل کی سطح پر 
\begin{align*}
\kvec{D}=\epsilon_0 \kvec{E}=\frac{-50\times 10^{-6} \az}{4\pi (43) ^{\frac{3}{2}}}=-14.13\times 10^{-9} \az
\end{align*}
حاصل ہوتا ہے جو سطح میں داخل ہونے کی سمت میں ہے۔یوں مساوات \حوالہ{مساوات_کپیسٹر_سرحد_موصل_خلاء_عمودی_میدان} کے تحت سطح پر 
\begin{align*}
\rho_S=-14.3  \si{\nano \coulomb \per \meter \squared}
\end{align*}
پایا جاتا ہے۔
\انتہا{مثال}
%====================

مندرجہ بالا مثال میں اگر \عددیء{N(5,7,8)} پر \عددیء{\SI{+5}{\micro\coulomb}}  پایا جاتا اور لامحدود سطح موجود نہ ہوتا تب \عددیء{M(2,4,3)} پر میدان \عددیء{\kvec{E}_+} ہوتا۔لامحدود موصل سطح کی موجودگی میں یہ قیمت تبدیل ہو کر مثال میں حاصل کی گئی \عددیء{\kvec{E}_{\textup{کل}}} ہو جاتی ہے۔درحقیقت سطح کے قریب چارج کی وجہ سے سطح پر سطحی چارج کثافت پیدا ہو جاتا ہے۔کسی بھی نقطے پر بیرونی چارج اور سطحی چارج دونوں کے میدان کا مجموعہ حقیقی میدان ہوتا ہے۔
%===================
\ابتدا{مثال}\شناخت{مثال_کپیسٹر_نقطہ_چارج_سے_لامحدود_سطح_میں_پیدا_کثافت}
لامحدود موصل سطح \عددیء{z=0} میں \عددیء{(0,0,z)} پر \عددیء{Q} نقطہ چارج سے پیدا کثافت سطحی چارج حاصل کریں۔ 

حل:اس مسئلے کو عکس کے ترکیب سے حل کرنے کی خاطر \عددیء{(0,0,-z)} پر \عددیء{-Q} چارج رکھتے ہوئے موصل سطح کو ہٹا کر حل کرتے ہیں۔ایسی صورت میں سطح کے مقام پر عمومی نقطہ \عددیء{(\rho, \phi,0)} پر \عددیء{Q} اور \عددیء{-Q} چارج
\begin{align*}
\kvec{E}_+&=\frac{Q (\rho \arho-z\az)}{4\pi \epsilon_0 (\rho^2+z^2)^{\frac{3}{2}}}\\
\kvec{E}_-&=\frac{-Q (\rho \arho+z\az)}{4\pi \epsilon_0 (\rho^2+z^2)^{\frac{3}{2}}}
\end{align*}
میدان پیدا کریں گے۔\عددیء{\kvec{D}=\epsilon_0 \kvec{E}} استعمال کرتے ہوئے  کُل 
\begin{align*}
\kvec{D}=\frac{-2Qz\az}{4\pi (\rho^2+z^2)^{\frac{3}{2}}}
\end{align*}  
حاصل ہوتا ہے جس کی سمت \عددیء{-\az} ہے جو موصل میں اوپر سے داخل ہونے کی سمت ہے۔یوں موصل سطح پر
\begin{align}\label{مساوات_کپیسٹر_لامحدود_سطح_پر_سطحی_کثافت}
\rho_S=\frac{-2Qz}{4\pi  (\rho^2+z^2)^{\frac{3}{2}}} \quad \quad \si{\coulomb \per \meter \squared}
\end{align}
پایا جائے گا۔شکل \حوالہ{شکل_کپیسٹر_لامحدود_سطح_پیدا-کثافت_چارج} میں چارج \عددیء{Q} اور موصل سطح پر \عددیء{\rho_S} دکھائے گئے ہیں۔
\begin{figure}
\centering
\includegraphics{figCapacitorChargeCreatedOnSurfaceDueToPointCharge}
\caption{نقطہ چارج سے لامحدود موصل سطح میں پیدا سطحی کثافت  چارج۔}
\label{شکل_کپیسٹر_لامحدود_سطح_پیدا-کثافت_چارج}
\end{figure}
\انتہا{مثال}
%======================

مساوات \حوالہ{مساوات_کپیسٹر_لامحدود_سطح_پر_سطحی_کثافت} کو استعمال کرتے ہوئے  لامحدود موصل سطح پر کل چارج حاصل کیا جا سکتا ہے۔یقینی طور پر اس کی مقدار \عددیء{-Q} ہی حاصل ہو گی۔

\حصہ{نیم موصل}
نیم موصل اشیاء مثلاً خالص سلیکان اور جرمینیم میں آزاد چارجوں کی تعداد موصل کی نسبت سے کم جبکہ غیر موصل کی نسبت سے زیادہ ہوتی ہے۔یوں ان کی موصلیت موصل اور غیر موصل کے موصلیت کے درمیان میں ہوتی ہے۔نیم موصل کی خاص بات یہ ہے کہ ان میں انتہائی کم مقدار کے \اصطلاح{ملاوٹ}\فرہنگ{ملاوٹ}\حاشیہب{doping}\فرہنگ{doping} سے ان کی موصلیت پر انتہائی گہرا اثر پڑتا ہے۔نیم موصل \اصطلاح{دوری جدول}\فرہنگ{دوری جدول}\حاشیہب{periodic table}\فرہنگ{periodic table} کے چوتھے  \اصطلاح{جماعت}\فرہنگ{جماعت}\حاشیہب{group}\فرہنگ{group} سے تعلق رکھتے ہیں۔دوری جدول کے پانچویں جماعت کے عناصر مثلاً  نائٹروجن اور فاسفورس کا ایٹم ایک عدد الیکٹران عطا کرنے کا رجحان رکھتا ہے۔یوں انہیں \اصطلاح{عطا کنندہ}\فرہنگ{عطا کنندہ}\حاشیہب{donor}\فرہنگ{donor} عناصر کہتے ہیں۔نیم موصل میں ایسا ہر عطا کنندہ ملاوٹی ایٹم ایک عدد آزاد الیکٹران کو جنم دیتا ہے۔ ایسے عنصر کی نہایت کم مقدار کی ملاوٹ سے نیم موصل میں آزاد الیکٹران کی تعداد بڑھ جاتی ہے جس سے ان کی موصلیت بہت بڑھ جاتی ہے۔ایسے نیم موصل جن میں آزاد الیکٹران کی تعداد بڑھا دی گئی ہو کو \عددیء{n} نیم موصل کہتے ہیں۔اس کے برعکس تیسرے جماعت کے عناصر مثلاً المونیم کا ایٹم ایک عدد الیکٹران قبول کرنے کا رجحان رکھتا ہے۔یوں المونیم کو \اصطلاح{قبول کنندہ}\فرہنگ{قبول کنندہ}\حاشیہب{acceptor}\فرہنگ{acceptor} عنصر کہا جاتا ہے۔ملاوٹی المونیم کا ایٹم نیم موصل کے ایٹم سے الیکٹران حاصل کرتے ہوئے  الیکٹران کی جگہ خالی جگہ پیدا کر دیتا ہے جسے \اصطلاح{خول}\فرہنگ{خول}\حاشیہب{hole}\فرہنگ{hole} کہا جاتا ہے۔نیم موصل میں ایسا ہر قبول کنندہ ملاوٹی ایٹم ایک عدد آزاد خول کو جنم دیتا ہے۔ایسا آزاد خول مثبت ذرے کی مانند معلوم ہوتا ہے جس کا چارج \عددیء{e} الیکٹران کے چارج \عددیء{-e} کے برابر مگر الٹ قطب کا ہوتا ہے اور جس کی کمیت \عددیء{m_h} لی جا سکتی ہے۔ اسی طرح آزاد خول کی حرکت پذیری \عددیء{\mu_h} لکھی جاتی ہے۔بالکل آزاد الیکٹران کی طرح برقی میدان کی موجودگی میں آزاد خول رفتار بہاو \عددیء{\kvec{v}_d=\mu_h \kvec{E}} سے حرکت کرتا ہے جو موصلیت \عددیء{\sigma=\rho_h  \mu_h}  کو جنم دیتا ہے۔یاد رہے کہ مثبت خول \عددیء{\kvec{E}} کی سمت میں ہی حرکت کرے گا لہٰذا اس کے رفتار بہاو کی سمت \عددیء{\kvec{E}} کی سمت ہی ہو گی۔تیسرے جماعت کے عناصر کی ملاوٹ کردہ نیم موصل کو \عددیء{p} نیم موصل کہا جاتا ہے۔
آزاد الیکٹران اور آزاد خول مل کر 
\begin{align}
\sigma =-\rho_e \mu_e+\rho_h \mu_h
\end{align} 
موصلیت پیدا کرتے ہیں جہاں \عددیء{\rho_h} آزاد خول کی حجمی چارج کثافت ہے۔خالص نیم موصل میں حرارتی توانائی سے نیم موصل کے ایٹم سے الیکٹران خارج ہو کر آزاد الیکٹران کی حیثیت اختیار کرتا ہے جبکہ ایسے الیکٹران کا خالی کردہ مقام آزاد خول کی حیثیت اختیار کرتا ہے۔یوں خالص نیم موصل میں آزاد الیکٹران اور آزاد خول کی تعداد برابر ہوتی ہے۔

خالص نیم موصل اوہم کے قانون کی نقطہ شکل پر پورا اترتا ہے۔یوں کسی ایک درجہ حرارت پر نیم موصل کی موصلیت تقریباً اٹل قیمت رکھتی ہے۔  

آپ کو یاد ہو گا کہ درجہ حرارت بڑھانے سے موصل میں آزاد الیکٹران کی رفتار بہاو کم ہوتی ہے جس سے موصلیت کم  ہو جاتی ہے۔درجہ حرارت کا موصل میں آزاد الیکٹران کے حجمی چارج کثافت پر خاص اثر نہیں ہوتا۔اگرچہ نیم موصل میں بھی درجہ حرارت بڑھانے سے آزاد چارج کی رفتار بہاو کم ہوتی ہے لیکن ساتھ ہی ساتھ آزاد چارج کی مقدار نسبتاً زیادہ مقدار میں بڑھتی ہے جس  کی وجہ سے نیم موصل کی  موصلیت درجہ حرارت بڑھانے سے بڑھتی ہے۔یہ موصل اور نیم موصل کے خصوصیات میں واضح فرق ہے۔
%======
\ابتدا{مشق}\شناخت{مشق_کپیسٹر_نیم_موصل_موصلیت}
\عددیء{\SI{300}{\kelvin}} درجہ حرارت پر خالص سلیکان میں آزاد الیکٹران اور آزاد خول کی تعداد \عددیء{\num{1.5e16}} فی مربع میٹر، الیکٹران کی رفتار بہاو \عددیء{\SI{0.12}{\meter \squared \per \volt  \per \second}} جبکہ خول کی رفتار بہاو \عددیء{\SI{0.025}{\meter \squared \per \volt \per \second}} ہے۔جرمینیم کے لئے یہی قیمتیں بالترتیب \عددیء{\num{2.4e19}} فی مربع میٹر، \عددیء{\SI{0.36}{\meter \squared \per \volt  \per \second}} اور \عددیء{\SI{0.17}{\meter \squared \per \volt \per \second}} ہیں۔ خالص سلیکان اور خالص جرمینیم کی موصلیت دریافت کریں۔

جوابات:\عددیء{\SI{0.348}{\milli \siemens \per \meter}} اور \عددیء{\SI{2}{\siemens \per \meter}}

\انتہا{مشق}
%===========================

\حصہ{ذو برق}
اس باب میں اب تک ہم موصل اور نیم موصل کی بات کر چکے ہیں جن میں آزاد چارج پائے جاتے ہیں۔یوں ایسے اشیاء پر برقی دباو لاگو کرنے سے ان میں برقرار برقی رو پیدا کی جا سکتی ہے۔آئیں ایسی اشیاء کی بات کریں جن میں آزاد چارج نہیں پائے جاتے لہٰذا ان میں برقرار برقی رو پیدا کرنا ممکن نہیں ہوتا۔

بعض اشیاء مثلاً پانی  کے مالیکیول میں قدرتی طور پر مثبت اور منفی مراکز پائے جاتے  ہیں۔ایسے مالیکیول کو \اصطلاح{قطببی}\فرہنگ{قطببی}\حاشیہب{polar}\فرہنگ{polar} مالیکیول کہتے ہیں۔قطببی مالیکیول کو جفت قطب\فرہنگ{جفت قطب} تصور کیا جا سکتا ہے۔بیرونی میدان کے غیر موجودگی میں کسی بھی چیز میں قطببی مالیکیول بلا ترتیب پائے جاتے ہیں۔ بیرونی میدان \عددیء{\kvec{E}} لاگو کرنے سے مالیکیول کے مثبت سرے پر میدان کی سمت میں جبکہ منفی سرے پر میدان کی الٹ سمت میں  قوت عمل کرتا ہے۔ان قوتوں کی وجہ سے مالیکیول کے مثبت اور منفی مراکز ان قوتوں کی سمتوں میں حرکت کرتے ہوئے گھوم جاتے ہیں اور ساتھ ہی ساتھ مراکز کے درمیان فاصلہ بھی بڑھ جاتا ہے۔ٹھوس قطببی اشیاء میں ایٹموں اور مالیکیول کے درمیان قوتیں ان حرکات کو روکنے کی کوشش کرتی ہیں۔اسی طرح مثبت اور منفی چارج کے مابین قوت کشش ان کے درمیان فاصلہ بڑھنے کو روکتا ہے۔جہاں یہ مخالف قوتیں برابر ہوں وہاں مثبت اور منفی مراکز رک جاتے ہیں۔بیرونی میدان ان تمام بلا ترتیب جفت قطب کو ایک سمت میں لانے کی کوشش کرتا ہے۔

بعض اشیاء میں قدرتی طور پر مثبت اور منفی مراکز نہیں پائے جاتے البتہ انہیں بیرونی میدان میں رکھنے سے ان میں ایسے مراکز پیدا ہو جاتے ہیں۔ایسے اشیاء کو \اصطلاح{غیر قطببی}\فرہنگ{غیر قطببی}\حاشیہب{non polar}\فرہنگ{non polar} کہتے ہیں۔ بیرونی میدان مالیکیول کے الیکٹرانوں کو ایک جانب کھینچ کر منفی مرکز جبکہ  بقایا ایٹم کو مثبت چھوڑ کر مثبت مرکز پیدا کرتا ہے۔مثبت اور منفی چارج کے مابین قوت کشش اس طرح مراکز پیدا ہونے کے خلاف عمل کرتا ہے۔جہاں یہ مخالف قوتیں برابر ہو جائیں وہیں پر چارج کے حرکت کا سلسلہ رک جاتا ہے۔یہ اشیاء قدرتی طور پر غیر قطببی ہیں البتہ انہیں بیرونی میدان قطبی بنا دیتا ہے۔پیدا کردہ جفت قطب بیرونی میدان کی سمت میں ہی ہوں گے۔

ایسے تمام اشیاء جو یا تو پہلے سے قطببی ہوں اور یا انہیں بیرونی میدان کی مدد سے قطببی بنایا جا سکے \اصطلاح{ذو برقی}\فرہنگ{ذو برق}\حاشیہب{dielectric}\فرہنگ{dielectric} کہلاتے ہیں۔

ذو برق میں بیرونی میدان سے مالیکیول کے اندر حرکت پیدا ہوتی ہے البتہ مالیکیول ازخود اسی جگہ رہتا ہے۔ایسا چارج جو بیرونی میدان کی وجہ سے اپنی جگہ پر معمولی حرکت کرتا ہو کو \اصطلاح{مقید چارج}\فرہنگ{چارج!مقید}\حاشیہب{bound charge}\فرہنگ{bound charge} کہتے ہیں۔ اس کے برعکس آزاد چارج بیرونی میدان میں مسلسل حرکت کرتا ہے۔ 

ذو برق کے جفت قطب کا معیار اثر کو صفحہ \حوالہصفحہ{مساوات_توانائی_سمتی_جفت_قطب} میں دئے مساوات \حوالہ{مساوات_توانائی_سمتی_جفت_قطب} 
\begin{align}
\kvec{p}=Q \kvec{d}
\end{align}
سے ظاہر کیا جا سکتا ہے جہاں \عددیء{Q} ذو برق کے جفت قطب میں مثبت مرکز کا چارج ہے۔

اگر اکائی حجم میں \عددیء{n} جفت قطب پائے جائیں تب \عددیء{\Delta v} حجم میں \عددیء{n \Delta v} جفت قطب ہوں گے جن کا کُل معیار اثر جفت قطب تمام کے سمتی مجموعے
\begin{align}
\kvec{p}_{\textup{کل}}=\sum_{i=1}^{n \Delta v} \kvec{p}_i
\end{align}
 کے برابر ہو گا جہاں انفرادی \عددیء{\kvec{p}} مختلف ہو سکتے ہیں۔\اصطلاح{تقطیب}\فرہنگ{تقطیب}\حاشیہب{polarization}\فرہنگ{polarization} سے مراد اکائی حجم میں کل معیار اثر جفت قطب ہے یعنی
\begin{align}
\kvec{P}=\lim_{\Delta v \to 0}\frac{1}{\Delta v} \sum_{i=1}^{n \Delta v} \kvec{p}_i
\end{align} 
جس کی اکائی کولمب فی مربع میٹر ہے۔\عددیء{\Delta v} کو کم سے کم\حاشیہد{یہ ایسے ہی ہے جیسے لمحاتی رفتار \عددیء{\tfrac{\Delta x}{\Delta t}} حاصل کرتے وقت  \عددیء{\Delta t \to 0} لیا جاتا ہے۔} کرتے ہوئے نقطے پر تقطیب حاصل کی گئی ہے۔حقیقت میں \عددیء{\Delta v} کو اتنا رکھا جاتا ہے کہ اس میں جفت قطب کی تعداد \عددیء{(n \Delta v)} اتنی ہو کہ انفرادی جفت قطب کے اثر کو نظر انداز کرنا ممکن ہو۔یوں تقطیب کو یکساں تفاعل تصور کیا جاتا ہے۔

آئیں ان حقائق کو استعمال کرتے ہوئے آگے بڑھیں۔
\begin{figure}
\centering
\includegraphics{figCapacitorMotionOfBoundChargesInDielectric}
\caption{بیرونی میدان کی موجودگی میں مقید چارج کی حرکت۔}
\label{شکل_کپیسٹر_مقید_چارج_حرکت}
\end{figure}

شکل  \حوالہ{شکل_کپیسٹر_مقید_چارج_حرکت} کو دیکھتے ہوئے آگے پڑھیں۔ تصور کریں کہ ذو برق میں غیر قطبی مالیکیول پائے جاتے ہیں جن کا مقام بیرونی میدان کی غیر موجودگی میں دائروں سے ظاہر کیا گیا ہے۔بیرونی میدان کے غیر موجودگی میں \عددیء{\kvec{P}=0} ہو گا۔ذو برق کے اندر تصوراتی سطح \عددیء{\Delta \kvec{S}} لیتے ہیں جسے موٹی گہری سیاہی کی لکیر سے ظاہر کیا گیا ہے۔اس کے دونوں جانب ہلکی سیاہی سے \عددیء{a} تا \عددیء{a'} لکیر بھی دکھائی گئی ہے۔بیرونی میدان لاگو کرنے سے  جفت قطب \عددیء{\kvec{p}=Q\kvec{d}} پیدا ہوتے ہیں  جن کا \عددیء{\kvec{d}} اور \عددیء{\kvec{p}} سطح  \عددیء{\Delta \kvec{S}} کے ساتھ \عددیء{\theta} زاویہ بناتے ہیں۔ان جفت قطب کو سمتیوں سے ظاہر کیا گیا ہے جہاں سمتیہ کی نوک مثبت جبکہ اس کی دم منفی چارج کا مقام دیتی ہے۔شکل کو دیکھتے ہوئے صاف ظاہر ہے  کہ \عددیء{aa'} سے \عددیء{\tfrac{d\cos \theta}{2}} فاصلے نیچے  تک تمام مثبت چارج بیرونی میدان لاگو کرنے سے  \عددیء{aa'} سے گزرتے ہوئے  اوپر چلے جائیں گے۔ اسی طرح \عددیء{aa'} سے \عددیء{\tfrac{d\cos \theta}{2}} فاصلے اوپر  تک تمام منفی چارج بیرونی میدان لاگو کرنے سے  \عددیء{aa'} سے گزرتے ہوئے  نیچے چلے جائیں گے۔یوں \عددیء{\Delta S} رقبہ اور \عددیء{d\cos \theta} گہرائی کے حجم \عددیء{d \Delta S \cos \theta} میں جتنے بھی جفت قطب ہوں ان تمام کا ایک سرا \عددیء{\Delta \kvec{S}} سے گزرے گا۔چونکہ اکائی حجم میں \عددیء{n} جفت قطب ہیں لہٰذا اتنی حجم میں \عددیء{n d \Delta S \cos \theta} جفت قطب ہوں گے۔یوں \عددیء{\tfrac{n Qd \Delta S \cos \theta}{2}} چارج \عددیء{\Delta S} سے گزر کر اوپر  جبکہ \عددیء{\tfrac{-n Qd \Delta S \cos \theta}{2}} چارج \عددیء{\Delta S} سے گزر کر نیچے جائے گا۔مثبت چارج کا اوپر جانب حرکت اور منفی چارج کا نیچے جانب حرکت ایک ہی معنی رکھتے ہیں لہٰذا  کل 
\begin{align}\label{مساوات_کپیسٹر_مقید_چارج_الف}
\Delta Q_m=nQd \Delta S \cos \theta=n Q  \kvec{d} \cdot \Delta \kvec{S}
\end{align}
چارج سطح سے گزرتے ہوئے اوپر جانب جائے گا جہاں \عددیء{\Delta Q_m} لکھتے ہوئے اس حقیقت کی یاد دہانی کرائی گئی ہے کہ ہم مقید چارج کی بات کر رہے ہیں۔چونکہ تمام جفت قطب ایک ہی سمت میں ہیں لہٰذا اس حجم کی تقطیب
\begin{align}
\kvec{P}=nQ\kvec{d}
\end{align}
ہو گی۔یوں مساوات \حوالہ{مساوات_کپیسٹر_مقید_چارج_الف} کو
\begin{align}\label{مساوات_کپیسٹر_مقید_چارج_ب}
\Delta Q_m=\kvec{P} \cdot \Delta \kvec{S}
\end{align}
لکھا جا سکتا ہے۔اگر \عددیء{\Delta \kvec{S}} کو بند سطح کا ٹکڑا سمجھا جائے جہاں \عددیء{\kvec{a}_S} بیرونی سمت کو ہو تب اس بند سطح سے کل چارج کا اخراج 
\begin{align*}
\oint_S \kvec{P} \cdot \dif \kvec{S}
\end{align*}
کے برابر ہو گا۔یوں بند سطح میں مقید چارج کا اضافہ
\begin{align}\label{مساوات_کپیسٹر_مقید_چارج}
Q_m=-\oint_S \kvec{P} \cdot \dif \kvec{S}
\end{align}
ہو گا۔یہ مساوات گاوس کے قانون کی شکل رکھتی ہے لہٰذا ہم کثافت برقی بہاو کی تعریف یوں تبدیل کرتے ہیں کہ یہ خالی خلاء کے علاوہ دیگر صورتوں میں بھی قابل استعمال ہو۔گاوس کا قانون صحہ \حوالہصفحہ{مساوات_گاوس_قانون_گاوس_بنیادی_شکل} پر مساوات \حوالہ{مساوات_گاوس_قانون_گاوس_بنیادی_شکل} میں دیا گیا ہے۔ہم پہلے اس قانون کو \عددیء{\epsilon_0\kvec{E}} اور کل گھیرے چارج \عددیء{Q_{\textup{کل}}} کی شکل میں لکھتے ہیں
\begin{align}\label{مساوات_کپیسٹر_گاوس_قانون}
Q_{\textup{کل}}=\oint_S \epsilon_0 \kvec{E} \cdot \dif \kvec{S}
\end{align}
جہاں 
\begin{align}\label{مساوات_کپیسٹر_کل_چارج}
Q_{\textup{کل}}=Q+Q_m
\end{align}
کے برابر ہے۔مساوات \حوالہ{مساوات_کپیسٹر_گاوس_قانون} میں بند سطح \عددیء{\kvec{S}} آزاد چارج \عددیء{Q} اور مقید چارج \عددیء{Q_m} کو گھیرے ہوئے ہے۔مساوات \حوالہ{مساوات_کپیسٹر_کل_چارج} میں مساوات \حوالہ{مساوات_کپیسٹر_مقید_چارج} اور مساوات \حوالہ{مساوات_کپیسٹر_گاوس_قانون} پر کرتے ہوئے
\begin{align}
Q=Q_{\textup{کل}}-Q_m=\oint_S (\epsilon_0 \kvec{E}+\kvec{P}) \cdot \dif \kvec{S}
\end{align}
حاصل ہوتا ہے۔

ہم کثافت برقی بہاو کو اب
\begin{align}\label{مساوات_کپیسٹر_برقی_بہاو_اور_تقطیب}
\kvec{D}=\epsilon_0 \kvec{E}+\kvec{P}
\end{align}
بیان کرتے ہیں جو زیادہ کارآمد اور عمومی مساوات ہے۔یوں ذو برق اشیاء کے لئے کثافت برقی بہاو میں اضافی جزو \عددیء{\kvec{P}} شامل ہو جاتا ہے۔اس طرح
\begin{align}\label{مساوات_کپیسٹر_گاوس_عمومی_مساوات}
Q=\oint_S \kvec{D} \cdot \dif \kvec{S}
\end{align}
لکھا جا سکتا ہے جہاں \عددیء{Q} گھیرا ہوا آزاد چارج ہے۔

ہم آزاد، مقید اور کُل چارجوں کے لئے آزاد، مقید اور کُل حجمی کثافت بیان کرتے ہوئے
\begin{align*}
Q&=\int_h \rho_h \dif h\\
Q_m&=\int_h \rho_m \dif h\\
Q_{\textup{کل}}&=\int_h \rho_{\textup{کل}} \dif h\\
\end{align*}
لکھ سکتے ہیں۔

مسئلہ پھیلاو کے استعمال سے مساوات \حوالہ{مساوات_کپیسٹر_مقید_چارج}، مساوات \حوالہ{مساوات_کپیسٹر_گاوس_قانون} اور مساوات \حوالہ{مساوات_کپیسٹر_گاوس_عمومی_مساوات} کے نقطہ اشکال
\begin{align*}
\nabla \cdot \kvec{P}&=-\rho_m \\
\epsilon_0 \nabla \cdot \kvec{E}&=\rho_{\textup{کل}} 
\end{align*}
اور
\begin{align}
\nabla \cdot \kvec{D}&=\rho_h
\end{align}

لکھے جا سکتے ہیں۔

قلم میں دوراتے طرز پر ایٹم پائے جاتے ہیں۔قلم عموماً کسی ایک سمت میں با آسانی جبکہ بقایا سمتوں میں مشکل سے تقطیب ہو پاتا ہے۔ایسے اشیاء جو ہر طرف یکساں خصوصیات نہیں رکھتے \اصطلاح{ناہم سموت}\فرہنگ{ناہم سموت}\حاشیہب{anisotropic}\فرہنگ{anisotropic} کہلاتے ہیں۔ساتھ ہی ساتھ یہ ضروری نہیں کہ بیرونی لاگو میدان اور تقطیب ایک ہی سمت میں ہوں۔کچھ ایسے اشیاء بھی پائے جاتے ہیں جو \اصطلاح{برقی چال}\فرہنگ{برقی چال}\حاشیہب{ferroelectric}\فرہنگ{ferroelectric} کی خاصیت رکھتے ہیں۔ان میں تقطیب کی قیمت ان اشیاء کی گزشتہ تاریخ پر مبنی ہوتی ہے۔یہ عمل بالکل مقناطیسی مادے کی مقناطیسی چال کے طرز کی خصوصیت ہے۔ 
   
کچھ ذو برق اشیاء میں لاگو بیرونی میدان \عددیء{\kvec{E}} اور تقطیب \عددیء{\kvec{P}} ہر صورت ایک ہی سمت میں ہوتے ہیں۔ ایسے اشیاء \اصطلاح{ہم سمتی}\فرہنگ{ہم سمتی}\حاشیہب{isotropic}\فرہنگ{isotropic} کہتے ہیں۔انجنیئرنگ میں استعمال ہونے والے  ذو برق اشیاء عموماً ایسے ہی ہوتے ہیں۔اس کتاب میں صرف انہیں پر تبصرہ کیا جائے گا۔ایسے اشیاء میں  تقطیب اور لاگو برقی میدان راست تناسب تعلق
\begin{gather}
\begin{aligned}
\kvec{P}&=\chi_e \epsilon_0 \kvec{E}\\
&=(\epsilon_R-1)\epsilon_0 \kvec{E}
\end{aligned}
\end{gather}
رکھتا ہے جہاں مساوات کے مستقل کو \عددیء{\chi_e \epsilon_0} یا \عددیء{(\epsilon_R-1)\epsilon_0} لکھا جاتا ہے۔یوں مساوات \حوالہ{مساوات_کپیسٹر_برقی_بہاو_اور_تقطیب}
\begin{align*}
\kvec{D}=\epsilon_0 \kvec{E}+(\epsilon_R-1)\epsilon_0 \kvec{E}
\end{align*}
یا
\begin{align}
\kvec{D}=\epsilon_R\epsilon_0 \kvec{E}=\epsilon \kvec{E}
\end{align}
شکل اختیار کرتا ہے جہاں
\begin{align}
\epsilon=\epsilon_R \epsilon_0
\end{align}
کے برابر ہے۔

\عددیء{\chi_e} \اصطلاح{ذو برقی مستقل}\فرہنگ{ذو برقی مستقل}\حاشیہب{susceptibility}\فرہنگ{susceptibility}، \عددیء{\epsilon_R} \اصطلاح{جزوی برقی مستقل}\فرہنگ{جزوی برقی مستقل}\حاشیہب{relative electric constant, relative permittivity}\فرہنگ{electric constant!relative}\فرہنگ{permittivity!relative}،  \عددیء{\epsilon_0} \اصطلاح{خالی خلاء کا برقی مستقل}\فرہنگ{برقی مستقل!خالی خلاء}\حاشیہب{permittivity of vacuum, electric constant of vacuum}\فرہنگ{electric constant!vacuum} جبکہ \عددیء{\epsilon} ان اشیاء کا \اصطلاح{برقی مستقل} کہلاتے ہیں اس کتاب کے آخر میں صفحہ \حوالہصفحہ{جدول_جدول_جزوی_برقی_مستقل_زاویہ_زیاع} پر  چند مخصوص اشیاء کے برقی مستقل جدول \حوالہ{جدول_جدول_جزوی_برقی_مستقل_زاویہ_زیاع} میں دئے گئے ہیں۔

ناہم سموت اشیاء اتنے سادہ مساوات سے نہیں نپٹے جاتے۔ان اشیاء میں \عددیء{\kvec{E}} کا ہر کارتیسی جزو \عددیء{\kvec{D}}   کے ہر کارتیسی جزو پر اثر انداز ہوتا ہے لہٰذا ان کا تعلق یوں
\begin{gather}
\begin{aligned}
D_x&=\epsilon_{xx} E_x +\epsilon_{xy} E_y+\epsilon_{xz} E_z\\
D_y&=\epsilon_{yx} E_x +\epsilon_{yy} E_y+\epsilon_{yz} E_z\\
D_z&=\epsilon_{zx} E_x +\epsilon_{zy} E_y+\epsilon_{zz} E_z
\end{aligned}
\end{gather}
لکھا جاتا ہے جہاں نو اعدادی \عددیء{\epsilon_{ij}} کو مجموعی طور پر \اصطلاح{تناوی مستقل}\فرہنگ{تناوی مستقل}\حاشیہب{tensor}\فرہنگ{tensor} کہا جاتا ہے۔ناہم سموت اشیاء میں \عددیء{\kvec{D}} اور  \عددیء{\kvec{E}} (اور \عددیء{\kvec{P}}) آپس میں متوازی نہیں  ہوتے اور اگرچہ \عددیء{\kvec{D}=\epsilon_0 \kvec{E}+\kvec{P}} ان کے لئے بھی درست ہے، \عددیء{\kvec{D}=\epsilon \kvec{E}} استعمال کرتے وقت  اس حقیقت کا خیال رکھنا ہو گا کہ \عددیء{\epsilon} اب تناوی مستقل ہے۔ناہم سموت اشیاء پر یہیں بحث روکتے ہیں۔


