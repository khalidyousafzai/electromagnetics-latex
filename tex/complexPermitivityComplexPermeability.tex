
برقی طاقت کے ضیاع کو \عددی{\sigma} سے ظاہر کیا جاتا ہے۔آئیں دیکھیں کہ \عددی{\sigma} از خود کن مستقل پر منحصر ہے۔کسی بھی خطے میں موجود ایٹم اور مالیکیول گزرتے موج پر اثر انداز ہوتے ہیں۔برقی موج پر ان اثرات کو \اصطلاح{مخلوط برقی مستقل}\فرہنگ{برقی مستقل!مخلوط}\حاشیہب{complex permittivity}\فرہنگ{permittivity!complex} 
\begin{align*}
\epsilon=\epsilon'-j \epsilon"=\epsilon_0 (\epsilon_R'-j\epsilon_R")
\end{align*}
سے ظاہر کیا جاتا ہے۔ بند الیکٹران یا آئن کے ارتعاش اور ذو برق کی سمت میں تبدیلی مخلوط برقی مستقل (اور برقی ضیاع) کو جنم دیتی ہیں۔اس کے علاوہ آزاد الیکٹران اور آزاد خول کی حرکت بھی مخلوط برقی مستقل کو جنم دیتی ہے۔بالکل مخلوط برقی مستقل کی طرح مقناطیسی موج کی برقی ضیاع کو \اصطلاح{مخلوط مقناطیسی مستقل}\فرہنگ{مخلوط مقناطیسی مستقل}\فرہنگ{مقناطیسی!مخلوط مستقل}\حاشیہب{complex magnetic constant}\فرہنگ{magnetic!complex constant}
\begin{align*}
\mu=\mu'-j \mu"=\mu_0(\mu_R'-j\mu_R")
\end{align*}
سے  ظاہر کیا جاتا ہے۔ہمیں جن اشیاء میں امواج میں دلچسپی  ہے ان میں مقناطیسی اثر قابل نظر انداز ہوتا ہے۔یوں ان میں عموماً \عددی{\mu \approx \mu_0} ہی لیا جاتا ہے۔اب 

اس کتاب میں \عددی{\mu} کو حقیقی تصور کرتے ہوئے موج کی برقی ضیاع کا جواز مخلوط برقی مستقل کو تصور کیا جائے گا۔ 

