
\حصہ{برقی دباو}
زمین اور کمیت \عددیء{m} کے درمیان قوت ثقل \عددیء{\kvec{F}=-\tfrac{GMm}{r^2}\ar} پایا جاتا ہے جس میں \عددیء{\tfrac{GM}{r^2}=g} لکھتے ہوئے  \عددیء{\kvec{F}=-mg \ar} لکھا جا سکتا ہے۔کمیت کو اونچائی \عددیء{\Delta h \ar} منتقل کرنے  کی خاطر اس قوت کے خلاف
\begin{align*}
\kvec{F}_{\textrm{لاگو}}=mg \ar
\end{align*}
لاگو کرتے ہوئے
\begin{align*}
\Delta W=\kvec{F}_{\textrm{لاگو}} \cdot \Delta h \ar=mg \Delta h
\end{align*}
 توانائی \عددیء{W} درکار ہو گی۔یہ توانائی کمیت میں منتقل ہو جاتی ہے جسے \ترچہ{مخففی توانائی}\فرہنگ{مخفففی توانائی}\حاشیہب{potential energy}\فرہنگ{potential energy} کہتے ہیں۔اگر \عددیء{\Delta h} کی قیمت  \عددیء{r} کی نسبت سے  بہت کم نہ ہو تب \عددیء{g} کو مستقل تصور کرنا ممکن نہ ہو گا اور مخففی توانائی تکملہ کے ذریعہ حاصل کی جائے گی۔\حاشیہط{sign is different}
\begin{align}
W =-\int_{\textrm{ابتدا}}^{\textrm{اختتام}} \kvec{F} \cdot \dif \kvec{r}=-\int_{\textrm{ابتدا}}^{\textrm{اختتام}} \frac{GMm }{r^2} dr
\end{align}
ثقلی میدان میں کمیت کو ابتدائی نقطے سے اختتامی نقطے تک پہنچاتے ہوئے کوئی بھی راستہ اختیار کیا جا سکتا ہے۔اختیار کردہ راستے کا درکار توانائی پر کسی قسم کا کوئی اثر نہیں ہوتا۔

چارجوں کے مسئلے کو بھی اسی طرح حل کیا جاتا ہے۔برقی میدان \سمتیہ{E} میں چارج \عددیء{q} پر قوت \عددیء{\kvec{F}=q \kvec{E}} عمل کرتا ہے۔چارج کو فاصلہ \عددیء{\dif \kvec{L}} ہلانے کی خاطر اس قوت کے خلاف بیرونی قوت
\begin{align}
\kvec{F}_{\textrm{بیرونی}} = -q \kvec{E}
\end{align}
لاگو کرتے ہوئے توانائی
\begin{align}
\dif W=-q \kvec{E} \cdot \dif \kvec{L}
\end{align}
خرچ ہو گی۔کسی بھی ابتدائی نقطے سے اختتامی نقطے تک یوں
\begin{align}
W=-q \int_{\textrm{ابتدا}}^{\textrm{اختتام}} \kvec{E} \cdot \dif\kvec{L}
\end{align}
توانائی درکار ہو گی۔

اس مساوات میں منتقلی کرتے ہوئے \سمتیہ{E} کو تغیر پذیر تصور کیا گیا ہے۔اس توانائی کا دارومدار مختلف مقامات پر برقی شدت اور چارج \عددیء{q} پر ہے۔ساکن برقی میدان\فرہنگ{ساکن برقی میدان} میں بھی اختیار کردہ راستے کا درکار توانائی پر کسی قسم کا کوئی اثر نہیں ہوتا۔برقی و برقیات میں عموماً اس توانائی سے زیادہ اہم وہ توانائی ہے جو اکائی چارج کو ابتدائی نقطے سے اختتامی نقطے تک منتقل کرنے کی خاطر درکار ہو۔اس توانائی کو \ترچہ{برقی دباو}\فرہنگ{برقی دباو}\حاشیہب{voltage}\فرہنگ{voltage} کہتے اور \عددیء{v} سے ظاہر کرتے ہیں۔چونکہ توانائی غیر سمتی یعنی مقداری ہے لہٰذا برقی دباو بھی مقداری ہے۔مندرجہ بالا مساوات سے برقی دباو یوں حاصل ہوتا ہے۔
\begin{align}
v=\frac{W}{q}=-\int_{\textrm{ابتدا}}^{\textrm{اختتام}} \kvec{E} \cdot \dif\kvec{L}
\end{align}
ہمارے ملک میں \عددیء{\SI{220}{\volt}} برقی دباو مہیا کی جاتی ہے۔یوں منفی تار سے مثبت تار تک \عددیء{\SI{1}{\coulomb}} کا چارج منتقل کرنے کی خاطر \عددیء{\SI{220}{\joule}} توانائی درکار ہو گی۔برقی دباو پر تفصیلاً آگے باب میں غور کیا جائے گا۔ 
%==============================

