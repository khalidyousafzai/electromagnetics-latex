\باب{مقناطیسی قوتیں، مقناطیسی مادے اور امالہ}
برقی چارج کے گرد برقی میدان پایا جاتا ہے جس میں موجود ساکن یا حرکت کرتے چارج پر قوت دفع یا قوت کشش پایا جاتا ہے۔مقناطیسی میدان برقی رو یعنی حرکت کرتے چارج سے پیدا ہوتا ہے اور اس میدان میں حرکت کرتے چارج پر قوت پائی جاتی ہے۔مقناطیسی میدان ساکن چارج پر قوت پیدا نہیں کرتا۔

اس باب میں برقی رو گزارتی تار پر قوت اور مروڑ  کا جائزہ لیا جائے گا۔ اس کے بعد مقناطیسی اشیاء اور آخر میں امالہ پر غور کیا جائے گا۔

\حصہ{متحرک چارج پر قوت}
تجربے سے  ثابت ہوتا ہے کہ برقی میدان میں چارج بردار ذرے  پر 
\begin{align}\label{مساوات_امالہ_برقی_قوت}
\kvec{F}=Q \kvec{E}
\end{align}
قوت اثر انداز ہوتی ہے۔مثبت چارج کی صورت میں یہ قوت برقی میدان کے شدت \عددیء{\kvec{E}} کی سمت میں ہوتی ہے۔قوت کی قیمت چارج \عددیء{Q} اور برقی میدان کی شدت \عددیء{\kvec{E}} کے حاصل ضرب کے برابر ہوتی ہے۔چارج ساکن ہو یا حرکت کر رہا ہو، اس پر قوت کی مقدار اسی مساوات سے حاصل ہوتی ہے۔

اسی طرح تجربے سے ثابت ہوتا ہے کہ مقناطیسی میدان میں ساکن چارج بردار ذرے  پر مقناطیسی میدان کوئی قوت پیدا نہیں کرتا البتہ متحرک چارج بردار ذرے  پر مقناطیسی میدان
\begin{align}\label{مساوات_امالہ_مقناطیسی_قوت}
\kvec{F}=Q \kvec{v} \times \kvec{B} 
\end{align}
قوت پیدا کرتا ہے۔یہ قوت چارج  کے براہ راست متناسب ہوتی ہے۔اسی طرح قوت چارج کے رفتار \عددیء{\kvec{v}}،  کثافت مقناطیسی میدان \عددیء{\kvec{B}} اور ان دو کے مابین زاویے کے سائن کے بھی براہ راست متناسب ہوتی ہے۔قوت کی سمت \عددیء{\kvec{v}} اور \عددیء{\kvec{B}} دونوں کے عمودی یعنی \عددیء{\kvec{v} \times \kvec{B}} سمت میں ہوتی ہے۔

مقناطیسی قوت رفتار کے عمودی ہے لہٰذا یہ رفتار کے قیمت پر اثر انداز نہیں ہوتا البتہ یہ اس کی سمت پر ضرور اثر ڈالتا ہے۔اس طرح مقناطیسی قوت چارج بردار ذرے کے متحرک توانائی میں تبدیلی لانے سے قاصر ہے۔ اس کے برعکس برقی قوت جسے مساوات \حوالہ{مساوات_امالہ_برقی_قوت} بیان کرتا ہے چارج بردار ذرے کی رفتار میں تبدیلی پیدا کرتے ہوئے حرکی توانائی میں تبدیلی پیدا کرتا ہے۔دونوں میدانوں میں یہ بنیادی فرق ہے کہ برقی میدان تبادلہ توانائی میں کردار ادا کرتا ہے جبکہ مقناطیسی میدان تبادلہ توانائی میں کردار ادا نہیں کرتا۔

دونوں میدانوں کے بیک وقت موجودگی میں چارج بردار ذرے پر کل قوت
\begin{align}\label{مساوات_امالہ_لورنز}
\kvec{F}=Q\left(\kvec{E}+\kvec{v} \times \kvec{B} \right)
\end{align}
دونوں میدانوں سے علیحدہ علیحدہ پیدا قوتوں کے مجموعے کے برابر ہے۔مساوات \حوالہ{مساوات_امالہ_لورنز} \اصطلاح{لورنز مساوات قوت}\فرہنگ{لورنز مساوات قوت}\حاشیہد{یہ مساوات ہینڈرک لورنز کے نام ہے۔}\حاشیہب{Lorentz force equation}\فرہنگ{Lorentz force equation} کہلاتی ہے۔برقی اور مقناطیسی میدانوں میں چارج بردار ذرے، مثلاً الیکٹران، کے راہ اسی مساوات کو حل کرتے ہوئے حاصل کئے جاتے ہیں۔
%===========

\ابتدا{مشق}
ایک عدد نقطہ چارج جس کی قیمت \عددیء{\SI{-3}{\coulomb}} اور رفتار \عددیء{\kvec{v}=2\ax-3\ay+\az} ہو پر مندرجہ ذیل میدانوں میں قوت کی حتمی قیمت حاصل کریں۔
(الف) \عددیء{\kvec{E}=3\ax-2\ay-5\az}، (ب) \عددیء{\kvec{B}=-2\ax-3\ay+6\az}، (پ) دونوں میدانوں کے بیک وقت موجودگی میں۔

جوابات:\عددی{\SI{18.49}{\newton}}،  \عددی{\SI{71.3}{\newton}}، \عددی{\SI{78.7}{\newton}}
\انتہا{مشق}
%=====================

\حصہ{تفرقی چارج پر قوت}
مقناطیسی میدان میں متحرک تفرقی چارج \عددیء{\dif Q} پر تفرقی قوت \عددیء{\dif \kvec{F}} عمل کرے گی۔  
\begin{align}\label{مساوات_امالہ_مقناطیسی_تفرقی_قوت}
\dif \kvec{F}=\dif Q \kvec{v} \times \kvec{B}
\end{align}

آپ جانتے ہیں کہ منفی چارج کی باریک ترین مقدار الیکٹران کا چارج ہے۔مثبت چارج کی باریک ترین قیمت بھی اتنی ہی لیکن مثبت قطب کی ہے۔منفی چارج کو مثال بناتے ہوئے، یوں مندرجہ بالا مساوات میں تفرقی چارج سے مراد کم از کم اتنا چارج ہے جس میں الیکٹرانوں کی تعداد اتنی ہو کہ کسی ایک الیکٹران کے چارج کا اثر قابل نظر انداز ہو۔اسی طرح اس تفرقی چارج کا حجم اگرچہ چھوٹا ہے لیکن اس حجم کی جسامت الیکٹرانوں کے مابین اوسط فاصلے سے بہت زیادہ ہے۔مساوات \حوالہ{مساوات_امالہ_مقناطیسی_تفرقی_قوت} تفرقی چارج پر کل قوت دیتا ہے۔یہاں یہ سمجھ لینا ضروری ہے کہ یہ قوت کسی ایک الیکٹران پر اثر انداز نہیں ہوتا بلکہ یہ تمام الیکٹرانوں پر علیحدہ علیحدہ قوتوں کا مجموعہ ہے۔ 

موصل تار میں برقی رو، الیکٹران کے حرکت کی بدولت ہے۔برقی رو گزارتے تار کو مقناطیسی میدان میں رکھنے سے تار میں ہر الیکٹران پر مقناطیسی قوت کا اثر پایا جائے گا۔اگرچہ کسی ایک الیکٹران پر انتہائی کم قیمت کا قوت پایا جاتا ہے لیکن موصل تار میں الیکٹرانوں کی تعداد انتہائی زیادہ ہوتی ہے۔یوں انتہائی زیادہ تعداد میں انتہائی کم قوتوں کا مجموعہ معقول قیمت کی قوت پیدا کرتا ہے۔آئیں دیکھتے ہیں کہ یہ مجموعی قوت تار تک کس طرح منتقل ہوتی ہے۔

موصل میں مثبت ایٹم یا آئن ساکن ہوتے ہیں جبکہ الیکٹران آزادی سے حرکت کر سکتے ہیں۔مقناطیسی میدان میں برقی رو گزارتے موصل تار میں حرکت پذیر منفی الیکٹران پر مقناطیسی قوت عمل کرتی ہے جس سے مثبت آئن اور منفی الیکٹران کے مابین فاصلوں میں تبدیلی رونما ہوتی ہے۔اب مثبت اور منفی چارج کے مابین کولومب قوتیں ایسی تبدیلی کو روکتے ہیں لہٰذا حرکت پذیر الیکٹران پر مقناطیسی قوت یوں ساکن آئن تک پہنچ پاتی ہیں جو بطور تار پر مقناطیسی قوت کی صورت میں رونما ہوتی ہے۔

\begin{figure}
\centering
\includegraphics{figInductanceHallEffect}
\caption{ہال اثر سے متحرک چارج کا قطب دریافت کیا جا سکتا ہے۔}
\label{شکل_امالہ_ہال_قطب_کا_حصول}
\end{figure}

مثبت آئن اور منفی الیکٹران کے مابین کولمب قوتیں انتہائی طاقتور ہوتی ہیں لہٰذا مقناطیسی میدان سے پیدا فاصلوں میں تبدیلی قابل ناپ نہیں ہوتی۔مثبت اور منفی چارجوں کے مابین فاصلے کی بنا پر انہیں دو چادر کپیسٹر تصور کیا جا سکتا ہے۔ہم جانتے ہیں کہ ایسے کپیسٹر کے چادروں کے مابین برقی دباو پایا جاتا ہے۔یوں الیکٹران کے حرکت اور مقناطیسی میدان دونوں کی سمتوں کے عمودی دو الٹ اطراف کے مابین تار پر معمولی برقی دباو پایا جاتا ہے جسے \اصطلاح{ہال اثر}\فرہنگ{ہال!اثر}\حاشیہب{Hall effect}\فرہنگ{Hall!effect} کے نام\حاشیہد{ایڈون حال نے اس اثر کو  1879 میں دریافت کیا۔} سے جانا جاتا ہے۔

ہال اثر کو شکل \حوالہ{شکل_امالہ_ہال_قطب_کا_حصول} کی مدد سے باآسانی سمجھا جا سکتا ہے۔شکل-الف میں موصل یا \عددیء{n} قسم کے نیم موصل برقی رو گزارتا تار دکھایا گیا ہے۔تار میں برقی رو \عددیء{I} کی سمت \عددیء{-\ax} ہے  لہٰذا تار میں آزاد منفی چارج اس کے الٹ یعنی \عددیء{\ax} سمت  میں حرکت کر رہے ہیں۔تار میں آزاد الیکٹران کو ہلکی سیاہی میں تیر کے نشان پر دائرے میں بند \عددیء{-} علامت سے ظاہر کیا گیا ہے جہاں تیر اس کے حرکت کی سمت ظاہر کرتا ہے۔یہ تار \عددیء{\az} سمت کے مقناطیسی میدان میں پڑی ہے۔تار میں آزاد چارج منفی قطب کے ہیں لہٰذا ان پر مساوات \حوالہ{مساوات_امالہ_مقناطیسی_قوت} کے تحت \عددیء{\ay} سمت میں قوت \عددیء{\kvec{F_-}} عمل کرے گا۔قوت کی علامت پر زیر نوشت میں منفی کی علامت یہ ظاہر کرتی ہے کہ یہ قوت متحرک منفی چارج پر اثر انداز ہوتا ہے۔یوں تار کے دائیں طرف پر منفی الیکٹرانوں کا انبار جمع ہوتا ہے جبکہ تار کے بائیں طرف پر الیکٹران کی تعداد کم ہو جاتی ہے جس سے اس جانب ساکن مثبت آئن \اصطلاح{بے پردہ}\فرہنگ{بے پردہ}\حاشیہب{uncovered}\فرہنگ{charges!uncovered} ہو جاتے ہیں۔شکل \حوالہ{شکل_امالہ_ہال_قطب_کا_حصول}-الف میں تار کے دائیں طرف \عددیء{-} اور بائیں طرف  \عددیء{+} کے علامات انہیں کو ظاہر کرتے ہیں۔آپ جانتے ہیں کہ مثبت اور  منفی چارج  کے مابین برقی میدان کی شدت \عددیء{\kvec{E}} اور یوں برقی دباو پایا جاتا ہے لہٰذا تار کے دائیں اور بائیں اطراف کے مابین \اصطلاح{ہال برقی دباو}\فرہنگ{ہال!برقی دباو}\حاشیہب{Hall voltage}\فرہنگ{Hall!voltage} پایا جائے گا۔تار کا بایاں طرف ہال برقی دباو کا مثبت سرا ہو گا۔

آئیں ایسی صورت دیکھیں جہاں متحرک مثبت چارج کی بدولت برقی رو پائی جائے۔شکل \حوالہ{شکل_امالہ_ہال_قطب_کا_حصول}-ب میں بقایا صورت حال بالکل شکل-الف کی طرح ہے البتہ یہاں تار \عددیء{p} قسم کے نیم موصل کا بنا ہوا ہے جس میں برقی رو مثبت \اصطلاح{آزاد خول}\فرہنگ{خول!آزاد}\حاشیہب{free holes}\فرہنگ{holes!free} کے حرکت سے پیدا ہوتی ہے۔یوں اگر برقی رو \عددیء{-\ax} سمت میں ہو تب آزاد خول بھی اسی سمت میں حرکت کریں گے۔جیسے شکل میں دکھایا گیا ہے یہاں بھی مقناطیسی قوت آزاد چارج کو دائیں جانب دھکیل رہے ہیں۔آپ دیکھ سکتے ہیں کہ اس بار ہال برقی دباو کا مثبت سرا تار کا دائیں طرف پایا جاتا ہے جو شکل-الف کے عین الٹ ہے۔اس حقیقت کو استعمال کرتے ہوئے یہ معلوم کیا جا سکتا ہے کہ آیا  نیم موصل \عددیء{n} یا \عددیء{p} قسم کا ہے۔

ہال اثر استعمال کرتے ہوئے مختلف پیمائشی آلات بنائے جاتے ہیں مثلاً \اصطلاح{یک سمتی رو پیما}\فرہنگ{ہال!یک سمتی رو پیما}\فرہنگ{Hall!effect current meter}، \اصطلاح{مقناطیسی بہاو پیما}\فرہنگ{ہال!مقناطیسی بہاو پیما}\حاشیہب{magnetic flux meter}\فرہنگ{Hall!magnetic flux meter} وغیرہ۔

سمتی رفتار \عددیء{\kvec{v}} سے حرکت کرتا ہوا حجمی کثافت چارج \عددیء{\rho_h}  کثافت برقی رو \عددیء{\kvec{J}}
\begin{align}
\kvec{J}=\rho_h \kvec{v}
\end{align}
کو جنم دیتا ہے۔اس مساوات کو صفحہ \حوالہصفحہ{مساوات_کپیسٹر_کثافت_رو_مساوی_کثافت_چارج_ضرب_رفتار} پر حاصل کیا گیا۔چھوٹے حجم \عددیء{\dif h} میں تھوڑے سے چارج کو
\begin{align}
\dif Q=\rho_h \dif h
\end{align}
لکھا جا سکتا ہے لہٰذا مساوات \حوالہ{مساوات_امالہ_مقناطیسی_تفرقی_قوت} کو
\begin{align*}
\dif \kvec{F}=\rho_h \dif h \kvec{v} \times \kvec{B}
\end{align*}
یا
\begin{align}\label{مساوات_امالہ_لورنز_تفرقی_الف}
\dif \kvec{F}=\kvec{J} \times \kvec{B} \dif h
\end{align}
لکھا جا سکتا ہے۔ہم مساوات \حوالہ{مساوات-مقناطیسی_تفرقی_برقی_رو_مختلف_صورت} میں دیکھ چکے ہیں کہ \عددیء{\kvec{J} \dif h} کو برقی رو گزارتے تار کا تفرقی حصہ تصور کیا جا سکتا ہے جسے 
\begin{align*}
\kvec{J} \dif h=\kvec{K} \dif S=I \dif \kvec{L}
\end{align*}
بھی لکھا جا سکتا ہے۔اس طرح مساوات \حوالہ{مساوات_امالہ_لورنز_تفرقی_الف} کو
\begin{align}\label{مساوات_امالہ_لورنز_تفرقی_ب}
\dif \kvec{F}=\kvec{K} \times \kvec{B} \dif S
\end{align}
یا 
\begin{align}\label{مساوات_امالہ_لورنز_تفرقی_پ}
\dif \kvec{F}=I \dif \kvec{L} \times \kvec{B}
\end{align}
بھی لکھا جا سکتا ہے۔

مساوات \حوالہ{مساوات_امالہ_لورنز_تفرقی_الف}، مساوات \حوالہ{مساوات_امالہ_لورنز_تفرقی_ب} اور مساوات \حوالہ{مساوات_امالہ_لورنز_تفرقی_پ} کے تکمل سے انہیں یوں
\begin{align}
\kvec{F}&=\int_h \kvec{J} \times \kvec{B} \dif h \label{مساوات_امالہ_لورنز_تتکمل_ت}\\
\kvec{F}&=\int_S \kvec{K} \times \kvec{B} \dif S \label{مساوات_امالہ_لورنز_تتکمل_ٹ}\\
\kvec{F}&=\oint I \dif \kvec{L} \times \kvec{B} \label{مساوات_امالہ_لورنز_تتکمل_پ}
\end{align}
لکھا جا سکتا ہے۔

مساوات \حوالہ{مساوات_امالہ_لورنز_تتکمل_پ} میں اگر سیدھی تار لی جائے جس کی لمبائی \عددیء{L} ہو  تو تکمل سے
\begin{align}\label{مساوات_امالہ_سیدھی_تار_قوت_الف}
\kvec{F}=I \kvec{L} \times \kvec{B}
\end{align}
حاصل ہوتا ہے جس میں قوت کی قیمت
\begin{align}\label{مساوات_امالہ_سیدھی_تار_قوت_ب}
F=I L B \sin \alpha
\end{align}
ہے جہاں تار اور مقناطیسی میدان کے درمیان زاویہ \عددیء{\alpha} ہے۔مساوات \حوالہ{مساوات_امالہ_سیدھی_تار_قوت_الف} اور مساوات \حوالہ{مساوات_امالہ_سیدھی_تار_قوت_ب} پورے دور کے کچھ حصے پر قوت دیتے ہیں۔دور کے بقایا حصوں پر بھی اسی طرح قوت حاصل کئے جا سکتے ہیں۔
%=====================

\حصہ{برقی رو گزارتے تفرقی تاروں کے مابین قوت}
شکل میں نقطہ \عددیء{N_1} پر تار کا ایک چھوٹا ٹکڑا \عددیء{\dif \kvec{L}_1} دکھایا گیا ہے جس میں \عددیء{I_1} برقی رو گزر رہی ہے  جبکہ نقطہ \عددیء{N_2} پر تار کا دوسرا چھوٹا ٹکڑا \عددیء{\dif \kvec{L}_2} دکھایا گیا ہے جس میں \عددیء{I_2} برقی رو گزر رہی ہے۔ نقطہ \عددیء{N_2} پر تار کے پہلے ٹکڑے سے پیدا مقناطیسی میدان مساوات \حوالہ{مساوات_بایوٹ_سیوارٹ_تفرق_شکل_ب} دیتا ہے۔

\begin{align*}
\dif \kvec{H}_2&=\frac{I_1 \dif \kvec{L}_1 \times \kvec{a}_{R21}}{4 \pi R_{21}^2}
\end{align*}
مساوات \حوالہ{مساوات_امالہ_لورنز_تفرقی_پ} مقناطیسی میدان \عددیء{\kvec{H}_2} میں تار کے تفرقی حصے پر تفرقی قوت دیتا ہے۔یہاں تفرقی مقناطیسی میدان \عددیء{\dif \kvec{H}_2} سے \عددیء{\dif \kvec{L}_2} پر  پیدا قوت درکار ہے۔اس قوت کو تفرقی قوت کا تفرقی حصہ \عددیء{\dif(\dif \kvec{F}_2)} لکھتے ہوئے مساوات \حوالہ{مساوات_امالہ_لورنز_تفرقی_پ} کو 
\begin{align*}
\dif \left(\dif \kvec{F}_2 \right)=I_2 \dif \kvec{L}_2 \times \dif \kvec{B}_2
\end{align*}
لکھا جا سکتا ہے جہاں \عددیء{\dif \kvec{B}_2=\mu_0 \dif \kvec{H}_2} کے برابر ہے۔مندرجہ بالا دو مساوات سے
\begin{align}\label{مساوات_امالہ_تفرقی_تفرقی_قوت}
\dif \left(\dif \kvec{F}_2 \right)=\mu_0 \frac{I_1 I_2}{4\pi R_{21}^2} \dif \kvec{L}_2 \times \left(\dif \kvec{L}_1 \times \kvec{a}_{R21} \right)
\end{align}
حاصل ہوتا ہے۔یاد رہے کہ کسی بھی نقطے پر برقی رو سے پیدا مقناطیسی میدان حاصل کرتے وقت ضروری ہے کہ پورے تار پر تکمل حاصل کیا جائے۔مندرجہ بالا مساوات میں نقطہ \عددیء{N_2} پر مکمل تکمل لیتے ہوئے میدان \عددیء{\kvec{H}_2} استعمال نہیں کیا گیا بلکہ تفرقی میدان \عددیء{\dif \kvec{H}_2} استعمال کیا گیا ہے۔یوں اگر اس مساوات سے قوتیں حاصل کی جائیں تو یہ درست نہیں ہوں گی۔یہ دیکھنے کے لئے تصور کریں کہ نقطہ \عددیء{(1,2,3)} پر \عددیء{I_1 \dif \kvec{L}_1=2\ay \si{\ampere \meter}} جبکہ نقطہ \عددیء{(-1,3,2)} پر \عددیء{I_2 \dif \kvec{L}_2=-4\az \si{\ampere \meter}} پایا جاتا ہے۔دوسرے نقطے پر قوت حاصل کرتے ہیں۔یہاں \عددیء{\kvec{R}_{21}=-2\ax+\ay-\az} ہے لہٰذا دوسرے تار پر قوت
\begin{align*}
\dif \left(\dif \kvec{F}_2 \right)&= \frac{4 \pi 10^{-7}}{4\pi {\left(2^2+1^1+1^2\right)}^\frac{3}{2}} (-4\az) \times \left[(2\ay) \times \left( -2\ax+\ay+2\az \right) \right]\\
&=-108.86 \ay \, \si{\nano \newton}
\end{align*}
ہو گا۔اب بالکل اسی طرح حل کرتے ہوئے پہلے نقطے پر
\begin{align*}
\dif \left(\dif \kvec{F}_1 \right)&= \frac{4 \pi 10^{-7}}{4\pi {\left(2^2+1^1+1^2\right)}^\frac{3}{2}} (2\ay)  \times \left[(-4\az)\times \left( 2\ax-\ay-2\az \right) \right]\\
&=54.4 \az \, \si{\nano \newton}
\end{align*}
 قوت حاصل ہوتی ہے جہاں \عددیء{\kvec{R}_{12}=-\kvec{R}_{21}} استعمال کیا گیا۔آپ کو یاد ہو گا کہ  چھوٹے سے چھوٹے مقدار کے دو چارجوں کے مابین ہر صورت قیمت میں برابر اور سمت میں الٹ قوتیں پائی جاتی ہیں۔مقناطیسی میدان میں ایسا نہیں ہے اور برقی رو گزارتے دو چھوٹے حصوں پر نا تو قوت کی قیمتیں برابر ہیں اور نا ہی ان کی سمتوں کا آپس میں کوئی تعلق ہے۔یہاں یہ سمجھ لینا ضروری ہے کہ مقناطیسی میدان میں مکمل بند دور حل کرتے ہوئے ہی صحیح جوابات حاصل ہوتے ہیں لہٰذا ایسا ہی کرتے ہیں۔

مساوات \حوالہ{مساوات_امالہ_تفرقی_تفرقی_قوت} کا دو درجی تکمل لیتے ہوئے
\begin{gather}
 \begin{aligned}\label{مساوات_امالہ_تاروں_کے_مابین_قوت}
\kvec{F}_2 &=\mu_0 \frac{I_1 I_2}{4\pi } \oint\left[\dif \kvec{L}_2 \times \oint\frac{\dif \kvec{L}_1 \times \kvec{a}_{R21}}{R_{21}^2}\right]\\
&=\mu_0 \frac{I_1 I_2}{4\pi } \oint\left[ \oint\frac{ \kvec{a}_{R21}\times \dif \kvec{L}_1}{R_{21}^2}\right] \times \dif \kvec{L}_2
\end{aligned}
\end{gather}
حاصل ہوتا ہے۔

مندرجہ بالا مساوات میں اندرونی تکمل نقطہ \عددیء{N_2} پر مقناطیسی میدان حاصل کرنے کے لئے درکار ہے جبکہ بیرونی تکمل اسی نقطے پر تار پر کل قوت حاصل کرنے کے لئے درکار ہے۔

\حصہ{قوت اور مروڑ}
مساوات \حوالہ{مساوات_امالہ_لورنز_تتکمل_پ}  مقناطیسی میدان میں برقی رو گزارتے تار پر قوت دیتا ہے جسے یکساں میدان میں \عددیء{\kvec{B}} کو تکمل کے باہر لے جاتے ہوئے
\begin{align*}
\kvec{F}=-\kvec{B} \times \oint \dif \kvec{L}
\end{align*}
 لکھا جا سکتا ہے۔اب کوئی بھی برقی دور مکمل بند دائرہ بناتا ہے۔کسی بھی شکل کے بند دائرے  کا لکیری تکمل \عددیء{\oint \dif \kvec{L}=0} ہوتا ہے لہٰذا یکساں میدان میں برقی دور کے پورے تار پر کل صفر قوت پایا جائے گا۔البتہ اگر میدان یکساں نہ ہو تب ضروری نہیں کہ پورے دور پر قوت صفر ہو۔

مساوات \حوالہ{مساوات_امالہ_لورنز_تتکمل_ت} اور مساوات \حوالہ{مساوات_امالہ_لورنز_تتکمل_ٹ} کے برقی رو کو بھی متعدد متوازی جڑے باریک تار نما ٹکڑوں میں تقسیم کیا جا سکتا ہے۔ایسے ہر باریک تار پر بھی یکساں میدان میں صفر قوت ہو گا لہٰذا ان اشکال کے برقی رو کے ادوار پر بھی کل صفر قوت ہی پایا جائے گا۔
\begin{figure}
\centering
\includegraphics{figInductanceTorqueDefined}
\caption{قوت کا معیار اثر۔}
\label{شکل_امالہ_قوت_کا_معیار_اثر_تعریف}
\end{figure}

یکساں میدان میں پورے دور پر صفر قوت پایا جاتا ہے البتہ دور پر \اصطلاح{مروڑ}\فرہنگ{مروڑ}\حاشیہب{torque}\فرہنگ{torque} یعنی \اصطلاح{قوت کا معیار اثر}\فرہنگ{قوت!معیار اثر}\فرہنگ{معیار اثر!قوت}\حاشیہب{moment of force}\فرہنگ{force!moment of}\فرہنگ{moment!of force} عموماً صفر نہیں ہوتا۔قوت کا معیار اثر حاصل کرنے کی خاطر قوت اور مروڑ کے \اصطلاح{محور} یعنی \اصطلاح{چُول}\فرہنگ{چُول}\حاشیہب{pivot}\فرہنگ{pivot} کا جاننا ضروری ہے۔شکل \حوالہ{شکل_امالہ_قوت_کا_معیار_اثر_تعریف}-الف میں نقطہ \عددیء{N} پر قوت \عددیء{\kvec{F}} عمل کر رہا ہے۔ہم نقطہ \عددیء{M} کو محور چنتے ہیں۔ نقطہ \عددیء{M} سے \عددیء{N} تک سمتی فاصلہ \عددیء{\kvec{R}} قوت کا \اصطلاح{بازو}\فرہنگ{قوت!کا بازو}\فرہنگ{بازو!قوت}\حاشیہب{moment arm}\فرہنگ{moment!arm}  کہلاتا ہے۔قوت کا معیار اثر \عددیء{\kvec{T}}
\begin{align}
\kvec{T}=\kvec{R} \times \kvec{F}
\end{align}
کے برابر ہے۔مروڑ کی قیمت، قوت کے بازو کی لمبائی ضرب قوت کی قیمت ضرب ان دو کے مابین زاویے کے سائن کے برابر ہے جبکہ اس کی سمت دونوں کے عمودی ہے جسے صلیبی ضرب سے حاصل کیا جا سکتا ہے۔

شکل \حوالہ{شکل_امالہ_قوت_کا_معیار_اثر_تعریف}-ب میں  پختہ شکل کے جسم پر دو مختلف نقطوں پر برابر مگر الٹ سمت کے قوت لاگو کئے گئے ہیں۔چونکہ اس جسم پر کل قوت صفر کے برابر ہے لہٰذا یہ کسی بھی سمت میں سیدھی حرکت نہیں کرے گی۔محور \عددیء{M} پر ان قوتوں کے مروڑ کا مجموعہ
\begin{align*}
\kvec{T}&=\kvec{R}_1 \times \kvec{F}_1+\kvec{R}_2 \times \kvec{F}_2\\
&=(\kvec{R}_1-\kvec{R}_2) \times \kvec{F}_1\\
&=\kvec{R}_{12} \times \kvec{F}_1
\end{align*}
ہو گا جہاں دوسرے قدم پر  \عددیء{\kvec{F}_2=-\kvec{F}_1} پر کیا گیا ہے۔اس مساوات میں قوتوں کے محور کا \عددیء{\kvec{R}_{12}} پر کوئی اثر نہیں ہے لہٰذا کل قوت صفر ہونے کی صورت میں مروڑ کی قیمت محور پر منحصر نہیں ہے۔اسی عمل کو زیادہ قوتوں پر بھی لاگو کیا جا سکتا ہے۔

چونکہ مروڑ کی قیمت محور پر منحصر نہیں ہے لہٰذا ہم محور اس مقام پر چن سکتے ہیں جس پر مروڑ کا حصول زیادہ آسان ہو۔ہم سطحی قوتوں کی صورت میں ایسا محور عموماً قوتوں کے ہم سطحی ، جسم  کے دھرے پر پایا جاتا ہے۔

\begin{figure}
\centering
\includegraphics{figInductanceTorqueOnSmallLoop}
\caption{مقناطیسی میدان میں برقی رو گزارتے تفرقی بند دائرے پر مروڑ۔}
\label{شکل_امالہ-تفرقی_دائرے_پر_مروڑ}
\end{figure}

آئیں شکل \حوالہ{شکل_امالہ-تفرقی_دائرے_پر_مروڑ} میں دئے  برقی رو گزارتے تار پر غیر یکساں مقناطیسی میدان \عددیء{\kvec{B}=B_x\ax+B_y \ay+B_z \az} میں مروڑ حاصل کریں۔تصور کریں کہ تار چول \عددیء{M} پر صرف گھوم سکتا ہے۔ اس تار کے اطراف \عددیء{\dif x} اور \عددیء{\dif y} ہیں جبکہ اس میں برقی رو \عددیء{I} کی سمت تیر کے نشان سے ظاہر کی گئی ہے۔اس چھوٹے رقبے کے وسط \عددیء{M} پر مقناطیسی میدان
\begin{align}
\kvec{B}_0=B_{x0}\ax+B_{y0}\ay+B_{z0}\az
\end{align}
کے برابر ہے۔ یوں وسط سے \عددیء{-\tfrac{\dif y}{2}} جانب نقطہ \عددیء{1} پر مقناطیسی میدان ٹیلر تسلسل\فرہنگ{ٹیلر تسلسل}\فرہنگ{Taylor series} سے
\begin{align*}
\kvec{B}_1&=\kvec{B}_0 -\frac{\partial \kvec{B}}{\partial y}\frac{\dif y}{2}+\cdots
\end{align*}
لکھا جا سکتا ہے جہاں تمام تفرق نقطہ \عددیء{M} پر حاصل کئے جاتے ہیں۔صرف ایک درجی تفرق رکھتے ہوئے یوں
\begin{align*}
\kvec{B}_1&=\left(B_{x0}-\frac{\partial B_{x}}{\partial y} \frac{\dif y}{2}\right)\ax+\left(B_{y0}-\frac{\partial B_{y}}{\partial y} \frac{\dif y}{2}\right)\ay+\left(B_{z0}-\frac{\partial B_{z}}{\partial y} \frac{\dif y}{2}\right)\az
\end{align*}
حاصل ہوتا ہے۔یوں راہ کے اس طرف کی  تفرقی لمبائی پر تفرقی قوت
\begin{align*}
\dif \kvec{F}_1=I \dif x \ax \times \kvec{B}_1
\end{align*}
یا
\begin{align*}
\dif \kvec{F}_1&=I \dif x \ax \times \left[\left(B_{x0}-\frac{\partial B_{x}}{\partial y} \frac{\dif y}{2}\right)\ax+\left(B_{y0}-\frac{\partial B_{y}}{\partial y} \frac{\dif y}{2}\right)\ay+\left(B_{z0}-\frac{\partial B_{z}}{\partial y} \frac{\dif y}{2}\right)\az \right]\\
&=I \dif x \left[\left(B_{y0}-\frac{\partial B_{y}}{\partial y} \frac{\dif y}{2}\right)\az-\left(B_{z0}-\frac{\partial B_{z}}{\partial y} \frac{\dif y}{2}\right)\ay \right]
\end{align*}
ہو گی۔اس قوت کا بازو مرکز سے اس طرف کے درمیانے نقطے تک ہو گا یعنی \عددیء{\kvec{R}_1=-\tfrac{\dif y}{2}\ay} لہٰذا اس قوت کا معیار اثر
\begin{align*}
\dif \kvec{T}_1&=\kvec{R}_1 \times \dif \kvec{F}_1\\
&=-\frac{\dif y}{2}\ay \times I \dif x \left[\left(B_{y0}-\frac{\partial B_{y}}{\partial y} \frac{\dif y}{2}\right)\az-\left(B_{z0}-\frac{\partial B_{z}}{\partial y} \frac{\dif y}{2}\right)\ay \right]\\
&=- \frac{I}{2}\left(B_{y0}-\frac{\partial B_{y}}{\partial y} \frac{\dif y}{2}\right)\dif x \dif y \ax
\end{align*}
ہو گا۔

اسی طرح وسط سے \عددیء{+\tfrac{\dif y}{2}} جانب نقطہ \عددیء{3} پر مقناطیسی میدان مکلارن تسلسل سے
\begin{align*}
\kvec{B}_3&=\kvec{B}_0 +\frac{\partial \kvec{B}}{\partial y}\frac{\dif y}{2}+\cdots
\end{align*}
لکھا جا سکتا ہے جہاں تمام تفرق نقطہ \عددیء{M} پر حاصل کئے جاتے ہیں۔صرف ایک درجی تفرق رکھتے ہوئے یوں
\begin{align*}
\kvec{B}_3&=\left(B_{x0}+\frac{\partial B_{x}}{\partial y} \frac{\dif y}{2}\right)\ax+\left(B_{y0}+\frac{\partial B_{y}}{\partial y} \frac{\dif y}{2}\right)\ay+\left(B_{z0}+\frac{\partial B_{z}}{\partial y} \frac{\dif y}{2}\right)\az
\end{align*}
حاصل ہوتا ہے۔یوں راہ کے اس طرف کی  تفرقی لمبائی پر تفرقی قوت
\begin{align*}
\dif \kvec{F}_3=-I \dif x \ax \times \kvec{B}_3
\end{align*}
یا
\begin{align*}
\dif \kvec{F}_3&=-I \dif x \ax \times \left[\left(B_{x0}+\frac{\partial B_{x}}{\partial y} \frac{\dif y}{2}\right)\ax+\left(B_{y0}+\frac{\partial B_{y}}{\partial y} \frac{\dif y}{2}\right)\ay+\left(B_{z0}+\frac{\partial B_{z}}{\partial y} \frac{\dif y}{2}\right)\az\right]\\
&=I \dif x \left[-\left(B_{y0}+\frac{\partial B_{y}}{\partial y} \frac{\dif y}{2}\right)\az+\left(B_{z0}+\frac{\partial B_{z}}{\partial y} \frac{\dif y}{2}\right)\ay \right]
\end{align*}
ہو گی۔اس قوت کا بازو مرکز سے اس طرف کے درمیان تک  یعنی \عددیء{\kvec{R}_3=\tfrac{\dif y}{2}\ay} ہے لہٰذا اس قوت کا معیار اثر
\begin{align*}
\dif \kvec{T}_3&=\kvec{R}_3 \times \dif \kvec{F}_3\\
&=\frac{\dif y}{2}\ay \times I \dif x \left[-\left(B_{y0}+\frac{\partial B_{y}}{\partial y} \frac{\dif y}{2}\right)\az+\left(B_{z0}+\frac{\partial B_{z}}{\partial y} \frac{\dif y}{2}\right)\ay \right]\\
&=-\frac{I}{2}\left(B_{y0}+\frac{\partial B_{y}}{\partial y} \frac{\dif y}{2}\right)\dif x \dif y\ax
\end{align*}
ہو گا۔ 

ان دو قوتوں کے معیار اثر کا مجموعہ
\begin{align*}
\dif \kvec{T}_1+\dif \kvec{T}_3=-IB_{y0}\dif x \dif y\ax
\end{align*}
کے برابر ہے۔بالکل اسی طرح تیسرے اور چھوتے اطراف کے قوتوں کے معیار اثر کا مجموعہ
\begin{align*}
\dif \kvec{T}_2+\dif \kvec{T}_4=IB_{x0}\dif x \dif y\ay
\end{align*}
حاصل ہوتا ہے۔یوں تمام اطراف کے قوتوں کے معیار اثر کا مجموعہ
\begin{align*}
\dif \kvec{T}=I \dif x \dif y \left(B_{x0}\ay-B_{y0}\ax \right)
\end{align*}
حاصل ہوتا ہے۔قوسین میں بند حصے کو صلیبی ضرب کی صورت میں لکھا جا سکتا ہے۔یوں
\begin{align*}
\dif \kvec{T}=I \dif x \dif y \left(\az \times \kvec{B}_0 \right)
\end{align*}
یا
\begin{align}\label{مساوات_امالہ_تفرقی_مروڑ_عمومی}
\dif \kvec{T}=I \dif \kvec{S} \times \kvec{B}
\end{align}
حاصل ہوتا ہے جہاں بند راہ سمتی رقبے \عددیء{\dif \kvec{S}} کو گھیرتی ہے۔مندرجہ بالا مساوات میں کثافت مقناطیسی بہاو \عددیء{\kvec{B}} لکھتے ہوئے زیر نوشت نہیں لکھا گیا۔

بند دائرے میں برقی رو ضرب چھوٹے سمتی رقبے  کا حاصل ضرب \اصطلاح{تفرقی مقناطیسی جفت قطب کے معیار اثر}\فرہنگ{معیار اثر!تفرقی مقناطیسی جفت قطب}\فرہنگ{جفت قطب!معیار اثر}\حاشیہب{differential magnetic dipole moment}\فرہنگ{moment!magnetic dipole} \عددیء{\dif \kvec{m}} کی تعریف ہے جس کی اکائی \عددیء{\si{\ampere \meter \squared}} ہے۔یوں
\begin{align}\label{مساوات-امالہ_مقناطیسی_جفت_قطب_معیار_اثر}
\dif \kvec{m}=I \dif \kvec{S}
\end{align}
اور
\begin{align}\label{مساوات_امالہ_تفرقی_مروڑ}
\dif \kvec{T} =\dif \kvec{m} \times \kvec{B}
\end{align}
لکھے جا سکتے ہیں۔

مساوات \حوالہ{مساوات_امالہ_تفرقی_مروڑ_عمومی}، مساوات \حوالہ{مساوات-امالہ_مقناطیسی_جفت_قطب_معیار_اثر} اور مساوات \حوالہ{مساوات_امالہ_تفرقی_مروڑ} عمومی مساوات ہیں جن میں چھوٹا رقبہ \عددیء{\dif \kvec{S}} مربع کے علاوہ کسی بھی شکل کا ہو سکتا ہے اور اس کی سمت کچھ بھی ہو سکتی ہے۔

غیر یکساں مقناطیسی میدان کی صورت میں تار پر کل قوت صفر نہیں ہو گی۔

\begin{figure}
\centering
\includegraphics{figInductanceTorqueOnElectricDipole}
\caption{برقی جفت قطب پر برقی میدان میں مروڑ۔}
\label{شکل_امالہ_برقی_جفت_قطب_مروڑ}
\end{figure}

شکل \حوالہ{شکل_امالہ_برقی_جفت_قطب_مروڑ} میں برقی میدان میں برقی جفت قطب دکھایا گیا ہے ۔مثبت چارج پر قوت \عددیء{\kvec{F}_1=Q \kvec{E}} اور منفی چارج پر قوت   \عددیء{\kvec{F}_2=-Q\kvec{E}} ہے۔ آپ دیکھ سکتے ہیں کہ اس جفت قطب پر  تفرقی مروڑ
\begin{align*}
\dif \kvec{T}&=\dif \kvec{L} \times Q \kvec{E}\\
&=\dif \kvec{p} \times \kvec{E}
\end{align*}
کے برابر ہے جہاں \عددیء{\dif \kvec{p}=Q \dif \kvec{L}} برقی جفت قطب ہے۔مروڑ کی سمت صفحہ کے اندر جانب کو ہے۔ آپ نے دیکھا کہ مقناطیسی اور برقی جفت قطب پر مروڑ کے مساوات یکساں ہیں۔بالکل مقناطیسی جفت قطب کی طرح یہاں بھی مروڑ کا تخمینہ لگاتے وقت جفت قطب کے احاطے میں میدان \عددیء{\kvec{E}} کے تبدیلی کو نظر انداز کیا جا سکتا ہے۔

%===============
\ابتدا{مثال}
شکل \حوالہ{شکل_امالہ-تفرقی_دائرے_پر_مروڑ} میں چھوٹے رقبے کو اتنا چھوٹا تصور کریں کہ اس پر مقناطیسی میدان یکساں تصور کرنا ممکن ہو۔ایسی صورت میں تفرقی مروڑ حاصل کریں۔

حل:یکساں میدان کی صورت میں
\begin{align*}
\dif \kvec{F}_1&=I \dif x \ax \times \left(B_{x0} \ax+B_{y0}\ay+B_{z0}\az\right)\\
&=I \dif x \left(B_{y0}\az-B_{z0}\ay \right)
\end{align*}
اور
\begin{align*}
\dif \kvec{T}_1&=-\frac{\dif y}{2}\ay \times I \dif x \left(B_{y0}\az-B_{z0}\ay \right)\\
&=-\frac{I}{2} \dif x \dif y B_{y0}\ax
\end{align*}
حاصل ہوتے ہیں۔اسی طرح
\begin{align*}
\dif \kvec{F}_3&=-I \dif x \ax \times \left(B_{x0} \ax+B_{y0}\ay+B_{z0}\az\right)\\
&=I \dif x \left(-B_{y0}\az+B_{z0}\ay \right)
\end{align*}
اور
\begin{align*}
\dif \kvec{T}_3&=\frac{\dif y}{2}\ay \times I \dif x \left(-B_{y0}\az+B_{z0}\ay \right)\\
&=-\frac{I}{2} \dif x \dif y B_{y0}\ax
\end{align*}
حاصل ہوتے ہیں۔یوں
\begin{align*}
\dif \kvec{T}_1+\dif \kvec{T}_3=-I \dif x \dif y B_{y0}\ax
\end{align*}
حاصل ہوتے ہیں۔اسی طرح
\begin{align*}
\dif \kvec{T}_2+\dif \kvec{T}_4=I\dif x \dif y  B_{x0}\ay
\end{align*}
حاصل ہوتا ہے۔ان نتائج سے کل مروڑ
\begin{align*}
\dif \kvec{T}=I \dif x \dif y  \left(B_{x0}\ay -B_{y0}\ax\right) 
\end{align*}
ہی حاصل ہوتا ہے۔
\انتہا{مثال}
%================

مندرجہ بالا مثال سے ثابت ہوتا ہے کہ غیر یکساں مقناطیسی میدان کی صورت میں مروڑ حاصل کرتے وقت چھوٹے رقبے پر میدان کی تبدیلی کو نظر انداز کیا جا سکتا ہے۔اس مثال سے یہ بھی ظاہر ہے کہ یکساں مقناطیسی میدان میں تار پر کل قوت صفر کے برابر ہوتی ہے۔ اگر مقناطیسی میدان حقیقت میں یکساں ہی ہو تب کسی بھی بڑے رقبے پر بھی مروڑ بالکل اسی مساوات
\begin{align}
\kvec{T}=I \kvec{S} \times \kvec{B}=\kvec{m} \times \kvec{B} \quad \quad \textrm{\RL{یکساں مقناطیسی میدان}}
\end{align}
سے حاصل ہو گا۔

غور کرنے سے معلوم ہوتا ہے کہ برقی رو گزارتے بند دائرے پر مروڑ اس سمت میں دائرے کو گھمانے کی کوشش کرتا ہے جس میں دائرے سے پیدا مقناطیسی میدان اور بیرونی لاگو مقناطیسی میدان کی سمتیں ایک ہی ہوں۔اس حقیقت کو شکل \حوالہ{شکل_امالہ_مروڑ_مقناطیسی_میدان_متوازی} کی مدد سے یاد رکھا جا سکتا ہے جہاں برقی رو گزارتے تار کی جگہ چھوٹا مقناطیس بیرونی میدان میں دکھایا گیا ہے۔چھوٹا مقناطیس اس سمت میں گھومتا ہے جہاں دونوں میدان متوازی ہوں۔
\begin{figure}
\centering
\includegraphics{figInductanceTorqueOnSmallMagnet}
\caption{مروڑ دونوں مقناطیسی میدان کو متوازی بنانے کی کوشش کرتا ہے۔}
\label{شکل_امالہ_مروڑ_مقناطیسی_میدان_متوازی}
\end{figure}

\حصہ{طاقتور مقناطیسی اشیاء اور مقناطیسی خطے}
شکل \حوالہ{شکل_امالہ_ایٹم_میں_گھومتا_الیکٹران} میں ایٹمی مرکز کے گرد مدار پر گھومتا الیکٹران دکھایا گیا ہے۔حرکت کرتا چارج برقی رو پیدا کرتا ہے۔الیکٹران منفی ہونے کی وجہ سے یہ   برقی  رو  \عددیء{\kvec{v}} کے الٹ سمت میں ہو گا۔اس الیکٹران کو بند گول دائرے پر برقی رو تصور کیا جا سکتا ہے جو مقناطیسی جفت قطب \عددیء{\kvec{m}} کو جنم دیتا ہے۔ایٹمی مسائل صرف \اصطلاح{کوانٹم میکانیات}\فرہنگ{کوانٹم میکانیات}\حاشیہب{quantum mechanics}\فرہنگ{quantum mechanics} سے ہی سمجھے جا سکتے ہیں۔یہاں صرف اتنا بتانا ضروری ہے کہ لوہا، نِکل\فرہنگ{نکل}\حاشیہب{nickel}\فرہنگ{nickel} اور کوبالٹ \فرہنگ{کوبالٹ}\حاشیہب{cobolt}\فرہنگ{cobolt} ایسے  عنصر ہیں جن کا  \عددیء{\kvec{m}} قدر زیادہ قیمت رکھتا ہے۔یہ اشیاء \اصطلاح{طاقتور مقناطیسی اشیاء}\فرہنگ{طاقتور مقناطیسی اشیاء}\حاشیہب{ferromagnetic}\فرہنگ{ferromagnetic} کہلاتے ہیں۔ہم انہیں اشیاء پر غور کرتے ہیں۔

طاقتور مقناطیسی اشیاء میں ایٹموں کے باہمی قوتوں کی وجہ سے  قریبی جفت قطب ایک ہی سمت میں رخ کر لیتے ہیں۔ایسے \اصطلاح{ہم صف}\فرہنگ{ہم صف}\حاشیہب{aligned}\فرہنگ{aligned} خطوں میں متعدد ایٹم شامل ہوتے ہیں۔ان خطوں کو \اصطلاح{مقناطیسی خطے}\فرہنگ{مقناطیسی!خطے}\حاشیہب{magnetic domain}\فرہنگ{domain} کہتے ہیں۔مقناطیسی خطے مختلف شکل کے ہو سکتے ہیں اور ان کی جسامت ایک مائیکرو میٹر تا  کئی سنٹی میٹر ممکن ہے۔کسی بھی قدرتی مقناطیسی شہ میں انفرادی مقناطیسی خطے کے مقناطیسی جفت قطب کا معیار اثر انتہائی بڑی مقدار کا ہوتا ہے  البتہ مختلف مقناطیسی خطوں کے جفت قطب کے رخ مختلف سمتوں میں ہوتے ہیں۔اسی وجہ سے پورا حجم از خود کوئی مقناطیسی معیار اثر نہیں رکھتا۔ہاں بیرونی مقناطیسی میدان \عددیء{\kvec{B}_0} لاگو کرنے سے  وہ مقناطیسی خطے جو  \عددیء{\kvec{B}_0} کے  ہی سمت میں رخ کئے ہوں کا حجم بڑھ جاتا ہے جبکہ بقایا مقناطیسی خطوں کا حجم کم ہو جاتا ہے۔یوں اندرونی مقناطیسی میدان بیرونی میدان سے کئی گنا بڑھ جاتا ہے۔بیرونی میدان ہٹا دینے سے تمام مقناطیسی خطے اپنی پرانی صورت اختیار نہیں کر پاتے۔یوں تمام مقناطیسی خطوں کا مجموعی بقایا مقناطیسی معیار اثر رہ جاتا ہے۔یہ حقیقت کہ مقناطیسی اشیاء کے خصوصیات گزشتہ حالات پر منحصر ہے،  \اصطلاح{مقناطیسی چال}\فرہنگ{مقناطیسی!چال}\حاشیہب{hysteresis}\فرہنگ{hysteresis}  کہلاتا ہے۔  
\begin{figure}
\centering
\includegraphics{figInductanceOrbitingElectronInExternalField}
\caption{مدار میں گھومتے الیکٹران کے مقناطیسی جفت قطب کے معیار اثر کو بیرونی میدان کے متوازی دکھایا گیا ہے۔}
\label{شکل_امالہ_ایٹم_میں_گھومتا_الیکٹران}
\end{figure}

\حصہ{مقناطیسیت اور مقناطیسی مستقل}
آئیں پہلے مقناطیسیت \عددیء{\kvec{M}} کی تعریف کریں۔تصور کریں کہ کسی مادے کے اکائی حجم میں \عددیء{n} مقناطیسی جفت قطب پائے جاتے ہوں۔ 
